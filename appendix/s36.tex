\documentclass[../main]{subfiles}
\begin{document}

\section{A Flatness Theorem}\label{sec:36}

\begin{parlemma}
\label{lem:36.01}
Let $A$ be a ring and $M$ be a $A$-module. Let $x$ be an element of $A$ which is $M$-regular and $A$-regular, and $N$ be an $A$-module with $xN=0$. Put $A' = A/xA$ and $M = M/xM$. Then:
\begin{enumerate}
    \item[(1)] $\Tor^{A'}_n(M,N) \simeq \Tor^{A'}_n(M',N)$ for all $n\ge 0$,
    \item[(2)] $\Ext^n_A(M,N) \simeq \Ext^n_{A'} (M', N)$ for all $n\ge 0$,
    \item[(3)] $\Ext^{n+1}_A(N,M)\simeq \Ext^{n+1}_{A'}(N,M')$ for all $n\ge 0$, and $\Hom_A(N,M) = 0$.
\end{enumerate}
\end{parlemma}
\begin{proof}\phantom{,}
\begin{enumerate}
    \item[(1) and (2)] The exact sequence \[0\longrightarrow A\varrightarrow{X} A\longrightarrow A\longrightarrow A'\longrightarrow 0\] is a free resolution of $A'$. Since \[0\longrightarrow M\varrightarrow{X} M\longrightarrow M\otimes_A A'\longrightarrow 0\] is also exact, we have $\Tor^A_i(M,A')=0$ for all $i>0$. Let $L_\bullet\longrightarrow M \longrightarrow 0$ be a free resolution of $M$. Since \[H_i(L_\bullet\otimes_A A') = \Tor^A_i(M,A') = 0\for{i>0},\] $L_\bullet\otimes_A A'$ is a free resolution of the $A'$-module $M'$. Now (1) and (2) are immediate. 
    \item[(3)]$\Hom_A(N,M)$ is obvious. For $n\ge 0$, put $T^n(N) = \Ext^{n+1}_A(N,M)$ and view them as functors on $A'$-modules. From \[0\longrightarrow M\varrightarrow{X} M \longrightarrow M'\longrightarrow 0\] we get $T^0(N) = \Hom_{A'}(N,M')$. Since $\ProjDim_A A' = 1$ we have $T^n(A') = 0$ for $n>0$, hence $T^n(N) = 0$ for $n>0$ if $N$ is projective over $A'$. If \[0\longrightarrow N'\longrightarrow N\longrightarrow N'' \longrightarrow 0\] is an exact sequence of $A'$-modules, then we have the long exact sequence \[\begin{aligned}0&\longrightarrow& T^0(N'') &\longrightarrow& T^0(N)&\longrightarrow& T^0(N')&\longrightarrow& T^1(N'')&\\ &\longrightarrow& T^1(N)&\longrightarrow& T^1(N')&\longrightarrow& T^2(N'')&\longrightarrow& \cdots.&\end{aligned}\]
    Thus $T^i(-)$ are the derived functors of $\Hom_{A'} (-, M')$, i.e. \[T^i(-) = \Ext^i_{A'}(-,M').\]
\end{enumerate}

\end{proof}

\newparagraph
Let $(A,\ideal{m})$ and $(B,\ideal{n})$ be Noetherian local rings and $\phi:A\varrightarrow{}B$ be a local homomorphism. Put $F = B/\ideal{m}B$. If $B$ is flat over $A$, we have \newline $\dim B = \dim A + \dim F$ by Th.~\ref{thm:019}. The converse is also true in some cases. (Cf. Th.~\ref{thm:046}.)

\begin{theorem}
\label{thm:081}
    Let the notation be as above. Assume that $A$ is regular, $B$ is Cohen-Macaulay and $\dim B = \dim A + \dim F$. Then $B$ is flat over $A$.
\end{theorem}
\begin{proof}
    Induction on $\dim A$. If $\dim A = 0$ then $A$ is a field. Suppose $\dim A>0$ and take $x\in \ideal{m}\setminus \ideal{m}^2$. Put $A' = A/xA$, $B'=B/xB$. Then \[\dim B' \le \dim A' + \dim F = \dim A - 1 + \dim F = \dim B - 1\] by Th.~\ref{thm:019}, but $\dim B'\ge \dim B - 1$ (by \ref{12.F}, or consider system of parameters over $B'$). Therefore $\dim B' = \dim B - 1$, $x$ is $B$-regular, and $B'$ is CM. Hence $B'$ is flat over $A'$ by induction hypothesis, and so $\Tor^{A'}_1(A/\ideal{m}, B)=0$. Since $x$ is $A$-regular and $B$-regular, we have $\Tor^A_1(A/\ideal{m}, B) = \Tor^{A'}_1(A/\ideal{m}, B') = 0$. Therefore $B$ is flat over $A$ by Th.~\ref{thm:049}. (Cf. \cite{egaIV} (6.1.5).)
\end{proof}

\end{document}