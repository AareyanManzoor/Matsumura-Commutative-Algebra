\documentclass[../main]{subfiles}
\begin{document}

\section{Jacobian Criteria and Excellent Rings}\label{sec:40}
\newparagraph Let $A$ be a ring and let $x_1, \ldots, x_r \in A$, $D_1, \ldots, D_s \in \Der(A)$. We shall denote the Jacobian matrix $(D_i x_j)$ by $J(x_1, \ldots, x_r; D_1, \ldots, D_s)$. If $P$ is an ideal of $A$, we shall write $J(x_1, \ldots, x_r; D_1, \ldots, D_s)(P)$ for $(D_i x_j \mod P)$. When $P$ is a prime ideal containing the $x$'s, the rank of the above matrix depends on the ideal $I = \sum A x_i$ rather than the elements $x_i$ themselves, so we denote it by $J(I; D_1, \ldots, D_s)(P)$. If $\Delta$ is the set of derivations of $A$ we define $\rank J(I; \Delta)(P)$ to be the supremum of $\rank J(I; D_1, \ldots, D_s)(P)$ when $\{D_1, \ldots, D_s\}$ runs over the set of all finite subsets of $\Delta$. 

When $A$ is an integral domain with quotient field $K$ and $M$ is an $A$-module, by $\rank M$ we understand $\rank_K M \otimes_A K$.

\begin{theorem}\label{thm:094}
Let $(R, \ideal m)$ be a regular local ring, $P$ be a prime ideal of height $r$ and $\Delta$ be a subset of $\Der(R)$. Then: 
\begin{enumerate}
    \item[i)] $\rank J(P; \Delta)(\ideal m) \le \rank J(P; \Delta)(P) \le r$,
    \item[ii)] if $\rank J(f_1, \ldots, f_r; D_1, \ldots, D_r)(\ideal m) = r$ and $f_1, \ldots, f_r \in P$, then $P = (f_1, \ldots, f_r)$ and $R/P$ is regular. 
\end{enumerate}
\end{theorem}

\begin{proof}
\begin{enumerate}[label = \roman*)]
    \item The first inequality is trivial, and the second is a consequence of the fact that $P R_P$ is generated by $r$ elements.
    \item The condition implies that the images of $f_i$'s are linearly independent over $R/\ideal m$ in $\ideal m/\ideal m^2$, hence the $f_i$'s generate a prime ideal of height $r$. Our assertion follows.
\end{enumerate}
\end{proof}

\begin{theorem}\label{thm:095}
Let $R$, $P$ and $\Delta$ be as in the preceding theorem. Then the following two conditions are equivalent: 

\begin{enumerate}[label = (\arabic*)]
    \item $\rank J(P; \Delta)(P) = \Ht P$,
    \item let $Q$ be a prime ideal contained in $P$, then $R_P/QR_P$ is regular iff \newline $\rank J(Q; \Delta)(P) = \Ht Q$.
\end{enumerate}
\end{theorem}

\begin{proof}
(1) is a special case $Q = P$ of (2). Conversely, suppose (1) holds. If $\rank J(Q; \Delta)(P) = \Ht Q$ then $R_P/Q R_P$ is regular by the preceding theorem. If $R_P/Q R_P$ is regular then there exists $f_1, \ldots, f_r \in P$ such that
\begin{itemize}
    \item $(f_1, \ldots, f_r)R_P = PR_P$,
    \item  $(f_1, \ldots, f_s)R_P = Q R_P$,
    \item  $r = \Ht P$,
    \item  $s = \Ht Q$.
\end{itemize}
 Then $\rank J(f_1, \ldots, f_r; \Delta)(P) = r$, and so $\rank J(f_1, \ldots, f_s; \Delta)(P) = s$.
\end{proof}

\newparagraph We shall say that a subfield $k'$ of a field $k$ is \defemph{cofinite} if $[k : k'] < \infty$.

\begin{lemma}
\label{lem:40.1}
Let $k \subseteq K$ be fields of characteristic $p$ and let $F = \{k_\alpha\}_{\alpha \in I}$ be a downwards-directed family of cofinite subfields of $K$ containing $k$. Then the following are equivalent:
\begin{enumerate}
    \item[(1)] $\bigcap\limits_\alpha k_\alpha K^p = k K^p$.
    \item[(2)] The natural map $\Omega_{K/k} \longrightarrow \varprojlim \Omega_{K/k_\alpha}$ is injective.
    \item[(3)] For every finite subset $\{u_1, \ldots, u_n\}$ of $K$ which is $p$-independent over $k$, there exists $k_\alpha \in F$ over which this set is $p$-independent. 
    \item[(4)] There exists a $p$-basis $B$ of $K$ over $k$ such that for each finite subset $F$ of $B$ there exists $k_\alpha \in F$ over which $F$ is $p$-independent. 
\end{enumerate}
\end{lemma}

\begin{proof}\phantom{,}
\begin{implyenumerate}
    \item[$(2) \iff (3)$] easy.
    \item[$(3) \implies (4)$] trivial.
    \item[$(1) \implies (3)$] The proof of \ref{30.C} Lemma~\ref{lem:30.01} applies mutatis mutandis. 
    \item[$(4) \implies (2)$] Let $0 \ne \omega \in \Omega_{K/k}$. Then $\omega = c_1 \dd b_1 + \ldots + c_n \dd b_n\for{b_i \in B,\, 0 \ne c_i \in K}$, and if $b_1, \ldots, b_n$ are $p$-independent over $k_\alpha$ then the image of $\omega$ in $\Omega_{K/k_\alpha}$ is not $0$.
    \item[$(3) \implies (1)$] Suppose $a \not \in k K^p$. Then $a$ is $p$-independent over $k$, therefore it is so over some $k_\alpha$, i.e. $a \not \in k_\alpha K^p$. 
\end{implyenumerate}
\end{proof}

\begin{lemma}
\label{lem:40.2}
Let $k$, $K$ and $F$ be as in lemma~\ref{lem:40.1} and let $L$ be a finitely generated extension over $K$. If $\bigcap\limits_\alpha k_\alpha K^p = k K^p$ holds, then $\bigcap\limits_\alpha k_\alpha L^p = k L^p$ holds also. 
\end{lemma}

\begin{proof}
It suffices to check the 4 cases of \ref{27.A}. 
\begin{enumerate}[label = \roman*)]
    \item If $L = K(t)$ with $t$ transcendental, then \[\bigcap k_\alpha L^p = \bigcap k_\alpha K^p(t^p) = k K^p(t^p) = k L^p\] is obvious. 
    \item If $L$ is separately algebraic over $K$ then a $p$-basis of $K$ over $k$ is also a $p$-basis of $L$ over $k$, and we can use the criterion (4) of Lemma~\ref{lem:40.1}.
    \item $L = K(t)$, $\, t^p = a \in K$, $\, \dd_{K/k} a = 0$. Then \[\Omega_{L/k} = \Omega_{K/k} \otimes L + L \dd t,\text{ and }\Omega_{L/k_\alpha} = \Omega_{K/k_\alpha} \otimes L + L \dd t.\] Therefore $\Omega_{L/k} \longrightarrow \varprojlim \Omega_{L/k_\alpha}$ is injective. 
    \item $L = K(t)$, $\, t^p = a \in K$, $\,\dd_{K/k} a \ne 0$. Then \[\Omega_{L/k} = (\Omega_{K/k} \otimes L)/L \dd_{K/k} a + L \dd t;\] if $B' \subset K$ is such that $\{a\} \cup B'$ is a $p$-basis of $K/k$ and $a \not \in B'$, then $\{t\} \cup B'$ is a $p$-basis of $L/k$. So if $b_1, \ldots, b_m \in B'$ and $\{a, b_1, \ldots, b_m\}$ are $p$-indep. in $K$ over $k_\alpha$, then $\{t, b_1, \ldots, b_m\}$ is $p$-indep. in $L$ over $k_\alpha$. 
\end{enumerate}

\end{proof}

\newparagraph Let $k$ be a field of characteristic $p$, $R = k[[X_1, \ldots, X_n]]$, $P \in \Spec(R)$ and $A = R/P$. Let $y_1, \ldots, y_r\for{r = \dim A}$ be a system of paramters of $A$ and put $B = k[[y_1, \ldots, y_r]]$. Then $A$ is finite over $B$. Let $k'$ be a cofinite subfield of $k$ and put $C' = k'[[y_1^p, \ldots, y_r^p]]$. Since every derivation $D \in \Der(A)$ is continuous (in any ideal-adic topology), we have $\Der_{k'}(A) = \Der_{C'}(A)$, and $A$ is finite over $C'$. Let $L$, $K$, $K'$ denote the quotient fields of $A$, $B$, $C'$. Then it is easy to see that \[\rank \Der_{k'}(A) = (L:K')_p = \rank \Omega_{L/K'},\] and similarly \[\rank \Der_{k'}(B) = (K : K')_p = \rank \Omega_{K/K'}.\] If $E$ is a $p$-basis of $k$ over $k'$ then $E \cup \{y_1, \ldots, y_r\}$ is a $p$-basis of $B$ over $C'$. Therefore $\rank \Omega_{K/K'} = \dim A + (k : k')_p$, and in general we have by Th.\ref{thm:059}
\[
\rank \Der_{k'}(A) = \rank \Omega_{L/K'} \ge \rank \Omega_{K/K'} = \dim A + (k : k')_p.
\]

\begin{theorem}\label{thm:096}
Let $k$, $R$ and $A$ be as above, and let $F = \{k_\alpha\}_{\alpha \in I}$ be a family of cofinite subfields of $k$, directed downwards, such that $\bigcap k_\alpha = k^p$. Then there exists $k_\alpha \in F$ such that, for every cofinite subfield $k'$ of $k_\alpha$, we have
\[
\rank \Der_{k'}(A) = \dim A + (k : k')_p.
\]
\end{theorem}

\begin{proof}
If $L = K$ then the theorem is obvious, so we will prove the existence of $\alpha$ such that $(L : K')_p = (K : K')_p$ for $k' \subseteq k_\alpha$ by induction on $(L : K)$. Suppose that our claim is proved for every proper subfield $L'$ of $L$ containing $K$, and let $L'$ be maximal among such subfields. If $L$ is separable over $L'$ then\newline $\Omega_{L/K'} = \Omega_{L'/K'} \otimes L$ and we are done. So we can suppose $L = L'(t)$, $\, t^p = a \in L'$. Then $a \not \in {L'}^p$. Put $K_\alpha = k_\alpha((y_1^p, \ldots, y_r^p))$. Then \[\bigcap K_\alpha = k^p((y_1^p, \ldots, y_r^p)) = K^p\] by \ref{prop:30.01}, hence $\bigcap K_\alpha {L'}^p = {L'}^p$ by Lemma~\ref{lem:40.2}. Therefore there exists $\alpha$ such that $a \not \in K_\alpha {L'}^p$ and such that $(L' : K')_p = (K : K')_p$ for $k' \subseteq k_\alpha$. Then for $k' \subseteq k_\alpha$ we have $a \not \in K' {L'}^p$, i.e. $\dd_{L'/K'} a \ne 0$, hence \[\Omega_{L/K'} = (\Omega_{L'/K'} \otimes L)/L \dd_{L'/K'} a + L \dd t,\] and so \[\rank \Omega_{L/K'} = \rank \Omega_{L'/K'} = \rank \Omega_{K/K'}.\]
\end{proof}

\begin{theorem}[Nagata]\label{thm:097}
Let $k$ be a field, $R = k[[X_1, \ldots, X_n]]$ and $P \in \Spec(R)$. Then $\rank J(P; \Der(R))(P) = \Ht P$.
\end{theorem}

\begin{proof}
Here we consider only the case $\ch(k) = p$. The case $\ch(k) = 0$ is easier, and we will prove a much more general result soon.

Put $A = R/P$ and $r = \dim A$. By the preceding theorem there exists a cofinite subfield $k'$ of $k$ such that
\[
\rank \Der_{k'}(A) = r + (k : k')_p.
\]
Put $s = (k : k')_p$. If $\{u_1, \ldots, u_s\}$ is a $p$-basis of $k/k'$ then $\{u_1, \ldots, u_s, X_1, \ldots, X_n\}$ is a $p$-basis of $R$ over $k'[[X_1^p, \ldots, X_n^p]]$. Let $\phi : R \longrightarrow A$ denote the natural map and put $X_i = u_{s + i}$, $D_i = \phi \circ \partial/\partial u_i\for{1 \le i \le n + s}$. Then $\Der_k(R, A)$ is a free $A$-module of rank $n + s$ with $D_1, \ldots, D_{n + s}$ as a basis. Let now $\overline D$ be an arbitrary element of $\Der_{k'}(A)$, and put $\overline D(\phi u_i) = \overline {c_i} \in A$. Then $\overline D$ is induced by $D = \sum \overline {c_i} D_i \in \Der_k(R, A)$ in the sense that $\overline D \circ \phi$. The derivation $\overline D$ is determined by $\overline {c_i}\for{1 \le i \le n + s}$, and these must satisfy
\[
\sum_{i = 1}^{n + s} \overline {c_i} D_i(f) = 0 \qquad \text {for each } f \in P.
\]
Conversely, if $\overline {c_i}$ satisfy these linear equations then $D = \sum \overline {c_i} D_i$ induces a derivation of $A$ over $k'$. Therefore \[r + s = \rank \Der_{k'}(A) = n + s - \rank J(P; \Der_{k'}(R))(P),\]
whence we get \[\rank J(P; \Der_{k'}(R))(P) = n - r = \Ht P.\] Since $\rank J(P; \Der(R))(P) \le \Ht P$ by Th.\ref{thm:094}, we are done. 
\end{proof}

\newparagraph Let $(A, \ideal m)$ be a Noetherian complete local ring containing a field. Let $k$ be a coefficient field of $A$ and let $\ideal m = (x_1, \ldots, x_n)$. Putting $R = k[[X_1, \ldots, X_n]]$ we then have $A = R/I$ with some ideal $I$ of $R$. Let $\ideal p = P/I \in \Spec(A)$. If $A_{\ideal p} = R_P/I R_P$ is regular, then $I R_P = Q R_P$ for some $Q \in \Spec(R)$, $Q \subseteq P$, and we have \[\rank J(I; \Der(R))(P) = \Ht Q = \Ht I R_P\] by Th.\ref{thm:095} and Th.\ref{thm:097}. Put $r = \Ht IR_P$ and let $f_1, \ldots, f_r \in I$ and\newline $D_i, \ldots, D_r \in \Der(A)$ be such that $\Det(D_i f_j) \not \in P$. Then \[I R_P = Q R_P = \sum f_i R_P,\] hence there exists $g \in R - P$ such that $IR_g = \sum_1^r f_i R_g$. Put $h = \Det(D_i f_j)$. If $P'/I = {\ideal p}' \in \Spec(A)$ is such that $h g \not \in P'$, then $R_{P'}/I R_{P'} = A_{{\ideal p}'}$ is regular by Th.\ref{thm:094} (note that $I R_{P'}$ is generated by $r$ elements). Thus $\Reg(A)$ is open in $\Spec(A)$, and we proved Cor. \ref{cor:29.02}. 

\begin{partheorem}\label{thm:098}
Let $(A, \ideal m)$ be a Noetherian local \defemph{domain} containing $Q$. Let $k$ be a quasi-coefficient field of $A$, i.e. a subfield of $A$ such that $A/\ideal m$ is algebraic over $k$. Then: 
\[
\rank \Der_k(A) \le \dim A.
\]
\end{partheorem}

\begin{proof}
We will prove that $\Der_k(A)$ is isomorphic to a submodule of $A^n$, where $n = \dim A$. Take a system of parameters $x_1, \ldots, x_n$ of $A$. We claim that the map $\phi : \Der_k(A) \longrightarrow A^n$ defined by $\phi(D) = (D x_1, \ldots, D x_n)$ is injective. Suppose that $D \in \Der_k(A)$ and $D x_1 = \ldots = D x_n = 0$. By continuity $D$ is uniquely extended to the completion $\completion A$. Now $\completion A$ is finite over the subring $k[[x_1, \ldots, x_n]]$, on which $D$ vanishes. Let $a \in A$. As an element of $\completion A$ it satisfies a polynomial relation $f(a) = 0$ with coefficients in $k[[x_1, \ldots, x_n]]$. Choose such a polynomial $f(T)$ of lowest degree. Then $0 = D(f(a)) = f'(a) Da$ and $f'(a) \ne 0$. Since $D a \in A$ and since the non-zero elements of $A$ are not zero divisors in $\completion A$, we must have $D a = 0$. Thus $D = 0$. 
\end{proof}

\begin{theorem}\label{thm:099}
Let $(R, \ideal m)$ be a regular local ring of dimension $n$ containing a field. Let $\completion R$ be the completion of $R$ and $k$ be a coefficient field of $\completion R$ containing a quasi-coefficient field $k_0$ of $R$. Let $x_1, \ldots, x_n$ be a regular system of parameters of $R$. Then $\completion R = k[[x_1, \ldots, x_n]]$, a formal power series ring over $k$, and $\Der_k(\completion R)$ is a free $\completion R$-module with the partial derivations $\partial/\partial x_1$, $\ldots$, $\partial/\partial x_n$ as a basis. Then the following conditions are all equivalent: 

\begin{enumerate}[label=(\arabic*)]
    \item $\partial/\partial x_i\for{1 \le i \le n}$ map $R$ into $R$, i.e. $\partial/\partial x_i \in \Der_{k_0}(R)$;
    \item there exist $D_1, \ldots, D_n \in \Der_{k_0}(R)$ and $a_1, \ldots, a_n \in R$ such that $D_i a_j = \delta_{ij}$;
    \item there exist $D_1, \ldots, D_n \in \Der_{k_0}(R)$ and $a_1, \ldots, a_n \in R$ such that $\Det(D_i a_j) \not \in \ideal m$;
    \item $\Der_{k_0}(R)$ is a free $R$-module of rank $n$;
    \item $\rank \Der_{k_0}(R) = n$.
\end{enumerate}
\end{theorem}

\begin{remark}
Since $\Der_{k_0}(R) = 0$ we have $\Der_{k_0}(R) = \Der_k(\completion R) \cap \Der(R)$. If we define $\Der_k(R)$ by $\Der_k(\completion R) \cap \Der(R)$ then Th.\ref{thm:098} and Th.\ref{thm:099} hold for any coefficient field $k$ of $\completion R$ and the mention quasi-coefficient field is superfluous. 
\end{remark}

\begin{proof}
Let $K$ and $L$ denote the quotient fields of $R$ and $\completion R$. The implications $(1) \implies (2) \implies (3)$ and $(4) \implies (5)$ are trivial.
\begin{implyenumerate}
    \item[$(3) \implies (4)$] Clearly $D_1, \ldots, D_n$ are linearly independent over $R$ as well as over $\completion R$. So every $D \in \Der_{k_0}(R)$ can be written as $D = \sum c_i D_i$ with $c_i \in L$. Solving the equations $D a_j = \sum c_i D_i a_j$, we get $c_i \in R$. 
    \item[$(5) \implies (1)$] Let $D_1, \ldots, D_n$ be linearly independent over $R$. This means that there exists $a_1, \ldots, a_n \in R$ with $\Det(D_i a_j) \ne 0$. Therefore $D_1, \ldots, D_n$ are linearly independent over $\completion R$ also. Hence $\partial/\partial x_i = \sum_j c_{ij} D_j$ with $c_{ij}$ in $L$. Then $\delta_{ik} = \sum_j c_{ij} D_j x_k$, therefore the matrix $(c_{ij})$ is the inverse $(D_j x_k)$ and so $c_{ij} \in K$. Therefore \[(\partial/\partial x_i)(R) \subseteq K \cap \completion R = R.\]
\end{implyenumerate}
\end{proof}

\newparagraph We will say that \hyperref[40.F]{(WJ)} ($=$ weak Jacobian condition)\index{WJ weak Jacobian condition@(WJ) (= weak Jacobian condition)} holds in a regular ring $R$ if $\rank J(P; \Der(R))(P) = \Ht P$ for every $P \in \Spec(R)$. The reasoning of \ref{40.D} and Th.\ref{thm:095} show that, if $A$ is a homomorphic image of a regular ring $R$ in which \hyperref[40.F]{(WJ)} holds, then $\Reg(A)$ is open in $\Spec(A)$. For the definition and the theory of the strong Jacobian condition (SJ)\index{SJ@(SJ)}, we refer to our article \cite{MATSUMURA1977279}.

\begin{theorem}\label{thm:100}
Let $(R, \ideal m, K)$ be a regular local ring of dimension $n$ containing a field $k$ of characteristic $0$. Assume that (1) $K$ is algebraic over $k$, and (2) $\rank \Der_k(R) = n$. Then: 
\begin{enumerate}
    \item[i)] \hyperref[40.F]{(WJ)} holds in $R$,
    \item[ii)] if $P \in \Spec(R)$ then every element of $\Der_k(R/P)$ is induced by an element of $\Der_k(R)$,
    \item[iii)] $\rank \Der_k(R/P) = \dim R/P$.
\end{enumerate}
\end{theorem}

\begin{proof}
The argument is essentially the same as in Th.\ref{thm:097}. We use the notation of Th.\ref{thm:099}. Then there exists $D_1, \ldots, D_n \in \Der_k(R)$ and $x_1, \ldots, x_n \in \ideal m$ such that $D_i x_j = \delta_{ij}$, and $\Der_k(R)$ is a free $R$-module with $D_1, \ldots, D_n$ as a basis. Put $A = R/P$ and let $\phi : R \longrightarrow A$ denote the natural map. Then $\Der_k(R, A)$ is a free $A$-module with $\phi \circ D_i\for{1 \le i \le n}$ as a basis. If $\overline D \in \Der_k(R)$, let $c_i \in R$ be such that $\phi(c_i) = \overline D \phi(x_i)$. Then $D = \sum c_i D_i \in \Der_k(R)$ induces $\overline D$ in the sense that $\phi \circ D = \overline D \circ \phi$. Let $(u_1, \ldots, u_n) \in A^n$. Then $\sum u_i \phi \circ D_i$ induces a derivation $D \in \Der_k(A)$ iff $\sum u_i \phi(D_i f) = 0$ for all $f \in P$. Thus
\[
\rank \Der_k(A) = n - \rank J(P; \Der_k(A))(P).
\]
The left hand-side is $\le \dim A = n - \Ht P$ by Th.\ref{thm:098}, and the right-hand side is $\ge n - \Ht P$ by Th.\ref{thm:094}. Therefore we have i) and iii). 
\end{proof}

\begin{theorem}\label{thm:101}
Let $R$ be a regular ring containing $Q$. If \hyperref[40.F]{(WJ)} holds in $R$, then $R$ is excellent.
\end{theorem}

\begin{proof}
Since $R$ is Cohen-Macaulay it is universally catenary. We have already remarked that \hyperref[40.F]{(WJ)} implies the openness of $\Reg(R/P)$ in $\Spec(P/R)$ for every $P \in \Spec(R)$, and as $R$ contains $Q$ this proves that $R$ is J-2 by \ref{thm:073}(3). To prove that $R$ is a $G$-ring we can assume that $R$ is a regular local ring, and we have to show that the formal fibres of $R$ are regular. Let $P$ be a prime ideal of the completion $\completion R$ and put $\ideal p = P \cap R$. Let $r = \Ht \ideal p$. Then there exist $D_1, \ldots, D_r \in \Der(R)$ such that $\rank J(\ideal p, D_1, \ldots, D_r)(\ideal p) = r$. We can extend the derivations $D_i$ to $\completion R$ and view the matrix $J(\ideal p; D_1, \ldots, D_r)(\ideal p)$ as $J(\ideal p \ideal R; D_1, \ldots, D_r)(P)$. On the other hand, we have $\Ht \ideal p \completion R = \Ht \ideal p = r$ by \ref{13.B}. Therefore $\completion {R_P}/{\ideal p}\completion{R_P}$ is regular.
\end{proof}

\begin{theorem}\label{thm:102}
Let $k$ be a field of characteristic $0$, and $R$ be a regular ring containing $k$. Suppose that
\begin{enumerate}[label = (\arabic*)]
    \item for any maximal ideal $\ideal m$ of $R$, the residue field $R/\ideal m$ is algebraic over $k$ and $\Ht \ideal m = n$, and
    \item there exist $D_1, \ldots, D_n \in \Der_k(R)$ and $x_1, \ldots, x_n \in R$ such that $D_i x_j = \delta_{ij}$.
\end{enumerate}
 Then $R$ is excellent. 
\end{theorem}

\begin{proof}
By Th.\ref{thm:100} it is clear that \hyperref[40.F]{(WJ)} holds in $R$. 
\end{proof}

\begin{remark*}
Convergent power series rings over $\bR$ or $\bC$, formal power series rings over a field $k$ of characteristic $0$, and more generally the rings of type \newline $k[X_1, \ldots, X_n][[Y_1, \ldots, Y_m]]$ where $k$ is a field of char.$0$, are examples of regular rings to which the theorem applies. Formal power series rings over a convergent power series ring also belong to the class. On the other hand there are excellent regular rings containing a coefficient field $k$ of char.$0$, such that $\Der_k(R) = 0$.
\end{remark*}

\begin{example}
Let $k$ be a field of char.$0$ and let $f(X)$ be a formal power series such that $f(X)$, $f'(X)$ and $X$ are algebraically independent over $k$ (e.g.\newline $f = \exp(\exp(X))$ will do). Let $f = \sum a_i X^i$, $a_i \in k$, and put \[y_i = \sum_{j = i}^\infty a_j x^{j - i}\for{i = 0, 1, 2, \ldots}.\] Then $y_0 = f(X)$ and $y_i = a_i + X y_{i + 1}$. Put $R = k[X, y_0, y_1, \ldots]$. Then $R/XR = k$, so that $XR$ is a prime ideal. Put $A = R_{XR}$. Since $A$ is a subring of $k[[X]]$ it is $X$-adically separated, so it is a regular local ring of dimension $1$ and $\ch(A) = 0$, hence $A$ is excellent. Its completion $\completion A$ is $k[[X]]$ and $\dd/\dd X$ maps $f$ to $f'$ which is not in $k(X, y_0)$, hence not in $A$. By Th.\ref{thm:099} we see that $\Der_k(A) = 0$. 
\end{example}

\begin{theorem}\label{thm:103}
Let $R$ be a regular ring. If \hyperref[40.F]{(WJ)} holds in $R[X_1, \ldots, X_n]$ for every $n \ge 0$, then $R$ is excellent. 
\end{theorem}

\begin{proof}
The condition implies that $\Reg(B)$ is open in $\Spec(B)$ for every finitely generated $R$-algebra $B$, i.e. that $R$ is J-2. To prove that $R$ is a $G$-ring we may assume that $R$ is local, and we have to prove that the formal fibres are \defemph{geometrically} regular. By \ref{33.E} Lemma~\ref{lem:33.03}, it suffices to prove that, if $C$ is a localization of a finite $R$-algebra which is a domain, and if $Q$ is a prime ideal of $C^\ast$ such that $Q \cap C = (0)$, then $C_Q^\ast$ is regular. Now $C$ is a homomorphic image of a localization of some $R[X_1, \ldots, X_n]$, and our assertion is proved by the same argument as in the proof of Th.\ref{thm:101}. 
\end{proof}

\begin{remark*}
It is easy to see that, if $R$ contains $Q$, then \hyperref[40.F]{(WJ)} in $R$ implies \hyperref[40.F]{(WJ)} in $R[X]$. But this is not so in the case of characteristic $p$. In fact, the ring $A$ of \ref{34.B} is a counterexample. 
\end{remark*}
\end{document}