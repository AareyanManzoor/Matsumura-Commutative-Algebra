\documentclass[../main]{subfiles}
\begin{document}

\section{Complement}\label{sec:43}

Grothendieck (EGA IV 19.7.1 \cite{egaIV}) proved the following important theorem:
\begin{enumerate}[label=$(43.*)$]
    \item\label{thm:43.star} Let $(A, \ideal m, k)$ and $(B, n, k^{\prime})$ be Noetherian local rings and $\phi: A\longrightarrow B$ be a local homomorphism. Then
    \begin{align*}
        \phi\text{ is formally smooth }\iff& B \otimes k\text{ is formally smooth over }k\text{,}\\
        &\text{ and }\phi\text{ is flat}.
    \end{align*}
\end{enumerate}

The most difficult part is the proof of flatness from formal smoothness. His proof is quite interesting but too long to include in this book.

Let $A$ be a ring, $B$ an $A$-algebra and $L$ a $B$-module. The set of isomorphism classes of extensions of $B$ by $L$ (\S\ref{sec:25}) has a natural structure of $A$-module, which was denoted by $\Exalcom_A(B, L)$ in EGA. The algebra $B$ is smooth over $A$ iff this module is zero for all $B$-modules $L$. When $A$ and $B$ are topological rings Grothendieck defined a variant of the above module, called $\Exalcomtop(B, L)$; $B$ is formally smooth over $A$ iff this last module vanishes for all $L$.

The functor $\Exalcom_A(B, L)$ has certain formal properties, which make it a $1$-dimensional cohomology functor in some sense. So several people tried to construct the higher cohomologies that should follow it. After the partial success of Gerstenhaber, Harrison and others, Michel André succeeded in constructing a satisfactory theory (\cite{andre1967methode}; \cite{andré1974homologie}). Let $A$, $B$ and $L$ be as above. He defines homology modules $H_n(A, B, L)$ and cohomology modules $H^n(A, B, L)$ for all $n \geqslant 0$. We have $H_0(A, B, L)=\Omega_{B / A} \otimes_B L$, $\, H^0(A, B, L)=\Der_A(B, L)$ and $H^1(A, B, L)=\Exalcom_A(B, L)$. When $A\longrightarrow B\longrightarrow C$ is a sequence of ring homomorphisms and $M$ is a $C$-module, we have the following long exact sequences called \defemph{Jacobi-Zariski sequences}:
\[\begin{aligned}
\cdots&\longrightarrow& H_n(A,B,M)&\longrightarrow& H_n(A,C,M)&\longrightarrow& H_n(B,C,M)&\\ &\longrightarrow& H_{n-1}(A,B,M)&\longrightarrow&\cdots\cdots\cdots\cdots&\longrightarrow& H_0(B,C,M)&\longrightarrow& 0,\end{aligned}\]
and
\[\begin{aligned}0&\longrightarrow& H^0(B,C,M)&\longrightarrow&\cdots\cdots\cdots\cdots&\longrightarrow& H^{n-1}(A,B,M) &\\ &\longrightarrow &H^n(B,C,M)&\longrightarrow &H^n(A,C,M)&\longrightarrow& H^n(A,B,M)&\longrightarrow&\cdots.\end{aligned}\]
Let $J$ be an ideal of $B$. The $A$-module $B$ with $J$-adic topology is formally smooth iff $H^1(A, B, W)=0$ for all $B / J$-module $W$. A Noetherian local ring $A$ is excellent iff $H^n(A, \completion{A}, W)=0$ for all $n>0$ and for every $\completion{A}$-module $W$.

André's homology and cohomology are connected with formal smoothness at $n=1$, with regularity at $n=2$ and with complete intersection at $n=3$ (and up). The theorem \ref{thm:43.star} cited above is proved rather naturally in André's theory.

A Noetherian local ring $A$ is called a \defemph{complete intersection}\index{complete intersection (=CI)} (CI for short) if its completion $\completion{A}$ is of the form $R / I$, where $R$ is a regular local ring and $I$ is an ideal generated by an $R$-sequence. This is characterized by $H_3(A, K, K)=0$, where $K$ is the residue field. Using this criterion it is easy to see that if $A$ is CI and $P \in\Spec(A)$, then $A_P$ is CI also. L.L.Avramov (\cite{avramov1975flat}) proved the following theorem using Andre's theory: Let $(A, \ideal m)$ and $B$ be Noetherian local rings and $f: A\longrightarrow B$ be a flat local homomorphism. Then
\begin{enumerate}[label=$(\dagger)$]
    \item\label{thm:43.dagger} $B$ is CI $\implies$ $A$ is CI, $A$ and $B/\ideal B$ are CI $\implies$ $B$ is CI.
\end{enumerate}
André (\cite{andre1974localisation}) proved the following useful theorem:

\begin{enumerate}[label=$(**)$]
    \item\label{thm:43.sstar} Let $f: A\longrightarrow B$ be a local homomorphism of Noetherian local rings. If $f$ is formally smooth and $A$ is excellent, then $f$ is regular.
\end{enumerate}
The question \ref{que:34.B} of \ref{34.C} was recently solved by C. Rotthaus in the case $A$ is semi-local (\cite{rotthaus1979komplettierung}). André's theorem \ref{thm:43.sstar} plays an important role in her proof. In the general case even the problem \ref{que:34.A} is open, but when $A$ is an algebra of finite type over a field Problem \ref{que:34.Aprime} was solved by P. Valabrega (\cite{valabrega1975on}). Later he generalized his result to the case where $k$ is a $1$-dimensional excellent domain of characteristic $0$. (\cite{valabrega1976a}).

L.J.Ratliff (\cite{ratliff1972catenary}) proved the following beautiful theorem: A Noetherian local domain $A$ is catenary iff $\Ht P+\dim R / P=\dim R$ holds for every $P \in\Spec(R)$. He has also characterized universally catenary rings in many different ways. (CF. \cite{ratliff1978chain} for references and for the definitions of his terminology.)

For excellent rings and Nagata rings, see also \cite{greco1976two}, and many articles by K. Langmann (in German Journals) and by H. Seydl (mostly in C.R. Acad. Sci. Paris)\footnote{The archives of C.R. Acad. Sci. Paris can be found in the archives of the France National Library.}. We also note that R.Y. Sharp defined acceptable rings by replacing ``regular'' by ``Gorenstein'' throughout the definition of excellent rings. (\cite{sharp1977acceptable}).

Finally, in connection with our Ch.6 we list a few important recent works: \cite{northcott1976finite}, \cite{peskineszpiro1973dimension}
\cite{hochster1975topics}.

\end{document}