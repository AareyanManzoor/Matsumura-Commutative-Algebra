\documentclass[../main]{subfiles}
\begin{document}

\section{Coefficient Rings}\label{sec:37}

In this section we will prove the Cohen structure theorem (p.\pageref{thm:cohen}) in the unequal characteristic case by the method of Grothendieck. 

\begin{theorem}
\label{thm:082}
    Let $(A,m,k)$ be a local ring and let $B$ be a flat $A$-algebra. Put $B_0 = B/mB = B\otimes_A k$. If $B_0$ is smooth over $k$ then $B$ is formally smooth over $A$ with respect to the $mB$-adic topology.
\end{theorem}
\begin{proof}
    By the definition of formal smoothness we have only to show that $B/\ideal{m}^i B$ is smooth over $A/\ideal{m}^i$ for every $i$. Thus we can assume that $\ideal{m}$ is nilpotent. Then $B$ is free over $A$ by \ref{3.G}, and so any $A$-algebra extension of $B$ by a $B$-module is a Hochschild extension, cf. \ref{25.C}. Therefore the proof of smoothness of $B$ reduces, as in \ref{28.H}, to showing that every symmetric 2-cocycle $f:B\times B\varrightarrow{} N$ with values in a $B$-module $N$ is a coboundary. Suppose first that $N$ satisfies $\ideal{m}N=0$. In this case $f$ is essentially a cocycle on $B_0$; namely, there exists a symmetric 2-cocycle $f_0:B_0\times B_0\varrightarrow{} N$ such that $f(x,y) = f_0(\overline{x},\overline{y})$. Since $B_0$ is smooth over $k$ we have $f_0 = \delta g_0$ for some $k$-linear map $g_0: B_0\varrightarrow{} N$. Putting $g(x) = g_0(\overline{x})$ we have $f=\delta g$. In the general case let $\phi: N\varrightarrow{} N/\ideal{m}N$ denote the natural map. Then \[\phi\circ f: B\times B \varrightarrow{} N/\ideal{m}N\] splits, i.e., there exists an $A$-linear map $\overline{g}: B\varrightarrow{} N/\ideal{m}N$ such that $\phi\circ f = \delta\overline{g}$. As $B$ is projective over $A$ the map $\overline{g}$ can be lifted to an $A$-linear map $g:B\varrightarrow{} N$, and $f-\delta g$ is a 2-cocycle with values in $mN$. Repeating the same argument, we can find $h:B\varrightarrow{} \ideal{m}N$ such that $f-\delta(g+h)$ has values in $\ideal{m}^2 N$, and so on. Since $\ideal{m}$ is nilpotent, we see that $f$ is a coboundary.
\end{proof}

\begin{theorem}
\label{thm:083}
    Let $(A,tA,k)$ be a principal valuation ring and $K$ be an extension field of $k$. Then there exists a principal valuation ring $B$ containing $A$ with maximal ideal generated by $t$ and with residue field $k$-isomorphic to $K$. 
\end{theorem}
\begin{proof}
    Let $\brc{x_\lambda}_{\lambda\in\Lambda}$ be a transcendency basis of $K$ over $k$ and put $k_1 = k(\brc{x_\lambda})$. Let $\brc{X_\lambda}_{\lambda\in\Lambda}$ be a set of independent indeterminates and put $A\sbr{\brc{X_\lambda}}=A'$, $A_1 = A'_{tA'}$. Then $A'$ is a free $A$-module, so that $A'$ and $A_1$ are separated in the $t$-adic topology. Therefore $A_1$ is a principal valuation ring with residue field $k_1$. So we can assume that $K$ is algebraic over $k$. Let $L$ be the algebraic closure of the quotient field of $A$. Let $\mathfrak{F}$ denote the set of the pairs $(B,\phi)$ of a subring $B$ of $L$ containing $A$ and an $A$-algebra homomorphism $\phi: B \varrightarrow{} K$ such that $B$ is a principal valuation ring with $\rad(B) = \Ker(\phi) = tB$, and define an order in $\mathfrak{F}$ by \[(B,\phi) < (C, \psi) \iff B\subset C \text{ and } \phi = \psi\mid B.\] One can easily check that $\mathfrak{F}$ satisfy the condition of Zorn's lemma, therefore there exists a maximal element $(B,\phi)$ in $\mathfrak{F}$. If $\phi(B) \neq K$, take an element $a \in K\setminus\phi(B)$, let $\overline{f}(X)$ be the irreducible equation of $a$ over $\phi(B)$ and lift it to a monic polynomial $f(X) \in B\sbr{X}$. Since $B$ is normal, $f$ is irreducible over the quotient field of $B$. Let $\chi$ be a root of $f$ in $L$ and put $B' = B\sbr{\alpha}$; then $B' = B\sbr{X}/(f)$, so that we have \[B'/tB' = B\sbr{X}/(t,f) = \phi(B)(a).\] Since $B'$ is integral over $B$ all maximal ideals of $B'$ must contain $tB'$, therefore $B'$ is a local ring with $tB'$ as maximal ideal. Clearly $B'$ is a Noetherian domain, so $B'$ must be a principal valuation ring. This contradicts the maximality of $(B,\phi)$ in $\mathfrak{F}$. Thus $\phi(B) = K$. 
\end{proof}
\begin{remark}
\label{rem:37.01}
    If $(A, tA)$ is a principal valuation ring and $M$ is an $A$-module, then $M$ is flat over $A$ iff $t$ is $M$-regular. This is an immediate consequence of \ref{3.A} Th.~\ref{thm:001} (3). In particular the ring $B$ of the above theorem is flat over $A$. 
\end{remark}
\begin{remark}
\label{rem:37.02}
    In \cite{egaIII} (10.3.1) the following more general theorem is proved: if $(A,\ideal{m},k)$ is a Noetherian local ring and $K$ is an extension field of $k$, then one can find a Noetherian local ring $B$ containing $A$ and flat over $A$ such that $\rad(B) = \ideal{m}B$, $B/\ideal{m}B \simeq K$.
\end{remark}
\begin{theorem}
\label{thm:084}
    Let $(A,\ideal{m},K)$ be a complete, separated local ring, $(R,pR,k)$ be a principal valuation ring and $\phi_0: k\varrightarrow{} K$ be a homomorphism of fields. Then there exists a local homomorphism $\phi: R \varrightarrow{} A$ which induces $\phi_0$.
\end{theorem}
\begin{proof}
    Put $S = \bZ_{p\bZ}$ and let $k_0$ be the prime field in $k$. Since $\ch(K) = \ch(k) = p$, the canonical homomorphism $\bZ \varrightarrow{} A$ can be extended to a local homomorphism $S\varrightarrow{} A$. Similarly $R$ is an $S$-algebra, which is flat by Remark \ref{rem:37.01}. Since $R/pR = k$ is separable (hence smooth) over $k_0$, $R$ is formally smooth over $S$ in the $pR$-adic topology by Th.~\ref{thm:082}. Therefore we can lift the map $R\longrightarrow k\longrightarrow K$ to $\phi: R\varrightarrow{}A$. 
    \begin{center}
    \begin{tikzcd}
        R \arrow[rrddd, "\phi", dashed] \arrow[r] & k \arrow[r, "\phi_0"] & K                        \\
                                          &                       & A/{\ideal m}^2 \arrow[u] \\
                                          &                       & \vdots \arrow[u]         \\
        S \arrow[uuu] \arrow[rr]                  &                       & A \arrow[u]             
    \end{tikzcd}
    \end{center}
\end{proof}

\begin{theorem}
\label{thm:085}
    A complete separated local ring has a coefficient ring\index{coefficient!\indexline ring}. (Cf.\ref{thm:cohen})
\end{theorem}
\begin{proof}
    This follows from Th.\ref{thm:083} and Th.\ref{thm:084}.
\end{proof}

\end{document}