%operators
\DeclareMathOperator{\Spec}{Spec} %spectrum
\DeclareMathOperator{\nil}{nil} %set of nilpotents
\DeclareMathOperator{\Hom}{Hom} %Hom
\DeclareMathOperator{\rad}{rad} %Jacobson radical
\newcommand{\length}{\ell} %length
\DeclareMathOperator{\Supp}{Supp} %support
\DeclareMathOperator{\Tor}{Tor} %torsion
\DeclareMathOperator{\Ext}{Ext} %Ext
\DeclareMathOperator{\Aut}{Aut} %Automorphism group
\DeclareMathOperator{\Ass}{Ass} %set of associated primes
\DeclareMathOperator{\Ann}{Ann} %annihilator
\DeclareMathOperator{\rank}{rank}
\DeclareMathOperator{\depth}{depth}
\DeclareMathOperator{\codepth}{codepth}
\DeclareMathOperator{\grade}{grade}
\DeclareMathOperator{\Der}{Der}
\DeclareMathOperator{\Reg}{Reg}
\DeclareMathOperator{\ord}{ord}
\DeclareMathOperator{\Ht}{ht}
\DeclareMathOperator{\gr}{gr}
\DeclareMathOperator{\ProjDim}{proj. dim}
\DeclareMathOperator{\InjDim}{inj. dim}
\DeclareMathOperator{\GlDim}{gl. dim}
\DeclareMathOperator{\TrDeg}{tr. deg}
\DeclareMathOperator{\Ker}{Ker}
\DeclareMathOperator{\Coker}{Coker}
\DeclareMathOperator{\ch}{ch}
\let\Im\relax   %remove original \Im command
\DeclareMathOperator{\Im}{Im}
\let\dim\relax  %remove original \dim command
\DeclareMathOperator{\dim}{dim}
\newcommand{\inv}{^{-1}}
\newcommand{\hilbertd}{\mathrm{d}}
\DeclareMathOperator{\Sing}{Sing}
\DeclareMathOperator{\Tr}{tr}
\DeclareMathOperator{\Det}{Det}
\DeclareMathOperator{\id}{id}
\newcommand{\reduced}[1]{#1_{\text{red}}} % the reduced ring of A defined in chapter 1, page 5 of original book
\DeclareMathOperator{\Nor}{Nor}
\DeclareMathOperator{\DimCoh}{dim.coh}
\DeclareMathOperator{\Exalcom}{Exalcom}
\DeclareMathOperator{\Exalcomtop}{Exalcomtop}

\newcommand{\bZ}{\mathbb{Z}}
\newcommand{\bQ}{\mathbb{Q}}
\newcommand{\bR}{\mathbb{R}}
\newcommand{\bC}{\mathbb{C}}

\providecommand{\dd}[1]{\mathrm{d}#1} %differential
\newcommand{\pdv}[2]{\dfrac{\partial #1}{\partial #2}} %partial derivative
\newcommand{\dv}[2]{\dfrac{\dd #1}{\dd #2}} %regular derivative
\newcommand{\dvn}[3]{\dfrac{\dd^#3 #1}{\dd^#3 #2}} %nth derivative

\newcommand{\ideal}[1]{\mathfrak{#1}} %for lower case ideals, so p,m,a etc.
\newcommand{\SpecInduced}[1]{#1^\ast} %induced maps between spectrum
\newcommand{\completion}[1]{\widehat{#1}} %for I-adic completions of rings.
\newcommand{\canonicalCompletion}{\varphi}
\newcommand{\defemph}[1]{\textbf{#1}}
%\renewcommand{\dots}{\dots}
\renewcommand{\ldots}{\dots}

\newcommand{\indexline}{---\ }



%%use this to overset/underset things to arrows, and in general use \varrightarrow{} for maps 

\makeatletter

\def\@@varrightarrow#1#2#3{\begingroup%
\setbox0=\hbox{$#1\xrightarrow[#3]{#2}$}%
\setbox1=\hbox{$#1\longrightarrow$}%
\ifdim\wd0<\wd1 \mathrel{\mathop{\longrightarrow}\limits^{#2}_{#3}}
\else \xrightarrow[#3]{#2} \fi\endgroup}

\def\@varrightarrow#1#2{\@@varrightarrow#1#2}

% \varrightarrow[under material]{over material} puts material over and under a right arrow.
% It automatically prints the longer option between an xrightarrow and a longrightarrow
\newcommand\varrightarrow[2][]{\mathpalette\@varrightarrow{{#2}{#1}}}

\newcommand{\biga}{\big} %parenthesis sizer
\newcommand{\bigb}{\Big} %parenthesis sizer
\newcommand{\bigc}{\bigg} %parenthesis sizer
\newcommand{\bigd}{\Bigg} %parenthesis sizer
\newcommand{\br}[1]{\left(#1\right)}
\newcommand{\sbr}[1]{\left[#1\right]}
\newcommand{\brc}[1]{\left\{#1\right\}}

\newcommand{\sse}{\subseteq} %lazy rokabe

\newcommand{\ep}{\varepsilon}

\renewcommand{\geq}{\geqslant}
\renewcommand{\ge}{\geqslant}
\renewcommand{\leq}{\leqslant}
\renewcommand{\le}{\leqslant}

%\newcommand{\xoverline}[1]{\mskip.5\thinmuskip\overline{\mskip-.5\thinmuskip {#1} \mskip-.5\thinmuskip}\mskip.5\thinmuskip} % overline short

\providecommand{\for}[1]{
\relax\if@display
    \qquad(#1)
  \else
    \quad(#1)
  \fi}