

\title{COMMUTATIVE ALGEBRA }


\author{HIDEYUKI MATSUMURA}
\date{}


\newcommand\longdiv[1]{\overline{\smash{)}#1}}

\DeclareUnicodeCharacter{2020}{$\dagger$}

\begin{document}
\maketitle
Hideyuki Matsumura,

Professor of Mathematics at Nagoya U University, received his graduate training at Kyoto University and was awarded his Ph.D. in 1959 . Formerly Associate Professor of Mathematics at this university, Professor Matsumura was a $\cdots$ research associate at the University of Pisa during 1962 and 1963. He was also Visiting Associate Professor at the University of Chicago (1962), at Johns Hopkins University (1963), at Columbia University (1966-1967), and at Brandeis University (1967-1968).

The author spent 1973 and 1974 as Visiting Professor at the University of Pennsylvania, 1974 and 1975 as Visiting Professor at the Politecnic of Torino and 1977 as Visiting Professor at the University of Münster.

\section{Commutative Algebra}
This book, based on the author's lectures at Brandeis University in 1967 and 1968 , is designed for use as a textbook on commutative algebra by students of modern algebraic geometry or abstract algebra.

Part I is devoted to basic concepts such as dimension, depth, normal rings, and regular local rings; Part II deals with the finer structure theory of noetherian rings initiated by Zariski and developed by Nagata and Grothendieck.

In this second edition, the chapter on Depth has been completely rewritten.. . There is also a new Appendix consisting of several sections, which are almost independent of each other. The Appendix has two purposes: to prove the theorems used but not proved in the text; to record some of the recent achievements in the areas connected with Part il.

For specialists in commutative algebra, this book will serve as an introduction to the more difficult and detailed books of Nagata and Grothendieck.

To geometers, it will be a convenient handbook of algebra.

\section{Review of the First Edition:}
"This is an excellent book which contains a wealth of material...Part I, for which the prerequisites are minimal, develops the main concepts, centralto modern commutative algebra... Part II, is considerably more advanced.."

\begin{itemize}
  \item American Mathematical Monthly
\end{itemize}
\includegraphics[max width=\textwidth]{2022_08_01_8d4eee36f1f42236b4f4g-001}

Commutative Algel SECONDEDITION Hideyuki Matsumura

BENJAMIN/CUMMINGS PUBLISHING COMP. Advanced Book Program

\section{MATHEMATICS LECTURE NOTE SERIES}
Volumes of the Series published from 1961 to 1973 are not officially numbered. The parenthetical numbers shown are designed to aid librarians and bibliographers to check the completeness of their holdings.

ISBN

\includegraphics[max width=\textwidth]{2022_08_01_8d4eee36f1f42236b4f4g-002}

Algebraic Functions, 1965

Lie Algebras and Lic Groups 1965 (3rd printing, with corrections, 1974)

Sections, 1974) Theory and the Continuum Sypothesis, 1966 (4th prinuine. 1977) 1966 (4th printing.

Rapport sur la cohomologie des groupes, 1966

A complexes, 1966

Class Field Theory, 1967 (2nd printing, 1974)

Characters of Finite Groups,

ectures on Algebraic Topology. 1966 (6th printing,

Foundations of Projective

Geometry, 1967 ( 3 rd printing

A Anglysis, 1968

Pings or Operacors 1968 Induced Representation of Groups and Quantum Mechan-

Abelian I-Adic Representations

Lectures on Analysis (3rd printing

Volume I. Integration and

Topological Vector Spaces, 1969

Volume 11. Representation Theory, 1969

Volume III. Infinite Dimensional

Measures and Problem Solutions,

1969 Cohomology Theories, 1969

\section{MATHEMATICS LECTURE NOTE SERIES (continued)}
$\begin{array}{rr}\text { ISBN } & \\ 0-8053-2420-8 & (28) \\ 0-8053-2421-6 & \\ 0-8053-2570-0 & (29) \\ 0-8053-2571-9 & \\ 0-8053-3080-1 & (30)\end{array}$

with corrections, 1979$)$ R. Ellis Lectures on Topological Dynamics, 1969

Invariant Theory, 1969

Algebraic Curves: An Introduction

o Algebraic Geometry, 1969 (5th printing, with

corrections, 1978)

Lectures on Forms in Many

Variables, 1969

Fourier Analysis on Groups and

Partial Wave Analysis, 1969

Piecewise Linear Topology, 1968

Completely O-Simple Semi-

of the Lattice of Congruences, 1969

Bifurcation Theory and Non-

Bifurcation Theory and Non-

Sinear Eigenvalue Problems, 169

Volume I. General Theory, 1969

Volume I. General Theory, lise Classification, 1969

Classification, 1869

Commutative Algebra, 1970

(2nd Edition-cf. Vol. 56)

Modular Forms and Dirichier

Series, 1969

Entropy and Generators

Ergodic Theory, 1969

Function Theory in Polydiscs, 1969

Celestal Mechanics Pant I, 1969

Celestial Mechanics Part II, 1969

Hopf Algebras, 1969

Lectures in Mathematical Physics

Volume I, 1970

Optimization Theory, 1970

Opiniza Then

Equations, 1971

Generalized Funci

Aeralysis, 1972

Lectures in Mathe

Volume II, 1972

Automorphic For

Groups, 1972

Saturated Model Theory, 1972

Martingate Inequalities: Seminar

Notes on Recent Progress. 1973

The Algebraic Theory of Quadratic

Forms, 1973 (2nd printing, with

revisions, 1980)

in Physics, Probabilisy,

and Number Theory, 197



Nagoya University, Nagoya, Japan

THE BENJAMIN/CUMMINGS PUBLISHING COMPANY, INC.

ADVANCED BOOK PROGRAM

Reading, Massachusetts

London - Amsterdam - Don Mills, Ontario . Sydney - Tokyo Chapter 8. FLATNESS II

\begin{enumerate}
  \setcounter{enumi}{20}
  \item Local Criteria of Flatness

  \item Fibres of Flat Morphisms

  \item Theorem of Generic Flatness

\end{enumerate}
Chapter 9. COMPLETION

\begin{enumerate}
  \setcounter{enumi}{23}
  \item Completion

  \item Zariski Rings

\end{enumerate}
PART II

Chapter 10. DERIVATION

\begin{enumerate}
  \setcounter{enumi}{25}
  \item Extension of a Ring by a Module

  \item Derivations and Differentials

  \item Separability

\end{enumerate}
Chapter 11. FORMAL SMOOTHNESS

\begin{enumerate}
  \setcounter{enumi}{28}
  \item Formal Smoothness I

  \item Jacobian Criteria

  \item Formal Smoothness II

\end{enumerate}
Chapter 12. NAGATA RINGS

\begin{enumerate}
  \setcounter{enumi}{31}
  \item Nagata Rings
\end{enumerate}
Chapter 13. EXCELLENT RINGS

\begin{enumerate}
  \setcounter{enumi}{32}
  \item Closedness of Singular Locu

  \item Formal Libres and $G$-Rings

  \item Excellent Rings

\end{enumerate}
\section{APPENDIX}
\begin{enumerate}
  \setcounter{enumi}{35}
  \item Eakin's Theorem

  \item A Flatness Theorem

  \item Coefficient Rings

  \item p-Basis

  \item Cartier's Equality and Geometric Regularity

  \item Cartier's Equality and Geometric Regular

  \item Jacobian Criteria and Excellent Rings

  \item Krull Rings and Marot's Theorem

  \item Complement

\end{enumerate}
\section{Index}
Index of Symbols 145

152

\section{Preface}
161

172

This book has evolved out of a graduate course in algebra I gave at Brandeis University during the academic year of 1967-1968. At that time M. Auslander taught algebraic geometry to the same group of students, and so I taught commutative algebra for use in algebraic geometry. Teaching a course in geometry and a course in commutative algebra in parallel seems to be a good way to introduce students to algebraic geometry.

Part $I$ is a self-contained exposition of basis concepts such as flatness, dimension, depth, normal rings, and regular local rings.

Part II deals with the finer structure theory of noetherian rings, which was initiated by Zariski (Sur la normalité analytique des variétés normales, Ann. Inst. Fourier 21950 ) and developed by Nagata and Grothendieck. Our purpose is to lead the reader as quickly as possible to Nagata's theory of pseudo-geometric rings (here called Nagata rings) and to Grothendieck's theory of excellent rings. rings (here called Nagata rings) and to Grothendieck's theory of excellent rings. The interested reader should advance to Nagata's book LOCAL RINGS and to Grothendieck's EGA, Ch. IV.

The theory of multiplicity was omitted because one has little to add on this subject to the lucid expositon of Serre's lecture notes (Algèbre locale. Multiplicité, Springer-Verlag).

Due to lack of space some important results on formal smoothness (especially its relation to flatness) had to be omitted also. For these, see EGA.

We assume that the reader is familiar with the elements of algebra (rings, modules, and Galois theory) and of homological algebra (Tor and Ext). Also, it is desirable but not indispensable to have some knowledge of scheme theory.

I thank my students at Brandeis, especially Robin Hur, for helpful comments.

Hideyuki Matsumura

Nagoya, Japan

November 1969 Nine years have passed since the publication of this book, during which time it has been awarded the warm reception of students of algebra and algebraic geometry in the United States, in Europe, as well as in Japan.

In this revised and enlarged edition, I have limited alternations on the original text to the minimum. Only Ch. 6 has been completely rewritten, and the other chapters have been left relatively untouched, with the exception of pages $37,38,160,176,216,252,258,259,260$.

On the other hand, I have added an Appendix consisting of several sections, which are almost independent of each other. Its purpose is twofold: one is to prove the theorems which were used but not proved in the text, namely Eakin's theorem, Cohen's existence theorem of coefficient rings for complete local rings of unequal characteristic, and Nagata's Jacobian criterion for formal power series rings. The other is to record some of the recent achievements in the area connected with PART II. They include Faltings' simple proof of formal smoothness of the geometrically regular local rings, Marot's theorem on Nagata rings, my theory on excellence of rings with enough derivations in characteristic 0 and Kunz' theorems on regularity and excellence of rings of characteristic $p$.

I should like to record my gratitude to my former students $M$. Mizutani and M. Nomura, who read this book carefully and proved Th.101 and Th.99.

Hideyuki Matsumura 1. All rings and algebras are tacitly assumed to be commutative with unit element.

\begin{enumerate}
  \setcounter{enumi}{2}
  \item If $\mathbf{F}: \mathrm{A} \rightarrow \mathrm{B}$ is a homomorphism of rings and if $I$ is an ideal of $B$, then the ideal $\mathrm{f}^{-1}(\mathrm{I})$ is denoted by InA.

  \item C means proper inclusion.

  \item We sometimes use the old-fashioned notation $I=\left(a_{1}, \ldots, a_{n}\right)$ for an ideal I generated by the elements $a_{i}$.

  \item By a finite A-module we mean a finitely generated A-module. By a finite A-algebra, we mean an algebra which is a finite A-module. By an A-algebra of finite type, we mean an algebra which is finitely generated as a ring over the canonical image of $\mathrm{A}$.

\end{enumerate}
\section{General Rings}
$(1, A)$ of elements $x$ in A some powers of which lie in or is an ideal of $A$, called the radical of or

An ideal $p$ is called a prime ideal of $A$ if $A / P$ is an integral domain ; in other words, if $p \neq A$ and if $A-p$ is closed under multiplication. If $p$ is prime, and if or and $q$ are ideals not contained in $p$, then or $\& \$ p$.

An ideal $q$ is called primary if $q \neq A$ and if the only zero divisors of $A / q$ are nilpotent elements, i.e. $x y \varepsilon q$, $x \notin q$ implies $y^{n} \varepsilon q$ for some $n$. If $q$ is primary then its radical $p$ is prime (but the converse is not true), and $p$ and $q$ are said to belong to each other. If $q(\neq A)$ is an ideal containing some power $w^{\mathrm{n}}$ of a maximal ideal w, then $q$ is a

\section{primary ideal belonging to w.}
The set of the prime ideals of $A$ is called the spectrum of $A$ and is denoted by $\operatorname{Spec}(A)$; the set of the maximal ideals of $A$ is called the maximal spectrum of $A$ and we denote it by $\Omega(A)$. The set Spec(A) is topologized as follows. For any subset $M$ of $A$, put $V(M)=\{p \in \operatorname{Spec}(A) \mid M \subseteq P\}$, and take as the closed sets in $\operatorname{Spec}(A)$ all subsets of the form $V(M)$. This topology is called the Zariski topology. If $f \varepsilon A$, we put $D(f)=\operatorname{Spec}(A)-V(f)$ and call it an elementary open set of Spec(A). The elementary open sets form a basis of open sets of the Zariski topology in $\operatorname{Spec}(A) .$

Let $\mathrm{f}: \mathrm{A} \rightarrow \mathrm{B}$ be a ring homomorphism. To each $P \varepsilon$

Spec (B) we associate the ideal $P \cap A$ (i.e. $f^{-1}(P)$ ) of $A$. Since

$P \cap A$ is prime in $A$, we then get a map $\operatorname{Spec}(B) \rightarrow \operatorname{Spec}(A)$, which is denoted by $a_{f}$. The map $a_{f}$ is continuous as one can easily check, It does not necessarily map $\Omega(B)$ into $\Omega(A)$.

When $P \varepsilon \operatorname{Spec}(B)$ and $P=P \cap A$, we say that $P$ lies over $P$.

(1.B) Let $A$ be a ring, and let $I, p_{1}, \ldots, P_{r}$ be ideals in A. Suppose that all but possibly two of the $P_{i}^{\prime}$ 's are prime ideals. Then, if $I \nsubseteq p_{i}$ for each $i$, the ideal $I$ is not contained in the set-theoretical union $U_{i} p_{i} \cdot$ $P_{j}$, we may suppose that there are no inclusion relations between the $P_{i}{ }^{\prime}$ s. We use induction on $r$. When $r=2$, suppose $I \subseteq p_{1} \cup p_{2}$. Take $x \varepsilon I-p_{2}$ and $s \varepsilon I-p_{1}$. Then $x \varepsilon p_{1}$, hence $s+x \notin p_{1}$, therefore both $s$ and $s+x$ must be in $p_{2}{ }^{\circ}$ Then $x \varepsilon p_{2}$ and we get a contradiction.

When $r>2$, assume that $p_{r}$ is prime. Then Ip $1 \ldots p_{r-1}$ $\pm p_{r}$; take an element $x \& p_{1} \ldots p_{r-1}$ which is not in $p_{r}$. Put $S=I-\left(p_{1} \cup \ldots \cup p_{r-1}\right)$. By induction hypothesis $S$ is not empty. Suppose IS $p_{1} \vee \ldots P_{r} . \quad$ Then $S$ is contained in $P_{r}$. But if $s \varepsilon S$ then $s+x \varepsilon s$ and therefore both $s$ and $s+x$ are in $P_{r}$, hence $x \in P_{r}$, contradiction.

Remark. When A contains an infinite field $k$, the condition that $p_{3}, \ldots, p_{r}$ be prime is superfluous, because the ideals are k-vector spaces and $I=U_{i}\left(I \cap p_{i}\right)$ cannot happen if In $p_{i}$ are proper subspaces of $I$.

(1.C) Let A be a ring, and $I_{1}, \ldots, I_{r}$ be ideals of $A$ such that $I_{i}+I_{j}=A \quad(i \neq j) . \quad$ Then $I_{1} \cap \ldots \cap I_{r}=I_{1} I_{2} \cdots_{r}$ and

$A /\left(\cap I_{i}\right) \simeq\left(A / I_{1}\right) \times \ldots \times\left(A / I_{r}\right)$

(1.D) Any ring $A \neq 0$ has at least one maximal ideal. In fact, the set $M=\{$ ideal $J$ of $A \mid l \notin J\}$ is not empty since (0) $\varepsilon M$, and one can apply Zorn's lemma to find a maximal element of M. It follows that $\operatorname{Spec}(\mathrm{A})$ is empty iff $\mathrm{A}=0$.

If $A \neq 0, \operatorname{Spec}(A)$ has also minimal elements (i.e. $A$

has minimal prime ideals). In fact, any prime $P \varepsilon \operatorname{Spec}(A)$

contains at least one minimal prime. This is proved by revers-

ing the inclusion-order of $\operatorname{Spec}(A)$ and applying Zorn's lemma.

If $J \neq A$ is an ideal, the map $\operatorname{Spec}(\mathrm{A} / J) \rightarrow \operatorname{Spec}(\mathrm{A})$

obtained from the natural homomorphism $\mathrm{A} \rightarrow \mathrm{A} / \mathrm{J}$ is an order-

preserving bijection from $\operatorname{Spec}(\mathrm{A} / J)$ onto $V(J)=\{p \in \operatorname{Spec}(\mathrm{A})$

$\mid p \geq J$. Therefore $V(J)$ has maximal as well as minimal ele-

ments. We shall call a minimal element of $\mathrm{V}(J)$ a minimal

prime over-ideal of $J .$

\section{(1.E) A subset $S$ of a ring A is called a multiplicative}
subset of A if $1 \varepsilon \mathrm{S}$ and if the products of elements of $S$ are again in $\mathrm{S}$.

\section{Let $s$ be a multiplicative subset of A not containing}
0 , and let $M$ be the set of the ideals of A which do not meet

S. Since (0) $\varepsilon M$ the set $M$ is not empty, and it has a maximal

element $p$ by Zorn's lemma. Such an ideal $p$ is prime ; in fact,

if $x \notin p$ and $y \notin p$, then both $A x+p$ and $A y+p$ meet $S$, hence

there exist elements $a, b \varepsilon A$ and $s, s^{\prime} \varepsilon S$ such that $a x \equiv s$,

by $\equiv s^{\prime}(\bmod p)$. Then abxy $\equiv s^{\top}(\bmod p), s^{\prime} \varepsilon s$, therefore

ss' $\notin P$ and hence $x y \neq P, Q . E . D$. A maximal element of $M$ is called a maximal ideal with respect to the multiplicative set S.

We list a few corollaries of the above result.

\begin{enumerate}
  \item If $S$ is a multiplicative subset of a ring $A$ and if $0 \notin S$, then there exists a prime $p$ of $A$ with $P_{n} S=\emptyset$
\end{enumerate}
1i) The set of nilpotent elements in A, $n i l(A)=\left\{a \in A \mid a^{n}=0\right.$ for some $\left.n>0\right\}$,

is the intersection of all prime ideals of A (hence also the

intersection of all minimal primes of A by (1.D)).

iii) Let $A$ be a ring and $J$ a proper ideal of $A$, Then the radical of $\mathrm{J}$ is the intersection of prime ideals of A containing $\mathrm{J}$

Proof. i) is already proved. ii): Clearly any prime ideal

contains $n i 1(A)$. Conversely, if a $\& \mathrm{nil}(\mathrm{A})$, then $\mathrm{S}=$

$\left\{1, a, a^{2}, \ldots\right\}$ is multiplicative and $0 \& \mathrm{~S}$, therefore there

exists a prime $p$ with a $\notin p \cdot$ iii) is nothing but ii) applied to $\mathrm{A} / \mathrm{J}$

We say a ring $A$ is reduced if it has no nilpotent elements except 0 , i.e. if nil $(A)=(0)$. This is equivalent to saying that (0) is an intersection of prime ideals. For any ring A, we put $A_{\text {red }}=A / n i 1(A)$. The ring $A_{\text {red }}$ is of course reduced. (1.F) Let $S$ be a multiplicative subset of a ring A. Then the localization (or quotient ring or ring of fractions) of A with respect to $S$, denoted by $\mathrm{S}^{-1} \mathrm{~A}$ or by $\mathrm{A}_{\mathrm{S}}$, is the ring
$$
\mathrm{S}^{-1} \mathrm{~A}=\left\{\frac{\mathrm{a}}{\mathrm{s}} \mid a \varepsilon A, \quad s \in \mathrm{S}\right\}
$$
where equality is defined by

$-a / s=a^{\prime} / s^{\prime} \Leftrightarrow s^{\prime \prime}\left(s^{\prime} a-s a^{\prime}\right)=0$ for some $s^{\prime \prime} \varepsilon s$ and the addition and the multiplication are defined by the usual formulas about fractions. We have $\mathrm{S}^{-1} \mathrm{~A}=0$ iff $0 \varepsilon \mathrm{S}$. The natural map $\phi: A \rightarrow S^{-1} A$ given by $\phi(a)=a / 1$ is a homomorphism, and its kernel is $\{a \varepsilon A \mid \exists s \varepsilon \mathrm{S}:$ sa $=0\}$. The A-algebra $\mathrm{S}^{-1} \mathrm{~A}$ has the following universal mapping property: if $f: A+B$ is a ring homomorphism such that the images of the elements of $S$ are invertible in $B$, then there exists a unique homomorphism $f_{S}: S^{-1} A \rightarrow B$ such that $E=f_{S} \bullet \phi$, where $\phi: A \rightarrow S^{-1} A$ is the natural map. Of course one can use this property as a definition of $\mathrm{S}^{-1} \mathrm{~A}$. It is the basis of all functorial properties of localization.

If $p$ is a prime (resp. primary) ideal of A such that $p \cap S=\emptyset$, then $p\left(\mathrm{~S}^{-1} \mathrm{~A}\right.$ ) is prime (resp. primary). Conversely, all the prime and the primary ideals of $\mathrm{S}^{-1} \mathrm{~A}$ are obtained in this way. For any ideal I of $\mathrm{S}^{-1} \mathrm{~A}$ we have $I=(\operatorname{I} \cap \mathrm{A})\left(\mathrm{S}^{-1} \mathrm{~A}\right)$. If $J$ is an ideal of $A$, then we have $J\left(S^{-1} A\right)=S^{-1} A$ iff $J \cap S \neq$

Ø. The canonical map $\operatorname{Spec}\left(\mathrm{S}^{-1} \mathrm{~A}\right) \rightarrow \operatorname{Spec}(\mathrm{A})$ is an order- preserving bijection and homeomorphism from $\operatorname{Spec}\left(\mathrm{S}^{-1} \mathrm{~A}\right)$ onto the subset $\{p \in \operatorname{spec}(A) \mid p \cap S=\emptyset\}$ of $\operatorname{Spec}(A)$

Let $S$ be a multiplicative subset of a ring $A$ and let $M$ be an A-module. One defines $\mathrm{S}^{-1} \mathrm{M}=\{\mathrm{x} / \mathrm{s} \mid \mathrm{x} \varepsilon \mathrm{M}, \mathrm{s} \varepsilon \mathrm{S}\}$ in the same way as $S^{-1} A$. The set $S^{-1} M$ is an $S^{-1} A-$ module, and there is a natural isomorphism of $\mathrm{S}^{-1}$ A-modules
$$
S^{-1} M \simeq S^{-1} A \otimes{ }_{A} M
$$
given by $x / s \mapsto(1 / s) \otimes x$

If $M$ and $N$ are $A$-modules, we have
$$
S^{-1}(M \otimes N)=\left(S^{-1} M\right) \otimes_{S^{-1}}^{A}\left(S^{-1} N\right)
$$
When $M$ is of finite presentation, 1.e. when there is an exact sequence of the form $A^{m} \rightarrow A^{n}+M \rightarrow 0$, we have also

\includegraphics[max width=\textwidth]{2022_08_01_8d4eee36f1f42236b4f4g-012}

(1.H) When $S=A-P$ with $p \in \operatorname{Spec}(A)$, we write $A_{p}, M_{P}$ for $S^{-1} A, S^{-1} M$.

LEMMA 1. If an element $x$ of $M$ is mapped to 0 in $M_{p}$ for all $p \varepsilon \Omega(A)$, then $x=0$. In other words, the natural map

$M \rightarrow \underset{\operatorname{al1} \max \cdot p}{M} M_{p}$ Proof. $\quad x=0 \operatorname{in} M_{p} \Leftrightarrow s \varepsilon A-p$ such that $s x=0$ in $M \Leftrightarrow$

\includegraphics[max width=\textwidth]{2022_08_01_8d4eee36f1f42236b4f4g-013}\\
all maximal ideals $p$, the annihilator Ann ( $x$ ) of $x$ is not contained in any maximal ideal and hence $\operatorname{Ann}(x)=A$. This implies $x=1 \cdot x=0$

LEMMA 2. When $A$ is an integral domain with quotient field $k$, a1l localizations of A can be viewed as subrings of $K$. In this sense, we have
$$
A=\prod_{a l 1} A_{p} A .
$$
Proof. Given $x \varepsilon K$, we put $D=\{$ a $\varepsilon A \mid$ ax $\varepsilon A\}$; we might call D the ideal of denominators of $x$. The element $x$ is in A iff $D=A$, and $x$ is in $A_{P}$ iff $D$. Therefore, if $x \notin A$, there exists a maximal ideal $P$ such that $D \subseteq P$, and $x \notin A_{P}$ for this $p$.

(1.I) Let $f: A \rightarrow B$ be a homomorphism of $r$ ings and $S$ a multiplicative subset of $A ;$ put $S^{\prime}=f(S)$. Then the localization $S^{-1} B$ of $B$ as an A-module coincides with $S^{-1} B$ :
$$
\text { (1.I.1) } \quad S^{\varphi^{-1}} B=S^{-1} B=\left(S^{-1} A\right) \otimes_{A} B \cdot
$$
In particular, if I is an ideal of $A$ and if $S^{\prime}$ is the image of $S$ in $A / I$, one obtains
$$
(1 . I .2) \quad S^{I^{-1}}(A / I)=S^{-1} A / I\left(S^{-1} A\right)
$$
In this sense, dividing by I commutes with localization.

(1.K) A ring A which has only one maximal ideal $m$ is

called a local ring, and $\mathrm{A} / \mathrm{is}$ called the residue field of

A. When we say that "(A, is a local ring "or " $(A, w, k)$

is a local ring", we mean that $A$ is a local ring, that is

the unique maximal ideal of $A$ and that $k$ is the residue field

of A. When $A$ is an arbitrary ring and $D E \operatorname{Spec}(A)$, the ring

$A_{p}$ is a local ring with maximal ideal $p_{p}{ }^{\circ}$ The residue field of $A_{p}$ is denoted by $k(p)$. Thus $k(p)=A_{p} / p A_{p}$, which is the quotient ield of the integral demain $\mathrm{A} / \mathrm{p}$ by (1.I.2).

If $(A, w, k)$ and $\left(B, m^{\prime}, k^{\prime}\right)$ are local rings, a homomorphism $\psi: A \rightarrow B$ is called a local homomorphism if $\psi(m) \leq m^{\prime}$.

In this case $\psi$ induces a homomorphism $k \rightarrow k^{\prime}$.
$$
\begin{aligned}
& \text { (1.J) Let } A \text { be a ring and } S \text { a multiplicative subset of } A \\
& \text { let } \mathrm{A} \rightarrow \mathrm{B} \rightarrow \mathrm{S}^{-1} \mathrm{~A} \text { be homomorphisms such that (1) gof is the } \\
& \text { natural map and (2) for any } b \varepsilon \mathrm{B} \text { there exists } s \& \mathrm{~S} \text { with } \\
& f(s) b \in f(A) \text {. Then } s^{-1} B=f(S)^{-1} B=s^{-1} A \text {, as one can easily } \\
& \text { check. In particular, let } A \text { be a domain, } P \in \text { Spec (A) and } B \\
& \text { a subring of } A_{p} \text { such that } A \subseteq B \subseteq A_{p} \text {. Then } A_{p}=B_{P}=B_{p^{\prime}} \text {, } \\
& \text { where } \mathrm{P}=P \mathrm{~A}_{p} \cap B \text { and } \mathrm{B}_{P}=B^{B} \otimes A_{p^{\circ}} 
\end{aligned}
$$
Consider the map $a_{\psi}: \operatorname{Spec}(B) \rightarrow \operatorname{Spec}(A)$. If $P \varepsilon \operatorname{Spec}(B)$ and ${ }^{a} \psi(P)=P \cap A=P$, we have $\psi(A-P) \subseteq B-P$, hence $\psi$ induces

a homomorphism $\psi_{p}: A_{p} \rightarrow{ }^{B}{ }_{P}$, which is a local homomorphism since $\psi_{p}\left(p A_{P}\right) \subseteq \psi(p) B_{P} \subseteq P_{P}{ }^{\circ}$. Note that $\psi_{p}$ can be factored as $A_{p} \rightarrow B_{p}=A_{p} \odot_{A} B \rightarrow B_{p}$ and $B_{p}$ is the localization of $B_{p}$ by ${ }^{P B_{P}} \cap B_{P^{\circ}}$ In general $B_{P}$ is not a local ring, and the maximal ideals of $B_{p}$ which contain $P B_{p}$ correspond to the pre-images of $P$ in $\operatorname{Spec}(B)$. (B $P$ can have maximal ideals other than these.) But if $B_{p}$ is a local ring, then $B_{p}=B_{p}$, because if $(R, m)$ is a local ring then $R-w$ is the set of units of $R$ and hence $\mathrm{R}_{\mu}=R_{1}$

(1.L) Definition. Let $A$ be a ring, $A \neq 0$. The Jacobson

radical of $A, \operatorname{rad}(\mathrm{A})$, is the intersection of a11 maximal ideals of A.

Thus, if $(A, m)$ is a local ring then $\mu=\operatorname{rad}(A)$. We say that a ring $A \neq 0$ is a semi-local ring if it has only a finite number of maximal ideals, say $N_{1}, \ldots, \ldots /{ }^{\circ}$ (We express this situation by saying "(A, $\left.m_{1}, \ldots, m_{r}\right)$ is a semi-local ring".) In this cass $\operatorname{rad}(\mathrm{A})=w_{1} \cap \ldots \cap w_{r}=\Pi w_{i}$ by (1.C).

Any element of the form $1+x, x \varepsilon \operatorname{rad}(A)$, is a unit in $A$, because $1+x$ is not contained in any maximal ideal.

Conversely, if $I$ is an ideal and if $1+\mathbf{x}$ is a unit for each $x \in I$, we have $I \subseteq \operatorname{rad}(A)$ (1.M) LEMMA $(N A K)^{*}$. Let A be a ring, M a finite A-module and I an ideal of A. Suppose that $I M=M$. Then there exists an element a $\varepsilon \mathrm{A}$ of the form $\mathrm{a}=1+\mathrm{x}, \mathrm{x} \varepsilon \mathrm{I}$, such that $\mathrm{aM}=0 .$ If moreover I $\subseteq \operatorname{rad}(\mathrm{A})$, then $\mathrm{M}=0$

Proof. Let $M=\mathrm{Aw}_{1}+\ldots+\mathrm{Aw}_{\mathrm{s}}$. We use induction on $s$. Put $M^{\prime}=$ M/Aw $s^{\circ}$ By induction hypothesis there exists $x \varepsilon$ I such that $(1+x) M^{\prime}=0, i_{0} .,(1+x) M \subseteq A_{s}$ (when $s=1$, take $x=0)$. Since $M=I M$, we have $(1+x) M=I(1+x) M \subseteq I\left(A w_{s}\right)$ = Iw $s^{2}$ hence we can write $(1+x) w_{s}=y_{s}$ for some y $\varepsilon$. Then $(1+x-y)(1+x) M=0$, and $(1+x-y)(1+x) \equiv 1$ Mod I, proving the first assertion. The second assertion follows from this and from (1.L).

This Lemma is often used in the following form.

COROLLARY. Let A be a ring, $M$ an A-module, $N$ and $N^{\prime}$ submodules of $M$, and I an ideal of A. Suppose that $M=N+I N^{\prime}$, and that either (a) I is nilpotent, or (b) I $\subseteq r a d$ (A) and $N^{\prime}$ is initely generated. Then $\mathrm{M}=\mathrm{N}$.

Proof. In case (a) we have $M / N=I(M / N)=I^{2}(M / N)=\ldots .0$. In case (b), apply NAK to $M / N$.

*) This simple but important lemma is due to $T$. Nakayama, . Azumaya and $W_{0}$ Krull. Priority is obscure, and although it is usually called the Lemma of Nakayama, late Prof. Nakayama did not like the name. (1.N) In particular, 1et $(A, m, k)$ be a local ring and $M$ an A-module. Suppose that either $\mathrm{N}$ is $n i l p o t e n t$ or $M$ is finite. Then a subset $G$ of $M$ generates $M$ iff its image $\bar{G}$ in $M / m M=M \otimes k$ generates $M \otimes k, \quad$ In fact, if $N$ is the submodule generated by $G$, and if $\bar{G}$ generates $M \otimes k$, then $M=N+M$, whence $M=N$ by the corollary. Since $M \otimes k$ is a vector space over the field $k$, it has a basis, say $\bar{G}$, and if we lift $\bar{G}$ arbitrarily to a subset $G$ of $M$ (i.e. choose a pre-image for each element of $\bar{G}$ ), then $G$ is a system of generators of $M$. Such a system of generators is called a minimal basis of $M$. Note that a minimal basis is not necessarily a basis of $M$ (but it is so in an important case, $c f .(3 . G)$ ).

(1.0) Let $A$ be a ring and $M$ an A-module. An element a of A is said M-regular if it is not a zero-divisor on $M, i . e .$, if $M \stackrel{a}{\rightarrow} M$ is injective. The set of the M-regular elements is a multiplicative subset of A.

Let $s_{0}$ be the set of A-regular elements. Then $s_{0}^{-1} A$ is called the total quotient ring of $A$. In this book we shall denote it by $\Phi \mathrm{A}$. When $\mathrm{A}$ is an integral domain, $\Phi \mathrm{A}$ is nothing but the quatient field of $A$.

(1.P) Let $A$ be a ring and $\alpha: Z \rightarrow A$ be the canonical homomorphism from the ring of integers $Z$ to $A$. Then $\operatorname{Ker}(\alpha)=n Z$ for some $n \geqslant 0$. We call $n$ the characteristic of $A$ and denote it by $\operatorname{ch}(\mathrm{A})$. If $\mathrm{A}$ is local the characteristic ch(A) is either 0 or a power of a prime number.

\section{Noetherian Rings and Artinian Rings}
(2.A) A ring is called noetherian (resp. artinian) if the ascending chain condition (resp. descending chain condition) for ideals holds in it. A ring $A$ is noetherian iff every ideal of A is a finite A-module.

If $A$ is a noetherian ring and $M$ a finite $A$-moeule, then the ascending chain condition for submodules holds in M and every submodule of $M$ is a finite A-module. From this, it follows easily that a finite module $M$ over a noetherian ring has a projective resolution $\ldots \rightarrow \mathrm{X}_{i} \rightarrow \mathrm{X}_{i-1} \rightarrow \ldots \rightarrow \mathrm{X}_{0}$ $\rightarrow M \rightarrow 0$ such that each $X_{i}$ is a finite free A-module. In particular, $M$ is of finite presentation.

A polynomial ring $A\left[x_{1}, \ldots, X_{n}\right]$ over a noetherian ring A is again noetherian. Similarly for a formal power series ring $A\left[\left[X_{1}, \ldots, X_{n}\right]\right]$. If $B$ is an A-algebra of finite type and if A is noetherian, then B is noetherian since it is a homomorphic image of $A\left[X_{1}, \ldots, X_{n}\right]$ for some $n$.

Any proper ideal I of a noetherian ring has a primary decomposition, i.e. $I=q_{1} \cap \ldots \cap q_{r}$ with primary

ideals $q_{i} \cdot$ (We shall discuss this topic again in Chap. 5)

\section{(2.C) PROPOSITION. A ring $A$ is artinian iff the length}
of $A$ as A-module is finite.

Proof. If length $A(A)<\infty$ then $A$ is certainly artinian

(and noetherian). Conversely, suppose A is artinian. Then

A has only a finite number of maximal ideals. Indeed, if

there were an infinite sequence of maximal ideals $p_{1}, p_{2}, \ldots$

then $p_{1}>p_{1} p_{2} \supset p_{1} p_{2} p_{3} \supset \ldots$ would be a strictly descend-

ing infinite chain of ideals, contradicting the hypothesis.

Let $p_{1}, \ldots, p_{r}$ be all the maximal ideals of $A$ (we may

assume $A \neq 0$, so $r>0)$, and put $I=p_{1} \ldots p_{r} \cdot$ The descend-

ing chain $I \geq I^{2} \geq I^{3} \geq \ldots$ stops, so there exists $s>0$

such that $I^{S}=I^{S+1}$. Put $\left((0): I^{S}\right)=J$. Then $(J: I)=$

$\left(\left((0): I^{s}\right): I\right)=\left((0): I^{s+1}\right)=J$. We claim $J=A$. Suppose the

contrary, and let $J^{\prime}$ be a minimal member of the set of ideals

strictly containing $\mathrm{J}$. Then $\mathrm{J}^{\prime}=\mathrm{Ax}+\mathrm{J}$ for any $\mathrm{x} \varepsilon \mathrm{J}^{\prime}-\mathrm{J}$.

Since $I=\operatorname{rad}(\mathrm{A})$, the ideal $\mathrm{Ix}+\mathrm{J}$ is not equal to $\mathrm{J}^{\prime}$ by NAK

(Cor. of $(1 . K))$. So we must have $I x+J=J$ by the minimal-

ity of $\mathrm{J}^{\prime}$, hence $\mathrm{Ix} \leqslant \mathrm{J}$ and $\mathrm{x} \varepsilon(\mathrm{J}: \mathrm{I})=\mathrm{J}$, contradiction.

Thus $J=A$, i.e. $l \cdot I^{S} \subseteq(0), i \cdot e \cdot I^{S}=(0)$.

Consider the descending chain ${ }^{A} \supseteq p_{1} \supseteq p_{1} p_{2} \supseteq \cdots \supseteq p_{1} \cdots p_{r-1} \supseteq I \supseteq I p_{1} \supseteq P_{1} p_{2} \supseteq$
$$
\ldots \supseteq \mathrm{I}^{2} \supseteq \mathrm{I}^{2} p_{1} \supseteq \ldots \supseteq \mathrm{I}^{\mathrm{s}}=(0) .
$$
Each factor module of this chain is a vector space over the field $\mathrm{A} / p_{i}=k_{1}$ for some $i$, and its subspaces correspond bijectively to the intermediate ideals. Thus, the descending chain condition in A implies that this factor module is of finite is ${ }^{\prime}$, therefore it is of finite length as A-module. A A tho $A \neq 0$ A

prime ideals are maximal (cf. 12.A).

COROLLARY. A ring $A \neq 0$ is artinian iff it is noetherian and of dimension zero.

Proof. If A is artinian, then it is noetherian since $\operatorname{length}_{A}(A)<\infty$.

Let $p$ be any prime ideal of A. In the notation of the above proof, we have $\left(p_{1} \ldots p_{r}\right)^{S}=I^{S}=(0) \subseteq p$, hence $p=p_{i}$ for

some i. Thus A is of dimension zero.

To prove the converse, let $(0)=q_{1} \cap \ldots \cap q_{r}$ be a primary decomposition of the zero ideal in $A$, and let $p_{i}=$ the radical of $q_{1} \cdot$ Since $p_{i}$ is finitely generated over A, there is a positive integer $n$ such that $p_{i}^{n} \subseteq q_{i}(1 \leqslant i \leqslant r)$

Then $\left(p_{1} \ldots p_{r}\right)^{n}=(0)$. After this point we can immitate the

CHAPTER 2. FLATNESS

last part of thepproof of the proposition to conclude that

length $_{A}(A)<\infty$

(2.D) I.S.Cohen proved that a ring is noetherian iff every

prime ideal is finitely generated (cf. Nagata, LOCAL RINGS,

p.8). Recently P.M.Eakin (Math. Annalen $177(1968), 278-282$ )

proved that, if $A$ is a ring and $A^{\prime}$ is a subring over which $A$

is finite, then $A^{\prime}$ is noetherian if (and of course only if)

\section{Flatness}
A is so, (The theorem was independently obtained by Nagata,

(3.A) DEFINITION. Let $A$ be $a$ ring and $M$ an A-module ;

when $S: \cdots \rightarrow N \rightarrow N^{\prime} \rightarrow N^{\prime \prime} \rightarrow \ldots$ is any sequence of A-modules

(and of A-1inear maps), let $S \otimes M$ denote the sequence $\cdots \rightarrow$

$N \otimes M \rightarrow N^{\prime} \otimes M \rightarrow N^{\prime \prime} \otimes M \rightarrow \cdots$ obtained by tensoring $S$ with $M .$

We say that $M$ is $f l a t$ over A, or A-flat, if $S \otimes M$ is exact

whenever $S$ is exact. We say that $M$ is faithfully flat (f.f.)

over A, if $S \otimes M$ is exact iff $S$ is exact.

Examples. Projective modules are flat. Free modules are f.f..

If $B$ and $C$ are rings and $A=B \times C$, then $B$ is a projective

module (hence flat) over A but not f.f. over A.

THEOREM 1. The following conditions are equivalent:

(1) M is A-flat;

but the priority is Eakin's.)

\section{Exercises to Chapter $1 .$}
\begin{enumerate}
  \item Let I and $\mathrm{J}$ be ideals of a ring A. What is the condition
\end{enumerate}
for $V(I)$ and $V(J)$ to be disjoint ?

\begin{enumerate}
  \setcounter{enumi}{2}
  \item Let $A$ be a ring and $M$ an A-module. Define the support
\end{enumerate}
of $M, \operatorname{Supp}(M)$, by
$$
\operatorname{Supp}(M)=\left\{p \varepsilon \operatorname{Spec}(A) \mid M_{p} \neq 0\right\} .
$$
If $M$ is finite over $A$, we have $\operatorname{Supp}(M)=V(\operatorname{Ann}(M))$ so that

the support is closed in $\operatorname{Spec}(A)$.

\begin{enumerate}
  \setcounter{enumi}{3}
  \item Let $A$ be a noetherian ring and $M$ a finite A-module. Let
\end{enumerate}
I be an ideal of $A$ such that $\operatorname{Supp}(M) \subseteq V(I)$. Then $I M=0$

for some $n>0$. (2) if $0 \rightarrow \mathrm{N}^{\prime} \rightarrow \mathrm{N}$ is an exact sequence of A-modules, then $0 \rightarrow \mathrm{N}^{1} \otimes \mathrm{M} \rightarrow \mathrm{N} \otimes \mathrm{M}$ is exact ;

(3) for any finitely generated ideal I of A, the sequence $0 \rightarrow I \otimes M \rightarrow M$ is exact, in other words we have $I \otimes M \simeq I M$;

(4) $\operatorname{Tor}_{1}^{A}(M, A / I)=0$ for any finitely generated ideal I of A ;

(5) $\operatorname{Tor}_{1}^{A}(M, N)=0$ for any finite A-module $N$;

(6) if $a_{i} \in A, x_{i} \in M \quad(1 \leqslant i \leqslant r)$ and $\sum_{1}^{r} a_{i} x_{i}=0$, then there exist an integer $s$ and elements $b_{i j} \varepsilon A$ and $y_{j}$

$\varepsilon M(1 \leqslant j \leqslant s)$ such that $\sum_{i} a_{i} b_{i j}=0$ for $a 11 j$ and $x_{i}=\sum_{j} b_{i j} y_{j}$ for all i.

Proof. The equivalence of the conditions (1) through (5) is well known ; one uses the fact that the inductive limit $(=$ direct Iimit) in the category of A-modules preserves exactness and commutes with $\operatorname{Tor}_{i}$. We omit the detail. As for (6), first suppose that $M$ is $f$ lat and $\Sigma_{1}^{a_{i} x_{i}}=0$. Consider the exact sequence
$$
\mathrm{K} \underset{\mathrm{g}}{\rightarrow} \mathrm{A}^{\mathrm{r}} \rightarrow \mathrm{A}
$$
where $f$ is defined by $f\left(b_{1}, \ldots, b_{r}\right)=\sum a_{i} b_{i}\left(b_{i} \varepsilon A\right), k=$

$\operatorname{Ker}(f)$ and $g$ is the inclusion map. Then $K Z M \rightarrow M^{r} \stackrel{f_{M}}{\rightarrow} M$

is exact, where $f_{M}\left(t_{1}, \ldots, t_{r}\right)=\sum a_{i} t_{i}\left(t_{i} \varepsilon M\right)$; therefore $\left(x_{1}, \ldots, x_{r}\right)=\Sigma_{1}^{S} \beta_{j} \otimes y_{j}$ with $\beta_{j} \varepsilon K, y_{j} \varepsilon M_{1}$. Writing $\beta_{j}=\left(b_{i j}, \ldots, b_{r j}\right) \quad\left(b_{i j} \varepsilon A\right)$, we get the wanted result. Next let us prove $(6) \Rightarrow(3)$. Let $a_{1}, \ldots, a_{r} \varepsilon$ I and $x_{1}, \ldots, x_{r} \varepsilon M$ be such that $\Sigma a_{i} x_{1}=0$. Then by assumption $x_{i}=\sum b_{i j} y_{j}, \sum a_{i} b_{i j}=0$, hence in $I \otimes M$ we have $\sum_{i} a_{i} \theta: x_{i}$ $=\Sigma_{i} a_{i} \otimes \Sigma_{j} b_{i j} y_{j}=\Sigma_{j}\left(\Sigma_{i} a_{i} b_{i j} \otimes y_{j}\right)=0$.

Q.E.D.

(3.B) (Transitivity) Let $\phi: A \rightarrow B$ be a homomorphism of rings and suppose that $\phi$ makes $B$ a flat A-module. (In this case we shall say that $\phi$ is a flat homomorphism.) Then a flat $B$-module $N$ is also flat over A.

Proof. Let $S$ be a sequence of A-modules. Then $S \otimes_{\mathrm{A}^{N}}=$ $S \otimes_{A}(B \otimes N)=\left(S \partial_{A} B\right) \otimes_{B}{ }^{N} \cdot$ Thus, $S$ is exact $\Rightarrow S \theta_{A} B$ is exact $\Rightarrow S Q_{A} N$ is exact.

(Change of base) Let $\phi: A \rightarrow B$ be any homomorphism of rings and let $M$ be a flat A-module. Then $M(B)=M Q_{A}$ is flat B-module.

Proof, Let $S$ be a sequence of B-modules. Then $S \otimes_{B}\left(B_{A} M\right)=$ ${ }_{A} Q^{M}$, which is exact if $S$ is exact.

(Localization) Let A be a ring, and $S$ a multiplicative subset of $A$. Then $S^{-1} A$ is flat over $A$. Proof. Let $M$ be an A-module and $N$ a submodule. We have $M \otimes \mathrm{s}^{-1} \mathrm{~A}=\mathrm{S}^{-1} \mathrm{M}$ and $\mathrm{N} \otimes \mathrm{S}^{-1} \mathrm{~A}=\mathrm{s}^{-1} \mathrm{~N}$. A typical element of $\mathrm{S}^{-1} \mathrm{~N}$ is of the form $\mathrm{x} / \mathrm{s}, \mathrm{x} \varepsilon \mathrm{N}, \mathrm{s} \varepsilon \mathrm{S}$; if $\mathrm{x} / \mathrm{s}=0$ in $\mathrm{s}^{-1} \mathrm{M}$, this means that there exists $s^{\prime} \varepsilon \mathrm{S}$ with $s^{\prime} x=0$ in $M$, which is equivalent to saying that $s^{\prime} x=0$ in $N$, hence $x / s=0$ in $\mathrm{s}^{-1} \mathrm{~N}$. Thus $0 \rightarrow \mathrm{s}^{-1} \mathrm{~N} \rightarrow \mathrm{s}^{-1} \mathrm{M}$ is exact.

(3.E) Let $\phi: A \rightarrow B$ be a flat homomorphism of rings, and let $M$ and $N$ be $A-$ modules. Then $\operatorname{Tor}_{i}^{A}(M, N) \otimes{ }_{A} B=$ $\operatorname{Tor}_{i}^{B}\left(M(B), N_{(B)}\right)$. If $A$ is noetherian and $M$ is finite over $A$, we also have $\left.\operatorname{Ext}_{A}^{i}(M, N) \otimes_{A} B=\operatorname{Ext}_{B}^{i}(M, B), N_{(B)}\right)$.

Proof. Let $\ldots \rightarrow X_{1} \rightarrow X_{0} \rightarrow M \rightarrow 0$ be a projective resolution of the A-module $M$. Then, since $B$ is $f 1 a t$, the sequence
$$
\text { (*) } \cdots+X_{1(B)}+X_{O(B)} \rightarrow M_{(B)} \rightarrow 0
$$
is a projective resolution of $M(B)$. We have therefore
$$
\begin{aligned}
&\operatorname{Tor}_{i}^{A}(M, N)=H_{i}(X, \otimes N), \\
&\operatorname{Tor}_{i}^{B}\left(M(B), N(B)=H_{i}\left(X \cdot \otimes_{A} N \otimes_{A} B\right),\right.
\end{aligned}
$$
But the exact functor $\otimes_{A} B$ commutes with taking homology, so that $H_{i}\left(X, \otimes_{A} N \otimes_{A} B\right)=H_{i}\left(X \cdot \otimes_{A} N\right) \otimes A_{A} B=\operatorname{Tor}_{i}^{A}(M, N) \otimes \otimes_{A} B \cdot$ If A is noetherian and $M$ is finite over $A$, we can assume that $\mathrm{I}_{i}$ s are finite free A-modules, Then $\operatorname{Hom}_{B}\left(\mathrm{X}_{1} \otimes \mathrm{B}, \mathrm{N} \otimes \mathrm{B}\right)$

$=\operatorname{Hom}_{A}\left(X_{i}, N\right) \otimes_{A} B$, and so the same reasoning as above proves the formula for Ext.

In particular, for $p \in \operatorname{Spec}(A)$, we have

the latter being valid for A noetherian and M finite.

Let $A$ be a ring and $M$ a flat A-module. Then an A-regular element a $\varepsilon$ A is also M-regular.

Proof. As $0 \rightarrow \mathrm{A} \rightarrow \mathrm{A}$ is exact, so is $0 \rightarrow \mathrm{M} \rightarrow \stackrel{\mathrm{a}}{\rightarrow} \mathrm{M} .$

PROPOSITION. Let $(A, m, k)$ be a local ring and $M$ an A-module. Suppose that either in is nilpotent or $M$ is finite over A. Then

$M$ is free $\Leftrightarrow M$ is projective $\Leftrightarrow M$ is flat.

Proof. We have only to prove that if $M$ is flat then it is free. We prove that any minimal basis of $M(c f .(1 . N))$ is basis of M. For that purpose it suffices to prove that, if $x_{1}, \ldots, x_{n} \in M$ are such that their images $\bar{x}_{1}, \ldots, \bar{x}_{n}$ in $M / M M=M \otimes_{A} k$ are linearly independent over $k$, then they are Iinearly independent over A. We use induction on $\mathrm{n}$. When $n=1$, let $a x=0 .$ Then there exist $y_{1}, \ldots, y_{r} \in M$ and
$$
\begin{aligned}
& \operatorname{Tor}_{i}^{A} p\left(M_{p}, N_{p}\right)=\operatorname{Tor}_{i}^{A}(M, N)_{p}, \\
& \operatorname{Ext}_{A_{p}}^{i}\left(M_{p}, N_{p}\right)=\operatorname{Ext}_{A}^{i}(M, N)_{p}, 
\end{aligned}
$$
$b_{1}, \ldots, b_{r} \in A$ such that $a_{i}=0$ for all $i$ and such that

$x=\sum b_{i} y_{i} .$ Since $\bar{x} \neq 0$ in $M / m M$, not all $b_{i}$ are in m.

Suppose $b_{1} \notin$. Then $b_{1}$ is a unit in $A$ and $a_{1}=0$, hence $a=0$

Suppose $n>1$ and $\Sigma_{1}^{n} a_{i} x_{i}=0$. Then there exist $y_{1}$, $\ldots y_{r} \varepsilon M$ and $b_{i j} \varepsilon A(1 \leqslant j \leqslant r)$ such that $x_{i}=\sum_{j} b_{i j} y_{j}$ and $\sum_{i} a_{i} b_{i j}=0$. Since $x_{n} \notin m M$ we have $b_{n j} \notin m$ for at least one $j$. Since $a_{1} b_{1 j}+\cdots+a_{n} b_{n j}=0$ and $b_{n j}$ is a unit, we have
$$
a_{n}=\Sigma_{1}^{n-1} c_{i} a_{i} \quad\left(c_{i}=-b_{i j} / b_{n j}\right)
$$
Then
$$
0=\sum_{1}^{n} a_{i} x_{i}=a_{1}\left(x_{1}+c_{1} x_{n}\right)+\cdots+a_{n-1}\left(x_{n-1}+c_{n-1} x_{n}\right) \text {. }
$$
Since the elements $\bar{x}_{1}+\bar{c}_{1} \bar{x}_{n}, \ldots, \bar{x}_{n-1}+\bar{c}_{n-1} \bar{x}_{n}$ are linear$1 y$ independent over $k$, by the induction hypothesis we get

$a_{1}=\ldots=a_{n-1}=0$, and $a_{n}=\Sigma_{1}^{n-1} c_{i} a_{i}=0$ Q.E.D.

REMARK. If $M$ is $f l a t$ but not $f$ inite, it is not necessarily free $(\mathrm{e} . \mathrm{g} . \mathrm{A}=\mathrm{Z}(\mathrm{p})$ and $\mathrm{M}=\mathrm{Q})$. On the other hand, any projective module over a local ring is free (I. Kaplansky: Projective Modules, Ann. of Math. $68(1958), 372-377)$. For more general rings, it is known that non-finitely generated projective modules are, under very mild hypotheses, free, (Cf. $\mathrm{H}_{\text {. }}$ Bass: Big Projective Modules Are Free, Ill. J. Math. 7 (1963) 24-31, and $\mathrm{Y}$. Hinohara: Projective Modules over Weakly Noetherian Rings, J. Math. Soc.Japan, 15 (1963), 75-88 and 474475). (3.H) Let $A \rightarrow B$ be a flat homomorphism of rings, and let $I_{1}$ and $I_{2}$ be ideals of $A$. Then

(1) $\left(I_{1} \cap I_{2}\right) B=I_{1} B \cap I_{2} B$,

(2) $\left(I_{1}: I_{2}\right) B=I_{1} B: I_{2} B$ if $I_{2}$ is finitely generated.

Proof. (1) Consider the exact sequence of A-modules
$$
\mathrm{I}_{1} \cap \mathrm{I}_{2} \rightarrow \mathrm{A} \rightarrow \mathrm{A} / \mathrm{I}_{1} \oplus \mathrm{A} / \mathrm{I}_{2}{ }^{\circ}
$$
Tensoring it with $B$, we get an exact sequence
$$
\left(I_{1} \cap I_{2}\right) \otimes_{A} B=\left(I_{1} \cap I_{2}\right) B \rightarrow B \rightarrow B / I_{1} B \oplus B / I_{2} B
$$
This means $\left(I_{1} \cap I_{2}\right) B=I_{1} B \cap I_{2} B$.

(2) When $I_{2}$ is a principal ideal aA, we use the exact sequence
$$
\left(I_{1}: a A\right) \stackrel{i}{\rightarrow} \stackrel{f}{\rightarrow} \mathrm{A}^{\mathrm{A}} \mathrm{I}_{1}
$$
where $i$ is the injection and $f(x)=\operatorname{ax}$ mod $I_{1}$. Tensoring it with $\mathrm{B}$ we get the formula $\left(I_{1}: \mathrm{aA}\right) \mathrm{B}=\left(\mathrm{I}_{1} \mathrm{~B}: \mathrm{aB}\right)$. In the general case, if $I_{2}=a \mathrm{~A}+\cdots+a_{n} A_{\text {, we have }}\left(I_{1}: I_{2}\right)=$ $\bigcap_{i}\left(I_{1}: a_{i}\right)$ so that by (1)

$\left(I_{1}: I_{2}\right) B=\cap\left(I_{1}: a_{1} A\right) B=\cap\left(I_{1} B: a_{1} B\right)=\left(I_{1} B: I_{2} B\right)$

(3.I) EXAMPLE 1. Let $A=k[x, y]$ be a polynomial ring over a field $k$, and put $B=A / x A \simeq k[y]$. Then $B$ is not flat over A by $(3 . F)$, Let $I_{1}=(x+y) A$ and $I_{2}=y A$. Then $I_{1} \cap I_{2}$ $=\left(x y+y^{2}\right) A, \quad I_{1} B=I_{2} B=y B, \quad\left(I_{1} \cap I_{2}\right) B=y^{2} B \neq I_{1} B \cap I_{2} B$ EXAMPLE 2. Let $k, x, y$ be as above and put $z=y / x, A=$ $k[x, y], \quad B=k[x, y, z]=k[x, z]$. Let $I_{1}=x A, \quad I_{2}=y A$. Then $I_{1} \cap I_{2}=x y A, \quad\left(I_{1} \cap I_{2}\right) B=x^{2} 2 B, \quad I_{1} B \cap I_{2} B=x z B$. Thus B is not flat over A. The map $\operatorname{Spec}(B) \rightarrow \operatorname{Spec}(A)$ corresponds to the projection to $(x, y)$-plane of the surface $F: x z=y$ in the $(x, y, z)$-space. Note $F$ contains the whole $z$-axis and hence does not look 'flat' over the (x,y)-plane,

EXAMPLE 3. Let $A=k[x, y]$ be as above and $B=k[x, y, z]$ with $z^{2}=f(x, y) \varepsilon A$. Then $B=A \oplus A z$ as an A-module, so that B is free, hence flat, over A. Geometrically, the surface $z^{2}=f(x, y)$ appears indeed to lie rather $f l a t l y$ over the (x, y)-plane. A word of caution: such intuitive pictures are not enough to guarantee flatness.

(3.J) Let $A \rightarrow B$ be a homomorphism of rings. Then the following conditions are equivalent:

(1) B is flat over $A$;

(2) $\mathrm{B}_{P}$ is flat over $\mathrm{A}_{p}(p=P \cap A)$ for all $P \varepsilon \operatorname{Spec}(B)$;

(3) $B_{P}$ is flat over $A_{p}(P=P \cap A)$ for all $P \varepsilon \Omega(B)$.

Proof $(1) \Rightarrow(2)$ : the ring $B_{p}=B \otimes A_{p}$ is flat over $A_{p}$ (base change), and $B_{P}$ is a localization of $B_{p}$, so that $B_{P}$ is $f$ lat over $A_{p}$ by transitivity. $(2) \Rightarrow(3): \operatorname{trivia1} \cdot \quad(3) \Rightarrow(1)$ : it suffices to show that $\operatorname{Tor}_{1}^{A}(B, N)=0$ for any $A$-module $N$. We use the following

LEMMA. Let $B$ be an A-algebra, $P$ a prime ideal of $B, P=P \cap A$ and $\mathrm{N}$ an A-module. Then
$$
\left(\operatorname{Tor}_{i}^{A}(B, N)\right)_{P}=\operatorname{Tor}_{i} P^{A}\left(B_{P}, N_{p}\right)
$$
Proof. Let $X_{\bullet}: \cdots \rightarrow X_{1}+X_{0}(\rightarrow N+0)$ be a free resolution of the A-module $N$. We have
$$
\operatorname{Tor}_{i}^{A}(B, N)=H_{i}\left(X, \otimes_{A} B\right) \text {, }
$$
$$
\begin{aligned}
& \operatorname{Tor}_{i}^{A}(B, N) \otimes_{B}{ }_{P}=H_{i}\left(X, \otimes_{A} B \otimes_{B} B_{P}\right)
\end{aligned}
$$
$$
=H_{i}\left(X \cdot \otimes_{A} B_{P}\right)=H_{i}\left(X \cdot \otimes_{A} A_{p} \otimes_{A_{P}}{ }^{B}{ }_{P}\right)
$$
and $\mathrm{X}_{0} \otimes \mathrm{A}_{P}$ is a free resolution of the $A_{P}$-module $\mathrm{N}_{P}$, hence the last expression is equal to $\left.\operatorname{Tor}_{i} P_{\left(B_{P}\right.}, N_{p}\right)$. Thus the lemma is proved.

Now, if ${ }_{P}$ is flat over $A_{p}$ for all $P \in \Omega(B)$, then $\left.\operatorname{Tor}_{1}^{A}(B, N)\right)_{P}=0$ for all $P \varepsilon \Omega(B)$ by the lemma, therefore $\operatorname{Tor}_{1}^{A}(B, N)=0$ by $(1 . H)$ as wanted.

\section{Faithful Flatness}
(4.A) THEOREM 2. Let $A$ be a ring and $M$ an A-module. The following conditions are equivalent: (i) $M$ is faithfully flat over A;

(ii) $M$ is flat over $A$, and for any A-module $N \neq 0$ we have $\mathrm{N} \partial \mathrm{M} \neq 0$

(iii) $M$ is flat over $A$, and for any maximal ideal th of A

we have $m M \neq M$,

Proof. (i) $\Rightarrow$ (ii): suppose $N \otimes M=0$. Let us consider the se-

quence $0+N \rightarrow 0$. As $0 \rightarrow N \otimes M \rightarrow 0$ is exact, so is $0 \rightarrow$

$\mathrm{N} \rightarrow 0$. Therefore $\mathrm{N}=0$.

(ii) $\Rightarrow$ (iii): since $A / m \neq 0$, we have $(A / m) \otimes M=$

$M / m M \neq O$ by hypothesis.

(iii) $\Rightarrow$ (ii): take an element $x \in \mathrm{N}, \mathrm{x} \neq 0$. The submodule Ax is a homomorphic image of A as A-module, hence

$\mathrm{Ax} \simeq \mathrm{A} / \mathrm{I}$ for some ideal $I \neq \mathrm{A}$. Let sh be a maximal ideal

of A containing I. Then $M \supset m M \supseteq I M$, therefore (A/I) $\ M=$

$M / I M \neq 0 . \quad$ By flatness $0 \rightarrow(A / I) \otimes M \rightarrow N \otimes M$ is exact, hence $\mathrm{N} \curvearrowright \mathrm{M} \neq 0$

(ii) $\Rightarrow$ (i): let $S: \mathrm{N}^{\prime} \rightarrow \mathrm{N} \rightarrow \mathrm{N}^{\prime \prime}$ be a sequence of $\mathrm{A}-$

modules, and suppose that

$S \otimes M: N^{\prime} \otimes M \stackrel{\mathrm{f}_{M}}{\rightarrow} \mathrm{N} \otimes M \stackrel{\mathrm{g}_{\mathrm{M}}}{\rightarrow} \mathrm{N}^{\prime \prime} \otimes \mathrm{M}$

is exact. As $M$ is flat, the exact functor $\otimes M$ transforms

kernel into kernel and image into image. Thus $\operatorname{Im}(g \circ f) \bigcirc M=$ $\operatorname{Im}\left(g_{M} \circ f_{M}\right)=0$, and by the assumption we get $\operatorname{Im}(g \circ f)=0$, homology (at $\mathrm{N})$, we have $H(S) \otimes M=H(S \otimes M)=0$. Using again the assumption (ii) we obtain $\mathrm{H}(S)=0$, which implies that $S$ is exact.

Q.E.D.

COROLLARY, Let $A$ and $B$ be local rings, and $\psi: A \rightarrow B$ a local

homomorphism. Let $M(\neq 0)$ be a finite B-module. Then

$M$ is flat over A $\Leftrightarrow M$ is f.f. over A.

In particular, $B$ is $f l a t$ over $A$ iff it is f.f. over $A$.

Proof, Let $m$ and in be the maximal ideals of $A$ and B respectively. Then onM $\subseteq M M$ since $\psi$ is local, and $M M \neq M$ by NAK, hence the assertion follows from the theorem.

(4.B) Just as flatness, faithful flatness is transitive

(B is f.f. A-algebra and $M$ is $f . f . B-m o d u l e \Rightarrow M$ is $f . f .$ over A) and is preserved by change of base ( $M$ is $f, f$. A-module and $B$ is any $A-a l g e b r a \Rightarrow M \otimes_{A} B$ is f.f. B-module).

Faithful flatness has, moreover, the following descent property: if $B$ is an A-algebra and if $M$ is a $f . f . B$-module which is also f.f. over $A$, then $B$ is $f . f$. over $A$.

Proofs are easy and left to the reader.

(4.C) Faithful flatness is particularly important in the case of a ring extension. Let $\psi: A \rightarrow B$ be a $f . f$. homomorph- ism of rings. Then:

(i) For any A-module $N$, the map $N \rightarrow N \otimes B$ defined by $x t+x \otimes 1$ is injective. In particular $\psi$ is injective and A can be viewed as a subring of $B$.

(ii) For any ideal I of $A$, we have $I B \cap A=I$.

(iii) $a_{\psi}: \operatorname{Spec}(B) \rightarrow \operatorname{Spec}(\mathrm{A})$ is surjective.

Proof. (i) Let $0 \neq x \varepsilon N$. Then $0 \neq A x \subseteq N$, hence $A x \otimes B$

$\subseteq N \otimes B$ by flatness of $B$. Then $A x \otimes B=(x \otimes 1) B$, therefore $x \otimes 1 \neq 0$ by $T h .2$

(ii) By change of base, $B \otimes{ }_{A}(A / I)=B / I B$ is $f . f$. over $A / I$. Now the assertion follows from (i).

(iii) Let $p \varepsilon \operatorname{Spec}(A)$. The ring $B_{p}=B \otimes A_{p}$ is $f . f$. over $A_{p}$, hence $p B_{p} \neq B_{p}$. Take a maximal ideal of of $B_{p}$ which contains $p B_{p}$. Then mnA $p \geq P A_{p}$, therefore in $n A_{p}=p A_{p}$ because $p A_{P}$ is maxima1. Putting $P=m, B$, we get $P \cap A=($ fr $\cap B) \cap A$ $=m \cap A=\left(w \cap A_{P}\right) \cap A=P A \cap A=P_{P}$ Q.E.D.

(4.D) THEOREM 3. Let $\psi: A \rightarrow B$ be a homomorphism of

rings, The following conditions are equivalent.

(1) $\psi$ is faithfully flat;

(2) $\psi$ is flat, and $\psi$ : $\operatorname{Spec}(B)+\operatorname{Spec}(A)$ is surjective;

(3) $\psi$ is flat, and for any maximal ideal of A there exists a maximal ideal th' of B lying over ur. Proof. $(1) \Rightarrow(2)$ is already proved.

$(2) \Rightarrow(3)$. By assumption there exists $p^{\prime} \varepsilon \operatorname{Spec}(B)$ with $P^{\prime} \cap A=$ Hr. If $W^{\prime}$ is any maximal ideal of $B$ containing $P^{\prime}$, we have $H^{\prime} \cap A=m$ as is maximal.

(3) $\Rightarrow$ (1). The existence of $w^{\prime}$ implies $\operatorname{riv} B \mathrm{~B}$. Therefore $B$ is $f . f$. over A by Th. 2 .

Remark. In algebraic geometry one says that a morphism $f$ : $X \rightarrow Y$ of preschemes is faithfully flat if $f$ is flat (i.e. for al1 $x \in X$ the associated homomorphisms $O_{Y, f(x)} \rightarrow O_{X, X}$ are $f l a t$ ) and surjective.

Let $A$ be a ring and $B$ a faithfully flat A-algebra. Let $M$ be an A-module. Then:

(i) $M$ is flat (resp.f.f.) over $A \Leftrightarrow M \otimes_{A} B$ is so over B,

(ii) when A is local and $M$ is finite over A we have $M$ is A-free $\Leftrightarrow M \otimes_{A} B$ is B-free.

Proof. (1). The implication $(\Rightarrow)$ is nothing but a change of base $((3 . C)$ and $(4 . B))$, while $(\Leftrightarrow)$ follows from the fact that, for any sequence $S$ of $A$-modules, we have $\left(S \otimes{ }_{A} M\right){ }_{A} B=$ $\left(S \otimes_{A} B\right) \otimes_{B}\left(M \otimes_{A} B\right) \cdot \quad(i i) . \quad(\Rightarrow)$ is trivial. $(\leftarrow)$ follows From (1) because, under the hypothesis, freeness of $M$ is equivalent to $f l a t n e s s$ as we saw in $(3 . G)$. (4.F) REMARK. Let $V$ be an algebraic variety over $C$ and let $x \in V$ (or more generally, let $V$ be an algebraic scheme over $C$ and let $x$ be a closed point on $V$ ). Let $V^{h}$ denote the complex space obtained from $V$ (for the precise definition see Serre's paper cited below), and let 0 and $0^{h}$ be the local rings of $x$ on $V$ and on $V^{h}$ respectively. Locally, one can Assune that $V$ is an algebraic subvariety of the affine $n-s p a c e$ $A_{n}$ Then $V$ is defined by an ideal I of $R=C\left[X_{1}, \ldots, X_{n}\right]$, and taking the coordinate system in such a way that $x$ is the origin we have $I \subseteq m=\left(x_{1}, \ldots, x_{n}\right)$ and $O=R_{m} / I R_{1 n}$. Prthermore, denoting the ring of convergent power series in $x_{1}, \cdots, x_{n}$ by $S=C\left\{\left\{x_{1}, \cdots, x_{n}\right\}\right\}$, we have $0^{h}=s /$ IS by definition. Let $F$ denote the formal power serfes ring: $F=$ $C\left[\left[X_{1}, \ldots, X_{n}\right]\right]$. It has been known long since that $O$ and $0^{h}$ are noetherian local rings. J. $-P$. Serre observed that the completion $\left(0^{h}\right)^{\wedge}\left(\mathrm{Cf}\right.$. Chap, 3) of $0^{\mathrm{h}}$ is the same as the completion $\hat{O}=F /$ IF of 0 , and that $\hat{O}$ is faithfully flat over 0 as well as over $0^{h}$. It follows by descent that $0^{h}$ is faithfully flat over $O$, and this fact was made the basis of Serre's famous paper GAGA (Géométrie algébrique et géométrie analytique, Ann. Inst. Fourier, Vo1.6, 1955/56). It was in the appendix to this paper that the notions of flatness and faithful

flatness were defined and studied for the first time. Exercise. Let $A$ be an integral domain and $B$ an integral domain containing A and having the same quotient field as A. Prove that $B$ is $f . f$. over $A$ only when $B=A$. (Geometrically, this means that if a birational morphism $f: X \rightarrow Y$ is flat at a point $x \in X$, then it is biregular at $x_{0}$ )

\section{Going-up and Going-down}
(5.A) Let $\phi: A \rightarrow B$ be a homomorphism of rings. We say that the going-up theorem holds for $\phi$ if the following condition is satisfied:

(GU) for any $p, p^{\prime} \varepsilon \operatorname{Spec}(\mathrm{A})$ such that $p \subset p^{\prime}$, and for any $P \varepsilon \operatorname{Spec}(B)$ lying over $p$, there exists $P^{\prime} \varepsilon \operatorname{Spec}(B) 1 y$ ing over $p^{\prime}$ such that $P \subset P^{\prime}$.

Similarly, we say that the going-down theorem holds for $\phi$ if the following condition is satisfied:

(GD) for any $p, p^{\prime} \varepsilon \operatorname{Spec}$ (A) such that $p \subset p^{\prime}$, and for any $P^{\prime} \varepsilon$ Spec (B) 1ying over $P^{\prime}$, there exists $P \varepsilon$ spec (B) 1ying over $P$ such that $P \subset P^{\dagger}$.

(5.B) The condition (GD) is equivalent to:

(GD') for any $p \in \operatorname{Spec}(\mathrm{A})$, and for any minimal prime overideal $P$ of $P B$, we have $\operatorname{Pn} A=p$. . Proof. (GD) $\Rightarrow\left(G D^{\prime}\right)$ : let $P$ and $P$ be as in (GD'). Then $P \cap A \geqslant P$ since $P \supseteq P B$. If $P \cap A \neq P$, by (GD) there exists $P_{1} \varepsilon \operatorname{spec}(\mathrm{B})$ such that $\mathrm{P}_{1} \cap \mathrm{A}=P$ and $\mathrm{P} \supset \mathrm{P}_{I}$. Then $P \supset P_{1} \geq$ $P B$, contradicting the minimality of $P$.

$\left(G^{\prime}\right) \Rightarrow(G D)$ : left to the reader.

Remark. Put $\mathrm{X}=\operatorname{Spec}(\mathrm{A}), \mathrm{Y}=\operatorname{Spec}(\mathrm{B}), \mathrm{f}=\mathrm{a}_{\phi: \mathrm{Y}} \rightarrow \mathrm{X}$, and

suppose B is noetherian. Then (GD') can be formulated geometrically as follows: let $p \varepsilon \mathrm{X}$, put $\mathrm{X}^{\prime}=\mathrm{V}(p) \subseteq \mathrm{X}$ and let $\mathrm{Y}^{\prime}$ be an arbitrary irreducible component of $\mathrm{f}^{-1}\left(\mathrm{X}^{\prime}\right)$. Then $\mathrm{f}$ maps $Y^{\prime}$ generically onto $X^{\%}$ in the sense that the generic

point of $Y^{\prime}$ is mapped to the generic point $p$ of $\left.X^{\prime} \cdot *\right)$

(5. C) EXAMPLE. Let $k[x]$ be a polynomial ring over a

field $k$, and put $x_{1}=x(x-1), \quad x_{2}=x^{2}(x-1)$. Then $k(x)$

$=k\left(x_{1}, x_{2}\right)$, and the inclusion $k\left[x_{1}, x_{2}\right] \subseteq k[x]$ induces

a birational morphism

$f: C=\operatorname{Spec}(k[x]) \rightarrow C^{p}=\operatorname{Spec}\left(k\left[x_{1}, x_{2}\right]\right)$

where $C$ is the affine line and $C^{\prime}$ is the affine curve $x_{1}^{3}-x_{2}^{2}+x_{1} x_{2}=0$. The morphism $f$ maps the points $Q_{1}$ :

$x=0$ and $Q_{2}: x=1$ of $C$ to the same point $P=(0,0)$ of

$C^{\prime}$, which is an ordinary double point of $C^{\prime}$, and $f$ maps

*) See (6.A) and (6.D) for the definitions of irreducible component and of generic point. $C-\left\{Q_{1}, Q_{2}\right\}$ bijectively onto $C-\{P\}$

Let $y$ be another indeterminate, and put $B=k[x, y]$, $A=k\left[x_{1}, x_{2}, y\right]$. Then $Y=\operatorname{Spec}(B)$ is a plane and $X=\operatorname{Spec}(A)$ is $C^{\prime} \times$ IIne; $X$ is obtained by identifying the lines $L_{1}$ : $x=0$ and $L_{2}: x=1$ on $Y \cdot$ Let $L_{3} \subset Y$ be the 1 ine defined by $y=a x$, a $\neq 0$. Let $g: Y \rightarrow X$ be the natural morphism. Then $g\left(L_{3}\right)=X^{\prime}$ is an irreducible curve on $X$, and $g^{-1}\left(X^{\prime}\right)=L_{3} \cup\{(0, a),(1,0)\}$

Therefore the going-down theorem does not hold for $A \subset B$.

THEOREM 4. Let $\phi: A \rightarrow B$ be a flat homomorphism of rings. Then the going-down theorem holds for $\phi$.

Proof. Let $p$ and $p^{\prime}$ be prime ideals in $A$ with $p^{\prime} \subset p$, and

\includegraphics[max width=\textwidth]{2022_08_01_8d4eee36f1f42236b4f4g-025}\\
over $A_{P}$ by $(3 . J)$, hence faithfully flat since $A_{P} \rightarrow B_{P}$ is local. Therefore $\operatorname{Spec}\left(B_{P}\right) \rightarrow \operatorname{Spec}\left(A_{p}\right)$ is surjective. Let $P^{\prime} *$ be a prime ideal of $B_{P}$ lying over $P^{\prime} A^{\prime} P^{\prime}$ Then $P^{\prime}=P^{\prime} * \cap B$ is a prime ideal of $B$ lying over $P^{\prime}$ and contained in $P . Q . E . D$.

(5.E) THEOREM 5. *) Let B be a ring and A a subring over which B is integral. Then:

\begin{enumerate}
  \item The canonical map $\operatorname{Spec}(B) \rightarrow \operatorname{Spec}(A)$ is surjective. *) This theorem is due to Krull, but is often called the Cohen-\\
Seidenberg theorem, ii) There is no inclusion relation between the prime ideals of B lying over a fixed prime ideal of A.
\end{enumerate}
iii) The going-up theorem holds for $\mathrm{A} \subset \mathrm{B}$.

iv) If $A$ is a local ring and $p$ is its maximal ideal, then the prime ideals of $B$ lying over $p$ are precisely the maximal ideals of $\mathrm{B} .$

Suppose furthermore that $A$ and $B$ are integral domains and that $A$ is integrally closed (in its quotient field $\Phi A$ ). Then we also have the following.

v) The going-down theorem holds for $A \subset B$.

vi) If $B$ is the integral closure of $A$ in a normal exten-

\includegraphics[max width=\textwidth]{2022_08_01_8d4eee36f1f42236b4f4g-026}\\
over the same prime $p \varepsilon \operatorname{Spec}(A)$ are conjugate to each other by some automorphism of L over $K$.

Proof. iv) First let $M$ be a maximal ideal of $B$ and put $\pi /$ $=\mathrm{M} \cap \mathrm{A} \cdot$ Then $\overline{\mathrm{B}}=\mathrm{B} / \mathrm{M}$ is a field which is integral over the subring $\bar{A}=\mathrm{A} / \mathrm{m}$. Let $0 \neq \mathrm{x} \varepsilon \overline{\mathrm{A}}$. Then $I / \mathrm{x} \varepsilon \overline{\mathrm{B}}$, hence
$$
(1 / x)^{n}+a_{1}(1 / x)^{n-1}+\cdots+a_{n}=0 \text { for some } a_{1} \varepsilon \overline{A_{0}}
$$
Multiplying by $x^{n-1}$ we get $1 / x=-\left(a_{1}+a_{2} x+\cdots+a_{n} x^{n-1}\right)$ $\varepsilon \bar{A}$. Therefore $\bar{A}$ is a field, i.e. $m=M \cap A$ is the maximal ideal $P$ of $A$. Next, let $P$ be a prime ideal of $B$ with $P \cap A=$ p. Then $\bar{B}=B / P$ is a domain which is integral over the field $\bar{A}=A / p . \quad$ Let $0 \neq y \in \bar{B} ;$ let $y^{n}+a_{1} y^{n-1}+\cdots+a_{n}=0$ (a $a_{i} \in \bar{A}$ ) be a relation of integral dependence for $y$, and assume that the degree $n$ is the smallest possible. Then $a_{n}$ $\neq 0$ (otherwise we could divide the equation by y to get a relation of degree $n-1)$. Then $y^{-1}=-\left(y^{n-1}+a_{1} y^{n-2}+\cdots\right.$ $\left.+a_{n-1}\right) / a_{n} \varepsilon \bar{B}$, hence $\bar{B}$ is a field and $P$ is maximal.

\begin{enumerate}
  \item and ii). Let $p \varepsilon \operatorname{spec}(\mathrm{A})$. Then $B_{p}=B_{A} \otimes_{p}=$ $(A-P)^{-1}$ is integral over $A_{P}$ and contains it as a subring. The prime ideals of $B$ lying over $p$ correspond to the prime ideals of $B_{p}$ lying over $p A_{p}$, which are the maximal ideals of ${ }^{B} p$ by iv). Since $A_{p} \neq 0, B_{p}$ is not zero and has maximal ideals. Of course there is no inclusion relation between maximal ideals. Thus i) and ii) are proved.
\end{enumerate}
iii). Let $P \subset p^{\prime}$ be in $\operatorname{Spec}(A)$ and $P$ be in $\operatorname{Spec}(B)$ such that $P \cap A=p$. Then $B / P$ contains, and is integral over, $A / P$. By i) there exists a prime $P^{\prime} / P$ ying over $p^{\prime} / p$. Then $P^{\prime}$ is a prime ideal of $B$ lying over $p^{\prime}$.

vi). Put $G=$ Aut $(L / K)=$ the group of automorphisms of L over $K$. First assume $L$ is finite over $K$. Then $G$ is finfte: $G=\left\{\sigma_{1}, \ldots, \sigma_{n}\right\}$. Let. $P$ and $P^{\prime}$ be prime ideals of $B$ such that Pn $A=P^{\prime} \cap A \cdot$ Put $\sigma_{i}(P)=P_{i^{\circ}} \quad$ (Note that $\sigma_{i}(B)=B$ so that $\left.P_{i} \varepsilon \operatorname{Spec}(\mathrm{B}) .\right)$ If $\mathrm{P}^{\prime} \neq P_{i}$ for $i=1, \ldots, n$, then $P^{\prime} \notin P_{i}$ by ii), and there exists an element $x \& P^{\prime}$ which is not in any $P_{i}$ by $(1 . B)$. Put $y=\left(\prod_{i}(x)\right)^{q}$, where $q=1$ if $\operatorname{ch}(\mathrm{K})=0$ and $q=p^{\nu}$ with sufficiently large $v$ if $\operatorname{ch}(K)=p$. Then $y \varepsilon \mathrm{K}$, and since A is integrally closed and $\mathrm{y} \varepsilon$ B we get $y \varepsilon A$. But $y \notin P$ (for, we have $x \notin \sigma_{i}^{-1}(P)$ hence $\sigma_{i}(x) \notin P$ ) while y $\varepsilon P^{\prime} \cap A=P \cap A$, contradiction.

When $L$ is indinite over $K$, let $K^{\prime}$ be the invariant subfield of $G$; then $L$ is Galois over $K^{\prime}$, and $K^{\prime}$ is purely inseparable over $K$. If $K^{\prime} \neq k$, let $p=c h(K)$. It is easy to see that the integral closure $B^{\prime}$ of $A$ in $K^{\prime}$ has one and only one prime $p^{\prime}$ which lies over $p$, namely $p^{\prime}=\left\{\mathrm{x} \varepsilon \mathrm{B}^{\prime} \mid \exists \mathrm{q}=\mathrm{p}^{\nu}\right.$ such that $\left.\mathrm{x}^{q} \in p\right\}$. Thus we can replace $K$ by $K^{\prime}$ and $p$ by $p$ in this case. Assume, therefore, that $\mathrm{L}$ is Galois over $\mathrm{k}$. Let $P$ and $P^{\prime}$ be in $\operatorname{Spec}(B)$ and let $P \cap A=P^{\prime} \cap A=P$. Let $L$ be any finite Galois extension of $\mathrm{K}$ contained in $\mathrm{L}$, and put $F\left(L^{\prime}\right)=\left\{\sigma \varepsilon G=\operatorname{Aut}(I / K) \mid \sigma\left(P \cap L^{\prime}\right)=P^{\prime} \cap L^{\prime}\right\}$

This set is not empty by what we have proved, and is closed in $G$ with respect to the Krull topology (for the Krull topology of an infinite Galois group, see Lang: Algebra, p.233 Clearly $F\left(L^{\prime}\right) \geq F\left(L^{\prime \prime}\right)$ if $L^{\prime} S$ exercise 19.) L". For any finite number of finite Galois extensions $L^{\prime}{ }_{i}$ $(1 \leqslant i \leqslant n)$ there exists a finite Galois extension L" containing all $L_{i}^{\prime}$, therefore $\bigcap_{i} F\left(L_{i}^{\prime}\right) \supseteq F\left(L^{\prime \prime}\right) \neq \emptyset$. As $G$ is compact this means all $L^{\prime} F\left(L^{\prime}\right) \neq \emptyset$. If $\sigma$ belongs to this intersection we get $\sigma(\mathrm{P})=\mathrm{P}^{\prime}$.

v) Let $\mathrm{L}_{1}=\Phi \mathrm{B}, \mathrm{K}=\Phi \mathrm{A}$, and let $\mathrm{L}$ be a normal extension of $K$ containing $L_{1}$; let $C$ denote the integral closure of $A$ (hence also of $B$ ) in L. Let $P \varepsilon \operatorname{Spec}(B), P=P \cap A, p^{\prime} \varepsilon$ $\operatorname{Spec}(A)$ and $P^{\prime} \subset P$. Take a prime ideal $Q^{\prime} \varepsilon$ Spec(C) lying over $p^{\prime}$, and, using the going-up theorem for $A \subset C$, take $Q_{1} \varepsilon$ Spec (C) lying over $P$ such that $Q^{\prime} C Q_{1} \cdot$ Let $Q$ be a prime ideal of $C$ lying over P. Then by vi) there exists $\sigma \varepsilon \operatorname{Aut}(\mathrm{L} / \mathrm{K})$ such that $\sigma\left(Q_{1}\right)=Q$. Put $\mathrm{P}^{\prime}=\sigma\left(Q^{\prime}\right) \cap \mathrm{B} .$ Then $P^{\prime} \subset P$ and $P^{\prime} \cap A=\sigma\left(Q^{\prime}\right) \cap A=Q^{\prime} \cap A=P^{\prime}$ O.E.D.

Remark. In the example of $(5 . C)$, the ring $B=k[x, y]$ is integral over $\mathrm{A}=\mathrm{k}\left[\mathrm{x}_{1}, \mathrm{x}_{2}, \mathrm{y}\right]$ since $\mathrm{x}^{2}-\mathrm{x}-\mathrm{x}_{1}=0$. Therefore the going-up theorem holds for $A \subset B$ while the going-down does not.

EXERCISES. 1. Let $A$ be a ring and $M$ an A-module. We shall say that $M$ is surjectively-free over $A$ if $A=\sum f(M)$ where sum is taken over $f \varepsilon \operatorname{Hom}_{A}(M, A)$. Thus, free $\Rightarrow$ surjectivelyfree. Prove that, if $B$ is a surjectively free A-algebra, then (i) for any ideal I of $A$ we have $I B \cap A=I$, and (ii) the canonical map $\operatorname{Spec}(\mathrm{B}) \rightarrow \operatorname{Spec}(\mathrm{A})$ is surjective. Prove also that, if $B$ is an A-algebra with retraction (i.e. an A-linear map $I: B \rightarrow A$ such that $r \circ 1=1 d$ (where $1: A \rightarrow B$ is the canonical map)) is surjectively-free over A.

\begin{enumerate}
  \setcounter{enumi}{2}
  \item Let $k$ be a field and $t$ and $x$ be two independent indeterminates. Put $A=k[t](t)$. Prove that $A[X]$ is free (hence faithfully flat) over A but that the going-up theorem does not hold for $A \subset A[X]$. Hint: consider the prime ideal $(t x-1)$ 3. Let $B$ be a ring, $A$ be a subring and $p \in \operatorname{Spec}(A)$. Suppose that $B$ is integral over $A$ and that there is only one prime ideal $P$ of $B$ lying over $p .$ Then $B_{P}=B_{P} \cdot$ (By ${ }_{P}$ we mean the localization of the A-module $B$ at $p$, i.e. $B_{p}=$ ${ }_{B} \otimes_{A} A^{\circ}$. Show that $B_{p}$ is a local ring with maximal ideal $P_{P^{\circ}}$ )
\end{enumerate}
\section{Constructible Sets}
(6.A) A topological space $X$ is said to be noetherian if the descending chain condition holds for the closed sets in X. The spectrum $\operatorname{Spec}(\mathrm{A})$ of a noetherian ring A is noetherian. If a space is covered by a finite number of noetherian subspaces then it is noetherian. Any subspace of a noetherian space is noetherian. A noetherian space is quasi-compact.

A closed set $Z$ in a topological space $X$ is irreducible if it is not expressible as the sum of two proper closed subsets. In a noetherian space $X$ any closed set $Z$ is uniqueIy decomposed into a finite number of irreducible closed sets : $Z=Z_{1} \cup \ldots Z_{r}$ such that $Z_{i} \neq Z_{j}$ for $i \neq j$. This follows easily from the definitions. The $\mathrm{Z}_{\mathbf{i}}$ 's are called the irreducible components of $\mathrm{Z}$.

(6.B)

Let $X$ be a topological space and $Z$ a subset of $X$. We say $Z$ is locally closed in $X$ if, for any point $z$ of $z$, there exists an open neighborhood $U$ of $z$ in $X$ such that Un $Z$ is closed in U. It is easy to see that $z$ is locally closed in $X$ iff it is expressible as the intersection of an open set in $X$ and a closed set in $X$.

Let $X$ be a noetherian space. We say a subset $Z$ of $X$ is a constructible set in $X$ if $Z$ is a finite union of locally closed sets in $\mathrm{X}$ :
$$
Z=\bigcup_{i=1}^{m}\left(U_{i} \wedge F_{i}\right), \quad U_{i} \text { open, } F_{i} \text { closed. }
$$
(When $X$ is not noetherian, the definftion of a constructible set is more complicated, cf. EGA $0_{\text {III }}$ )

If $Z$ and $Z^{\prime}$ are constructible in $X$, so are $Z \cup Z^{\prime}$, $Z_{n} Z^{\prime}$ and $Z-Z^{\prime}$. This is clesr for $Z \cup Z^{\prime}$. Repeated use of the formula
$$
(U \cap F)-\left(U^{\prime} \cap F^{\prime}\right)=U \cap F \cap\left(C\left(U^{\prime}\right) \cup C\left(F^{\prime}\right)\right)
$$
$=\left[U \cap\left\{F \cap C\left(U^{\prime}\right)\right\}\right] \cup\left[\left\{U \cap C\left(F^{\prime}\right)\right\} \cap F\right]$, where $C\left(\right.$ denotes the complement in $X$, shows that $Z-Z^{\prime}$ is constructible. Taking $\mathrm{Z}=\mathrm{X}$ we see the complement of a constructible set is constructible. Finally, $Z \cap Z^{\prime}=C(C(Z)$ $\left.U C\left(Z^{\prime}\right)\right)$ is constructible.

We say a subset $Z$ of a noetherian space $X$ is proconstructible (resp, ind-constructible) if it is the intersection (resp, union) of an arbitrary collection of construct-

\section{ible sets in $X .$}
\section{(6.C) PROPOSITION, Let $\mathrm{X}$ be a noetherian space and $\mathrm{Z}$ a}
subset of $X$. Then $Z$ is constructible in $X$ if the following condition is satisfied.

(*) For each irreducible closed set $X_{0}$ in $X$, either $X_{0} \cap Z$ is not dense in $X_{0}$, or $X_{0} \cap Z$ contains a non-empty open set of $\mathrm{X}_{0}$

Proof. (Necessity,) If $z$ is constructible we can write
$$
x_{0} \cap Z=\bigcup_{i=1}^{m}\left(U_{i} \cap F_{i}\right)
$$
where $U_{i}$ is open in $X, F_{i}$ is closed and irreducible in $X$ and $U_{i} \cap F_{i}$ is not empty for each $i$. Then $\overline{U_{i} \cap F_{i}}=F_{i}$ since $F_{i}$ is irreducible, therefore $\overline{X_{0} \cap Z}=\bigcup_{i} F_{i}$. If $X_{0} \cap Z$ is dense in $X_{0}$, we have $X_{0}=\bigcup F_{1}$ so that some $F_{i}$, say $F_{1}$, is equal to $X_{0}$. Then $U_{1} \cap X_{0}=U_{1} \cap F_{1}$ is a non-empty open set of $X_{0}$ contained in $x_{0} \cap 2$

(Sufficiency.) Suppose (\textit{) holds. We prove the constructibility of $Z$ by induction on the smallness of $\bar{Z}$, using the fact that $X$ is noetherian. The empty set being constructible, we suppose that $Z \neq \emptyset$ and that any subset $Z^{\prime}$ of $Z$ which satisfies (}) and is such that $\overline{Z^{\top}} \subset \bar{Z}$ is constructible.

Let $\bar{Z}=F_{1} \cup \ldots \cup F_{r}$ be the decomposition of $\bar{Z}$ into the irreducible components. Then $F_{1} \cap Z$ is dense in $F_{1}$ as one can easily check, whence there exists, by $(*)$, a proper closed subset $F^{\prime}$ of $F_{1}$ such that $F_{1}-F^{\prime} \subseteq Z$. Then, putting $F^{*}=$ $F^{\prime} \cup F_{2} \cup . . F_{r^{\prime}}$ we have $Z=\left(F_{1}-F^{\prime}\right) \cup\left(Z \cap F^{*}\right)$. The set $\mathrm{F}_{I}-\mathrm{F}^{*}$ is locally closed in $\mathrm{X}$. On the other hand $\mathrm{Z} \cap \mathrm{F}^{*}$ satisfies the condition $(*)$ because, if $X_{0}$ is irreducible and if $\bar{Z}^{n} F^{*} \cap X_{0}=X_{0}$, the closed set $F^{*}$ must contain $X_{0}$ and so $2 \cap F * \cap X_{0}=2 \cap X_{0}$ Since $\overline{Z n F *} \subseteq F^{*} \subset \bar{Z}$, the set $Z n F *$ (s constructible by the induction hypothesis. Therefore 2 is constructible.

(6.D) LEMMA 1. Let $A$ be a ring and $F$ a closed subset of $X=\operatorname{spec}(A)$. Then $F$ is irreducible iff $F=V(p)$ for some prime ldeal $p$. This $p$ is unique and is called the generic point of F.

Proof. Suppose that $F$ is irreducible. Since it is closed it can be written $F=V(I)$ with $I=\bigcap_{p \varepsilon F}$. If I is not prime we would have elements $a$ and $b$ of $A$ - I such that $a b \varepsilon I$. Then $F \leqslant V(a), F \leqslant V(b)$ and $F \subseteq V(a) \cup V(b)=V(a b)$, hence $F=(F \cap V(a)) \cup(F \cap V(b))$, which contradicts the irreducibility. The converse is proved by noting $p \in V(p)$. The uniqueness comes from the fact that $P$ is the smallest element of $V(P)$.

LEMMA 2. Let $\phi: A \rightarrow B$ be a homomorphism of rings, Put $X=$ $\operatorname{Spec}(A), Y=\operatorname{Spec}(B)$ and $f={ }^{a} \phi: Y+X . \quad$ Then $f(Y)$ is dense in $x$ iff $\operatorname{Ker}(\phi) \subseteq n i l(A)$. If, in particular, A is reduced, $f(Y)$ is dense in $X$ iff $\phi$ is injective.

Proof. The closure $\overline{f(Y)}$ in $\operatorname{Spec}(A)$ is the closed set $V(I)$ defined by the ideal $I=\bigcap_{p \varepsilon Y} \phi^{-1}(p)=\phi^{-1}\left(\bigcap_{p \in Y} p\right)$, which is equal to $\phi^{-1}(n i 1(B))$ by $(1 . E) .$ Clearly $\operatorname{Ker}(\phi) \subseteq I$. Suppose that $f(Y)$ is dense in $X$. Then $V(I)=X$, whence $I=\operatorname{nil}(A)$ by ( $I . E)$. Therefore $\operatorname{Ker}(\phi) \subseteq \operatorname{nil}(A)$. Conversely, suppose $\operatorname{Ker}(\phi) \subseteq$ nil(A). Then it is clear that $I=\phi^{-1}(\operatorname{nil}(B))=\operatorname{nil}(A)$ which means $\overline{f(Y)}=V(I)=X$.

(6.E) THEOREM 6. (Chevalley). Let A be a noetherian ring and $\mathrm{B}$ an A-algebra of finite type. Let $\phi: \mathrm{A} \rightarrow \mathrm{B}$ be the canonical homomorphism; put $X=\operatorname{Spec}(A), Y=\operatorname{Spec}(B)$ and $f={ }^{a} \phi$ : $Y \rightarrow X$. Then the image $f\left(Y^{\prime}\right)$ of a constructable set $Y^{\prime}$ in $Y$ is constructable in $\mathrm{X}$.

Proof. First we show $(6.0)$ can be applied to the case when $Y^{\prime}=Y$. Let $X_{0}$ be an irreducible closed set in $X_{0}$. Then $X_{0}$ $=V(p)$ for some $p \varepsilon \operatorname{spec}(\mathrm{A})$. Put $\mathrm{A}^{+}=\mathrm{A} / p$, and $\mathrm{B}^{\prime}=\mathrm{B} / p \mathrm{~B}$. Suppose that $X_{0} \cap f(Y)$ is dense in $X_{0}$. The map $\phi^{\prime}: A^{\prime} \rightarrow B^{\prime}$ induced by $\phi$ is then injective by Lemma 2. We want to show $X_{0} \cap f(Y)$ contains a non-empty open subset of $x_{0}$. By replacing $A, B$ and $\phi$ by $A^{\prime}, B^{\prime}$ and $\phi^{\prime}$ respectively, it is enough to prove the following assertion :

(*) If $A$ is a noetherian domain, and if $B$ is a ring which contains A and which is finitely generated over $A$, there exists $0 \neq$ a $\varepsilon$ A such that the elementary open set $D(a)$ of $X=\operatorname{Spec}(A)$ is contained in $f(Y)$, where $Y=\operatorname{Spec}(B)$ and $f$ : $Y \rightarrow X$ is the canonical map.

Write $B=A\left[x_{1}, \ldots, x_{n}\right]$, and suppose that $x_{1}, \ldots, x_{r}$ are algebraically independent over $A$ while each $x_{j}(r<j \leqslant n)$ satisfies algebraic relations over $\mathrm{A}\left[\mathrm{x}_{1}, \ldots, \mathrm{x}_{r}\right]$. Put $\mathrm{A} *=$ $A\left[x_{1}, \ldots, x_{r}\right]$, and choose for each $r<j \leqslant n$ a relation
$$
g_{j 0}(x) \cdot x_{j}{ }_{j}+g_{j 1}(x) \cdot x_{j}{ }_{j}^{-1}+\ldots=0
$$
where $g_{j v}(x) \varepsilon A^{*}, g_{j 0}(x) \neq 0$. Then $\Pi_{j=r+1}^{n} g_{j 0}\left(x_{1}, \ldots, x_{r}\right)$ is a non-zero polynomial in $x_{1}, \ldots, x_{r}$ with coefficients in A. Let a $\varepsilon$ A be any one of the non-zero coefficients of this polynomial. We claim that this element satisfies the requirement. In fact, suppose $p \varepsilon \operatorname{spec}(A)$, a $\& p$, and put $p^{*}=p A^{*}$ $=p\left[x_{1}, \ldots, x_{r}\right]$. Then $\prod_{j 0} \notin p^{*}$, so that $B_{p *}$ is integral over $A^{*} p^{*}$ Thus there exists a prime $P$ of $B^{p^{*}}$ lying over $p^{*} A^{*} p^{* \cdot}$ We have $P \cap A=P \cap A^{*} \cap A=p\left[x_{1}, \ldots, x_{r}\right] \cap A=p$, therefore $P=P \cap A=(P \cap B) \cap A \in f(\operatorname{Spec}(B))$. Thus $(*)$ is proved. The general case follows from the special case treated above and from the following

LEMMA. Let B be a noetherian ring and 1 et $Y^{\prime}$ be a constructible set in $Y=\operatorname{spec}(B)$. Then there exists a $B$-algebra of finite type $B^{\prime}$ such that the image of $\operatorname{Spec}\left(B^{\prime}\right)$ in $\operatorname{Spec}(B)$ is exactly $Y^{\prime}$

Proof. First suppose $\mathrm{Y}^{\prime}=\mathrm{U} \cap \mathrm{F}$, where $\mathrm{U}$ is an elementary

open set $U=D(b), b \varepsilon B$, and $F$ is a closed set $V(I)$ defined

by an ideal I of $B .$ Put $S=\left\{1, b, b^{2}, \ldots .\right\}$ and $B^{\prime}=$

$S^{-1}(B / I)$. Then $B^{\prime}$ is a B-algebra of finite type generated

by $1 / \bar{b}$, where $\bar{b}=$ the image of $b$ in $B^{\prime}$, and the image of

Spec $\left(B^{\prime}\right)$ in $\operatorname{Spec}(B)$ is clearly UnF.

When $Y^{\prime}$ is an arbitrary constructible set, we can write

it as a finite union of locally closed sets $U_{i} \cap F_{i}(1 \leqslant i \leqslant m)$

with $U_{i}$ elementary open, because any open set in the noether-

ian space $Y$ is a finite union of elementary open sets. Choose

a B-algebra $B_{i}$ of finite type such that $U_{i} \sim F_{i}$ is the image

of $\operatorname{Spec}\left(\mathrm{B}_{i}^{\prime}{ }_{i}\right.$ ) for each $i$, and put $B^{\prime}=B^{\prime}{ }_{1} \times \cdots \times B^{\top}{ }^{\top}$. Then

we can view $\operatorname{Spec}\left(\mathrm{B}^{\prime}\right)$ as the disjoint union of $\operatorname{Spec}\left(\mathrm{B}_{i}{ }_{i}\right)^{\prime} s$,

so the image of $\operatorname{Spec}(B)$ in $Y$ is $Y^{\prime}$ as wanted.

(6.F) PROPOSITION. Let $A$ be a noetherian ring, $\phi: A \rightarrow B$ a homomorphism of $r$ ings, $X=\operatorname{Spec}(A), Y=\operatorname{Spec}(B)$, and $f=$

$a_{\phi:} Y \rightarrow X$. Then $f(Y)$ is pro-constructible in $X$.

Proof. We have $\mathrm{B}=\lim _{\lambda} \mathrm{B}_{\lambda}$, where the $\mathrm{B}_{\lambda}{ }^{\prime}$ s are the subalgebras of $B$ which are finitely generated over $A$. Put $Y_{\lambda}=\operatorname{Spec}\left(B_{\lambda}\right)$ and let $g_{\lambda}: \mathrm{Y} \rightarrow \mathrm{Y}_{\lambda}$ and $\mathrm{f}_{\lambda}: \mathrm{Y}_{\lambda} \rightarrow \mathrm{X}$ denote the canonica1 maps. Clearly $f(Y) \subseteq \bigcap_{\lambda}\left(Y_{\lambda}\right)$. Actually the equality holds, for suppose that $p \varepsilon X-f(Y)$. Then $p B B_{p}=B_{p}$, so that there exist elements $\pi_{\alpha} \varepsilon p, b_{\alpha} \varepsilon$ B $(1 \leqslant \alpha \leqslant m)$ and $s \varepsilon A-p$ such that $\sum_{\alpha=1} \pi_{\alpha}\left(b_{\alpha} / s\right)=1$ in B $p^{\prime} i_{.} e_{,} s^{\prime}\left(\sum_{\alpha} b_{\alpha}-s\right)=0$ in B for some $s^{\prime} \varepsilon A-p$. If $B_{\lambda}$ contains $b_{1}, \ldots, b_{m}$ we have $1 \varepsilon p\left(B_{\lambda}\right) p$, therefore $p \notin f_{\lambda}\left(Y_{\lambda}\right)$ for such $\lambda_{\text {. Thus we have }}$ proved $f(Y)=\bigcap f_{\lambda}\left(Y_{\lambda}\right)$. Since each $f_{\lambda}\left(Y_{\lambda}\right)$ is constructible by Th. $6, f(Y)$ is pro-constructible.

Q.E.D.

(Remark. [EGA Ch.IV, §I] contains many other results on constructible sets, including generalization to non-noetherian case.)

(6.G) Let $A$ be a ring and let $p, p^{\prime} \varepsilon$ spec (A). We say that $P^{\prime}$ is a specialization of $P$ and that $P$ is a generalization of $P^{\prime}$ iff $P \subseteq P^{\prime}$. If a subset $Z$ of $\operatorname{Spec}(A)$ contains all specializations (resp. generalizations) of its points, we say $Z$ is stable under specialization (resp. generalization). A

closed (resp. open) set in Spec(A) is stable under speciali- zation (resp, generalization).

LEMMA. Let $A$ be a noetherian ring and $X=\operatorname{Spec}(A)$. Let $Z$ be a pro-constructible set in $\mathrm{X}$ stable under specialization. Then $\mathrm{Z}$ is closed in $\mathrm{X}$.

Proof. Let $z=\cap E_{\lambda}$ with $E_{\lambda}$ constructible in $x$. Let $W$ be an irreducible component of $\bar{z}$ and let $x$ be its generic point. Then $W \cap Z$ is dense in $W$, hence a fortiori $W \cap E_{\lambda}$ is dense in W. Therefore $W \cap E_{\lambda}$ contains a non-empty open set of $W$ by $(6 . c)$, so that $x \varepsilon E_{\lambda}$ Thus $x \varepsilon \cap_{E_{\lambda}}=2$. This means $\mathrm{W} \subseteq \mathrm{Z}$ by our assumption, and so we obtain $\mathrm{Z}=\overline{\mathrm{Z}}_{\text {• }} \quad$ Q.E.D.

(6.H) Let $\phi: A \rightarrow B$ be a homomorphism of rings, and put $X=\operatorname{Spec}(A), Y=\operatorname{Spec}(B)$ and $f={ }^{a} \phi: Y \rightarrow X$. We say that $f$ is (or: $\phi$ is) submersive if $f$ is surjective and if the topology of $X$ is the quotient of that of $Y$ (i.e. a subset $X^{\prime}$ of $X$ is closed in $X$ iff $f^{-1}\left(X^{\prime}\right)$ is closed in $Y$ ). We say $f$ is (or: $\phi$ is) universally submersive if, for any A-algebra $C$, the homomorphism $\phi_{C}: C \rightarrow B \otimes_{A} C$ is submersive, (Submersiveness and universal submersiveness for morphisms of preschemes are defined in the same way, cf. EGA IV $(15.7 .8) .)$

THEOREM 7. Let $A, B, \phi, X, Y$ and $f$ be as above. Suppose that (1) A is noetherian, (2) $f$ is surjective and (3) the going-down theorem holds for $\phi: A \rightarrow B$. Then $\phi$ is submersive Remark. The conditions (2) and (3) are satisfied, e.g., in the following cases:

\includegraphics[max width=\textwidth]{2022_08_01_8d4eee36f1f42236b4f4g-032}

(B) when $\phi$ is injective, assume $B$ is an integral domain over $A$ and $A$ is an integrally closed integral domain. In the case $(\alpha), \phi$ is even universally submersive since faithful flatness is preserved by change of base. $*$ )

Proof of $T h .7$. Let $X^{\prime} \subseteq X$ be such that $\mathrm{f}^{-1}\left(X^{\prime}\right)$ is closed. We have to prove $X^{\prime}$ is closed. Take an ideal $J$ of $B$ such that $f^{-1}\left(X^{\prime}\right)=V(J)$. As $X^{\prime}=f\left(f^{-1}\left(X^{\prime}\right)\right)$ by (2), application of $(6 . F)$ to the composite map $A \rightarrow B \rightarrow B / J$ shows $X^{\prime}$ is pro-constructible. Therefore it suffices, by $(6 . G)$, to prove that $X^{\prime}$ is stable under specialization. For that purpose, let $P_{1}, P_{2} \varepsilon \operatorname{spec}(\mathrm{A}), p_{1} \supset P_{2} \varepsilon X^{\prime}$. Take $P_{1} \varepsilon Y$ lying over $P_{1}$ (by (2)) and $P_{2} \varepsilon Y$ lying over $P_{2}$ such that $P_{1} \supset P_{2}$ (by (3)). Then $P_{2}$ is in the closed set $f^{-1}\left(X^{\prime}\right)$, so $P_{1}$ is also in $f^{-1}\left(X^{p}\right)$. Thus $p_{1}=f\left(P_{1}\right) \varepsilon f\left(f^{-1}\left(X^{1}\right)\right)=X^{\prime}$, as wanted.

*) In algebraic geometry, there are two important classes of universally submersive morphisms. Namely, the faithfu1ly versal submersiveness of ther and surjective ones. The unidefinitions, whiles of the latter is immediate from the just proved. While that of the former is essentially what we just proved. (6.I) THEOREM 8. Let $A$ be a noetherian $r$ ing and $B$ an

A-algebra of finite type. Suppose that the going-down theorem

holds between $A$ and $B$. Then the canonical map $f: S p e c(B)+$

CHAPTER 3. ASSOCIATED PRIMES

$\operatorname{spec}(A)$ is an open map (1.e. sends open sets to open sets).

Proof. Let $U$ be an open set in $\operatorname{Spec}(B)$. Then $f(U)$ is a

In this chapter we consider noetherian rings

constructible set (Th.6). On the other hand the going-down only. theorem shows that $f(U)$ is stable under generalization.

Therefore, applying $(6 . G)$ to $\operatorname{Spec}(A)-f(U)$ we see that $f(U)$

is open.

Q.E.D.

$(6 . J)$

Let $A$ and $B$ be rings and $\phi: A \rightarrow B$ a homomorphism.

Suppose B is noetherian and that the going-up theorem holds

for $\phi .$ Then ${ }^{a} \phi: \operatorname{Spec}(B) \rightarrow \operatorname{Spec}(A)$ is a closed map (i.e.

sends closed sets to closed sets).

Proof. Left to the reader as an easy exercise. (It has

nothing to do with constructible sets.) 7. Ass(M)

(7.A) Throughout this section let A denote a noetherian ring and $M$ an $A$-module. We say a prime ideal $P$ of $A$ is an associated prime of $M$, if one of the following equivalent conditions holds :

(i) there exists an element $x \in M$ with $\operatorname{Ann}(x)=p$;

(ii) M contains a submodule isomorphic to $A / p$.

The set of the associated primes of $M$ is denoted by Ass ${ }_{A}(M)$ or by Ass (M)

(7.B) PROPOSITION. Let $p$ be a maximal element of the set of ideals $\{$ Ann $(x) \mid x \in M, x \neq 0\}$. Then $p \varepsilon \operatorname{Ass}(M)$

Proof. We have to show that $p$ is a prime. Let $p=\operatorname{Ann}(x)$, and suppose $a b \varepsilon p, b \notin p$. Then $b x \neq 0$ and $a b x=0$. Since $\operatorname{Ann}(\mathrm{bx}) \geq \operatorname{Ann}(\mathrm{x})=p$, we have $\operatorname{Ann}(\mathrm{bx})=p$ by the

maximality of $p$. Thus a $\varepsilon p$.

COROLLARY 1. Ass $(M)=\emptyset \Longleftrightarrow M=0$

COROLLARY 2. The set of the zero-divisors for $M$ is the

union of the associated primes of $M$.

(7.C) LEMMA. Let $S$ be a multiplicative subset of $A$,

and put $A^{\prime}=S^{-1} A, M^{\prime}=S^{-1} M$. Then
$$
\text { Ass }_{A}\left(M^{r}\right)=f\left(A s s_{A^{\prime}}\left(M^{\prime}\right)\right)=\text { Ass }_{A}(M) \cap\{p \mid p \cap S=\emptyset\},
$$
where $f$ is the natural map $\operatorname{Spec}\left(A^{\prime}\right) \rightarrow \operatorname{Spec}(A)$.

Proof. Left to the reader. One must use the fact that any

ideal of A is finitely generated.

(7.D) THEOREM 9. Let $A$ be a noetherian ring and $M$ an

A-module. Then Ass $(M) \subseteq \operatorname{Supp}(M)$, and any minimal element

of $\operatorname{Supp}(M)$ is in $\operatorname{Ass}(M)$

Proof. If $p \varepsilon$ Ass $(M)$ there exists an exact sequence

$0 \rightarrow A / P \rightarrow M$, and since $A_{p}$ is flat over $A$ the sequence $0 \rightarrow A_{p} / p A_{p}+M_{p}$ is also exact. As $A_{p} / p A_{p} \neq 0$ we have $M_{p}$ $\neq 0$, i.e. $p \varepsilon \operatorname{Supp}(M)$. Next let $p$ be a minimal element of $\operatorname{Supp}(M)$. By (7.C), $p \varepsilon \operatorname{Ass}(M)$ iff $p A_{p} \in \operatorname{Ass}_{A_{p}}\left(M_{p}\right)$, there-

fore replacing $A$ and $M$ by $A_{P}$ and $M_{p}$ we can assume that $(A, p)$ is a local ring, that $M \neq 0$ and that $M=0$ for any prime $q \subset p .$ Thus $\operatorname{Supp}(M)=\{p\}$. Since Ass $(M)$ is not empty and is contained in $\operatorname{Supp}(M)$, we must have $p \varepsilon$ Ass $(M)$, Q.E.D. COROLLARY. Let I. be an ideal. Then the minimal associated primes of the A-module A/I are precisely the minimal prime over-ideals of $I$.

Remark. By the above theorem the minimal associated primes M are the minimal elements of Supp(M). Associated primes which are not minimal are called embedded primes.

THEOREM 10. Let A be a noetherian ring and $M$ a finite A-module, $M \neq 0$. Then there exists a chain of submodules $(0)=M_{0} \subset \cdots \subset M_{n-1} \subset M_{n}=M$ such that $M_{i} / M_{i-1}$ $\simeq A / p_{i}$ for some $p_{i} \varepsilon \operatorname{spec}(A) \quad(1 \leqslant i \leqslant n)$.

Proof. Since $M \neq 0$ we can choose $M_{1} \leqslant M$ such that $M_{1} \simeq A / P_{1}$ for some $p_{1} \varepsilon$ Ass $(M)$. If $M_{1} \neq M$ then we apply the same procedure to $M / M_{1}$ to find $M_{2}$, and so on. Since the ascending chain condition for submodules holds in $M$, the process must stop in finite steps.

(7.F) LEMMA. If $0 \rightarrow M^{\prime} \rightarrow M \rightarrow M^{\prime \prime}$ is an exact sequence of A-modules, then Ass $(M) \subseteq \operatorname{Ass}\left(M^{\prime}\right) \cup \operatorname{Ass}\left(M^{\prime \prime}\right)$ Proof. Take $P \varepsilon$ Ass $(M)$ and choose a submodule $N$ of $M$ isomorphic to $A / p$. If $N \cap M^{\top}=(0)$ then $N$ is isomorphic to a submodule of $M^{\prime \prime}$, so that $p \in$ Ass $\left(M^{\prime \prime}\right)$. If $N \cap M^{\prime} \neq(0)$, pick $0 \neq x \in N \cap M^{\prime}$. Since $N \simeq A / p$ and since $A / p$ is a domain we have $\operatorname{Ann}(x)=p$, therefore $p \in \operatorname{Ass}\left(M^{\prime}\right) .$

(7.G) PROPOSITION. Let A be a noetherian ring and $M$ a

finite A-module. Then Ass $(M)$ is a finite set.

Proof. Using the notation of Th.10, we have $\operatorname{Ass}(M) \subseteq \operatorname{Ass}\left(M_{1}\right) \cup_{\operatorname{Ass}}\left(M_{2} / M_{1}\right) \cup \ldots \cup_{\operatorname{Ass}\left(M_{n} / M_{n-1}\right)}$

by the lemma, On the other hand we have Ass $\left(M_{i} / M_{i-1}\right)=$ $\operatorname{Ass}\left(\mathrm{A} / p_{i}\right)=\left\{p_{i}\right\}$, therefore $\operatorname{Ass}(M) \subseteq\left\{p_{1}, \ldots, p_{n}\right\}$.

\section{Primary Decomposition}
As in the preceding section, A denotes a noetherian ring and $M$ an A-module.

\section{(8.A) DEFINITIONS. An A-module is said to be co-primary}
if it has only one associated prime. A submodule $\mathrm{N}$ of $M$ is said to be a primary submodule of $M$ if $M / N$ is co-primary.

If $\operatorname{Ass}(M / N)=\{p\}$, we say $N$ is $p$-primary or that $N$ belongs to $p$

(8.B) PROPOSITION The following are equivalent: (1) the module $M$ is co-primary;

(2) $M \neq 0$, and if a $\varepsilon$ A is a zero-divisor for $M$ then a Is locally nilpotent on $M$ (by this we mean that, for each $x \varepsilon M$, there exists an integer $n>0$ such that $\left.a^{n} x=0\right)$

Proof. $(1) \rightarrow(2) .$ Suppose Ass $(M)=\{p\} .$ If $0 \neq x \in M$, then Ass $(A x)=\{p\}$ and hence $p$ is the unique minimal element of $\operatorname{Supp}(A x)=V(\operatorname{Ann}(x))$ by (7.D). Thus $p$ is the radical of Ann $(x)$, therefore a $\varepsilon p$ implies $a^{n} x=0$ for some $n>0$.

(2) $\rightarrow$ (1). Put $P=\left\{\begin{array}{ll}a & A \mid \\ \text { a }\end{array}\right.$ is locally nilpotent on $\left.M\right\}$. Clearly this is an ideal. Let $q \varepsilon \operatorname{Ass}(M)$. Then there exists an element $x$ of $M$ with $A n n(x)=q$, therefore $p \leq q$ by the definition of $p$. Conversely, since $p$ coincides with the union of the associated primes by assumption, we get $q \subseteq p$. Thus $p=q$ and $A s(M)=\{p\}$, so that $M$ is co-primary.

Remark. When $M=A / 9$, the condition (2) reads as follows: $\left(2^{8}\right)$ all zero-divisors of the ring $\mathrm{A} / q$ are nilpotent. This is precisely the classical definition of a primary ideal $q, \operatorname{cf} \cdot(1 . \mathrm{A})$

Exercise. Prove that, if $M$ is a finitely generated co-primary A-module with Ass $(M)=\{p\}$, then the annihilator Ann(M) is a p-primary ideal of A. (8.C) Let $P$ be a prime of $A$, and let $Q_{1}$ and $Q_{2}$ be $p$-primary submodules of $M$. Then the intersection $Q_{1} \cap Q_{2}$ is also $p$ primary.

Proof. There is an obvious monomorphism $M / Q_{1} \cap Q_{2} \rightarrow M / Q_{1}: M / Q_{2}$.

Hence $\emptyset \neq \operatorname{Ass}\left(M / Q_{1} \cap Q_{2}\right) \subseteq \operatorname{Ass}\left(M / Q_{1}\right) \cup \operatorname{Ass}\left(M / Q_{2}\right)=\{p\}$.

(8.D) Let $N$ be a submodule of $M$. A primary decomposition of $N$ is an equation $N=Q_{1} n \ldots \cap Q_{r}$ with $Q_{i}$ primary in $M$. Such a decomposition is said to be irredundant if no $Q_{i}$ can be omitted and if the associated primes of $M / Q_{i}(1 \leqslant i \leqslant r)$ are all distinct. Clearly any primary decomposition can be simplified to an irredundant one.

(8.E) LEMMA. If $N=Q_{1} \cap \ldots \cap Q_{r}$ is an irredundant

primary decomposition and if $Q_{i}$ belongs to $P_{i}$, then we have
$$
\operatorname{Ass}(M / N)=\left\{p_{1}, \ldots, p_{r}\right\}
$$
Proof. There is a natural monomorphism $M / N \rightarrow M / Q_{1} \in \ldots \because M / Q_{r}$, whence $\operatorname{Ass}(M / N) \subseteq U_{i}$ Ass $\left(M / Q_{1}\right)=\left\{p_{1}, \ldots, P_{r}\right\}$. Conversely, $\left(Q_{2} n \ldots \cap Q_{r}\right) / N$ is isomorphic to a non-zero submodule of $M / Q_{1}$ so that $A s s\left(Q_{2} \cap \ldots \cap Q_{r} / N\right)=\left\{p_{1}\right\}$, and since $Q_{2} \cap \ldots \cap Q_{r} / N \simeq$ M/N we have $p_{1} \varepsilon$ Ass $(M / N) .$ Similarly for other $p_{i}^{\prime} s$.

PROPOSITION. Let $N$ be a p-primary submodule of an A-module $M$, and let $p^{\prime}$ be a prime ideal. Put $M^{\prime}=M_{P^{\prime}}$ and $N^{r}=N_{P^{\prime}}$ and let $\nu: M \rightarrow M^{\prime}$ be the canonical map. Then

(i) $\mathrm{N}^{\prime}=\mathrm{M}^{\prime}$ if $p \leq p^{\prime}$,

(ii) $N=v^{-1}\left(N^{\prime}\right)$ if $p \leqslant p^{\prime}$ (symbolically one may write $\left.N=M \cap N^{\prime}\right)$

Proof. (1) We have $M^{\prime} / N^{\prime}=(M / N) P^{\prime}$ and Ass ${ }_{A}\left(M^{\prime} / N^{\prime}\right)=$

Ass $_{A}(M / N) r_{1}\left\{\right.$ primes contained in $\left.P^{\prime}\right\}=\emptyset$. Hence $M^{\prime} / N^{\prime}=0$.

(ii) Since Ass $(M / N)=\{p\}$ and since $p \subseteq p^{\prime}$, the multiplicative set $A-p^{\prime}$ does not contain zero-divisors for $M / N$. Therefore the natural $\operatorname{map} M / N \rightarrow(M / N) p^{\prime}=M^{\prime} / N^{\prime}$ is injective, COROLLARY. Let $N=Q_{1} n \ldots m Q_{T}$ be an irredundant primary decomposition of a submodule $N$ of $M$, let $Q_{1}$ be $p_{1}$-primary and suppose $P_{1}$ is minimal in Ass(M/N). Then $Q_{1}=M \cap N_{P_{1}}$, hence the primary component $Q_{1}$ is uniquely determined by $N$ and by $p_{1}$

Remark. If $P_{i}$ is an embedded prime of $M / N$ then the corresponding primary component $Q_{1}$ is not necessarily unique.

(8.G) THEOREM 11. Let $A$ be a noetherian ring and $M$ an A-module. Then one can choose a $p$-primary submodule $Q(p)$ for each $P \varepsilon \operatorname{Ass}(M)$ in such a way that $(0)=\prod_{\text {peAss }(M)} Q(P)$. Proof. Fix an associated prime $p$ of $M$, and consider the set of submodules $N=\{N \subseteq M \mid P \notin \operatorname{Ass}(N)\}$. This set is not empty since (0) is in it, and if $N^{\prime}=\left\{N_{\lambda}\right\}_{\lambda}$ is a linearly ordered subset of $N$ then $U N_{\lambda}$ is an element of $N$ (because Ass $\left(W_{\lambda}\right)$ $=U \operatorname{Ass}\left(\mathrm{N}_{\lambda}\right)$ by the definition of Ass). Therefore $N$ has maximal elements by Zorn; choose one of them and call it $Q=$ $Q(p)$. Since $P$ is associated to $M$ and not to $Q$ we have $M \neq Q .$ On the other hand, if $M / Q$ had an associated prime $p^{\prime}$ other than $P$, then $M / Q$ would contain a submodule $Q^{\prime} / Q \simeq A / P^{\prime}$ and then $Q^{\prime}$ would belong to $N$ contradicting the maximality of $Q$. Thus $Q=Q(p)$ is a $p$-primary submodule of $M$. As Ass $\left(\bigcap_{n} Q(p)\right)$ $=\bigcap_{\operatorname{Ass}}(Q(p))=\emptyset$ we have $\cap Q(p)=(0)$

COROLLARY. If $M$ is finitely generated then any submodule $N$ of $M$ has a primary decomposition.

Proof. Apply the theorem to $M / N$ and notice that Ass $(M / N)$ is finite.

(8.H) Let $P$ be a prime ideal of a noetherian ring $A$, and let $n>0$ be an integer. Then $p$ is the unique minimal prime over-ideal of $p^{n}$, therefore the $p$-primary component of $p^{n}$ is uniquely determined; this is called the $n$-th symbolic power of $p$ and is denoted by $p^{(n)}$. Thus $p^{(n)}=p^{n} A_{p} \cap A$. It can happen that $p^{n} \neq p^{(n)}$. Example: let $k$ be a fleld and $B=$ $k[x, y]$ the polynomial ring in the indeterminates $x$ and $y$ Put $A=k\left[x, x y, y^{2}, y^{3}\right]$ and $P=y B \cap A=\left(x y, y^{2}, y^{3}\right)$. Then $p^{2}=\left(x^{2} y^{2}, x y^{3}, y^{4}, y^{5}\right)$. Since $y=x y / x \varepsilon A_{p}$, we have $B=$ $k[x, y] \subseteq A_{P}$ and hence $A_{P}=B_{y B} \cdot$ Thus $p^{(2)}=y^{2} B_{y B} \cap A=$ $y^{2} B \cap A=\left(y^{2}, y^{3}\right) \neq p^{2}$. An irredundant primary decomposition of $p^{2}$ is given by $p^{2}=\left(y^{2}, y^{3}\right) \cap\left(x^{2}, x y^{3}, y^{4}, y^{5}\right)$

\section{Homomorphisms and Ass}
\section{(9.A) PROPOSITION. Let $\phi: A \rightarrow B$ be a homomorphism of}
noetherian rings and $M$ a B-module. We can view $M$ as an $A-$ module by means of $\phi$. Then
$$
\left.\operatorname{Ass}_{A}(M)=a_{\phi\left(\operatorname{Ass}_{B}\right.}(M)\right) .
$$
Proof. Let $P \in$ Ass $_{B}(M)$. Then there exists an element $x$ of $M$ such that $\operatorname{Ann}_{B}(x)=P$. Since $A_{A}(x)=A n_{B}(x) \cap A=P \cap A$ we have $P \cap A \varepsilon$ Ass $_{A}(M)$. Conversely, let $p \varepsilon \operatorname{Ass}_{A}(M)$ and take an element $x \& M$ such that $A n_{A}(x)=P$. Put $\operatorname{Ann}_{B}(x)=I$, let $I=Q_{1} \cap \ldots \cap Q_{r}$ be an irredundant primary decomposition of the ideal I and let $Q_{i}$ be $P_{i}$-primary. Since $M \supseteq B x \simeq B / I$ the set Ass $(M)$ contains Ass $(B / I)=\left\{P_{I}, \ldots, P_{r}\right\}$. We will Prove $i^{n A}=p$ for some i. since I $n A=p$ we have $P_{i} \cap A Z$ p for al1 1. Suppose $P_{i} \cap A \neq p$ for all $i$. Then there exists $i \quad i \quad A_{i}$ such that $a_{i} p$, for each i. Then $a_{i} \varepsilon Q_{i}$ for a11 $i$ if $m$ is sufficiently large, hence $a=\Pi_{i} a_{i} m \in \operatorname{In} A=p$, contradiction. Thus $P_{i} \cap A=p$ for some $i$ and $p \varepsilon^{a} \phi\left(A s s_{B}(M)\right)$.

(9.B) THEOREM 12, (Bourbaki). Let $\phi: A \rightarrow B$ be a homo-

morphism of noetherian rings, $E$ an A-module and $F$ a B-module.

Suppose F is flat as an A-module. Then:

(i) for any prime ideal $P$ of $A$,
$$
\begin{aligned}
&\left.{ }^{a} \phi\left(\operatorname{Ass}_{B}(F / p F)\right)=\operatorname{Ass}_{A}(F / p F)\right)=\left\{\begin{array}{ccc}
\{p\} & \text { if } & F / p F \neq 0 \\
\emptyset & \text { if } & F / p F=0
\end{array}\right. \\
&A s s_{B}\left(E \otimes_{A} F\right)=\bigcup_{p \in A s s} A_{B}(E)
\end{aligned}
$$
COROLLARY, Let A and B be as above and suppose $B$ is A-flat. Then
$$
\operatorname{Ass}_{B}(B)=\bigcup_{\operatorname{peAss}(A)} \operatorname{Ass}_{B}(B / F B),
$$
\includegraphics[max width=\textwidth]{2022_08_01_8d4eee36f1f42236b4f4g-038}

$=\operatorname{Ass}(\mathrm{A})$ if $\mathrm{B}$ is faithfully flat over $\mathrm{A}$.

Proof of Theorem 12, (i) The module $F / p F$ is flat over $A / P$

(base change), and $A / p$ is a domain, therefore $F / p F$ is torsionfree as an A/p-module by (3.F). The assertion follows from

this, (ii) The inclusion $\geq$ is immediate: if $p \varepsilon$ Ass(E)

then $\mathrm{E}$ contains a submodule isomorphic to $\mathrm{A} / p$, whence $\mathrm{E} \beta \mathrm{F}$

contains a submodule isomorphic to $(\mathrm{A} / p) \otimes_{\mathrm{A}} F=F / p F$ by the flatness of $F$. Therefore Ass $(F / p F) \subseteq A_{B}(\Xi \xi F)$. To prove

the other inclusion $\supseteq$ is more difficult. Step 1. Suppose $E$ is finitely generated and coprimary with Ass $(E)=\{p\}$. Then any associated prime $P \varepsilon \operatorname{Ass}_{B}(E \& F)$ lies over $p$. In fact, the elements of $p$ are locally nilpotent (on E, hence) on $\mathrm{E} Q \mathrm{~F}$, therefore $p \subseteq \mathrm{P} \cap \mathrm{A}$. On the other hand the elements of $A-P$ are E-regular, hence E $\otimes F$-regular by the flatness of $F$. Therefore $A$ - $P$ does not meet $P$, so that $P \cap A=p \cdot \quad$ Now, take a chain of submodules
$$
E=E_{0} \supset E_{1} \supset \ldots \supset E_{r}=(0)
$$
such that $E_{i} / E_{i+1} \simeq A / p_{1}$ for some prime ideal $p_{1}$. Then $E \cdot \Delta F=E_{0} \otimes F \supseteq E_{I} \otimes F \supseteq \ldots \supseteq E_{r} \otimes F=(0)$ and $E_{i}: \Delta F / E_{i+1} \otimes F$ $\simeq F / P_{i} F$, so that $\operatorname{Ass}_{B}(E \Delta F) \subseteq \bigcup_{i} A_{s s_{B}}\left(F / P_{i} F\right)$. But if $P \varepsilon$ $\operatorname{Ass}_{B}\left(F / p_{i} F\right)$ and if $p_{i} \neq p$ then $P \cap A=p_{i}$ (by (i)) $\neq p$, hence $P \notin \operatorname{Ass}_{B}(E \otimes F)$ by what we have just proved. Therefore $\operatorname{Ass}_{B}(E \partial F) \subseteq \operatorname{Ass}_{B}(F / p F)$ as wanted.

Step 2. Suppose $\mathrm{E}$ is finitely generated. Let $(0)=Q_{1} \cap \ldots$ $n Q_{r}$ be an irredundant primary decomposition of (0) in $E$. Then $E$ is isomorphic to a submodule of $E / Q_{1}: \exists . E / Q_{r}$, and so $E Q F$ is isomorphic to a submodule of the direct sum of the $E / Q_{i} \otimes F$ 's. Then $A_{B}(E \partial F) \subseteq U s_{B}\left(E / Q_{i} \otimes F\right)=U$ Ass $_{B}\left(F / p_{i} F\right)$.

Step 3. General case. Write $E=\bigcup_{\lambda} E_{\lambda}$ with finitely generated submodules $\mathrm{E}_{\lambda}$. Then it follows from the definition of the associated primes that $\operatorname{Ass}(E)=\bigcup \operatorname{Ass}\left(E_{\lambda}\right)$ and $\operatorname{Ass}\left(E^{\prime} \partial \mathrm{F}\right)=\operatorname{Ass}\left(\mathrm{E}_{\lambda} \nabla \mathrm{F}\right)=U \operatorname{Ass}\left(\mathrm{E}_{\lambda} \Delta \mathrm{F}\right)$. Therefore the proof is reduced to the case of finitely generated $E$.

THEOREM 13. Let $A \rightarrow B$ be a flat homomorphism of

CHAPTER 4. GRADED RINGS

noetherian rings; let $q$ be a $p$-primary ideal of $A$ and assume

that $p B$ is prime. Then $q B$ is $p B$-primary.

Proof. Replacing $A$ by $A / q$ and $B$ by $B / q B$, one may assume $q=$

(0). Then $\operatorname{Ass}(A)=\{p\}$, whence $\operatorname{Ass}(B)=\operatorname{Ass}_{B}(B / p B)=\{p B\}$

by the preceding theorem.

(9.D) We say a homomorphism $\phi: A \rightarrow B$ of noetherian rings is

non-degenerate if ${ }^{\phi} \phi$ maps Ass(B) into Ass(A). A flat homo-

morphism is non-degenerate by the Cor. of Th. 12 .

PROPOSITION. Let $f: A \rightarrow B$ and $g: A \rightarrow C$ be homomorphisms of noetherian rings. Suppose 1) B $8 \mathrm{~A}$ C is noetherian, 2) $\mathrm{f}$ is flat and 3) $g$ is non-degenerate. Then $I_{B} \otimes g: B \rightarrow B \otimes C$ is

also non-degenerate, (In short, the property of being non-

degenerate is preserved by flat base change.)

Proof. Left to the reader as an exercise.

\section{Graded Rings and Modules}
(10,A) A graded ring is a ring A equipped with a direct decomposition of the underlying additive group; $A=E D A_{n}$, such that $A_{n} A_{m} \subseteq A_{n+m}$ A graded A-module is an A-module $M$, together with a direct decomposition as a group $M=\bigoplus_{n \in Z} M_{n}$ such that $A_{n} M_{n} \subseteq M^{\circ}$ Elements of $A_{n}$ (or $M_{n}$ ) are called homogeneous elements of degree $n$. A submodule $N$ of $M$ is said to be a graded (or homogeneous) submodule if $\mathrm{N}=\theta\left(\mathrm{N} \cap \mathrm{M}_{\mathrm{n}}\right)$.

It Is easy to see that this condition is equivalent to

(*) $\mathrm{N}$ is generated over A by homogeneous elements, and also to

(**) if $x=x_{r}+x_{r+1}+\cdots+x_{s} \varepsilon N, \quad x_{i} \varepsilon M_{i}$ (a11 i), then each $x_{i}$ is in $N$.

If $N$ is a graded submodule of $M$, then $M / N$ is also a graded A-module, in fact $M / N=\bigoplus M_{n} / N \cap M_{n}$.

(10.B) PROPOSITION, Let A be a noetherian graded ring, and $M$ a graded A-module. Then

i) any associated prime $p$ of $M$ is a graded ideal, and there exists a homogeneous element $x$ of $M$ such that $p=$ $\operatorname{Ann}(x)$

ii) one can choose a $p$-primary graded submodule $Q(p)$ for each $p \varepsilon$ Ass $(M)$ in such a way that $(0)=\bigcap_{p \varepsilon \text { Ass }(M)} Q(p)$.

Proof. i) Let $p \varepsilon \operatorname{Ass}(M)$. Then $p=\operatorname{Ann}(\mathrm{x})$ for some $x \in M$. Write $x=x_{e}+x_{e-1}+\ldots+x_{0}, \quad x_{i} \varepsilon M_{i} \cdot$ Let $f=f_{r}+$ $f_{r-1}+\ldots+f_{0} \varepsilon p, f_{i} \varepsilon A_{i}$. We shall prove that all $f_{i}$ are in $p$. We have
$$
\begin{aligned}
0=f x=& f_{r} x_{e}+\left(f_{r-1} x_{e}+f_{r} x e_{e-1}\right)+\ldots+\left(\sum_{i+j=p} f_{i} x_{j}\right) \\
+&+f_{0} x \text { o }
\end{aligned}
$$
$$
\begin{aligned}
& +\ldots+\mathrm{f}_{0} \mathrm{x}_{0}
\end{aligned}
$$
Hence $\mathrm{f}_{\mathrm{r}} \mathrm{x}_{\mathrm{e}}=0, \mathrm{f}_{\mathrm{r}-1} \mathrm{x}_{\mathrm{e}}+\mathrm{f}_{\mathrm{r}} \mathrm{x}_{\mathrm{e}-1}=0, \ldots, \mathrm{f}_{\mathrm{r}}-\mathrm{e}^{\mathrm{x}} \mathrm{e}+\ldots+$ $\mathrm{f}_{\mathrm{r}^{\mathrm{x}_{0}}}=0$ (we put $\mathrm{f}_{\mathrm{i}}=0$ for $\mathbf{i}<0$ ). It follows that $\mathrm{f}_{\mathrm{r}}{ }^{\mathrm{x}_{i}}$ $=0$ for $0 \leqslant i \leqslant e$. Hence $\mathrm{f}_{\mathrm{r}}^{\mathrm{e}} \mathrm{x}=0, \mathrm{f}_{\mathrm{r}}^{\mathrm{e}} \varepsilon p$, therefore $\mathrm{f}_{\mathrm{r}} \varepsilon$ p. By descending induction we see that all $f_{i}$ are in $P$, so that $p$ is a graded ideal. Then $p \in \operatorname{Ann}\left(x_{i}\right)$ for all $i$, and clearly $p=\bigcap^{e} \operatorname{Ann}\left(x_{i}\right)$. Since $p$ is prime this means $p=$ $\operatorname{Ann}\left(x_{i}\right)$ for some 1 .

ii) A slight modification of the proof of (8.G) Th.I1 proves the assertion. Alternatively, we can derive it from Th.1l and from the following Lemma: Let $P$ be a graded ideal and let $Q \subset M$ be a p-primary submodule. Then the largest graded submodule $Q^{\prime}$ contained in $Q$ (i.e. the submodule generated by the homogeneous elements in Q) is again p-primary. Proof: let $p^{\prime}$ be an associated prime of $\mathrm{M} / Q^{\prime}$. Since both $p$ and $p^{\prime}$ are graded, $p^{\prime}=p$ iff $p^{\prime} \cap H=p \cap H$ where $H$ is the set of homogeneous elements of A. If a $\varepsilon p \cap H$ then a is locally nilpotent on $M / Q^{\prime}$. If a $\varepsilon \mathrm{H}$, a $\notin P$, then for $\times \varepsilon \mathrm{M}$ satisfying ax $\varepsilon Q^{\prime}, x=\sum x_{i}, x_{i} \varepsilon M_{1}$, we have $\operatorname{ax}_{i} \varepsilon Q^{\prime} \subseteq Q$ for each 1 , hence $x_{i} \varepsilon$ Q for each $i$, hence $x \varepsilon Q^{\prime}$. Thus $a \neq p^{\prime}$

(10.C) In this book we define a filtration of a ring A to be a descending sequence of ideals

(*) $\quad A=J_{0} \supseteq J_{1} \supseteq J_{2} \supseteq \ldots$

satisfying $J_{n} J_{m} \subseteq J_{n+m}$. Given a filtration (*), we construct a graded ring $A^{\prime}$ as follows. The underlying additive group is
$$
A^{\prime}=\bigoplus_{n=0}^{\infty} J_{n} / J_{n+1},
$$
and if $\xi \varepsilon A_{n}^{\prime}=J_{n} / J_{n+1}$ and $n \varepsilon A_{m}^{\prime}=J_{m} / J_{m+1}$, then choose $x \in J_{n}$ and $y \in J_{m}$ such that $\xi=\left(x \bmod J_{n+1}\right)$ and $n=(y \bmod$ $\mathrm{J}_{\mathrm{m}+1}$ ) and put $\xi n=\left(x y \bmod \mathrm{J}_{n+m+1}\right)^{\text {. }}$. This multiplication is well defined and makes $A^{\prime}$ a graded ring. When I is an ideal of $A$, its powers define a filtration $A=I^{0} \supseteq I \supseteq I^{2} \supseteq \ldots$. This is called the I-adic filtration, and its associated graded ring is denoted by $g r^{I}(A)$.

(10.D) PROPOSITION. If $A$ is a noetherian ring and I an ideal, then $\mathrm{gr}^{\mathrm{I}}$ (A) is noetherian.

Proof. Write $g r^{I}(A)=\bigoplus_{n=0}^{\infty} A_{n}^{\prime}, A_{n}^{\prime}=I^{n} / I^{n+1} \cdot$ Then $A_{0}^{\prime}=$

$A / I$ is a noetherian ring. Let $I=a_{1} A+\ldots+a_{r} A$ and let

$\bar{a}_{i}$ denote the image of $a_{i}$ in $I / I^{2}$. Then $g r^{I}(A)$ is generated

by $\bar{a}_{1}, \ldots, \bar{a}_{r}$ over $A_{0}^{+}$, therefore is noetherian.

(10.E) Let $A$ be an artinian ring, and $B=A\left[X_{1}, \ldots, X_{m}\right]$ the polynomial ring with its natural grading. Let $M=\bigoplus_{n=0}^{\infty} M_{n}$ be a finitely generated, graded $B$-module. Put $F_{M}(n)=\ell\left(M_{n}\right)$ for $\mathbf{n} \geqslant 0$, where $\ell($ ) denotes the length of A-module. The numerical function $F_{M}$ measures the largeness of $M$. The number $F_{M}(n)$ is finite for any $n$, because there exists a degree-preserving epimorphism of $B$-modules

\includegraphics[max width=\textwidth]{2022_08_01_8d4eee36f1f42236b4f4g-041}

where $B(d)=B$ as a module but $B(d)_{n}=B_{n-d}$ (in fact, if $M$ is generated over $B$ by homogeneous elements $\xi_{1}, \ldots, \xi_{p}$ with $\operatorname{deg}\left(\xi_{i}\right)=d_{i}$ then the map $f: \oplus B\left(d_{i}\right) \rightarrow M$ such that $f\left(b_{1}, \ldots, b_{p}\right)=\sum b_{i} \xi_{i}$ satisfies the requirement), so that $\ell\left(M_{n}\right) \leqslant \sum \ell\left(B_{n-d_{i}}\right)<\infty$. Note that, since the number of the monomials of degree $n$ in $x_{1}, \ldots, x_{m}$ is $\left(\begin{array}{c}n+m-1 \\ m-1\end{array}\right)$, we have $F_{B}(n)=\ell\left(B_{n}\right)=\left(\begin{array}{c}n+m-1 \\ m-1\end{array}\right) \ell(A)$

$(10, F)$

THEOREM 14. Let $A, B$ and $M$ be as above. Then

there is a polynomial $\mathrm{f}_{M}(\mathrm{x})$ in one variable with rational

coefficients such that $F_{M}(n)=f_{M}(n)$ for $n \gg 0$ (i.e. for all

sufficiently large $n$ ).

Proof. Let $P(M)$ denote the assertion for $M$. We consider the graded submodules $N$ of $M$ and we will prove $P(M / N)$ by induction on the largeness of $N$ (note that $M$ satisfies the maximum condition for submodules). For $N=M$ the assertion is obvious. Supposing $P\left(M / N^{\prime}\right)$ is true for any graded submodule $N^{\prime}$ of $M$ properly containing $N$, we prove $P(M / N)$.

Case 1. If $N=N_{1} \cap N_{2}$ with $N_{1} \supset N(i=1,2)$, then using $\mathrm{N}_{1}+\mathrm{N}_{2} / \mathrm{N}_{1} \simeq \mathrm{N}_{2} / \mathrm{N}$ we get

$F_{M / N}=F_{M / N_{2}}+F_{N_{1}}+N_{2} / N_{1}$

$=\mathrm{F}_{\mathrm{M}} / \mathrm{N}_{2}+\mathrm{F}_{\mathrm{M}_{1} \mathrm{~N}_{1}}-\mathrm{F}_{\mathrm{M} / \mathrm{N}_{1}}+\mathrm{N}_{2}$

and the assertion $P(M / N)$ follows from $P\left(M / N_{1}\right), P\left(M / N_{2}\right)$ and $\mathrm{P}\left(\mathrm{M} / \mathrm{N}_{1}+\mathrm{N}_{2}\right)$

Case 2. If $\mathrm{N}$ is irreducible (in the sense that it is not the intersection of two larger submodules) then $\mathrm{N}$ is a primary submodule of $M ;$ let $\operatorname{Ass}(M / N)=\{p\}$. Put $I=X_{1} B+$ $\ldots+X_{m} B$ and $M^{\prime}=M / N$. If I $\subseteq p$ then we claim that $M_{n}^{\prime}=$ O for large $n$. In fact, if $\left\{\xi_{1}, \ldots, \xi_{p}\right\}$ is a set of homogeneous generators of $M^{\prime}$ over $B$ and if $d=\max \left(\operatorname{deg} \xi_{i}\right)$, then $M_{d+n}^{\prime}=I^{n} M_{d}^{\prime}$. On the other hand we have $p^{P^{\top}}=(0)$ for some $p>0$. Thus $M_{n}^{1}=0$ for $n>p+d$, and $P\left(M^{\prime}\right)$ holds with $f_{M^{\prime}}$ $=0$. It remains to show the case I $£ p$. We may suppose that $x_{1} \notin p$. Then the sequence
$$
0 \rightarrow(M / N)_{n-1} \stackrel{X_{1}}{\rightarrow}(M / N)_{n} \rightarrow\left(M / N+X_{1} Y_{n} \rightarrow 0\right.
$$
is exact for $n>0$. Since $N+X_{1} M \supset N$ there is a polynomial $f(x)=a_{d} x^{d}+\ldots+a_{0}$ with rational coefficients satisfying $P\left(M / N+X_{1} M\right)$. Thus there is an integer $n_{0}>0$ such that
$$
F_{M / N}(n)-F_{M / N}(n-1)=a_{d} n^{d}+\ldots+a_{0}\left(n>n_{0}\right) .
$$
Then
$$
\begin{aligned}
F_{M / N}(n)=& a_{d}\left(\sum_{i=n_{0}+1}^{n} i^{d}\right)+a_{d-1}\left(\sum_{i=n_{0}+1}^{n} i^{d-1}\right)+\\
& \ldots+a_{0}\left(n-n_{0}\right)+F_{M / N}\left(n_{0}\right) \quad\left(n>n_{0}\right),
\end{aligned}
$$
which means (cf. the remark below) that $\mathrm{F}_{M / N}(n)$ is a polynomial of degree $d+1$ in $n$ for $n>n_{0}$, as wanted.

Remark 1. Put $\left(\begin{array}{l}x \\ r\end{array}\right)=x(x-1) \cdots(x-r+1) / r !,\left(\begin{array}{l}x \\ 0\end{array}\right)=1$.

Then any polynomial $f(x)$ of degree $d$ in $Q[x]$ can be written
$$
f(x)=c_{d}\left(\begin{array}{c}
x+d \\
d
\end{array}\right)+c_{d-1}\left(\begin{array}{c}
x+d-1 \\
d-1
\end{array}\right)+\ldots+c_{0}\left(\begin{array}{l}
x \\
0
\end{array}\right), c_{i} \varepsilon Q
$$
Moreover, since $\left(\begin{array}{c}x+r \\ r\end{array}\right)-\left(\begin{array}{c}x+r-1 \\ r\end{array}\right)=\left(\begin{array}{c}x+r-1 \\ r-1\end{array}\right)$, we have $f(x)-f(x-1)=c_{d}\left(\begin{array}{c}x+d-1 \\ d-1\end{array}\right)+\ldots+c_{1}\left(\begin{array}{l}x \\ 0\end{array}\right)$. It follows by induction on d that, if $f(n) \in Z$ for $n \gg 0$, we have $c_{i} \varepsilon Z$ for all i (and so $f(n) \in Z$ for all $n \varepsilon Z$ ). It also follows that, if $F(n)$ is a numerical function such that
$$
F(n)-F(n-1)=f(n) \text { for } n>n_{0}
$$
then $F(n)=c_{d}\left(\begin{array}{c}n+d+1 \\ d+1\end{array}\right)+\ldots+c_{0}\left(\begin{array}{c}n+1 \\ 1\end{array}\right)+$ const for $n>n_{0}$.

Remark 2. The polynomial $\mathrm{f}_{\mathrm{M}}(\mathrm{x})$ of the theorem is called the Hilbert polynomial or the Hilbert characteristic function of $M$

\section{Artin-Rees Theorem}
(11.A) Let $A$ be a ring, I an ideal of $A$ and $M$ an A-module. We define a filtration of $M$ to be a descending sequence of submodules

(*) $M=M_{0} \supseteq M_{1} \geq M_{2} \supseteq \ldots$

The filtration is said to be I-admissible if $I M_{i} \subseteq M_{i+1}$ for a11 $i$, I-adic if $M_{i}=I^{i} M$, and essentially I-adic if it is I-admissible and if there is an integer $i_{0}$ such that $I M_{i}=M_{i+1}$ for $i>i_{0}$

Given a filtration (*), we can define a topology on $M$ by taking $\left\{x+M_{n} \mid n=1,2, \ldots\right\}$ as a fundamental system of neighborhoods of $x$ for each $x \varepsilon M$. This topology is separated iff $\bigcap_{n}^{\infty}=(0)$. The topology defined by the I-adic filtration is called the I-adic topology of M. An essentially I-adic filtration defines the I-adic topology on $M$, since $I^{i} M \subseteq M_{i}$ $\subseteq I^{i-i} O_{M_{0}} \subseteq I^{i-i_{0}} M_{M}$

(11.B) LEMMA. Let $A, I$ and $M$ be as above. Let $M=M_{0}$ $\supseteq M_{1} \supseteq M_{2} \supseteq \ldots$ be an I-admissible filtration such that all $M_{i}$ are finite A-modules, let $X$ be an indeterminate and put $A^{\prime}=\sum I^{n} X^{n}$ and $M^{*}=\sum M_{n} X^{n}$. Then the filtration is essentially I-adic iff $\mathrm{M}^{\prime}$ is finitely ganerated over $A^{\prime}$.

Proof. $A^{\prime}$ is a graded subring of $A[X]$ and $M^{\prime}$ is a subgroup of $M \otimes_{A} A[X]$ such that $A^{\prime} M^{\prime} \subseteq M^{\prime}$, hence $M^{\prime}$ is a graded $A^{\prime}-$ module. If $M^{\prime}=A^{\prime} \xi+\ldots+A^{\prime} \xi_{r}, \quad \xi_{i} \in M_{d_{i}^{\prime}}$, then $M_{n}^{\prime}=$ (IX) $M_{n-1}^{\prime}$ (hence $M_{n}=M_{n-1}$ ) for $n>\max d_{i} \cdot$ Conversely, if $M_{n}=M_{n-1}$ for $n>d$, then $M^{\prime}$ is generated over $A^{\top}$ by $M_{d-1} x^{d-1}+\ldots+M_{1} x+M_{0}$, which is, in turn, generated by a finite number of elements over A.

(11.C) THEOREM 15. (Artin-Rees) Let A be a noetherian ring, I an ideal, $M$ a finite A-module and $\mathrm{N}$ a submodule.

Then there exists an integer $r>0$ such that $I^{n} M \cap N=I^{n-r}\left(I^{r} M \cap N\right)$ for $n>r$. Proof. In other words, the theorem asserts that the filtration ( $\left.I^{n} M \cap N\right) \quad n=0,1,2, \ldots$ of $N$ (induced on $N$ by the I-adic filtration of $M$ ) is essentially I-adic. The filtration is I-admissible, and $N^{\prime}=\sum\left(I^{n} M \cap N\right) X^{n}$ is a submodule of the finite $A^{\prime}$-module $M^{\prime}=\Sigma I_{M} X^{n}$, where $A^{\prime}=\sum I^{n} X^{n}$. If $I=$ $a_{1} A+\ldots+a_{r} A$ then $A^{*}=A\left[a_{1} X, \ldots, a_{r} X\right]$, so that $A^{*}$ is noetherian. Therefore $N^{\prime}$ is finite over $A^{\prime}$. Thus the assertion follows from the preceding lemma.

Remark. It follows that the I-adic topology on $M$ induces the I-adic topology on $N$. This is not always true if $M$ is infinite over A.

(11.D) THEOREM 16. (Intersection theorem). Let A, I and

$M$ be as in the preceding theorem, and put $N=\bigcap^{\infty} I^{n}$. Then we have $I N=N$.

Proof. For sufficiently large $n$ we get $N=I^{n} M \cap N=$ $\mathrm{I}^{\mathrm{n}-\mathrm{r}}\left(I^{\mathrm{r}} \mathrm{M} \cap \mathrm{N}\right) \subseteq \mathrm{IN} \subseteq \mathrm{N}$

COROLLARY 1. If I $\subseteq \operatorname{rad}(A)$ then $\cap^{\infty} M=(0)$. In other words $M$ is I-adically separated in that case.

COROLLARY 2. (Krull) Let $A$ be a noetherian ring and $I=$ $\operatorname{rad}(A)$. Then $\bigcap^{\infty}=(0)$. COROLLARY 3. (Krul1) Let A be a noetherian domain and let

I be any proper ideal. Then $\bigcap^{n}=(0)$.

CHAPTER 5. DIMENSION

Proof. Putting $N=\bigcap I^{n}$ we have $I N=N$, whence there exists

$x \in l$ such that $(1+x) N=(0)$ by $(1 . M)$. Since $A$ is an in-

tegral domain and since $1+x \neq 0$, we have $N=(0)$.

(11.E) PROPOSITION. Let $A$ be a noetherian ring, $M$ a

Einite A-module, I and ideal, and $J$ an ideal generated by

M-regular elements. Then there exists $r>0$ such that
$$
I^{n} M: J=I^{n-r}\left(I^{r} M: J\right) \text { for } n>r .
$$

\section{Dimension}
Proof. Let $J=a_{1} A+\ldots+a_{p} A$ where the $a_{i}$ are M-regular.

Let $S$ be the multiplicative subset of A generated by $a_{1}, \ldots$,

$a_{p}$, and consider the A-submodules $a_{j}{ }^{-1} M$ of $\mathrm{S}^{-1}$. Put . P $=$ $a_{1}^{-1} M \oplus a_{p}^{-1} M$ and let $\Delta_{M}$ be the image of the diagonal

$\operatorname{map} x \rightarrow(x, x, \ldots, x)$ from $M$ to $L$. Then $M \simeq \Delta_{M}$, and $I^{n} M: J=\bigcap_{j}\left(I^{n} M: a_{j}\right)=\cap\left(I^{n} a_{j}^{-1} M \cap M\right) \simeq I^{n} L \cap \Delta_{M}$,

so that the assertion follows from the Artin-Rees theorem

applied to $\mathrm{L}$ and $\Delta_{\mathrm{M}^{\circ}}$ (12.A) Let $A$ be a ring, $A \neq 0$. A finite sequence of $n+1$ prime ideals $p_{0} \supset p_{1} \supset \ldots \supset p_{n}$ is called a prime chain of length $n$, If $p \in \operatorname{spec}(\mathrm{A})$, the supremum of the lengths of the prime chains with $p=p_{0}$ is called the height of $p$ and denoted by $h t(p)$. Thus $h t(p)=0$ means that $p$ is a minimal prime ideal of $A$.

Let $I$ be a proper ideal of $A$. We define the height of I to be the minimum of the heights of the prime ideals containing I: $h t(I)=\inf \{h t(p) \mid p \supseteq I\}$.

The dimension of A is defined to be the supremum of the heights of the prime ideals in A:

$\operatorname{dim}(A)=\sup \{h t(p) \mid p \in \operatorname{Spec}(A)\}$

It is also called the Krull dimension of $A$. If $\operatorname{dim}(A)$ is finite then it is equal to the length of the longest prinie chains in A. For example, a principal ideal domain has

dimension one.

It follows from the definition that
$$
h t(p)=\operatorname{dim}\left(A_{p}\right) \quad(p \in \operatorname{Spec}(A)),
$$
and that, for any ideal I of $A$,
$$
\operatorname{dim}(A / I)+h t(I) \leqslant \operatorname{dim}(A) .
$$
(12.B) Let $M \neq 0$ be an A-module, We define the dimension of $M$ by
$$
\operatorname{dim}(M)=\operatorname{dim}(A / \operatorname{Ann}(M)) .
$$
(When $M=0$ we put $\operatorname{dim}(M)=-1$, ) Under the assumption that

$A$ is noetherian and $M \neq 0$ is finite over $A$, the following conditions are equivalent:

(1) $M$ is an A-module of finite length,

(2) the ring A/Ann(M) is artinian,

(3) $\operatorname{dim}(M)=0$.

In fact, (3) $\Leftrightarrow(2) \Rightarrow(1)$ is obvious by $(2 . C)$. Let us prove (1) $\Rightarrow(3)$. We suppose $\ell(M)$ is finite, and replacing A by $A / A n n(M)$ we assume that $\operatorname{Ann}(M)=(0)$. If $\operatorname{dim}(A)>0$, take a minimal prime $p$ of $A$ which is not maximal. Since $M$ is finite over $A$ and since $\operatorname{Ann}(M)=(0)$, we easily see that $M_{p} \neq 0$. Hence $p$ is a minimal member of $\operatorname{Supp}(M)$, so that $p \in \operatorname{Ass}(M)$. Then $M$ contains a submodule isomorphic to $\mathrm{A} / p$, and since $\operatorname{dim}(A / p)>0$ we have $\ell(A / P)=\infty$, contradiction.

Therefore $\operatorname{dim}(\mathrm{A})(=\operatorname{dim}(\mathrm{M}))=0$.

(12.C) Let $A$ be a noetherian semi-local ring, and $m=$ $\operatorname{rad}(A)$. An ideal I is called an ideal of definition or $A$ if $m^{\nu} \subseteq I \subseteq m$ for some $v>0$. This is equivalent to saying that

$I \subseteq m$, and $A / I$ is artinian.

Let I be an ideal of definition and M a finite A-modlle. Put $A^{*}=\operatorname{gr}^{I}(A)=\oplus I^{n} / I^{n+I}$,

and $M^{*}=g^{I}(M)=\oplus I^{n} M^{\prime} I^{n+1} M$

Let $I=\mathrm{Ax}_{1}+\ldots+\mathrm{Ax}_{\mathrm{r}}$. Then the graded ring $\mathrm{A}^{*}$ is a homomorphic image of $B=(A / I)\left[X_{1}, \ldots, X_{r}\right]$, and $M^{*}$ is a finite, graded $A^{*}-\operatorname{module}$, Therefore $F_{M^{*}}(n)=\ell\left(I^{n} M / I^{n+1} M\right)$ is a polynomial in $n$, of degree $\leqslant r-1$, for $n \gg 0$. It follows that the function
$$
X(M, I ; n)=\ell\left(M / I^{n} M\right)=\sum_{j=0}^{n-1} F_{M^{*}}(j)
$$
is also a polynomial in $n$, of degree $\leqslant r$, for $n \gg 0$. The polynomial which represents $X(M, I ; n)$ for $n \gg 0$ is called the Hilbert polynomial of $M$ with respect to $I$. If $J$ is another ideal of definition of $A$, then $\mathrm{J}^{\mathrm{S}} \subseteq \mathrm{I}$ for some $s>0$, so that we have $X(M, I ; n) \leqslant X(M, J ; s n)$. Thus, if $X(M, I ; n)$ $=a_{d^{n}} \mathrm{~d}^{\mathrm{d}}+\ldots+a_{0}$ and $x(M, J ; n)=b_{d}, n^{d^{\prime}}+\ldots+b_{0}$, then $d \leqslant d^{\prime}$. By symmetry we get $d=d^{\prime}$. Thus the degree $d$ of the Hilbert polynomial is independent of the choice of I.

We denote it by $\mathrm{d}(M)$. Remember that, if there exists an ideal

of definition of A generated by $r$ elements, then $d(M) \leqslant r$.

(12.D) PROPOSITION. Let A be a noetherian semi-local

ring, I an ideal of definition of $A$ and
$$
0 \rightarrow M^{\prime} \rightarrow M \rightarrow M^{\prime \prime} \rightarrow 0
$$
an exact sequence of finite A-modules. Then $d(M)=\max$

$\left(d\left(M^{\prime}\right), d\left(M^{\prime \prime}\right)\right)$. Moreover, $X(M, I ; n)-X\left(M^{\prime}, I ; n\right)-X\left(M^{\prime \prime}, I ; n\right)$

is a polynomial of degree $\left\langle d\left(M^{\prime}\right)\right.$ for $n \gg 0$

Proof. Since $\ell\left(M^{\prime \prime} / I^{n} M^{\prime \prime}\right)=\ell\left(M / M^{\prime}+I^{n} M\right) \leqslant \ell\left(M / I^{n} M\right)$, we get

$d\left(M^{\prime \prime}\right) \leqslant d(M)$. Furthermore, $\chi(M, I ; n)-\chi\left(M^{\prime \prime}, I ; n\right)=$

$\ell\left(M / I^{n} M\right)-\ell\left(M / M^{\prime}+I^{n} M\right)=\ell\left(M^{+}+I^{n} M / I^{n} M\right)=\ell\left(M^{\dagger} / M^{\dagger} \cap I^{n} M\right)$,

and there exists $r>0$ such that $M^{\prime} \cap I^{n} M \subseteq I^{n-r} M^{\prime}$ for $n>r$

by Artin-Rees. Thus $\ell\left(M^{\prime} / I^{n} M^{\prime}\right) \geqslant \ell\left(M^{\prime} / M^{\prime} \cap I^{n} M\right) \geqslant \ell\left(M^{\prime} / I^{n-r} M^{\prime}\right)$.

This means that $X(M, I ; n)-X\left(M^{\prime \prime}, I ; n\right)$ and $X\left(M^{\prime}, I ; n\right)$ have

the same degree and the same leading term.

(12.E) LEMMA 1. Let A be a noetherian semi-1ocal ring.

Then $d(A) \geqslant \operatorname{dim}(A)$.

Proof. Induction on $\mathrm{d}(\mathrm{A})$. If $\mathrm{d}(\mathrm{A})=0$ then $\mu^{\nu}=\boldsymbol{K}^{\nu+1}=\cdots$

for some $v>0$. By the intersection theorem ((11.D) Cor.l),

this implies $m^{V}=(0)$. Hence $l(A)$ is finite and $\operatorname{dim}(A)=0$. Suppose $d(A)>0$. As the case $\operatorname{dim}(A)=0$ is trivial, we assume $\operatorname{dim}(\mathrm{A})>0$. Let $p_{0} \supset \ldots \supset p_{e-1} \supset p_{e}=p$ be a prime chain of length $e>0$, and take an element $x \varepsilon P_{e-1}$ such that $x \notin p .$ Then $\operatorname{dim}(\mathrm{A} / \mathrm{xA}+p) \geqslant e-1$. Applying the preceding proposition to the exact sequence
$$
0 \rightarrow \mathrm{A} / p \rightarrow \mathrm{A} \rightarrow \mathrm{A} / \mathrm{A} / \mathrm{XA}+p \rightarrow 0
$$
we have $d(A / x A+p)<d(A / p) \leqslant d(A)$. Thus, by induction

hypothesis we get $e-1 \leqslant \operatorname{dim}(A / x A+p) \leqslant d(A / x A+p)<d(A)$.

Hence $e \leqslant d(A)$, therefore $\operatorname{dim}(A) \leqslant d(A)$.

Remark. The lemma shows that the dimension of A is finite. When A is an arbitrary noetherian ring and $P$ is a prime ideal, we have $h t(p)=\operatorname{dim}\left(A_{p}\right)$ so that $h t(p)$ is finite. (This was first proved by Krull by a different method.) Thus the descending chain condition holds for prime ideals in a noetherian ring. On the other hand, there are noetherian rings with infinite dimension.

$(12 . F)$

LEMMA 2. Let $A$ be a noetherian semi-local ring, $M \neq 0$ a finite A-module, and $x \in \operatorname{rad}(A)$. Then
$$
d(M) \geqslant d(M / x M) \geqslant d(M)-1 .
$$
Proof. Let I be an ideal of definition containing $x$. Then $X(M / X M, I ; n)=\ell\left(M / X M+I^{n} M\right)=\ell\left(M / I^{n} M\right)-\ell\left(X M+I^{n} M / I^{n} M\right)$ and $x M+I^{n} M / I^{n} M \simeq x M / x M n I^{n} M \simeq M /\left(I^{n} M: x\right)$ and $I^{n-1} M \subseteq$ $\left(I^{n} M: x\right)$, therefore
$$
\begin{aligned}
x(M / X M, I ; n) & \geqslant \ell\left(M / I^{n} M\right)-\ell\left(M / I^{n-1} M\right) \\
&=X(M, I ; n)-x(M, I ; n-1)
\end{aligned}
$$
It follows that $d(M / x M) \geqslant d(M)-1$.

(12.G) LEMMA 3. Let $A$ and $M$ be as above, and let dim(M)

$=r$. Then there exist $r$ elements $x_{1}, \ldots, x_{r}$ of rad(A) such

that $l\left(M / x_{1} M+\ldots+x_{r} M\right)<\infty$.

Proof. Let I be an ideal of definition of A. When $r=0$

we have $\ell(M)<\infty$ and the assertion holds. Suppose $r>0$, and let $p_{1}, \ldots, p_{t}$ be those minimal prime over-ideals of

$\operatorname{Ann}(M)$ which satisfy $\operatorname{dim}\left(A / P_{i}\right)=r$. Then no maximal ideals

are contained in any $p_{i}$, hence $\operatorname{rad}(A) \notin p_{i}(1 \leqslant i \leqslant t)$. Thus

by (1.B) there exists $x_{1} \varepsilon$ rad (A) which is not contained in

any $P_{i}$. Then $\operatorname{dim}\left(M / x_{1} M\right) \leqslant r-1$, and the assertion follows

by induction on $\operatorname{dim}(M)$.

(12. H) THEOREM 17. Let A be a noetherian semi-local ring,

$m=\operatorname{rad}(A)$ and $M \neq 0$ a finite A-module. Then $d(M)=\operatorname{dim}(M)$

$=$ the smallest integer $r$ such that there exist elements $x_{1}$,

$\ldots, x_{r}$ of $m$ satisfying $\ell\left(M / x_{1} M+\ldots+x_{r} M\right)<\infty .$

Proof. If $l\left(M / x_{1} M+\ldots+x_{r} M\right)<\infty$ we have $d(M) \leqslant r$ by

Lemma 2. When $r$ is the smallest possible we aave $r \leqslant \operatorname{dim}(M)$ by Lemma 3. It remains to prove $\operatorname{dim}(M) \leqslant d(M)$. Take a sequence of submodules $M=M_{1} \supset M_{2} \supset \ldots \supset M_{k+1}=(0)$ such that $M_{i} / M_{i+1} \simeq A / p_{i}, p_{i} \varepsilon \operatorname{Spec}(\mathrm{A}) .$ Then $p_{i} \supseteq$ Ann(M) and $\operatorname{Ass}(M) \subseteq\left\{p_{1}, \ldots, p_{k}\right\}$. Since $\operatorname{Supp}(M)=V(\operatorname{Ann}(M))$ all the minimal over-ideals of $A n n(M)$ are in Ass(M) (hence also in $\left.\left\{p_{1}, \ldots, p_{k}\right\}\right)$ by (7.D). Therefore
$$
\begin{array}{rlr}
\mathrm{d}(M) & =\max d\left(A / P_{i}\right) & \text { by }(12 \cdot D) \\
& \geqslant \max \operatorname{dim}\left(A / P_{i}\right) & \text { by Lemma } 1 \\
& =\operatorname{dim}(A / \operatorname{Ann}(M))=\operatorname{dim}(M)
\end{array}
$$
which completes the proof.

(12.I) THEOREM 18. Let A be a noetherian ring and $I=$ $\left(a_{1}, \ldots, a_{r}\right)$ be an ideal generated by $r$ elements. Then any minimal prime over-ideal $p$ of $I$ has height $\leqslant r$. In particular, ht $(I) \leqslant r$

Proof. Since $p A_{p}$ is the only prime ideal of $A_{p}$ containing IA $_{p}$, the ring $A_{p} / I_{p}=A_{p} /\left(a_{1} A_{p}+\ldots+a_{r} A_{p}\right)$ is artinian. Therefore $h t(p)=\operatorname{dim}\left(A_{p}\right) \leqslant r$ by Th. $17 .$

(12.J) Let $(A, m, k)$ be a noetherian local ring of dimension d. In this case, an ideal of definition of A and a primary ideal belonging to $m$ are the same thing. We know (Th.17) that no ideals of definition are generated by less than d elements, and that there are ideals of definition generated by exactly $d$ elements. If $\left(x_{1}, \ldots, x_{d}\right)$ is an ideal of definition then we say that $\left\{x_{1}, \ldots, x_{d}\right\}$ is a system of parameters of $A$. If there exists a system of parameters generating the maximal ideal $\mathrm{m}$, then we say that $A$ is a regular local ring and we call such a system of parameters a regular system of parameters. Since the number of elements of a minimal basis of is equal to $\operatorname{rank}_{k} m / m^{2}$, we have in general
$$
\operatorname{dim}(A) \leq \operatorname{rank}_{k} m / \mathrm{m}^{2} \text {, }
$$
and the equality holds iff $A$ is regular.

(12.K) PROPOSITION. Let $(A, m)$ be a noetherian local

ring and $x_{1}, \ldots, x_{d}$ a system of parameters of A. Then
$$
\operatorname{dim}\left(A /\left(x_{1}, \ldots, x_{i}\right)\right)=d-i=\operatorname{dim}(A)-i
$$
for each $1 \leqslant i \leqslant d$.

Proof. Put $\bar{A}=A /\left(x_{1}, \ldots, x_{i}\right)$. Then $\operatorname{dim}(\bar{A}) \leqslant d-i$ since $\bar{x}_{i+1}, \ldots, \bar{x}_{d}$ generate an ideal of definition of $\bar{A}$. On the other hand, if $\operatorname{dim}(\bar{A})=p$ and if $\bar{y}_{1}, \ldots, \bar{y}_{p}$ is a system of parameters of $\bar{A}$, then $x_{1}, \ldots, x_{i}, y_{1}, \ldots, y_{p}$ generate an ideal of definition of $A$ so that $p+i \geqslant d$, that is, $p \geqslant d-1$.

\section{Homomorphism and Dimension}
(13.A) Let $\phi: \mathrm{A} \rightarrow \mathrm{B}$ be a homomorphism of rings. Let $p \varepsilon$ $\operatorname{Spec}(A)$, and put $K(p)=A_{P} / p A_{P}$. Then $\operatorname{Spec}\left(B \otimes_{A} K(p)\right)$ is called the fibre over $p$ (of the canonical map $a_{\phi:} \operatorname{spec}(B) \rightarrow$

\includegraphics[max width=\textwidth]{2022_08_01_8d4eee36f1f42236b4f4g-048}\\
and $\operatorname{Spec}(B \otimes k(p))$. If $\mathrm{P}$ is a prime ideal of $\mathrm{B}$ lying over $p$, the corresponding prime of $\mathrm{B} Q K(p)=\mathrm{B}_{p} / p \mathrm{~B}_{p}$ is $\mathrm{PB}_{p} / p B_{p}$; denote it by $P^{*}$. Then the local ring $\left(B \otimes_{A} K(p)\right)_{P *}$ can be identified with ${ }^{B_{P}} / P B_{P}=B_{P} \otimes{ }_{A} K(P)$; in fact, we have $\left(B_{P}{ }_{P} P_{P}=B_{P}\right.$ and so $(\mathrm{B} \otimes K(p))_{\mathrm{P} *}=\left(\mathrm{B}_{p} / p \mathrm{~B}_{p}\right)_{\mathrm{PB}_{p} / p \mathrm{~B}_{p}}=\mathrm{B}_{\mathrm{P}} / p B_{P}$ by (1.I.2). Now we have the following theorem.

THEOREM 19. Let $\phi: A \rightarrow B$ be a homomorphism of noetherian rings; let $P \in \operatorname{Spec}(\mathrm{B})$ and $P=P \cap A$. Then

(1) ht $(P) \leqslant h t(p)+h t(P / p B)$, in other words $\operatorname{dim}\left(B_{P}\right) \leqslant \operatorname{dim}\left(A_{p}\right)+\operatorname{dim}\left(B_{P} \otimes K(p)\right)$

(2) the equality holds in (1) if the going-down theorem holds for $\phi$ (e.g. if $\phi$ is flat);

(3) if if $_{\phi}: \operatorname{Spec}(B) \rightarrow \operatorname{Spec}(A)$ is surjective and if the going-down theorem holds, then we have i) $\operatorname{dim}(B) \geqslant \operatorname{dim}(A)$ and ii) $h t(I)=h t(I B)$ for any ideal I of $A$.

Proof. (1) Replacing $A$ and $B$ by $A_{p}$ and $B_{P}$, we may suppose that $(A, P)$ and $(B, P)$ are local rings such that $P \cap A=P$. We have to prove $\operatorname{dim}(B) \leqslant \operatorname{dim}(A)+\operatorname{dim}(B / p B)$. Let $a_{1}, \ldots, a_{r}$ be a system of parameters of $A$ and put $I=\sum a_{i} A$. Then $p^{n} \subseteq$ I for some $n>0$, so that $p^{n} B \subseteq I B \subseteq p B$. Thus the ideals $p B$ and IB have the same radical. Therefore it follows from the definition that $\operatorname{dim}(B / P B)=\operatorname{dim}(B / I B) \cdot$ If $\operatorname{dim}(B / I B)=s$ and if $\left\{\bar{b}_{1}, \ldots, \bar{b}_{\mathrm{s}}\right\}$ is a system of parameters of $B / I B$, then $b_{1}, \ldots, b_{s}, a_{1}, \ldots, a_{r}$ generate an ideal of definition of $B$. Hence $\operatorname{dim}(B) \leqslant r+s$

(2) We use the same notation as above. If ht $(P / p B)=s$ there exists a prime chain of length $s, P=P_{0} \supset P_{1} \supset \ldots P_{s}$, such that $P_{s} \supseteq p B$. As $p=P \cap A \geq P_{i} \cap A \supseteq p$, all the $P_{i}$ lie over p. If $h t(p)=r$ then there exists a prime chain $p \supset$ $p_{1} \supset \ldots>p_{r}$ in $A$, and by going-down there exists a prime chain $P_{s}=Q_{0} \supset Q_{1} \supset \ldots \supset Q_{r}$ of $B$ such that $Q_{i} \cap A=P_{i} \cdot$ Thus $P=P_{0} \supset P_{1}>\ldots>P_{s} \supset Q_{1}>\ldots>Q_{r}$ is a prime chain of length $r+s$, therefore $h t(P) \geqslant r+s$

(3) i) follows from (2). ii) Take a minimal prime overideal $P$ of IB such that $h t(P)=h t(I B)$, and put $P=P \cap A$. Then $h t(P / P B)=0$, hence by (2) we get $h t(I B)=h t(P)=h t(p)$ $\geqslant h t(I)$. Conversely, let $p$ be a minimal prime over-ideal of I such that $h t(P)=h t(I)$, and take a prime $P$ of $B$ lying over $P$. Replacing $P$ if necessary we may suppose that $P$ is

a minimal prime over-ideal of $p B$. Then $h t(I)=h t(p)=h t(P)$ $\geqslant h t(I B)$ (13.C) THEOREM 20. Let B be a noetherian ring, and let A be a noetherian subring over which $B$ is integral. Then

(1) $\operatorname{dim}(A)=\operatorname{dim}(B)$,

(2) for any $P \in \operatorname{Spec}(B)$ we have $h t(P) \leqslant h t(P \cap A)$,

(3) if, moreover, the going-down theorem holds between $A$ and $B$, then for any ideal $J$ of $B$ we have $h t(J)=h t(J \cap A)$.

Proof. Since $P_{1} \subset P_{2}$ implies $P_{1} \wedge A \subset P_{2} \cap A$ by $(5 . E)$ ii), we have $\operatorname{dim}(B) \leqslant \operatorname{dim}(A)$. On the other hand the going-up theorem proves $\operatorname{dim}(B) \geqslant \operatorname{dim}(A)$. Thus $\operatorname{dim}(B)=\operatorname{dim}(A)$. The inequality $h t(P) \leqslant h t(P \cap A)$ follows from $T h .19(1)$, since $h t(P /(P \cap A) B)=0$ by $(5 . E)$ ii). To prove (3), first take a prime ideal $P$ of $B$ containing $J$ such that $h t(P)=h t(J)$ Then $h t(P)=h t(P \cap A)$ by $T h .19(3)$, so that $h t(J)=h t(P)=$ $h t(P \cap A) \geqslant h t(J \cap A)$. Next let $P$ be a prime ideal of A containing $J \cap A$ such that $h t(p)=h t(J \cap A)$. Since $B / J$ is integral over the subring A/JnA, there exists a prime $P$ of B containing $J$ and lying over $p .$ Then $h t(J \cap A)=h t(p)=$ $h t(P) \geqslant h t(J)$

(13.D) THEOREM 21. Let $\phi: A \rightarrow B$ be a homomorphism of noetherian rings and suppose that the going-up theorem holds for $\phi$. Let $p$ and $q$ be prime ideals of $A$ such that $p \supset q$.

Then $\operatorname{dim}\left(B \otimes_{A} K(p)\right) \geqslant \operatorname{dim}\left(B \otimes_{A} K(q)\right)$. Proof. Put $r=\operatorname{dim}\left(B \otimes_{A} K(q)\right)$ and $s=h t(p / q)$. Take a prime

B $Q_{r+s} \supset \ldots Q^{r}$ $\mathrm{Q}_{\mathbf{i}} \cap \mathrm{A}=q$ for all $i$, and a prime chain $q=p_{0} \subset p_{1} \subset \cdots \subset p_{s}=p$ in A. By going-up we can find a in $B$ such that $Q_{r+j} A=P_{r}$ Then $Q_{r+s}$ lies over $p$ and $h t\left(Q_{r+s} / Q_{0}\right)$

A $\left.p=p_{s}\right) \ldots, q$ Pr+s. Applying th. 19 (1) to $\mathrm{A} / \mathrm{G}$ $\rightarrow \mathrm{B} / \mathrm{Q}_{0}$ we get ht $\left(Q_{r+s} / Q_{0}\right) \leqslant s+$

ht $\left(Q_{\mathrm{r}+\mathrm{s}} / Q_{0}+p B\right) \leqslant s+h t\left(Q_{r+s} / p B\right) \leqslant s+\operatorname{dim}(B \otimes K(p))$. Thus $r \leqslant \operatorname{dim}(B \otimes K(p))$, Q.E.D.

(13.E) Remark. The 1ocal form of theorem 21 is inconvenient for applications in algebraic geometry. The global counterpart of the going-up theorem is the closedness of a morphism. Thus, we have the following geometric theorem: Let $\mathrm{f}: \mathrm{X} \rightarrow \mathrm{Y}$ be a closed morphism (e.g. a proper morphism) between noetherian schemes, and let $y$ and $y^{\prime}$ be points of $Y$ such that $y^{\prime}$ is a specialization of $y$. Then $\operatorname{dim} f^{-1}\left(y^{\prime}\right) \geqslant \operatorname{dim} f^{-1}(y)$.

The proof is essentially the same as above. 14. Finitely Generated Extensions

(14.A) THEOREM 22. Let A be a noetherian ring and let $A\left[x_{1}, \ldots, x_{n}\right]$ be a polynomial ring in $n$ variables. Then $\operatorname{dim} A\left[x_{1}, \ldots, x_{n}\right]=\operatorname{dim} A+n$

Proof. Enough to prove the case $n=1$. Put $B=A[X]$. Let $P$ be a prime ideal of $A$ and let $P$ be a prime ideal of $B$ which is maximal among the prime ideals lying over $p$. We claim that $h t(P / p B)=1$. In fact, localizing $A$ and $B$ by the multiplicative set A - $p$ we can assume that $p$ is a maximal ideal, and then $B / P B=(A / p)[X]$ is a polynomial ring in one variable over a field. Therefore $B / p B$ is a principal ideal domain and every maximal ideal has height one. Thus $h t(P / p B)=1$.

Since $B$ is free over A we have $h t(P)=h t(p)+1$ by Th.19 (2). As the map $\operatorname{Spec}(B) \rightarrow \operatorname{Spec}(A)$ is surjective, we obtain $\operatorname{dim} B$ $=\operatorname{dim~} \mathrm{A}+1$

COROLLARY. Let $k$ be a field. Then $\operatorname{dim} k\left[x_{1}, \ldots, x_{n}\right]=n$, and the ideal $\left(x_{1}, \ldots, x_{i}\right)$ is a prime ideal of height $i$, for $1 \leqslant i \leqslant n$

Proof $.$ Since $(0) \subset\left(x_{1}\right) \subset\left(x_{1}, x_{2}\right) \subset \ldots \subset\left(x_{1}, \ldots, x_{i}\right) \subset$ $\cdots \subset\left(x_{1}, \ldots, x_{n}\right)$ is a prime chain of length $n$ and since $\operatorname{dim} k\left[x_{1}, \ldots, x_{n}\right]=n$, the assertion is obvious. (14.B) A ring A is said to be catenary if, for each pair of prime ideals $p, q$ with $p \supset q, h t(p / q)$ is finite and is equal to the length of any maximal prime chain between $p$ and q. (When A is noetherian, the condition $h t(p / q)<\infty$ is automatically satisfied.) Thus if $A$ is a noetherian domain the following conditions are equivalent:

(1) A is catenary,

(2) for any pair of prime ideals $p, q$ such that $p \supset q$, we have $h t(p)=h t(q)+h t(p / q)$,

(3) for any pair of prime ideals $p, q$ such that $p \supset q$ with $h t(p / q)=1$, we have $h t(p)=h t(q)+1$

If $A$ is catenary, then clearly any localization $\mathrm{S}^{-1} \mathrm{~A}$ and any homomorphic image $\mathrm{A} / \mathrm{I}$ of $\mathrm{A}$ are also catenary.

A ring $A$ is said to be universally catenary (u.c. for short) if A is noetherian and if every A-algebra of finite type is catenary. Since any A-algebra of finite type is a homomorphic image of $A\left[X_{1}, \ldots, X_{n}\right]$ for some $n$, a noetherian ring A is universally catenary iff $A\left[X_{1}, \ldots, X_{n}\right]$ is catenary for every $n \geqslant 0$

If $A$ is $u . c$, so are the localizations of $A$, the homomorphic images of A and any A-algebras of finite type.

(14.C) THEOREM 23. Let $A$ be a noetherian domain, and let $B$ be a finitely generated overdomain of $A$. Let $P \varepsilon \operatorname{Spec}(B)$

\section{and $p=P \cap A$. Then we have}
(*) $h t(P) \leqslant h t(p)+t r \cdot d e g \cdot A B-t r \cdot d e g \cdot{ }_{K}(p){ }^{K}(P) .$ And the equality holds if A is universally catenary, or if $B$ is a polynomial ring $A\left[X_{1}, \ldots, X_{n}\right]$, (Here, tr.deg. $A^{B}$ means the transcendence degree of the quotient field of $B$ over that of $A$, and $K(P)$ is the quotient field of $B / P .)$

Proof. Let $B=A\left[x_{1}, \ldots, x_{n}\right]$. By induction on $n$ it is enough to consider the case $n=1$. So let $B=A[x]$. Replacing $A$ by $A_{P}$, and $B$ by $B_{P}=A_{P}[x]$, we assume that $(A, p)$ is a local ring. Put $k=k(P)=A / P$ and $I=\{f(X) \varepsilon A[X] \mid f(x)=0\}$. Thus $\mathrm{B}=\mathrm{A}[\mathrm{X}] / \mathrm{I}$.

Case 1. $I=(0)$. Then $B=A[X]$, tr.deg $\cdot{ }_{A} B=1$ and $\mathrm{B} / \mathrm{pB}^{\mathrm{B}}=\mathrm{k}[\mathrm{X}]$. Therefore $h t(\mathrm{P} / p \mathrm{p})=1$ or 0 according as $\mathrm{P} \supset p^{\mathrm{B}}$ (then tr.deg $k_{k} k(P)=0$ ) or $P=p B$ (then $\operatorname{tr} \cdot d e g \cdot k^{k}(P)=1$ ). In other words $h t(P / p B)=1-\operatorname{tr} \cdot \operatorname{deg} \cdot k_{k}^{K(P)}$. On the other hand, $h t(P)=h t(p)+h t(P / p B)$ by Th.19. Thus the equality holds in $(*)$.

Case 2. I $\neq(0)$. Then tr.deg ${ }_{A}^{B}=0$. Let $P *$ be the inverse image of $P$ in $A[X]$, so that $P=P * / I$ and $\kappa(P)=K\left(P^{*}\right)$.

Since $A$ is a subring of $B=A[X] / I$ we have $A \cap I=(0)$.

Therefore, if $\mathrm{K}$ denotes the quotient field of $\mathrm{A}$ then $h \mathrm{t}(\mathrm{I})=$ $h t(\operatorname{IK}[X]) \leqslant \operatorname{dim} K[X]=1$. Since $I f(0)$ we have $h t(I)=1$.

Hence $h t(P) \leqslant h t(P *)-h t(I)=h t\left(P^{*}\right)-1$, where the equality holds if A is u.c.. On the other hand we have ht $\left(P^{*}\right)=$

$h t(p)+1-t r \cdot d e g \cdot k^{K}\left(P^{*}\right)$ by case 1 , and $k\left(P^{*}\right)=K(P)$. Our

assertions follow immediately from these,

Definition. We shall call the inequality (*) the dimension

inequality. If $B$ is a finitely generated overdomain of A

and if the equality in $(*)$ holds for any prime ideal of $B$,

then we say that the dimension formula holds between $A$ and $B$.

(14.D) COROLLARY. A noetherian ring A is universally

catenary iff the following is true: A is catenary, and

for any prime $p$ of $A$ and for any finitely generated over-

domain $B$ of $A / p$, the dimension formula holds between $A / P$ and $B$.

Proof. If A (hence $A / p$ ) is u.c., then the condition holds by

the theorem. Conversely, suppose the condition holds. Let

B be any A-algebra of finite type and let $Q^{\prime} D Q$ be prime

ideals of $B$. We have to show that all maximal prime chains

between $Q^{\prime}$ and $Q$ have the same length. Replacing B by $B / Q$

and $A$ by $A / A \cap Q$ we can assume that $B$ is a finitely generated

overdomain of A. We are going to prove that for any prime

ideals $P$ and $P^{\prime}$ of $B$ such that $P \supset P^{\prime}$ we have $h t(P)=h t\left(P^{\top}\right)$

$+h t\left(P / P^{\prime}\right) \cdot$ Put $p=P \cap A, P^{\prime}=P^{\prime} \cap A$ and $n=t r \cdot d e g \cdot A^{B}$

Then $h t(P)=h t(p)+n-t r \cdot d e g \cdot k(p)^{k}(P), h t\left(P^{\prime}\right)=h t\left(p^{\prime}\right)+$

$n-t r \cdot d e g \cdot K\left(p^{\prime}\right)^{K\left(P^{\prime}\right)}$, and by the assumption applied to $B / P^{\prime}$ and $A / p^{\prime}$, we also have $h t\left(P / P^{\prime}\right)=h t\left(p / p^{\prime}\right)+t r$.deg. $k\left(p^{\prime}\right)^{K\left(P^{\prime}\right)}$

\begin{itemize}
  \item tr.deg. $K(p)^{K(P)}$. Since A is catenary we have $h t(p)=$
\end{itemize}
$h t\left(p^{\prime}\right)+h t\left(p / p^{\prime}\right)$. It follows that $h t(P)=h t\left(P^{\prime}\right)+h t\left(P / P^{\prime}\right)$.

(14.E) EXAMPLE. A11 noetherian rings that appear in alge-

braic geometry are catenary. And many algebraists had in

vain tried to know if all noetherian rings are catenary,

until Nagata constructed counterexamples in 1956 (cf. Local

Rings, p.203, Example 2). In particular, he produced a

noetherian local domain which is catenary but not univer-

sally catenary. We will sketch here his construction in

its simplest form.

Let $k$ be a field and let $s=k[[x]]$ be the formal power series ring over $k$ in one variable $x$. Take an element $z=$ $\sum_{i} a_{i} x^{i}$ of $S$ which is algebraically independent over $k(x)$. $=1$

It is well known that the quotient field of $S$ has an infinite transcendence degree over $k(x)$. Cf. e.g. Zariski-Samuel, Commutative Algebra, Vol.II, p.220.) Put $z_{j}=\left(z-\sum_{i<j} a_{i} x^{i}\right) /$ $x^{j-1}$ for $j=1,2, \ldots$, (note that $z_{1}=z$ ), and let $R$ be the subring of $\mathrm{S}$ which is generated over $\mathrm{k}$ by $\mathrm{x}$ and by all the $z_{j}^{\prime} s: \quad R=k\left[x, z_{1}, z_{2}, \ldots\right]$. Consider the ideals $m=(x)$ and $\mathcal{M}=\left(x-1, z_{1}, z_{2}, \ldots\right)$ of $R$. Since $x\left(z_{j+1}+a_{j}\right)=z_{j}$ we have $z_{j} \varepsilon$ on for all $j$, and $m$ is a maximal ideal of $R$ with $R / m=k$. The local ring $R_{m}$ is a subring of $S$ and and $m R_{m}=x R_{m} \subset x S$. Hence $\bigcap_{n} x^{n} \subseteq \bigcap_{n}^{n} s=(0)$. Then

it is easy to see that any ideal $(\neq(0))$ of $R_{M \text { is }}$ is the

form $x^{i} R_{m}$. Thus $R_{m}$ is noetherian, and is a regular local

ring of dimension 1. On the other hand, $R$ is a subring of the rational function field in two variables $k(x, z)$, and so we have $R /(x-1)=k[x, z, z, z, \ldots] /(x-1) \simeq k[z]$, hence $\mu=$ $(x-1, z)$ and $R / U \simeq k$. The local ring $R_{M}$ contains $x$ and hence it is a localization of the ring $R\left[x^{-1}\right]=k\left[x, x^{-1}, z\right]$. This shows that $R_{\mu}$ is noetherian. Clearly $R_{n}$ is a regular local ring of dimension 2, Let $B$ be the localization of $R$ with respect to the multiplicatively closed subset $(R-M) \cap$ $(R-M) .$ Then $\mu B$ and $\mu B$ are the only maximal ideals of $B$ by $(1 . B)$, and the local rings $B_{\mu} B=R_{M}$ and $B_{M B}=R_{M}$ are noetherian, It follows easily (using $(1 . \mathrm{H})$ ) that any ideal of $B$ is finitely generated. Thus $B$ is a semi-local noetherian domain. Put $I=r a d(B)$ and $A=k+I$. Then $A$ is a subring of $B$, and It is easy to see that $(A, I)$ is a local ring. As $B / I \simeq B / M B \oplus B / M B \simeq k \oplus k$ the ring $B$ is a finite $A$-module. It follows (e.g. by Eakin's theorem cited in (2.D)) that $A$ is also noetherian. We have ht $(\mu B)=1$ and $h t(\mu B)=2$, hence dim $A=$ dim $B=2$ by $(13.0)$ Th.20 (1). If A were u.c. then we would have $h t(m B)=h t(\operatorname{m} \cdot B \cap A)=h t t_{A}(I)=\operatorname{dim} A$

$=2$ by the dimension formula. Therefore A is not u.c.. But

A is catenary because it is a local domain of dimension 2 . (14.F) THEOREM 24. Let $A=k\left[x_{1}, \ldots, x_{n}\right]$ be a polynomial ring over a field $k$, and let I be an ideal of A with ht(I)

$=r$. Then we can choose $Y_{1}, \ldots, Y_{n} \in A$ in such a way that

\begin{enumerate}
  \item A is integral over $k[Y]=k\left[Y_{1}, \ldots, Y_{n}\right]$, and

  \item $I \cap k[Y]=\left(Y_{1}, \ldots, Y_{r}\right)$.

\end{enumerate}
Proof. Induction on $\mathrm{r}$. If $\mathrm{r}=0$ then $\mathrm{I}=(0)$ and we car take $\mathrm{Y}_{i}=\mathrm{X}_{i}$. When $\mathrm{r}=1$, let $\mathrm{Y}_{1}=\mathrm{f}(\mathrm{X})$ be any non-zero element of $I$. Write $f(X)=\sum_{i=1} a_{i} M_{i}(X)$, where $0 \neq a_{i} \varepsilon k$ and $M_{i}(X)$ are distinct monomials in $X_{1}, \ldots, X_{n}$, and take $n$ : positive integers $d_{1}=1, d_{2}, \ldots, d_{n}$, If $M(X)=X_{i} a_{1}$ then let us call the integer $\Sigma a_{i} d_{i}$ the weight of the monomial $M(X)$. By a suitable choice of $d_{2}, \ldots, d_{n}$ we can see to it that no two of the monomials $M, \ldots, M$ that appear in $f(X)$ have the weight, (If $p$ is a given prime number, we can take $2, \ldots, d_{s}=p v_{s}$ where $v_{i}-v_{i-1}\left(i=2, \ldots, s ; v_{1}=(0)\right.$ are large integers. This remark will be useful for some PPlications.) Put $Y_{i} \times X_{i}-X_{1} d_{i}(i=2, \ldots, n)$. Then $\left.1=I(X)=I X_{1}, Y_{2}+X_{1}, \cdots Y_{n}+X_{1} d_{n}\right)=a_{i} X_{1}{ }^{e}+$ $8\left(X_{1}, Y_{2}, \ldots, Y_{n}\right)$ where $g$ is a polynomial whose degree in 1 is less than $e$ and $a_{1}$ is the coefficient of the term with highest weight $\ln \mathrm{F}(\mathrm{X})$. Then $\mathrm{X}_{1}$ is integral over $k[Y], \mathrm{an}^{+} \mathrm{d}$ $X_{i}=Y_{i}+X_{1} d_{1}(i=2, \ldots, n)$ are also integral over I 1 of $k[Y]$ is prime of height $1,\left(Y_{1}\right) \subseteq$ that $k[Y]$ is integrally closed and so the going-down theorem holds between $k[X]$ and $k[Y],$, Therefore $\left(Y_{1}\right)=\operatorname{In} k[Y]$, as wanted. When $r>1$, let $J$ be an ideal of $k[X]$ such that $J \subset I, h t(J)=r-1$. (The existence of such $J$ is easy to prove for any noetherian ring and for any ideal I of height r. Take $\mathrm{f}_{1} \varepsilon$ I from outside of the minimal prime ideals, and $\mathrm{f}_{2} \varepsilon$ I from outside of the minimal prime over-ideals of $\left(f_{1}\right)$, and $f_{3} \varepsilon$ I from outside of the minimal prime over-ideals of $\left(\mathrm{f}_{1}, \mathrm{f}_{2}\right)$, and so on, and put $\mathrm{J}=\left(\mathrm{f}_{1}, \ldots, \mathrm{f}_{\mathrm{r}-1}\right) . \mathrm{Th} .18$ is the basis of this construction.) By induction hypothesis there exist $Z_{1}, \ldots, Z_{n} \varepsilon k[X]$ such that $k[X]$ is integral over $k[Z]$ and that $k[Z] \cap J=\left(Z_{1}, \ldots, Z_{r-1}\right)$. Put $I^{\prime}=I \cap k[Z]$. Then $h t\left(I^{\prime}\right)=h t(I)=r$, and so $I^{\prime} D\left(Z_{1}, \ldots, Z_{r-1}\right)$. Thus we can choose an element $0 \neq f\left(z_{r}, \ldots, z_{n}\right)$ of $I^{\prime} \cdot$ Following the method we used for the case $r=1$, we put $Y_{i}=Z_{i}(i<r)$, $\mathrm{Y}_{r}=f\left(Z_{r}, \ldots, Z_{n}\right), Y_{r+j}=Z_{r+j}-Z_{r}{ }_{j}(1 \leqslant j \leqslant n-r)$. Then, for a suitable choice of $e_{1}, \ldots, e_{n-r}, k[Z]$ is integral over $k[Y]$. Moreover, Ink[Y] contains the prime ideal $\left(Y_{1}, \ldots, Y_{r}\right)$ of height $\mathrm{r}$ and so coincides with it. The proof is completed.

Remark. The above proof shows that we can choose the $Y_{i}^{\prime}$ s in such a way that $Y_{r+1}, \ldots, Y_{n}$ have the form $Y_{r+j}=X_{r+j}+$ $F_{j}\left(X_{1}, \ldots, X_{r}\right)$, where $F_{j}$ is a polynomial with coefficients in the prime subring $k_{0}$ of $k$ (i.e. the canonical image of $Z$ in k). If $\operatorname{ch}(\mathrm{k})=\mathrm{p}>0$ then we can see to it that $\mathrm{F}_{\mathrm{j}}\left(\mathrm{X}_{1}, \ldots, \mathrm{X}_{\mathrm{r}}\right)$ $\varepsilon k_{0}\left[x_{1}^{p}, \ldots, x_{r}^{p}\right]$ for all $j$.

(14.G) COROLLARY.1. (Normalization theorem of E.Noether) Let $A=k\left[x_{1}, \ldots, x_{n}\right]$ be a finitely generated algebra over a field $k_{0}$ Then there exist $y_{1}, \ldots, y_{r} \varepsilon$ A which are algebraically independent over $k$ such that $A$ is integral over $k\left[y_{1}\right.$, $\left.\ldots, y_{r}\right]$. We have $r=$ dim A. If $A$ is a domain we also have $r=t r \cdot d e g \cdot k$

Proof. Write $A=k\left[X_{1}, \ldots, X_{n}\right] / I$, and put $h t(I)=n-r$. According to the theorem there exist elements $Y_{1}, \ldots, Y_{n}$ of $k\left[X_{1}, \ldots, X_{n}\right]$ such that $k[X]$ is integral over $k[Y]$ and that $I \cap k[Y]=\left(Y_{r+1}, \ldots, Y_{n}\right) \cdot$ Putting $y_{i}=Y_{i} \bmod I(1 \leqslant i \leqslant r)$ we get the required result. The equality $r=\operatorname{dim}$ A follows from Th.20. The last assertion is obvious, as A is algebraic over $k\left(y_{1}, \ldots, y_{r}\right)$

COROLLARY 2. Let $k$ be an algebraically closed field. Then any maximal ideal wh of $k\left[x_{1}, \ldots, X_{n}\right]$ is of the form $w=\left(x_{1}-a_{1}, \ldots, x_{n}-a_{n}\right), a_{i} \varepsilon k .$

Proof. Since $0=\operatorname{dim}(\mathrm{A} / \mathrm{m})=\mathrm{tr} \cdot \mathrm{deg} \cdot \mathrm{k} / \mathrm{A}$, we get $\mathrm{A} / \mu \mathrm{k}=\mathrm{k}$. Hence $X_{i} \equiv a_{i}(1 n)$ for some $a_{i} \in k$ for each $i$. Since $\left(x_{1}-a_{1}, \ldots, x_{n}-a_{n}\right)$ is obviously a maximal ideal, it is 1 . (14.H) COROLLARY 3. Let A be a finitely generated algebra

over a field $k$. Then (1) if $A$ is an integral domain, we

have $\operatorname{dim}(A / P)+h t(p)=\operatorname{dim} A$ for any prime ideal $P$ of $A$

and (2) A is universally catenary.

Proof. (1) Take $y_{1}, \ldots, y_{r} \in A$ as in Cor.l, and put $p^{\prime}=$

$p \cap k[y]$. Then $\operatorname{dim} A=r, \operatorname{dim}(A / p)=\operatorname{dim}\left(k[y] / p^{\prime}\right)$ and $h t(p)=$

ht $\left(p^{\prime}\right)$. As $k[y]$ is isomorphic to the polynomial ring in $r$

variables, we have $h t\left(p^{\prime}\right)+\operatorname{dim}\left(k[y] / p^{\prime}\right)=r$ by the theorem.

(2) It suffices to prove that $k$ is universally cate-

nary. This is a consequence of (1) and (14.D), but we will

give a direct proof. We are going to prove $k\left[X_{1}, \ldots, X_{n}\right]$ is

catenary. Let $\mathrm{P} \supset Q$ be prime ideals of $k[X]=k\left[x_{1}, \ldots, X, X_{n}\right]$.

Then we have
$$
h t(P)=n-\operatorname{dim}(k[X] / P)
$$
$$
h t(Q)=n-\operatorname{dim}(k[X] / Q),
$$
$$
\begin{aligned}
& h t(P / Q)=\operatorname{dim}(k[X] / Q)-\operatorname{dim}(k[X] / P)
\end{aligned}
$$
and by (1)

Therefore $h t(P / Q)=h t(P)-h t(Q)$,

COROLLARY 4. (Dimension of intersection in an

affine space). Let $p_{1}$ and $p_{2}$ be prime ideals in a polynomial

ring $R=k\left[x_{1}, \ldots, x_{n}\right]$ over a field $k$, with $d i m\left(R / p_{1}\right)=r$,

$\operatorname{dim}\left(R / p_{2}\right)=s$. Let $q$ be any minimal prime over-ideal of

$p_{1}+p_{2} \cdot$ Then $\operatorname{dim}(\mathrm{R} / q) \geqslant r+s-n .$

(In geometric terms this means that, if $V_{1}$ and $V_{2}$ are irredu- cible slosed sets of dimension $r$ and s respectively, in an affine $n-$ space $\operatorname{spec}\left(k\left[x_{1}, \ldots, x_{n}\right]\right)$. Then any irreducible component of $\mathrm{V}_{1} \cap \mathrm{V}_{2}$ has dimension not less than $\left.\mathrm{r}+\mathrm{s}-\mathrm{n} .\right)$

Proof. Let $\mathrm{Y}_{1}, \ldots, \mathrm{Y}_{n}$ be another set of $\mathrm{n}$ indeterminates and let $p_{2}^{\prime}$ be the image of $p_{2}$ in $k\left[Y_{1}, \ldots, Y_{n}\right]$ by the isomorphism $k[X] \simeq k[Y]$ over $k$ which maps $X_{i}$ to $Y_{i}(1 \leqslant i \leqslant n)$. Let I be the ideal of $k[X, Y]=k\left[X_{1}, \ldots, X_{n}, Y_{1}, \ldots, Y_{n}\right]$ generated by $p_{1}$ and $P_{2}$, and $D$ the ideal $\left(X_{1}-Y_{1}, \ldots, X_{n}-Y_{n}\right)$ of $k[X, Y]$

Then $k[X, Y] / I \simeq\left(R / p_{1}\right) \otimes_{k}\left(R / p_{2}\right), \quad k[X, Y] / D \simeq k[X], \quad$ Take $\xi_{1}, \ldots, \xi_{r} \varepsilon R / p_{1}$ and $n_{1}, \ldots, n_{s} \varepsilon R / p_{2}$ such that $R / p_{1}$ (resp, $R / p_{2}$ ) is integral over $k[\xi]$ (resp. over $k[\eta]$ ). Then $k[X, Y] / I$ is integral over $k[\xi, n]$ which is a polynomial ring in r+s variables. Thus $\operatorname{dim}(k[X, Y] / I)=\operatorname{dim} k[\xi, \eta]=r+s$. Writing $k[X, Y] / I=k[x, y]$ we have $k[X, Y] / D+I=k[x, y] /$ $\left(x_{1}-y_{1}, \ldots, x_{n}-y_{n}\right)$. Since $k[X, Y] / I+D \simeq k[X] / p_{1}+p_{2}$, the prime $q$ of $k[x]$ corresponds to a minimal prime over-ideal $Q$ of $I+D$ in $k[X, Y]$ such that $k[X] / q \simeq k[X, Y] / Q$. Then $Q / I$ is a minimal prime over-ideal of $\left(x_{1}-y_{1}, \ldots, x_{n}-y_{n}\right)$ of $k[x, y]$, hence $h t(Q / I) \leqslant n$ by Th.18. Therefore $\operatorname{dim} k[x] / q$

$=\operatorname{dim} k[x, y] /(Q / I)=\operatorname{dim} k[x, y]-h t(Q / I) \geqslant r+s-n$ by the preceding corollary.

(14.L) THEOREM 25. (Zero-point theorem of Hilbert). Let $k$ be a field, A be a finitely generated $k$-algebra and $I$

be a proper ideal of A. Then the radical of I is the inter-

section of all maximal ideals containing $I$.

CHAPTER 6. DEPTH

Proof. Let $\mathrm{N}$ denote the intersection of all maximal ideals

containing $I$, and suppose that there is an element a $\varepsilon \mathrm{N}$

which is not in the radical of $I$. Put $S=\left\{1, a, a^{2}, \ldots\right\}$ and

$A^{\prime}=S^{-1} A$. Then $I^{\prime} \neq(1)$, so there is a maximal ideal $P^{\prime}$

\begin{enumerate}
  \setcounter{enumi}{15}
  \item M-regular Sequences
\end{enumerate}
of $A^{\prime}$ containing $I A^{\prime}$. Since $A^{\prime}$ is also finitely generated

over $k$, we have $0=\operatorname{dim~} A^{\prime} / P^{\prime}=t r \cdot d e g \cdot k^{\prime} / P^{\prime} \cdot$ Putting

(15.A) Let A be a ring, $M$ be an A-module and $a_{1}, \ldots, a_{r}$ be

$A \cap P^{r}=P$ we have $k \subseteq A / P \subseteq A^{\prime} / P^{\prime}$, hence $0=t r \cdot d e g \cdot k / P$

a sequence of elements of A. We write (a) for the ideal

$\left(a_{1}, \ldots, a_{r}\right)$, and $\underline{a} M$ for the submodule $\sum_{a_{i}} M=(\underline{a}) M$.

We say $a_{1}, \ldots, a_{r}$ is an M-regular sequence (or simply

M-sequence) if the following conditions are satisfied:

(1) for each $1 \leqslant i \leqslant r, a_{i}$ is not a zero-divisor on $M /\left(a_{1}, \ldots, a_{i-1}\right) M$, and

(2) $M \neq a M$.

When all $a_{1}$ belong to an ideal I we say $a_{1}, \ldots, a_{r}$ is an

M-regular sequence in I. If, moreover, there is no $b \varepsilon \mathrm{I}$

such that $a_{1}, \ldots, a_{r}, b$ is M-regular, then $a_{1}, \ldots, a_{r}$ is said

to be a maximal M-regular sequence in I. Notice that the

notion of M-regular sequence depends on the order of the

elements in the sequence.

$=\operatorname{dim} \mathrm{A} / \mathrm{P}$. Thus $\mathrm{P}$ is a maximal ideal of $\mathrm{A}$ containing I,

and a \&P, contradiction.

Remark. The theorem can be stated as follows: if $\mathrm{A}$ is a $\mathrm{k}-$

algebra of finite type, then the correspondence which maps

each closed set $V(I)$ of $\operatorname{spec}(A)$ to $V(I) \cap \Omega(A)$ is a bijection

between the closed sets of $\operatorname{Spec}(\mathrm{A})$ and the closed sets of

$\Omega(A)$. When $k$ is algebraically closed and $A \simeq k\left[X_{1}, \ldots, X, X_{n}\right] / I$

one can identify $\Omega(A)$ with the algebraic variety in $k^{n}$

defined by the ideal I (i.e. the set of zero-points of I in $\left.k^{n}\right)$ LEMMA 1. Suppose that $a_{1}, \ldots, a_{r}$ is M-regular and
$$
a_{1} \xi_{1}+\ldots+a_{r} \xi_{r}=0, \quad \xi_{i} \varepsilon M .
$$
Then $\xi_{i} \varepsilon \underline{a M}$ for a1l $i$.

Proof. Induction on $r$. For $r=1, a_{1} \xi_{1}=0$ implies $\xi_{1}=0$.

Let $r>1 .$ Since $a_{r}$ is $M /\left(a_{1}, \ldots, a_{r-1}\right) M$ - regular we have $\xi_{r}=\sum_{i=1}^{r-1} a_{i} n_{i}$, hence $\sum_{i=1}^{r-1} a_{i}\left(\xi_{i}+a_{r} n_{i}\right)=0$. By induction hypothesis, for $i<r$ we get $\xi_{i}+a_{r} n_{i} \varepsilon\left(a_{1}, \ldots, a_{r-1}\right) M$, so that $\xi_{i} \in\left(a_{1}, \ldots, a_{r}\right) M$.

THEOREM 26. Let $A, M$ be as above and $a_{1}, \ldots, a_{r} \varepsilon$ A be an M-regular sequence. Then for every sequence $v_{1}, \ldots, v_{r}$ of integers $>0$, the sequence $a_{1}^{v_{1}}, \ldots, a_{r}^{{ }_{r}^{r}}$ is M-regular.

Proof. It suffices to prove that $a_{1}{ }^{v}, a_{2}, \ldots, a_{r}$ is M-regular, because then $a_{2}, \ldots, a_{r}$ will be $M / a_{1}{ }^{M}$-regular and we can repeat the argument. We use induction on $v$, the case $v=1$ being true by assumption. Let $v>1$ and assume that $a_{1}^{v-1}, a_{2}$, $\ldots, a_{r}$ is M-regular. $a_{1}{ }^{v}$ is certainly M-regular. Let $i>1$ and assume that $a_{1}^{v}, a_{2}, \ldots, a_{i-1}$ is an M-regular sequence.

Let $a_{i} w=a_{1}{ }^{v}{ }_{1}+a_{2} \xi_{2}+\ldots+a_{i-1} \xi_{1-1}$. Then $w=a_{1}^{v-1} n_{1}+$ $a_{2} n_{2}+\ldots+a_{i-1} n_{i-1}$ by the induction hypothesis. So

$a_{1}^{v-1}\left(a_{1} \xi_{1}-a_{i} n_{1}\right)+a_{2}\left(\xi_{2}-a_{i} n_{2}\right)+\ldots+a_{i-1}\left(\xi_{i-1}-a_{i} n_{i-1}\right)$

$=0$, hence $a_{1} \xi_{1}-a_{i} n_{1} \varepsilon\left(a_{1}^{v-1}, a_{2}, \ldots, a_{i-1}\right) M$ by Lemma 1. It follows that $a_{i} n_{1} \varepsilon\left(a_{1}, a_{2}, \ldots, a_{i-1}\right) M$, hence $n_{1} \varepsilon\left(a_{1}, \ldots, a_{i-1}\right) M$ and so $\omega \varepsilon\left(a_{1}^{\nu}, a_{2}, \ldots, a_{i-1}\right) M$.

(15,B) Let $A$ be a ring, $X_{1}, \ldots, X_{n}$ be indeterminates over $A$ and $M$ be an A-module. An element of $M Q_{A} A\left[X_{1}, \ldots, X_{n}\right]$ can be viewed as a polynomial $F(X)=F\left(X_{1}, \ldots, X_{n}\right)$ with coefficients in $M$. Therefore we write $M\left[X_{1}, \ldots, X_{n}\right]$ for $M Q_{A} A\left[X_{1}\right.$, $\left.\ldots, x_{n}\right]$. If $a_{1}, \ldots, a_{n} \in A$ then $F(a) \varepsilon M$.

Let $a_{1}, \ldots, a_{n} \varepsilon A, I=$ (a) . We say that $a_{1}, \ldots, a_{n}$ is an M-quasiregular sequence if the following condition is satisfied.

(*) For every $v>0$ and for every homogeneous polynomial $F(X) \varepsilon M\left[X_{1}, \ldots, X_{n}\right]$ of degree $v$ such that $F(a) \varepsilon$ $\mathrm{I}^{\nu+1} \mathrm{M}$, we have $\mathrm{F} \varepsilon \operatorname{IM}[\mathrm{X}]$

Obviously this concept does not depend on the order of the elements. But $a_{1}, \ldots, a_{i}(i<n)$ need not be M-quasiregular. The condition (*) can be stated in the following form.

(**) If $F(X) \in M\left[X_{1}, \ldots, X_{n}\right]$ is homogeneous and $F(a)=0$, then the coefficients of $F$ are in IM.

Define a map $\phi:(M / I M)\left[x_{1}, \ldots, x_{n}\right] \rightarrow g r^{I} M=\bigoplus_{I} I_{M} / I^{\nu+1} M$ as follows. If $F(X) \in M[X]$ is homogeneous of degree $v$, let $\psi(F)=$ the image of $F(a)$ in $I^{U} M / I^{\nu+1} M$. Then $\psi$ is a degreepreserving additive map from $M[X]$ to $\mathrm{gr}^{I}(M)$, and since it maps $\operatorname{IM}[X]$ to 0 it induces $\phi:(M / I M)[X] \rightarrow \operatorname{gr}^{I}(M)$. This is

clearly surjective, and $(*)$ is equivalent to

$(* * *) \phi$ is an isomorphism: $(M / I M)\left[x_{1}, \ldots, X_{n}\right] \simeq g r^{I}(M) .$

THEOREM 27. Let $A$ be a ring, $M$ an A-module, $a_{1}, \ldots, a_{n} \in A$

and $\mathrm{I}=\underline{\text { aM. }}$. Then:

i) if $a_{1}, \ldots, a_{n}$ is $M$-quasiregular and $x \in A, \quad I M: x=I M$, then $I^{\nu} M: x=I^{\nu} M$ for a11 $v>0$,

ii) if $a_{1}, \ldots, a_{n}$ is M-regular then it is M-quasiregular;

iii) if $M, M / a_{1} M, M /\left(a_{1}, a_{2}\right) M, \ldots, M /\left(a_{1}, \ldots, a_{n-1}\right) M$ are

separated in the I-adic topology, then the converse of ii)

is also true.

Remark. The separation condition of iii) is satisfied in

either of the following cases:

(a) A is noetherian, $M$ is finitely generated and IS $\subseteq$ ad(A),

(B) A is a graded ring $A=\bigoplus_{\nu \geqslant 0} A_{V}$, $M$ is a graded $A-$

module $M=\underset{v \geqslant 0}{\bigoplus}, M^{\prime}$ and each $a_{i}$ is homogeneous of degree $>0 .$

Proof. i) Induction on $v$. Let $v>1, \xi \varepsilon M$ and suppose $x \xi \varepsilon I^{V}$. Then $\xi \varepsilon I^{V-1} M$, hence there exists a homogeneous polynomial $F(X) \varepsilon M\left[X_{1}, \ldots, X_{n}\right]$ of degree $v-1$ such that $\xi=$ $F(a)$. Since $x \xi=x F(a) \varepsilon I^{\nu} M$, the coefficients of $F$ are in IM : $x=I M$. Therefore $\xi=F(a) \varepsilon I^{\nu} M$.

ii) Induction on $n$. For $n=1$ it is easy to check. Let $n>1$. By induction hypothesis $a_{1}, \ldots, a_{n-1}$ is M-quasiregular. Let $F(X) \varepsilon M\left[X_{1}, \ldots, X_{n}\right]$ be homogeneous of degree $v$ such that $F(a)=0$. We will prove $F \in I M[X]$ by induction on $\nu$. Write $F(X)=G\left(X_{1}, \ldots, X_{n-1}\right)+X_{n} H\left(X_{1}, \ldots, X_{n}\right)$. Then $G$ and $H$ are homogeneous of degree $v$ and $v-1$, respectively, By i) we have $H$ (a) $\varepsilon\left(a_{1}, \ldots, a_{n-1}\right)^{\nu} M: a_{n}=\left(a_{1}, \ldots, a_{n-1}\right)^{\nu} S I^{V} M$, there-

\includegraphics[max width=\textwidth]{2022_08_01_8d4eee36f1f42236b4f4g-058}\\
Since $H(a) \varepsilon\left(a_{1}, \ldots, a_{n-1}\right)^{V} M$ there exists $h(X) \varepsilon M\left[X_{1}, \ldots, X X_{n-1}\right]$ which is homogeneous of degree $V$ such that $H(a)=h(a)$. Putting $G\left(x_{1}, \ldots, x_{n-1}\right)+a_{n} h\left(x_{1}, \ldots, x_{n-1}\right)=g(x)$ we have $g\left(a_{1}, \ldots, a_{n-1}\right)=0$, hence by the induction hypothesis on $n$ we have $g \varepsilon \operatorname{IM}[X]$, hence $G \varepsilon \operatorname{IM}[X]$ and so $F \varepsilon \operatorname{IM}[X]$.

iii) If $a_{1} \xi=0$ then $\xi \varepsilon$ IM, hence $\xi=\sum a_{i} n_{i}$ and $\sum a_{1} a_{i} n_{i}$ $=0$, hence $\eta_{i} \varepsilon$ IM and $\xi \varepsilon I^{2} M$. In this way we see $\xi \varepsilon$ A $^{v} M=0$. Thus $a_{1}$ is M-regular. Put $M_{1}=M / a_{1} M$. If $a_{2}$, $\ldots, a_{n}$ is $M_{1}$-quasiregular then our assertion will be proved by induction on $n$. $(M \neq I M$ follows from the separation condition.) Let $F\left(x_{2}, \ldots, X_{n}\right) \varepsilon M\left[x_{2}, \ldots, x_{n}\right]$ be homogeneous of degree $v$ such that $F\left(a_{2}, \ldots, a_{n}\right) \varepsilon a_{1}$ M. Put $F\left(a_{2}, \ldots, a_{n}\right)=$ $a_{1} \omega$, and assume $\omega \varepsilon I^{i} M$. Then $\omega=G\left(a_{1}, \ldots, a_{n}\right)$ for some homogeneous polynomial of degree i, and

(†) $\quad F\left(a_{2}, \ldots, a_{n}\right)=a_{1} G\left(a_{1}, \ldots, a_{n}\right)$.

If $i<v-1$ then $G \in I M[X]$ and so $\omega \varepsilon I^{i+1}$. We thus conclude that $\omega \varepsilon I^{v-1} M$. If $i=v-1$ in $(t)$, then $F\left(X_{2}, \ldots, x_{n}\right)-X_{1} G(X)$ $\varepsilon \mathrm{IM}[\mathrm{X}]$, and since $F$ does not contain $\mathrm{X}_{1}$ we have $F \varepsilon \operatorname{IM}[X]$.

Therefore $F \bmod a_{1} M[X] \varepsilon\left(a_{2}, \ldots, a_{n}\right) M_{1}[X]$.

The theorem shows that, under the assumptions of ii1),

any permutation of an M-regular sequence is M-regular.

Examples. 1. Let $k$ be a field and $A=k[X, Y, Z]$. Put $a_{1}=$

$\mathrm{X}(\mathrm{Y}-1), \mathrm{a}_{2}=\mathrm{Y}$ and $\mathrm{a}_{3}=\mathrm{Z}(\mathrm{Y}-1)$. Then $\mathrm{a}_{1}, \mathrm{a}_{2}, \mathrm{a}_{3}$ is an

A-regular sequence, while $a_{1}, a_{3}, a_{2}$ is not.

\begin{enumerate}
  \setcounter{enumi}{2}
  \item There exists a non-noetherian local ring (A,w) such
\end{enumerate}
that $=\left(x_{1}, x_{2}\right)$ where $x_{1}, x_{2}$ is an A-regular sequence but

$x_{2}$ is a zero-divisor in A. (J. Dieudonne, Nagoya Math. J.

$27-1(1966), 355-356 .)$

(15.C) If $a_{1}, a_{2}, \ldots \varepsilon$ A is an M-regular sequence then the

sequence of submodules $a_{1} M,\left(a_{1}, a_{2}\right) M, \ldots$ is strictly increas-

ing, hence the sequence of ideals $\left(a_{1}\right),\left(a_{1}, a_{2}\right), \ldots$ is also

strictly increasing. If A is noetherian such a sequence must

stop. Therefore each M-regular sequence in I can be extended

to a maximal M-regular sequence in I. The next theorem shows

that any two maximal M-regular sequences in I have the same

length if $M$ is finitely generated.

THEOREM 28. Let A be a noetherian ring, $M$ a finite A-module

and $I$ an ideal of $A$ with $I M \neq M$. Let $n>0$ be an integer. Then the following are equivalent:

(1) $\operatorname{Ext}_{A}^{i}(N, M)=0 \quad(i<n)$ for every finite A-module $N$ with $\operatorname{Supp}(\mathrm{N}) \subseteq \mathrm{V}(\mathrm{I})$;

(2) $\operatorname{Ext}_{A}^{i}(A / I, M)=0 \quad(1<n)$;

(3) there exists a finite A-module $N$ with $\operatorname{Supp}(N)=V(I)$ such that $\operatorname{Ext}_{A}^{i}(N, M)=0 \quad(i<n)$;

(4) there exists an M-regular sequence $a_{1}, \ldots, a_{n}$ of length $\mathrm{n}$ in I

Proof $:(1) \Rightarrow(2) \Rightarrow(3)$ is trivial. $(3) \Rightarrow(4)$ : We have $\operatorname{Ext}_{A}^{O}(N, M)=\operatorname{Hom}_{A}(N, M)=0$. If no elements of I are $M-$ regular, then I is contained in the join of the associated primes of $M$, hence in one of them by $(I . B): I \subseteq P$ for some $P \in \operatorname{Ass}(\mathrm{M})$. Then there exists an injection $\mathrm{A} / \mathrm{P} \rightarrow \mathrm{M} .$ Localizing at $P$ we get $\operatorname{Hom}_{A_{P}}\left(k, M_{P}\right) \neq 0$, where $k=A_{P} / P A_{P} \cdot$ Since $P \varepsilon V(I)=\operatorname{Supp}(\mathrm{N})$, we have $\mathrm{N}_{P} \neq 0$ and so $\mathrm{N}_{P} / \mathrm{PN}_{P}=\mathrm{N} Q_{A} k \neq 0$ by NAK. Then $\operatorname{Hom}_{k}(N \otimes k, k) \neq 0$. Therefore ${ }^{H}{ }_{A} A_{P}\left(N_{P}, M_{P}\right) \neq$ 0. But the left hand side is a localization of $\operatorname{Hom}_{A}(N, M)$ Which is o. This is a contradiction, therefore there exists an M-regular element $a_{1} \varepsilon$ I. If $n>1$, put $M=M / a_{1} M$

From the exact sequence
$$
0 \rightarrow \mathrm{M}^{\mathrm{a}} \rightarrow_{\mathrm{M}} \mathrm{M} \rightarrow \mathrm{M}_{1} \rightarrow 0
$$
we get the long exact sequence
$$
\ldots \rightarrow \operatorname{Ext}_{A}^{i}(N, M) \rightarrow \operatorname{Ext}_{A}^{i}\left(N, M_{1}\right) \rightarrow \operatorname{Ext}_{A}^{i+1}(N, M) \rightarrow \ldots
$$
which shows that $\operatorname{Ext}_{\mathrm{A}}^{1}\left(\mathrm{~N}, \mathrm{M}_{1}\right)=0(i<n-1)$. So by induction on $n$ there exists an $M_{1}$-regular sequence $a_{2}, \ldots, a_{n}$ in $I$.

$(4) \Rightarrow(1): \quad$ Put $M_{1}=M / a_{1} M .$ Then $\operatorname{Ext}_{A}^{i}\left(N, M_{1}\right)=0(i<n-1)$ by induction on $n$. From (*) we get exact sequences
$$
0 \rightarrow \operatorname{Ext}_{A}^{i}(\mathrm{~N}, \mathrm{M}) \stackrel{\rightarrow^{a}}{\operatorname{Ext}_{A}^{i}}(\mathrm{~N}, \mathrm{M}) \quad(1<\mathrm{n}) .
$$
But $\operatorname{Supp}(N)=V(\operatorname{Ann}(N)) \subseteq V(I)$, hence $I \subseteq$ radical of $\operatorname{Ann}(N)$ and so $a_{1}^{r} N=0$ for some $r>0$. Therefore $a_{1}^{r}$ annihilates $\operatorname{Ext}_{A}^{i}(N, M)$ as well. Thus we have $\operatorname{Ext}_{A}^{i}(N, M)=0(i<n)$.

\section{Under the assumptions of the theorem, we call the}
length of the maximal M-regular sequences in I the I-depth

of $M$ and denote it by depth $(M)$. The theorem shows that
$$
\operatorname{depth}_{I}(M)=\min \left\{i \mid \operatorname{Ext}_{A}^{i}(A / I, M) \neq 0\right\} \text {. }
$$
When $(A, m)$ is a local ring we write depth $M$ or depth $A_{A}$

for depth $(M)$ and call it simply the depth of M. Thus

depth $M=0$ iff $\operatorname{me} A s s(M)$. If $A$ is an arbitrary noether-

ian ring and $P E \operatorname{Spec}(A)$, we have depth $M_{P}=0 \Leftrightarrow P A_{P} \varepsilon$

Ass $_{A_{P}}\left(M_{P}\right) \Leftrightarrow P \in$ Ass $_{A}(M) \Rightarrow \operatorname{depth}_{P}(M)=0$. In general

we have depth $A_{P}\left(M_{P}\right) \geqslant \operatorname{depth}_{P}(M)$, because localization pre-

serves exactness. When $I M=M$ we define depth $I M)=\infty$.

For instance $\operatorname{depth}_{\mathrm{I}}(\mathrm{M})=0$ if $\mathrm{M}=0$.

(15.D) D. Rees introduced the notion of grade, which is

closely related to depth, in 1957. (The grade of an ideal or module, Proc. Camb. Phil. Soc. 53, 28-42.) Let A be a noetherian ring, $M \neq 0$ be a finite A-module and $I=A n n(M)$.

Then he puts
$$
\text { grade } M=\inf \left\{1 \mid \operatorname{Ext}_{A}^{i}(M, A) \neq 0\right\} .
$$
According to the above theorem, we have
$$
\text { grade } M=\operatorname{depth}_{I}(A), \quad I=\operatorname{Ann}(M) \text {. }
$$
Also, it follows from the definition that
$$
\text { grade } M \leqslant \operatorname{proj} \cdot d i m M .
$$
When I is an ideal of $A$, grade(A/I) is called the grade of I. [Thus grade I can have two meanings according to whether I is viewed as an ideal or as a module. When confusion can arise, the depth notation should be used.] The grade of an ideal I is depth $\mathrm{I}_{\mathrm{I}}(\mathrm{A})$, the length of maximal A-sequence in $I$. If $a_{1}, \ldots, a_{r}$ is an A-regular sequence it is easy to see that $h t\left(a_{1}, \ldots, a_{r}\right)=r$. Therefore

grade I $\leqslant h t$ I.

PROPOSITION. Let $A$ be a noetherian ring, $M(\neq 0)$ and $\mathrm{N}$ be finite A-modules, grade $M=k$ and proj.dim $N=l<k$. Then
$$
\operatorname{Ext}_{\mathrm{A}}^{i}(M, N)=0 \quad(i<k-l) .
$$
Proof. Induction on $\ell$. If $\ell=0$ then $N$ is a direct summand of a free module. Since our assertion holds for A by definition, it holds for $N$ also. If $l>0$ take an exact sequence $0 \rightarrow N^{\prime} \rightarrow L \rightarrow N \rightarrow 0$ with L free. Then proj.dim $N^{\prime}=l-1$

and our assertion is proved by induction.

(15.E) LEMMA 2. (Ischebeck) Let $(A, \ldots)$ be a noetherian

local ring and $M \neq 0$ and $N \neq 0$ be finite A-modules. Put

depth $M=k, \operatorname{dim} N=r$. Then
$$
\operatorname{Ext}_{A}^{i}(N, M)=0 \quad(i<k-r) .
$$
Proof. Induction on $r$. If $r=0$ then $\operatorname{Supp}(N)=\{m\}$ and

the assertion follows from Th.28. Let $r>0$. By p.51 Th. I0

we can easily reduce to the case $N=A / P, P \varepsilon$ Spec(A). Since

$\mathrm{r}=\operatorname{dim} \mathrm{A} / \mathrm{P}>0$ we can pick $\mathrm{x} \varepsilon \mathrm{m}-\mathrm{P}$, and then $0 \rightarrow \mathrm{N} \rightarrow$

$N \rightarrow N^{\prime} \rightarrow 0$ is exact, where $N^{\top}=A /(P+A x)$ has dimension $<r$.

Then using induction hypothesis we get exact sequences
$$
0 \rightarrow \operatorname{Ext}_{A}^{i}(N, M) \stackrel{x}{\rightarrow} \operatorname{Ext}_{A}^{i}(N, M) \rightarrow \operatorname{Ext}_{A}^{i+1}\left(N^{\prime}, M\right)=0
$$
for $i<k-r$, and these Ext must vanish by NAK. Q.E.D.

THEOREM 29. Let $(A, \psi)$ be a noetherian local ring and let

$M \neq 0$ be a finite A-module. Then we have

depth $M \leqslant \operatorname{dim}(\mathrm{A} / \mathrm{P})$ for every $\mathrm{P} \varepsilon \operatorname{Ass}(\mathrm{M}) .$

Proof. If $P \varepsilon \operatorname{Ass}(M)$ then $\operatorname{Hom}_{A}(A / P, M) \neq 0$, hence depth $M$

$\leqslant \operatorname{dim}(\mathrm{A} / \mathrm{P})$ by Lemma $2 .$

LEMMA 3. Let A be a ring, and let $E$ and $F$ be finite A-modules. Then $\operatorname{Supp}(E \otimes F)=\operatorname{Supp}(E) \cap \operatorname{Supp}(F)$.

Proof. Fot $P \varepsilon \operatorname{Spec}(A)$ we have $(E \otimes F)_{P}=\left(E \otimes_{A} F\right) \otimes_{A} A_{P}=$ ${ }^{E_{P}}{ }_{A_{P}}{ }_{P}$. Therefore the assertion is equivalent to the following: Let $(A, m, k)$ be a local ring and $E$ and $F$ be finite A-modules. Then $\mathrm{E} \otimes \mathrm{F} \neq 0 \Longleftrightarrow \mathrm{E} \neq 0$ and $\mathrm{F} \neq 0$. Now $\Rightarrow$ is trivial. Conversely, if $E \neq 0$ and $F \neq 0$ then $E \otimes k=$ $E / M E \neq 0$ by NAK. Similarly $F \otimes k \neq 0 . \quad$ Since $k$ is a field we get $(E \otimes F) \otimes k=(E \otimes k) \otimes_{k}(F \otimes k) \neq 0$, so $E \otimes F \neq 0$.

LEMMA 4. Let A be a noetherian local ring and $M$ be a finite A-module. Let $a_{1}, \ldots, a_{r}$ be an M-regular sequence. Then $\operatorname{dim} M /\left(a_{1}, \ldots, a_{r}\right) M=\operatorname{dim} M-r .$

Proof. We have $\operatorname{dim} M / \underline{a M} \geqslant \operatorname{dim} M-\mathrm{r}$ by Th.17. On the other hand, suppose $f$ is an M-regular element. We have $\operatorname{Supp}(M / f M)=\operatorname{Supp}(M) \cap \operatorname{Supp}(A / F A)=\operatorname{Supp}(M) \cap V(f)$ by Lemma 3 , and $f$ is not in any minimal element of $\operatorname{Supp}(M)$, in other words $V(f)$ does not contain any irreducible component of $\operatorname{Supp}(M) \cdot$ Hence $\operatorname{dim}(M / f M)<\operatorname{dim} M .$ This proves $\operatorname{dim} M / a M \leqslant$ $\operatorname{dim} M-r$

PROPOSITION. Let A be a noetherian ring, $M$ a finite A-module and I an ideal. Then

$\operatorname{depth}_{I}(M)=\inf \left\{\operatorname{depth} M_{P} \mid P \varepsilon V(I)\right\} .$ Proof. Let $n$ denote the value of the right hand side. If $n=0$ then depth $M_{P}=0$ for some $P \supseteq I$, and then $I \subseteq P \varepsilon$ $\operatorname{Ass}(M)$. Thus $\operatorname{depth}(M)=0 .$ If $0<n<\infty$, then I is not contained in any associated prime of $M$, and so there exists by (1.B) an M-regular element a $\varepsilon I .$ Put $M^{\dagger}=M / a M .$ Then depth $\left(M^{\prime}\right)_{P}=\operatorname{depth} M_{P} / M_{P}=\operatorname{depth} M_{P}-1$ for $P \geq I$, and $\operatorname{depth}_{I}\left(M^{\prime}\right)=\operatorname{depth}_{I}(M)-1$. Therefore our assertion is proved by induction on $n$. If $n=\infty$ then $\mathrm{PM}_{P}=M_{P}$ for a11 $P$ $\varepsilon V(I)$. If IM $\neq M$ we would have $(M / I M)_{P} \neq 0$ for every $P \varepsilon$ $\operatorname{Supp}(M / I M)=V(I) \cap \operatorname{Supp}(M)$. If $P$ is a minimal element of Supp(M/IM) then $\operatorname{Supp}_{A_{P}}(M / I M)_{P}=\left\{P_{P}\right\}$, hence the $A_{P}$-module $(M / I M)_{P}=M_{P} / I M_{P}$ is coprimary in $M_{P}$ and $P^{S} M_{P} \subseteq M_{P}$ for some $s>0$ by $(8 . B)$. Hence $P_{P} \neq M_{P}$, contradiction. Therefore $I M=M$ and $\operatorname{depth}_{I}(M)=\infty$.



CHAPTER 7. NORMAL RINGS and REGULAR RINGS

\begin{enumerate}
  \setcounter{enumi}{4}
  \item Let $A$ be a noetherian local ring and $M$ be a finte
\end{enumerate}
the following formula due to Auslander-Buchsbaum:

proj.dim $M+$ depth $M=$ depth $A$.

[Hint: Use induction on proj.dim $M$. For the case proj.dim $M$

$=1$, use the exercise 3 above. I

\begin{enumerate}
  \setcounter{enumi}{5}
  \item Let $A$ be as above and let $P \in$ Spec A. Show that
\end{enumerate}
i) depth $A \leqslant \operatorname{depth}_{P}(A)+\operatorname{dim} A / P$,

ii) Put codepth $\mathrm{A}=\operatorname{dim} \mathrm{A}-\operatorname{depth} \mathrm{A}$. Then

\begin{enumerate}
  \setcounter{enumi}{17}
  \item Classical Theory
\end{enumerate}
codepth $A \geqslant$ codepth $A_{P}$.

Further References.

The concept of depth has striking applications in

(17. A) Let $A$ be an integral domain, and $K$ be its quotient field. We say that A is normal if it is integrally closed in $K$. If A is normal, so is the localization $\mathrm{S}^{-1} \mathrm{~A}$ for every multiplicatively closed subset $S$ of $A$ not containing 0 .

Since $A=A_{\text {all }} A_{P}$ by $(1 . H)$, the domain A is normal iff $A_{p}$ is normal for every maximal ideal $p$.

An element $u$ of $K$ is said to be almost integral over $A$ if there exists an element a of $A(a \neq 0)$ such that $a^{n} \varepsilon A$ for all $n>0$. If $u$ and $v$ are almost integral over $A$, so are $u+v$ and uv. If $u \varepsilon \mathrm{K}$ is integral over A then it is almost integral over A. The converse is also true when A is noetherian, In fact, if $a \neq 0$ and $a u^{n} \varepsilon A(n=1,2, \ldots)$, then

$A[u]$ is a submodule of the finite A-module $a^{-1} A$, whence $A[u]$

\section{unexpected areas:}
\begin{enumerate}
  \item R. Hartshorne: Complete intersections and connected-
\end{enumerate}
ness. Amer. J. Math. $84(1962), 497-508 .$

For instance he proves that, if $A$ is a noeth. local ring and

if $\operatorname{Spec}(\mathrm{A})-\{m\}$ is disconnected, then depth $A \leqslant 1$.

\begin{enumerate}
  \setcounter{enumi}{2}
  \item D.Buchsbaum- D.Eisenbud: What makes a complex exact ?
\end{enumerate}
J. of Alg. 25(1973), 259-268.

They show that if C. : $0 \rightarrow F_{n} \rightarrow F_{n-1} \rightarrow \ldots+F_{0}$ is a complex of finite free modules over a noetherian ring, and if $\mathrm{E}_{\mathbf{i}}$ de-

note the matrix of the map $F_{i} \rightarrow F_{i-1}$, then the exactness of C. can be fully expressed in terms of the ranks of the modules and maps and depth $I_{i}$, where $I_{i}$ is the ideal generated by certain minors of the matrix $E_{i}(1 \leqslant i \leqslant n)$. For applications of their theorem, cf. D.Eisenbud: Some directions on recent

progress in comm. algebra, in Proc.Symp.Pure Math.29 (1975). itself is finite over $A$ and $u$ is integral over $A$. We say that A is completely normal if every element $u$ of $K$ which is almost integral over A belongs to A. For a noetherian domain normality and complete normality coincide. Valuation rings of rank (= Krull dimension) greater than one (cf. Nagata: LOCAL RINGS or Zariski-Samuel: COMM. ALG. vo1.II) are normal but not completely normal.



\section{Homological Theory}
(18. A) Let A be a ring. The projective (resp. injective) dimension of an A-module $M$ is the length of a shortest projective (resp. injective) resolution of $M$.

LEMMA 1. (i) An A-module $M$ is projective iff $\operatorname{Ext}_{A}^{1}(M, N)=0$ for all A-modules $\mathrm{N}$.

(ii) $M$ is injective iff $\operatorname{Ext}_{\mathrm{A}}^{1}(\mathrm{~A} / \mathrm{I}, \mathrm{M})=0$ for all ideals I of A.

Proof. Immediate from the definitions. In (ii) we use the fact (which is proved by Zorn's lemma) that if any homomorphism $f: N \rightarrow M$ can be extended to any A-module $N^{\prime}$ containing $N$ such that $N^{\prime}=N+A \xi$ for some $\xi \varepsilon N^{\prime}$, then $M$ is injective.

LEMMA 2. Let $A$ be a ring and $n$ be a non-negative integer.

Then the following conditions are equivalent:

(1) proj.dim $M \leqslant n$ for all A-modules $M$,

(2) proj.dim $M \leqslant n$ for all finite A-modules $M$,

(3) inj. dim $M \leqslant n$ for all A-modules $M$,

(4) $\operatorname{Ext}_{A}^{n+1}(M, N)=0$ for all A-modules $M$ and $N$.

Proof. $(1) \Rightarrow(2)$ : trivial. $(2) \Rightarrow(3):$ take an exact sequence $0 \rightarrow \mathrm{M} \rightarrow \mathrm{U}_{0} \rightarrow \mathrm{U}_{1} \rightarrow \ldots \rightarrow \mathrm{U}_{\mathrm{n}-1} \rightarrow \mathrm{C} \rightarrow 0$ with $\mathrm{U}_{\mathrm{j}}$ injective for al1 $j$. Let $I$ be any idea1. Then we have $\operatorname{Ext}_{A}^{1}(A / I, C) \simeq$ Ext $_{A}^{n+1}(A / I, M)$, which is zero by (2) since $A / I$ is a finite A-module, (4) $\Rightarrow$ (1) is proved similarly, with "projective"

(3) $\Rightarrow(4)$ is trivial, as one can calculate Ext ${ }_{A}^{*}(M, N)$ using an injective resolution of $\mathrm{N}$

By virtue of Lemma 2 we have

$\sup _{M}(\operatorname{proj} \cdot \operatorname{dim} M)=\sup _{M}(\operatorname{inj} \cdot \operatorname{dim} M) .$

We call this common value (which may be $\infty$ ) the global dimension of $A$ and denote it by gl. $\mathrm{d} i \mathrm{~m}$ A. (In EGA it is denoted by dim. $\operatorname{coh}(\mathrm{A}) \cdot)$

(18.B) LEMMA 3. Let A be a noetherian ring and $M$ a finite A-module. Then $M$ is projective iff $\operatorname{Ext}_{A}^{1}(M, N)=0$ for a11 finite A-modules $\mathrm{N}$.

Proof. Take a resolution $0 \rightarrow \mathrm{R} \rightarrow \mathrm{F} \rightarrow \mathrm{M} \rightarrow 0$ with $\mathrm{F}$ finite and free. Then $R$ is also finite, hence we have $\operatorname{Ext}^{1}(M, R)=0$. Thus $\operatorname{Hom}(F, R)+\operatorname{Hom}(R, R) \rightarrow 0$ is exact, and so there exists $s: F \rightarrow R$ with soi $=i d_{R}, 1 . e$. the sequence $0 \rightarrow R \rightarrow F \rightarrow M$

$\rightarrow 0$ splits. Then $M$ is a direct summand of a free module.

LEMMA 4. Let $(A, m, k)$ be a noetherian local ring, and $M$ be a finite A-module. Then proj. $\operatorname{dim} M \leqslant n \quad \Leftrightarrow \quad \operatorname{Tor}_{n+1}^{A}(M, k)=0 .$

Proof. $(\Rightarrow)$ Trivial. $(\Leftarrow)$ The general case is easily reduced to the case $n=0$. If $\operatorname{Tor}_{1}(M, K)=0$, let $0+R$ $\rightarrow F \rightarrow M \rightarrow 0$ be exact with u minimal (cf. P.113 Ex.3). Then $0 \rightarrow \mathrm{R} \otimes \mathrm{k} \rightarrow \mathrm{F} \otimes \mathrm{k} \rightarrow \mathrm{M} \otimes \mathrm{k} \rightarrow 0$ is exact and $\bar{u}$ is an isomorphism, hence $R \otimes k=0$ and so $R=0$ by NAK. Therefore $M$ is free, as wanted.

LEMMA 5. (I) Let A be a noetherian ring and M a finite Amodule. Then (i) proj. dim M is equal to the supremum of proj. $\operatorname{dim} M_{p}$ (as $A_{P}-$ module) for the maximal ideals $P$ of $A$, and (ii) we have proj. dim $M \leqq n$ iff $\operatorname{Tor}_{n+1}^{A}(M, A / p)=0$

for all maximal ideals $p$ of A.

(II) The following conditions about a noetherian ring A are equivalent:

(1) g1. $\operatorname{dim} A \leqslant n$,

(2) proj. $\operatorname{dim} M \leqslant n$ for all finite A-modules $M$,

(3) inj. $\operatorname{dim} M \leqslant n$ for all finite A-modules $M$,

(4) $\operatorname{Ext}_{A}^{n+1}(M, N)=0$ for all finite A-modules $M$ and $N$,

(5) $\operatorname{Tor}_{n+1}^{A}(M, N)=0$ for all finite A-modules $M$ and $N$.

(III) For any noetherian ring $A$, we have
$$
\operatorname{gl.} \operatorname{dim} A=\sup _{\max \cdot p} g 1 \cdot \operatorname{dim}\left(A_{p}\right)
$$




REMARKS TO CHAPTER $7 .$


  \item  A normal domain $A$ is called a krull ring if (1) for any non-zero element $x$ of $A$, the number of the prime ideals of $A$ of height one containing $x$ is finite, and (2) $A=$

CIAPTER 8. FLATNESS II. $h t(p)=1{ }^{A} p$. Noetherian normal rings are Krull, but not conversely, If $A$ is a noetherian domain, then the integral

closure of A in the quotient field of $A$ is a Krull ring

(Theorem of $\mathrm{Y}$. Mori, cf. Nagata: Local Rings). On Krul1 Iings, cf. Bourbaki: Alg. Comm. Ch.7.

\begin{enumerate}
  \setcounter{enumi}{3}
  \item P. Samuel has made an extensive study on the subject of unique factorization. Cf. his Tata lecture note. 20. Local Criteria of Flatness

  \item We did not discuss valuation theory. On this topic

\end{enumerate}
the following paper contains important results in connection

(20.A) In (18.B) Lemma 4 we proved the following. with algebraic geometry. Abhyankar: On the valuations centerLet $(A, M)$ be a noetherian local ring and $M$ a finite $A$-module. Then $M$ is $f l a t$ iff $\operatorname{Tor}_{1}(M, A / M)$ ed in a local domain, Amer. J. Math. $78(1956), 321-348 .$

The condition that $M$ is finite over $A$ is too strong; in geometric application it is of ten necéssary to prove flatness of infinite modules. In this section we shall learn several criteria of flatness, due to Bourbaki, which are very useful.

Let $A$ be a ring, I an ideal of $A$ and $M$ an A-module. We say that $M$ is idealwise separated (i.s. for short) for I if, for each finitely generated ideal $q$ of $A$, the A-module $q_{A} \otimes_{A}$ is separated in the I-adic topology.

Example 1. Let B be a noetherian A-algebra such that IB $\subseteq$ rad(B), and let $M$ be a finite $B$-module. Then $M$ is i.s. for $I$ as an $A$-module: since $q \otimes_{A} M$ is a finite $B$-module and since the I-adic topology on $q \otimes M$ is nothing but the IB-adic topo$\log y$, we can apply (11.D) Cor.l.

Example 2. When A is a principal ideal domain, any I-adically

separated A-module $M$ is i.s. for $I$.

Example 3. Let $M$ be an I-adically separated flat A-module.

Then $M$ is i.s. for $I$. In fact we have $q \otimes M \simeq q M \subseteq M$.

\includegraphics[max width=\textwidth]{2022_08_01_8d4eee36f1f42236b4f4g-082}

Then the following are equivalent:

(1) M is A-flat;

(2) $\operatorname{Tor}_{1}^{A}(\mathrm{~N}, \mathrm{M})=0$ for all $\mathrm{A}_{0}$-modules $\mathrm{N}$; (3) $M_{0}$ is $A_{0}-f l a t$, and $I \otimes_{A} M \simeq I M$ by the natural map, (note that, if I is a maximal ideal, the flatness over $A_{0}$ is trivial);

(3') $\mathrm{M}_{0}$ is $\mathrm{A}_{0}-\mathrm{flat}$ and $\operatorname{Tor}_{1}^{\mathrm{A}_{1}}\left(\mathrm{~A}_{0}, \mathrm{M}\right)=0$;

(4) $\mathrm{M}_{0}$ is $\mathrm{A}_{0}-\mathrm{flat}$, and the canonical maps
$$
\gamma_{n}: I^{n} / I^{n+1} \otimes_{A_{0}}{ }_{0} \rightarrow I^{n} M / I^{n+1} M
$$
are isomorphisms;

(5) $M_{n}=M / I^{n+1} M$ is $f$ lat over $A_{n}=A / I^{n+1}$, for each $n \geqslant 0$.

(The implications $(1) \Rightarrow(2) \Leftrightarrow(3) \Leftrightarrow\left(3^{\top}\right) \Rightarrow(4) \Rightarrow(5)$ are true without any assumption on M.)

Proof. We first prove the equivalence of (1) and (5) under the assumption $(\alpha)$ or $(\beta)$. The implication $(1) \Rightarrow(5)$ is just a change of base $(\mathrm{cf},(3, \mathrm{C}))$.

(5) $\Rightarrow(1)$ : The nilpotent case $(\alpha)$ is trivia1 $\left(A=A_{n}\right.$ for some $n)$. In the case $(B)$, we prove the flatness of $M$ by showing that, for every ideal $q$ of $A$, the canonical map $j: q \otimes M \rightarrow M$ Is injective. Since $q \otimes M$ is I-adically separated it suffices to prove that $\operatorname{Ker}(j) \subseteq I(q \otimes M)$ for all $\mathrm{n}>0, \quad$ Fix an $n$ Then there exists, by Artin-Rees, an integer $k>n$ such that $q \cap I^{k} \subseteq I^{n} q \cdot$ Consider the natural maps

$q \otimes M \stackrel{f}{\rightarrow}\left(q / I^{k} n q\right) \otimes M \stackrel{g}{\rightarrow}\left(q / I^{n} q\right) \otimes M=(q \otimes M) / I^{n}(q \otimes M) .$ Since $M_{k-1}$ is $A_{k-1}-f l a t$, the natural map $q /\left(I^{k} n q\right) \otimes_{A}=$ $q /\left(I^{k} \cap q\right) \otimes_{A_{k-1} M_{k-1}} \rightarrow M_{k-1}$ is injective. Therefore $\operatorname{Ker}(j) \subseteq \operatorname{Ker}(f)$, and a fortiori $\operatorname{Ker}(j) \subseteq \operatorname{Ker}(g f)=I^{\mathrm{n}}(q \otimes M)$.

Thus our assertion is proved.

Next we prove $(1) \Rightarrow(2) \Leftrightarrow(3) \Leftrightarrow\left(3^{\prime}\right) \Rightarrow(4) \Rightarrow(5)$ for arbitrary $M_{\bullet}(1) \Rightarrow(2)$ is trivial.

$(2) \Rightarrow(3): \quad$ Let $0 \rightarrow N^{\prime} \rightarrow N \rightarrow N^{\prime \prime} \rightarrow 0$ be an exact sequence of $A_{0}$-modules. Then $0=\operatorname{Tor}_{1}^{A}\left(N^{\prime \prime}, M\right) \rightarrow N^{\prime} \otimes_{A} M=N^{\prime} \otimes_{A_{0}}{ }^{M}{ }^{M} \rightarrow$ $N \otimes_{A} M=N \otimes_{A_{0}} M_{0}$ is exact, so $M_{0}$ is $A_{0}-f l a t$. From the exact sequence $0 \rightarrow I \rightarrow A \rightarrow A_{0} \rightarrow 0$ we get $0=\operatorname{ToI}_{1}\left(A_{0}, M\right) \rightarrow I$ W $M$

$(3) \Rightarrow\left(3^{*}\right):$ immediate.

$\left(3^{\prime}\right) \Rightarrow(2)$ : let $\mathrm{N}$ be an $\mathrm{A}_{0}$-module and take an exact sequence of $\mathrm{A}_{0}$-modules $0 \rightarrow R \rightarrow \mathrm{F}_{0} \rightarrow \mathrm{N} \rightarrow 0$ where $\mathrm{F}_{0}$ is $\mathrm{A}_{0}$-free, Then $\operatorname{Tor}_{1}^{A}(F, M)=0 \rightarrow \operatorname{Tor}_{1}^{A}(N, M) \rightarrow R \otimes_{A_{0}} M_{0} \rightarrow F_{0} \otimes_{A_{0}}{ }^{M}{ }_{0}$ is exact and $M_{0}$ is $A_{0}-f l a t$, hence $\operatorname{Tor}_{1}^{A}(N, M)=0$.

(2) $\Rightarrow(4)$ : consider the exact sequences
$$
0 \rightarrow I^{n+1} \rightarrow I^{n} \rightarrow I^{n} / I^{n+1} \rightarrow 0
$$
and the commutative diagrams

\includegraphics[max width=\textwidth]{2022_08_01_8d4eee36f1f42236b4f4g-083}

where $\alpha_{1}, \alpha_{2}, \ldots$ are the natural epimorphisms, the first row is exact by (2) and the second row is of course exact. Since $\alpha_{1}$ is injective by (3) we see inductively that all $\alpha_{n}$ are injective. Thus they are isomorphisms, and consequently the $Y_{n}$ are also isomorphisms.

Before proving $(4) \Rightarrow(5)$ we remark the following fact: if (2) holds then, for any $n \geqslant 0$ and for any $A_{n}-m o d u l e N$, we have $\operatorname{Tor}_{1}^{A}(N, M)=0$. In fact, if $N$ is an $A$-module and $n>0$, then IN and N/IN are $A_{n-1}$-modules, so that the assertion is proved by induction on $n$.

(4) $\Rightarrow(5)$ : we $f i x$ an integer $n \geqslant 0$ and we are going to prove that $M_{n}$ is $A_{n}-f 1 a t$. For $n=0$ this is included in the assumptions, so we suppose $n>0$. Put $I_{n}=I / I^{n+1}$

Consider the commutative diagrams with exact rows:

\includegraphics[max width=\textwidth]{2022_08_01_8d4eee36f1f42236b4f4g-083(1)}

$0 \rightarrow I^{i+1} M_{n}=I^{i+1} M / I^{n+1} M \rightarrow I_{M}^{i}=I^{i} M / I^{n+1} M \rightarrow I_{M}^{i} I^{i+1} M \rightarrow 0$

for $i=1,2, \ldots$, n. Since the $Y_{i}$ are isomorphisms by assumption, and since $\bar{\alpha}_{n+1}=0$, we see by descending induction on $i$ that all $\bar{\alpha}_{1}$ are isomorphisms. In particular, $\bar{\alpha}_{1}: I / I^{n+1}$ $\otimes_{A} M=I_{n} \otimes_{A_{n}} M_{n} \rightarrow I_{n}$ is an isomorphism. Therefore the condition (3) (hence also (2)) holds for $A_{n}, I_{n}$ and $M_{n}$. From this and from what we have just remarked it follows that $\operatorname{Tor}_{1}{ }^{n}\left(N, M_{n}\right)=0$ for all $A_{n}$-modules $N$, hence $M_{n}$ is $A_{n}-f l a t$. Q.E.D. (20.D) APPLICATION 1 (Hartshorne). Let $(B, n)$ be a noetherian local ring containing a field $k$ and let $x_{1}, \ldots, x_{n}$ be a B-regular sequence in $w$. Then the subring $k\left[x_{1}, \ldots, x_{n}\right]$ of $B$ is isomorphic to the polynomial ring $A=k\left[X_{1}, \ldots, X_{n}\right]$, and $B$ is flat over it.

Proof. Considering the k-algebra homomorphism $\phi: \mathrm{A} \rightarrow \mathrm{B}$ such that $\phi\left(X_{i}\right)=x_{i}$, we view $B$ as an A-algebra. It suffices to prove $B$ is flat over $A$. In fact, any non-zero element $y$ of $A$ is A-regular, so under the assumption of flatness it is a1so B-regular, hence $\phi(y) \neq 0$.

We apply the criterion $\left(3^{\prime}\right)$ of $T h .49$ to $A, I=\sum_{1} X_{i} A$ and $M=B$. The A-module $B$ is idealwise separated for I as IB $\subseteq$ $\operatorname{rad}(B)$. Since $A / I=k$ is a field we have only to prove $\operatorname{Tor}_{1}^{A}(k, B)=0$. Now the Koszu1 complex $K .\left(X_{1}, \ldots, X_{n} ; A\right)$ is a free resolution of the A-module $k=A / I$ by Cor. to Th.43. So we have $\operatorname{Tor}_{i}^{A}(k, B)=H_{i}\left(K_{0}\left(X_{1}, \ldots, X_{n} ; A\right) \otimes_{A} B\right)=H_{i}\left(K_{0}\left(x_{1}, \ldots,\right.\right.$, $\left.\left.x_{n} ; B\right)\right)$, which is zero for $i>0$ as $x_{1}, \ldots, x_{n}$ is a B-regular sequence.

(20.E) APPLICATION 2 (EGA $\left.O_{\text {II }}(10,2.4)\right)$. Let $(A, m, k)$ and $\left(B, 1, k^{\prime}\right.$ ) be noetherian local rings and $A \rightarrow B$ a local homomorphism. Let $u: M \rightarrow N$ be a homomorphism of finite B-modules, and assume that $\mathrm{N}$ is A-flat. Then the following are equi- valent: (a) $u$ is injective, and $N / u(M)$ is A-flat;

(b) $\bar{u}: M \otimes_{A} k \rightarrow N_{A} k$ is injective.

Proof. $(a) \Rightarrow(b)$. Immediate.

(b) $\Rightarrow$ (a). Let $x \in \operatorname{Ker}(u)$. Then $x \otimes 1=0 \quad$ in $M \otimes k$

$=M / M r M$, therefore $x \varepsilon$ m. We will show $x \in \bigcap_{n} m^{n} M=(0)$ by induction. Suppose $x \in M^{n} M$, let $\left\{a_{1}, \ldots, a_{p}\right\}$ be a minimal basis of the ideal $m^{n}$ and write $x=\sum a_{i} x_{i}, x_{i} \varepsilon M .$ Then $u(x)=\sum a_{i} u\left(x_{i}\right)=0$ in $N_{0}$ By flatness of $N$ there exists $c_{i j} \varepsilon A$ and $x_{j}^{\prime} \varepsilon N$ such that $\sum_{i} a_{i j} c_{i j}=0$ (for all $j$ ) and such that $u\left(x_{i}\right)=\sum_{j} c_{j} x_{j}^{i}$ (for all $i$ ). By the choice of $a_{1}$, $\ldots, a_{p}$ all the $c_{i j}$ must belong to Wh. Thus $u\left(x_{i}\right) \in M N$, in other words $\bar{u}\left(x_{i} \otimes 1\right)=0$. Since $\bar{u}$ is injective we get $x_{1} \varepsilon$ $m M$, hence $x \in m^{n+1} M$. Thus $u$ is injective and we get an exact sequence $0 \rightarrow M \rightarrow N \rightarrow N / u(M) \rightarrow 0$. From this and from the hypotheses $i t$ follows that $\operatorname{Tor}_{1}^{A}(k, N / u(M))=0$, which shows the flatness of $N / u(M)$ by Th. 49 .

(20.F) COROLLARY 1. Let A be a noetherian ring, B a noetherian A-algebra, $M$ a finite $B$-module and $f \in B$. Suppose that (i) $M$ is A-flat, and (ii) for each maximal ideal $P$ of $B$, the element $f$ is $M /(P \cap A) M$-regular, Then $f$ is M-regular and M/fM is A-flat.

Proof. If $K$ denotes the kernel of $M \rightarrow M$, then $K=0$ iff $\mathrm{K}_{\mathrm{p}}=0$ for al1 maximal ideals $\mathrm{P}$ of $\mathrm{B}$. Similarly, by an obvious extension of (3.J), $M / f M$ is A-flat iff $M_{P} / f M_{P}$ is flat over $A_{(P \cap A)}$ for all maximal $P$. The assumptions are also stable under localization. So we may assume that $(A, 4, k)$ and $\left(B, H, k^{\prime}\right)$ are noetherian local rings and $A \rightarrow B$ is a local

homomorphism. Then the assertion follows from (20.E).

COROLLARY 2. Let $A$ be a noetherian ring and $B=A\left[X_{1}, \ldots, X_{n}\right]$ a polynomial ring over $A$. Let $f(X) \in B$ be such that its coefficients generate over A the unit ideal A. Then $f$ is not a zero-divisor of $B$, and $B / f B$ is $A-f l a t$.

(20.G) APPLICATION 3. Let $\mathrm{A} \rightarrow \mathrm{B} \rightarrow \mathrm{C}$ be local homomorphisms of noetherian local rings and M be a finite C-module. Suppose B is A-flat, Let $k$ denote the residue field of A. Then

$M$ is $B-f l a t \Leftrightarrow M$ is $A-f l$ at and $M \otimes_{A} k$ is $B \otimes_{A} k-f l a t .$ Proof. $(\Rightarrow)$ Trivial. $(\Leftarrow)$ Use the criterion (4) of Th.49.

For more applications of Th.49, cf. EGA $0_{\text {III }}(10.2)$.

\section{Fibres of Flat Morphisms}
(21. A) Let $\phi: A \rightarrow B$ be a homomorphism of noetherian rings;

let $P \varepsilon \operatorname{spec}(B), P=P \cap A$ and $K(p)=$ the residue field of $A_{p}$. Then the 'fibre over $p^{\prime}$ is $\operatorname{spec}\left(B \otimes_{A} K(p)\right)$, and 'the local

ring of $P$ on the fibre' is $B_{P} / P B_{P}=B_{P} \otimes_{A} K(P) \quad(c f . P .79)$.

Suppose $B$ is flat over A. Then we have
$$
\operatorname{dim}\left(B_{P}\right)=\operatorname{dim}\left(A_{p}\right)+\operatorname{dim}\left(B_{P} \otimes K(p)\right)
$$
by (13.B) Th. 19

(21. B) THEOREM 50. Let $(A, m, k)$ and $\left(B, n, k^{\prime}\right)$ be noetherian local rings, and let $A \rightarrow B$ a local homomorphism. Let $M$ be a finite A-module and $\mathrm{N}$ be a finite B-module which is A-flat.

Then we have
$$
\operatorname{depth}_{B}\left(M \otimes_{A} N\right)=\operatorname{depth} A_{A}+\operatorname{depth}_{B \otimes k}(N \otimes k) .
$$
Proof. Induction on $\mathrm{n}=\operatorname{depth} \mathrm{M}+\operatorname{depth} \mathrm{N} \otimes \mathrm{k}$.

Case 1: $n=0$. Then $M E A s s_{A}(M)$ and $M \in A_{B}(N \otimes k)$, and we know (p.58) that
$$
\text { Ass }_{B}\left(M \otimes_{A} N\right)=\bigcup_{p \in A_{A}}(M) A_{B}(N \dot{\otimes} / p) .
$$
Hence $M \in$ Ass $_{B}(M \otimes N)$, i.e. depth $\left.B \otimes N\right)=0$.

Case 2: depth $M>0$. Easy and left to the reader.

Case 3: depth $N \otimes k>0$. Take $y \in N$ which is N\&k-regular.

By (20.E) y is N-regular and $\mathrm{N} / \mathrm{yN}$ is A-flat. From the exact sequence $0 \rightarrow \mathrm{N} \rightarrow \mathrm{N} \rightarrow \mathrm{N} / \mathrm{yN} \rightarrow 0$ it then follows that

\includegraphics[max width=\textwidth]{2022_08_01_8d4eee36f1f42236b4f4g-085}

is exact. Putting $\bar{N}=N / y N$ we get $\operatorname{depth}(M \otimes N)-1=$

$\operatorname{depth}_{B}(M \otimes \bar{N})$, and depth $\operatorname{dok}_{\mathrm{k}}(\mathrm{N} \otimes \mathrm{k})-1=\operatorname{depth} \boldsymbol{B}_{\mathrm{k}}(\bar{N} \otimes k)$. From these and from the induction hypothesis on $\bar{N}$ we get the desired formula.

(21.C) COROLLARY 1. Let $A \rightarrow B$ be as above and suppose that:

$B$ is A-flat. Then we have

$\operatorname{depth} B=\operatorname{depth} A+\operatorname{depth} B \otimes k$

and

$B$ is $C . M . \Longleftrightarrow A$ and $B \otimes k$ are C.M..

COROLLARY 2. Let $A$ and $B$ be noetherian rings and $A \rightarrow B$ be a

faithfully flat homomorphism. Let $i$ be a positive integer.

Then (1) if B satisfies the condition $\left(S_{i}\right)$ of $(17 . I)$, so

does A;

(2) if A satisfies $\left(S_{i}\right)$ and if all fibres satisfy $\left(S_{i}\right)$

(i.e. $B \otimes K(p)$ satisfies $\left(S_{i}\right)$ for every $p \in \operatorname{Spec}(A)$ )

then B satisfies $\left(S_{i}\right)$

Proof. (1) Given $P \in \operatorname{Spec}(A)$, take $P \in \operatorname{Spec}(B)$ which is mini-

mal among prime ideals of $B$ lying over $p$, and put $k=k(p)$.

Then $\operatorname{dim} B_{P} \otimes k=\operatorname{depth} B_{P} \otimes k=0$, whence depth $B_{P}=\operatorname{depth} A_{n}$

and $\operatorname{dim} B_{P}=\operatorname{dim} A_{P}$. Therefore

$\operatorname{depth} A_{p}=\operatorname{depth} B_{P} \geqslant \inf \left(i, \operatorname{dim} B_{P}\right)=\inf \left(i, \operatorname{dim} A_{P}\right)$.

(2) Given $P \in \operatorname{Spec}(B)$, put $P=P \cap A$ and $k=k(p)$.

Then
$$
\begin{aligned}
\text { depth } B_{P} &=\operatorname{depth} A_{p}+\operatorname{depth}\left(B_{p} \otimes k\right) \\
& \geqslant \inf \left(i, \operatorname{dim} A_{p}\right)+\inf \left(i, \operatorname{dim} B_{P} \otimes k\right)
\end{aligned}
$$
$$
\begin{aligned}
&\geqslant \inf \left(i, \operatorname{dim} A_{p}+\operatorname{dim} B_{p} \otimes k\right) \\
&=\inf \left(i, \operatorname{dim} B_{p}\right) .
\end{aligned}
$$
(21.D) THEOREM 51. Let $(A, m, k)$ and $\left(B, w, k^{\prime}\right)$ be noetherian local rings and $\phi: A \rightarrow B$ a local homomorphism. Then:

(i) if $B$ is flat over $A$ and regular, then $A$ is regular.

(ii) if $\mathrm{dim} B=\operatorname{dim} A+d i m B \otimes k$ holds, and if $A$ and $B \otimes k=B / m B$ are regular, then $B$ is flat over $A$ and regular.

Proof. (i) Since a flat base change commutes with homology, we have $\operatorname{Tor}_{q}^{A}(k, k) \otimes_{A} B=\operatorname{Tor}_{q}^{B}(k \otimes B, k \otimes B)=0$ for $q>\operatorname{dim} B$. Since $B$ is faithfully flat over A this implies $\operatorname{Tor}_{q}^{A}(k, k)=0$, hence gl.dim A is finite, i.e. A is regular.

(ii) If $\left\{x_{1}, \ldots, x_{r}\right\}$ is a regular system of parameters of $A$ and if $\mathrm{y}_{1}, \ldots, \mathrm{y}_{s} \in M$ are such that their images form a regular system of parameters of $B / M B$, then $\left\{\phi\left(x_{1}\right), \ldots, \phi\left(x_{r}\right), y_{1}\right.$, $\left.\ldots, y_{s}\right\}$ generates $M$, and $r+s=\operatorname{dim} B$ by hypothesis. Thus $B$ is regular, To prove flatness it suffices, by the criterion (3') of $\mathrm{Th} .49$, to prove $\operatorname{Tor}_{1}^{\mathrm{A}}(\mathrm{k}, \mathrm{B})=0$. The Koszul complex $k_{0}\left(x_{1}, \ldots . x_{r} ; A\right)$ is a free resolution of the A-module $k$, hence we have $\operatorname{Tor}_{1}^{A}(k, B)=H_{1}\left(K \cdot(x ; A) \otimes_{A} B\right)=H_{1}(K \cdot(x ; B))$. Since the sequence $\phi\left(x_{1}\right), \ldots, \phi\left(x_{r}\right)$ is a part of a regular system of parameters of B it is a B-regular sequence. Hence we have $\mathrm{H}_{i}\left(K_{0}(\underline{\mathrm{x}} ; \mathrm{B})\right)=0$ for all $\mathrm{I}>0$, and we are done. Remark. Even if $B$ is regular and $A-f l a t$, the local ring $B \otimes k$ on the fibre is not necessarily regular. Example: put $k=$ a field, $k[x, y]=k[x, y] /\left((x-1)^{2}+Y^{2}-1\right), B=k[x, y](x, y)$, $\mathrm{A}=\mathrm{k}[\mathrm{x}]_{(\mathrm{x})}$ and $m=x \mathrm{x}$. Then $\mathrm{B} \otimes(\mathrm{A} / \mathrm{m}) \simeq \mathrm{k}[\mathrm{Y}] /\left(\mathrm{Y}^{2}\right)$ has nilpotent elements.

(21.E) COROLLARY. Let $A$ and $B$ be noetherian rings and $A \rightarrow B$

a faithfully flat homomorphism. Then

i) if B satisfies $\left(R_{i}\right)$, so does A;

ii) if $A$ and all fibres $B \otimes K(P)(P \in \operatorname{Spec}(A))$ satisfy $\left(R_{i}\right)$, then B satisfies $\left(R_{i}\right)$;

iii) if $B$ is normal (resp. C.M., resp. reduced), so is A. Conversely, if A and all fibres are normal (resp. ...) then

$B$ is normal (resp. ...).

Proof. i) and ii) are immediate from Th.51. As for iii), it is enough to recall (17.I) that normal $\Leftrightarrow\left(\mathrm{R}_{1}\right)+\left(\mathrm{S}_{2}\right)$, $\underline{C . M .} \Leftrightarrow$ all $\left(\mathrm{S}_{i}\right)$, and reduced $\Leftrightarrow\left(R_{0}\right)+\left(S_{1}\right)$.

\section{Theorem of Generic Flatness}
(22. A) LEMMA 1. Let $A$ be a noetherian domain, $B$ an A-algebra of finite type and $M$ a finite $B-$ module. Then there exists $0 \neq E \varepsilon A$ such that $M_{f}=M \otimes_{A} A_{f}$ is $A_{f}$-free (where $A_{f}$ is the localization of $A$ with respect to $\left.\left\{1, f, f^{2}, \ldots\right\}\right)$. Proof. We may suppose that $M \neq 0$. Then, by (7.E) Th.10 there exists a chain of submodules $0=M_{0} \subset M_{1} \subset \ldots \subset M_{n}=M$ with $M_{i} / M_{i-1} \simeq B / P_{i}, P_{i} \varepsilon \operatorname{Spec}(B)$. Since an extension of free modules is again free, it suffices to prove the lemma for the case that $B$ is a domain and $M=B$. If the canonical map $A \rightarrow$ B has a non-trivial kernel then $B_{f}=0$ for any non-zero element $f$ of the kernel, and our assertion is trivial. So we may assume that $A$ is a subring of the domain $B$. Let $K$ be the quotient field of $A \cdot$ Then $B \otimes K=B K$ is a domain (contained in the quotient field of $B$ ) and is finitely generated as an algebra over $K$. Hence $\mathrm{dim} B K=t r . d e g_{K} B K<\infty .$ Put theorem $(14 . G)$, the ring BK contains $n$ algebraically independent elements $y_{1}, \ldots, y_{n}$ such that $B K$ is Integral over $K[y]$ We may assume that $y_{i} \varepsilon B$. Since $B$ is finitely generated over A there exists $0 f g E A$ such that $B_{g}=B \cdot A$ is integral over $\mathrm{g}^{[y]}$. Replacing $A$ and $B$ by $A_{g}$ and $B_{g}$ respectively, and puttig $C=A[y]$, we have that $B$ is a finite module over the polynomial ring $c$. Let $b_{1}, \ldots, b_{m}$ be a maximal set of linearly
$$
0 \rightarrow C^{m} \rightarrow B \rightarrow B^{\top} \rightarrow 0
$$
where $B^{\prime}$ is a finitely generated torsion C-module. Since

$(\mathrm{C} / p) \otimes \mathrm{K}=\mathrm{CK} / \mathrm{pK}$ has a smaller dimension than $\mathrm{n}=\mathrm{dim} \mathrm{CK}$ for any non-zero prime ideal $p$ of $C$, there exists by the induction assumption a non-zero element $f$ of $A$ such that $B^{\prime} f{ }^{i}$ A $_{f}$-free. Since $C_{f}^{m}=\left(A_{f}\left[y_{1}, \ldots, y_{n}\right]\right)^{m}$ is also $A_{f}$-free, the localization $\mathrm{B}_{\mathrm{f}}$ is $\mathrm{A}_{\mathrm{f}}$-free.

Q.E.D.

An important special case of the Lemma is the following THEOREM 52. Let A be a noetherian domain and B an A-algebra of finite type. Suppose that the canonical map $\phi: A \rightarrow B$ is injective. Then there exists $0 \neq f \varepsilon$ A such that $B_{f}$ is $A_{f}-$ free and $\neq 0 .$ Thus, the map $\phi: \operatorname{Spec}(B) \rightarrow \operatorname{Spec}(A)$ is faithfully flat over the non-empty open set $D(f)=\operatorname{Spec}(A)-V(f)$ of $\operatorname{spec}(A)$, that is, ${ }_{\phi}^{-1}(D(f)) \rightarrow D(f)$ is faithfully flat.

(22.B) LEMMA 2. Let $B$ be a noetherian ring and let $U$ be a subset of $\operatorname{Spec}(B)$. Then $U$ is open iff the following conditions are satisfied.

(1) $U$ is stable under generalization,

(2) if $P \& U$ then $U$ contains a non-empty open set of the irreducible closed set $V(P)$.

Proof. Assume the conditions, and let $F$ be the complement of $U$ and $P_{i}(1 \leqslant i \leqslant s)$ be the generic points of the irreducible components of the closure $\bar{F}$ of $F$. Then (2) implies that $P_{i}$ cannot lie in $U$. Hence $P_{i} \varepsilon F$, and so $F=\bar{F}$ by (1). O.E.D. THEOREM 53. Let $A$ be a noetherian ring, $B$ an A-algebra of finite type and $M$ a finite $B-m o d u l e . ~ P u t \quad U=\{P \in \operatorname{Spec}(B) \mid$ $M_{P}$ is flat over $\left.A\right\}$. Then $U$ is open in $\operatorname{Spec}(B) .$

Remark 1. The set U may be empty.

Remark 2. It follows from (6.I) Th.8 that a flat morphism of finite type between noetherian preschemes is an open map.

Therefore the image of $U$ in $\operatorname{Spec}(A)$ is open in $\operatorname{Spec}(A)$.

Proof. Let $P \supset Q$ be prime ideals of $B$ with $M_{P}$ flat over A. For any A-module $N$ we have $N \otimes_{A} M_{Q}=\left(N \otimes_{A} M_{P}\right) \otimes_{B} B_{Q}$, therefore $M_{Q}$ is flat over A and the condition (1) of Lemma 2 is verified for U. As for the condition (2), let $P \varepsilon U$ and put $P=P \cap A$ and $\bar{A}=A / P$. Let $Q \in V(P)$. Then $p B C \operatorname{rad}\left(B_{Q}\right)$, so we can apply the local criterion of flatness that $M$ is flat over A if ${ }^{M} Q / M_{Q}$ is flat over $A$ and $\operatorname{Tor}_{1}(M, A)=0$. Applying Lemma 1 to $(\bar{A}, B / P B, M / P M)$ we see that there exists a neighborhood of $P$ in $V(p B)$ such that $M_{Q} / p M_{Q}$ is flat over A for each point $Q$ in it. On the other hand, since $0=\operatorname{Tor}_{1}^{A}\left(M_{P}, \bar{A}\right)$ $=\operatorname{Tor}_{1}(M, \bar{A}) \bigotimes_{B}{ }_{P}$ and since $\operatorname{Tor}_{1}(M, \bar{A})$ is a finite B-module, there exists a neighborhood of $P$ in $\operatorname{Spec}(B)$ in which $\operatorname{Tor}_{1}\left(M_{Q}, \bar{A}\right)=0$. Therefore there exists a non-empty open set of $V(P)$ in which $M_{Q}$ is A-flat for all points $Q$, in other words the set $U$ in question contains a non-empty open set of $V(P)$.

Thus the theorem is proved. (22.C) Let $\underline{P}$ be a property on noetherian local rings and let

$\underline{P}(A)$ denote the set $\left\{p \in \operatorname{spec}(A) \mid A_{p}\right.$ has the property $\underline{P}\}$.

CHAPTER 9. Completion

Consider the following statement.

(NC) If $A$ is a noetherian ring and if, for every $P \varepsilon$

$\operatorname{Spec}(A), \underline{P}(A / p)$ contains a non-empty open set of

$\operatorname{Spec}(A / \rho)$, then $\underline{P}(A)$ is open in $\operatorname{Spec}(A) .$

While Lemma 2 of $(22 . B)$ was topological, (NC) is ring-theoreti-

cal and its validity of course depends on $\underline{P}$. Both are inven-

tions of Nagata ( $\mathrm{NC}$ means Nagata criterion), who proved ( $\mathrm{NC})$

for $\underline{P}=$ regular $($ cf. p.245). As an example we prove

Proof. $C M(A)$ is stable under generalization. We will prove (2) of Lemma 2. If $P \in C M(A)$ and ht $P=n$, we can take an $A_{p}$-regular sequence $y_{1}, \ldots, y_{n}$ from $P$. Replacing A by $A_{a}$ with suitable a $\varepsilon A-P$, we may assume that $y_{1}, \ldots, y_{n}$ is an Aregular sequence and $I=\sum y_{i} A$ is a P-primary idea1. Then for $Q \varepsilon V(P), A_{0}$ is $C M$ iff $A_{0} / I A_{0}$ is so. Hence we can replace $A$ by $A / I$ and assume that $(0)$ is $P$-primary. So we have $P^{r}=0$ for some $r>0$. Since $\mathrm{P}^{i} / \mathrm{P}^{i+1}$ is a finite $\mathrm{A} / \mathrm{P}$-module for each $0 \leq i<r$, we may assume (replacing A by some $A_{a}$ ) that the $\mathrm{P}^{i} / \mathrm{P}^{i+1}$ are free $A / P$-modules. Then it is easy to see that

a sequence $x_{1}, \ldots, x_{n} \varepsilon A$ is A-regular if it is $A / P$-regular. By the hypothesis of $(N C)$ we may assume further that $A / P$ is CM. Then depth $A_{0}=\operatorname{depth} A_{0} / P A_{0}=\operatorname{dim} A_{0} / P A_{0}=\operatorname{dim} A_{0}$,

hence $Q \in C M(A)$.

EXERCISE. If $A$ is a homomorphic image of a CM ring, then $\mathrm{CM}(\mathrm{A})$ is open.

\section{Completion}
(23. A) Let $A$ be a ring, and let $F$ be a set of ideals of $A$ such that for any two ideals $I_{1}, I_{2} \in F$ there exists $I_{3} \varepsilon F$ contained in $I_{1} \cap I_{2}$. Then one can define a topology on A by taking $\{x+I \mid I E F\}$ as a fundamenta1 system of neighborhoods of $x$ for each $x \in A$. One sees immediately that in this topology the addition, the multiplication and the map $x \mapsto-x$ are continuous; in other words $A$ is a topological ring. A topology on a ring obtained in this manner is called a linear topology. When $M$ is an A-module one defined a linear topology on $M$ in the same way, the only difference being that 'ideals' are replaced by "submodules " Let $M=\left\{M_{\lambda}\right\}$ be a set of submodules which defines the topology, Then $M$ is separated (i.e.

PROPOSITION. (NC) is valid for $\underline{p}=C M$. iff $\cap\left(M_{\lambda}+N\right)=N$, the left hand side being the closure of $N$.

(23.B) Let $A$ be a ring, $M$ an A-module linearly topologized by a set of submodules $\left\{M_{\lambda}\right\}$ and $N$ a submodule of $M$. Let $\bar{M}_{\lambda}$ be the image of $M_{\lambda}$ in $M / N$. Then the linear topology on $M / N$ defined by $\left\{\bar{M}_{\lambda}\right\}$ is nothing but the quotient topology of the topology on $M$, as one can easily check. When we say "the quotient module $M / N$ ", we shall always mean the module $M / N$ with the quotient topology. It is separated iff $\mathrm{N}$ is closed.

(23.C) For simplicity, we shall consider in the following only such linear topologies that are defined by a countable set of submodules. This is equivalent to saying that the topology satisfies the first axiom of countability. If a linear topology on $M$ is defined by $\left\{M_{1}, M_{2}, \ldots\right\}$, then the set $\left\{M_{1}, M_{1} \cap M_{2}, M_{1} \cap M_{2} \cap M_{3}, \ldots\right\}$ defines the same topology. Therefore we can assume without loss of generality that $M_{1} \geq$ $M_{2} \supseteq M_{3} \supseteq \ldots$ (in other words, the topology is defined by a filtration of $M$, cf. p.67). A sequence $\left(x_{n}\right)$ of elements of $M$ is a Cauchy sequence if, for every open submodule $N$ of $M$,

there exists an integer $n_{0}$ such that

(*)
$$
x_{n}-x_{m} \in N \text { for all } n, m>n_{0} \text {. }
$$
Since $N$ is a submodule, the condition (*) can also be written as $x_{n+1}-x_{n} \in N$ for all $n>n_{0}$. Therefore a sequence $\left(x_{n}\right)$ is Caucy iff $x_{n+1}-x_{n}$ converges to zero when $n$ tends to infinity. A continuous homomorphism of linearly topologized modules maps Cauchy sequences into Cauchy sequences. A topological A-module $M$ is said to be complete if every Cauchy sequence in $M$ has a limit in $M$. Note that the limit of a Cauchy sequence is not uniquely determined if $M$ is not separated.

(23.D) PROPOSITION. Let $A$ be a ring and let $M$ be an A-module with a linear topology defined by a filtration $\mathrm{M}_{1} \geq \mathrm{M}_{2} \supseteq$. ; let $\mathrm{N}$ be a submodule of $M$. If $M$ is complete, then the quotient module $M / N$ is also complete.

Proof. Let $\left(\bar{x}_{n}\right)$ be a Cauchy sequence in M/N. For each $\vec{x}_{n}$ choose a pre-image $x_{n}$ in $M$. We have $\bar{x}_{n+1}-\bar{x}_{n} \varepsilon \bar{M}_{i(n)}$ with $i(n) \rightarrow \infty$, therefore we can write
$$
x_{n+1}-x_{n}=y_{n}+z_{n}, \quad y_{n} \in M_{i(n)}, \quad z_{n} \in N,
$$
and the sequence $\left(y_{n}\right)$ converges to zero in $M$. Let $s \varepsilon M$ be a limit of the Cauchy sequence $x_{1}, x_{1}+y_{1}, x_{1}+y_{1}+y_{2}, \ldots$; then its image $\bar{s}$ in $M / N$ is a limit of the sequence $\left(\bar{x}_{n}\right)$.

Thus $M / N$ is complete.

(23.E) Let $A$ be a ring, I an ideal and $M$ an A-module. The set of submodules $\left\{I^{\mathrm{M}} \mid \mathrm{n}=1,2, \ldots\right\}$ defines the I-adic topology of M. We also say that the topology is adic and that I is an ideal of definition for the topology. Clearly, any ideal $J$ such that $I^{\mathrm{n}} \subseteq \mathrm{J}$ and $\mathrm{J}^{\mathrm{m}} \subseteq \mathrm{I}$ for some $\mathrm{n}, \mathrm{m}>0$ is an ideal of definition for the same topology. When both A and $M$ are I-adically topologized, the map $(a, x) \longmapsto a x(a \varepsilon A, x \varepsilon M)$ is a continuous map from $A \times M$ to $M .$ When $A$ is a semi-local ring with $\operatorname{rad}(A)=M$ then it is viewed as an $M$-adic topological ring, unless the contrary is explicitly stated.

(23.F) Let $k$ be a ring, and let $A$ and $B$ be $k$-algebras with linear topology defined by $m=\left\{\mathrm{I}_{\mathrm{n}}\right\}$ and $\eta=\left\{\mathrm{J}_{\mathrm{m}}\right\}$ respectively. Put $\mathrm{C}=\mathrm{A} \otimes_{\mathrm{k}}$. Then a linear topology can be defined on $\mathrm{C}$ by means of the set of ideals $\left\{I_{n} C+J_{m} C\right\}_{n, m}$. This is called the topology of tensor product. If A has the I-adic topology and B the J-adic topology, where I (resp. J) is an ideal of A (resp. B), then the topology of tensor product on $\mathrm{C}$ is the (IC $+$ JC)-adic topology, for we have

$(I C+J C)^{n+m-1} \subseteq I^{n} C+J^{m} C$ and $I^{n} C+J^{n} C \subseteq(I C+J C)^{n}$

(23.G) PROPOSITION. Let $A$ be a ring and I an ideal of A.

Suppose that A is complete and separated for the I-adic topology. Then any element of the form $u+x$, where $u$ is a unit

in $A$ and $x$ is an element of $I$, is a unit in A. The ideal I

is contained in the Jacobson radical of $A$. Proof. We have $u+x=u(1-y)$, where $y=-u^{-1} x \in I$. The infinite series $1+y+y^{2}+\ldots$ converges in $A$, and we have $(1-y)\left(1+y+y^{2}+\ldots\right)=1$ since $A$ is separated. Thus 1 - $y$ (hence also $u+x$ ) is a unit. The second assertion is easy.

(23.H) Let A be a ring and M a linearly topologized A-module. The completion of $M$ is, by definition, an A-module $M^{*}$ with a complete separated linear topology, together with a continuous homomorphism $\varphi: M \rightarrow M^{*}$, having the following universal mapping property: for any A-module $M^{\prime}$ with a complete separated linear topology and for any contimuous homomorphism $f: M \rightarrow M^{r}$, there exists a unique continuous homomorphism $I *$ : $M * \rightarrow M^{\prime}$ satisfying $f * g=f$. The completion of $M$ exists, and is unique up to isomorphisms. In fact the uniqueness is clear from the definition, while the existence can be proved by several methods. First of all, note that, if $\mathrm{K}$ is the intersection of all open submodules of $M$, the canonical map $\rho: M \rightarrow M *$ must Factor through $M=M / K$ (which is called the Hausdorffization of $M$ ) and hence $M$ and $M^{h}$ have the same completion.

\begin{enumerate}
  \item Take the completion of the uniform space $M^{h}$ and call it $M^{*}$. The topological space $M^{*}$ becomes a linearly topologized A-module by extending the A-module structure of $M^{h}$ to $M^{*}$ by uniform continuity. The universal mapping property of $M^{*}$ follows immediately, continuous homomorphisms $f: M+M^{\prime}$ being uniformly continuous.

  \item Let $W$ be the set of Cauchy sequences in $M$, and make it an A-module by defining the addition and the scalar multiplication termwise. Then the set $W_{0}$ of the null sequences (i.e. the sequences which have zero as a limit) is a submodule of W. Put $M^{*}=W / W_{0}$, and define the canonical map $\varphi: M \rightarrow M^{*}$ in the obvious way. For any open submodule $N$ of $M$, let $N$ denote the image in $M^{*}$ of the set of Cauchy sequences in $N$. Then $\hat{\mathrm{N}}$ is a submodule of $\mathrm{M}^{*}$. The set of all such $\hat{\mathrm{N}}$ defines a linear topology in $M^{*}$, and $\hat{N}$ is the closure of $\varphi(N)$ in this topology. It is easy to see that $M^{*}$ is complete and separated and has the universal mapping property.

  \item Denote by $M^{*}$ the inverse limit of the discrete $A$-modules $M / M_{n}$, where $\left(M_{n}\right)$ is a filtration of $M$ defining the topology, and put the inverse limit topology (i.e. the topology as a subspace of the product space $T\left(M / M_{n}\right)$ on it. Let $\varphi: M \rightarrow M^{*}$ be defined in' the obvious way, and let $M_{n}^{*}$ denote the closure of $\varphi\left(M_{n}\right)$ in $M^{*}$. Then $M_{n}^{*}$ consists of those vectors of $M^{*}$ of which the first $n$ coordinates are zero, and the set of submodules $\left\{M_{n} * \mid \mathrm{n}=1,2, \ldots\right\}$ defines a complete separated linear topology on $M^{*}$. Let $M^{\prime}$ be an A-module with a complete separated linear topology and $\mathrm{f}: \mathrm{M} \rightarrow \mathrm{M}^{\prime}$ a continuous homomorphism. For any element $x^{*}=\left(\bar{x}_{1}, \bar{x}_{2}, \ldots\right)$ of $M *\left(\bar{x}_{n} \varepsilon M / M_{n}\right)$, choose a pre-image $x_{n}$ of $\vec{x}_{n}$ in $M$ for each $n$. Then the sequence $x_{1}$, $x_{2}, \ldots$ is a Cauchy sequence in $M$, hence the image sequence $f\left(x_{1}\right), f\left(x_{2}\right), \ldots$ is a Cauchy sequence in $M^{\prime} \cdot$ Therefore $\lim _{n \rightarrow \infty} f\left(x_{n}\right)$ exists in $M^{\prime}$, and this limit is easily seen to be (ndependent of the choice of the pre-inages $x^{n}$ Putting if $M$ is separated.

\end{enumerate}
(23. I) If $f: M \rightarrow N$ is a continuous homomorphism of linearly topologized A-modules $M$ and $N$, and if $\varphi_{M}: M \rightarrow M^{*}$ and $\varphi_{N}$ : $N \rightarrow N^{*}$ are the canonical homomorphisms into the completions, then there exists a unique continuous homomorphism $f *: M^{*} \rightarrow$ $N^{*}$ with $\varphi_{N^{f}}=f * \varphi_{M}$; this is a formal consequence of the definition. The map $f$ t is called the completion of $f$. Taking completions is, therefore, an additive covariant functor.

PROPOSITION. Let $M$ be a linearly topologized A-module, N a submodule and $\varphi: M \rightarrow M *$ the canonical map to the completion. Then (i) the completion of $N$ (for the topology induced from $M$ ) is the closure $\overline{\varphi(N)}$ of $\varphi(N)$ in $M^{*}$, and (ii) the quotient module $M^{*} / \varphi(N)$ is the completion of the quotient module $M / N$.

Proof. (i) This follows, e.g., from the second construction of completion in $(23 . \mathrm{H})$.

(ii) The quotient module $M * \longdiv { \Phi ( N ) }$ is separated by (23.B), and complete by $(23 . D)$. The canonical map $M \rightarrow M$ induces a $\operatorname{map} M / N \rightarrow M * / \overline{\varphi(N)}$, and the universal property of this map is easily proved by a formal argument.

Remark 1. Taking $N=M$ we see that $\varphi(M)$ is dense in $M^{*} .$

$\underline{\text { Remark 2. If }} \mathrm{N}$ is an open submodule of $M$ then $M / N$ is $d i s-$ crete, hence complete and separated. Thus $M / N \simeq M * / \varphi(N)$.

THEOREM 54. Let A be a noetherian ring and I an ideal. Let $0 \rightarrow \mathrm{L} \rightarrow \mathrm{M} \rightarrow \mathrm{N} \rightarrow 0$ be an exact sequence of finite A-modules, and let * denote the I-adic completion. Then the sequence $0 \rightarrow \mathrm{L}^{*} \rightarrow \mathrm{M}^{*} \rightarrow \mathrm{N}^{*} \rightarrow 0$ is also exact.

Proof. By Artin-Rees theorem, the I-adic topology of I coincides with the topology induced by the I-adic topology of $M$. Therefore the assertion follows from the preceding proposition.

(23.J) Let A be a linearly topologized ring. Then the completion $A^{*}$ of $A$ is not only an A-module but also a ring, the multiplication in A being extended to A* by continuity. If $\varphi: A \rightarrow A^{*}$ is the canonical map and $I$ is an ideal of $A$, then the closure $\overline{\varphi(I)}$ of $\varphi(I)$ in $A^{*}$ is an ideal of $A^{*}$. Thus $A^{*}$ is a linearly topologized ring. Example: Let $k$ be a ring. Put $A=k\left[X_{1}, \ldots, X_{n}\right]$ and $I=\sum_{1}^{n} A_{1} .$ Then the ring of formal power series $k\left[\left[x_{1}, \ldots, x_{n}\right]\right]$ is the I-adic completion of $A$.

(23.K) Let $A$ be a ring, I a finitely generated ideal of $A$, $A^{*}$ the I-adic completion of $A$ and $\varphi: A \rightarrow A^{*}$ the canonical map. Then, for any element $x^{*}$ of $A^{*}$ there exists a Cauchy sequence $\left(x_{n}\right)=\left(x_{0}, x_{1}, \ldots\right)$ in A such that $x^{*}=11 m \varphi\left(x_{n}\right)$. Replacing $\left(x_{n}\right)$ by a suitable subsequence we may assume that $x_{n 41}-x_{n} \in I \quad(n=0,1,2, \ldots)$. Let $a_{1}, \ldots, a_{m}$ generate $I$, and put $a_{i}^{\prime}=\varphi\left(a_{i}\right)$. Then $x_{n+1}-x_{n}$ is a homogeneous polynomial of degree $n$ in $a_{1}, \ldots, a_{m}$. Thus $x^{*}=\varphi\left(x_{0}\right)+\sum_{n=0}^{\infty} \varphi\left(x_{n+1}\right.$ $\left.-x_{n}\right)$ has a power series expansion in $a_{1}^{\prime}, \ldots, a_{m}^{\prime}$. with coefficients in $\varphi(A)$. Consider the formal power series ring $A[[X]]$ $=A\left[\left[x_{1}, \ldots, X_{m}\right]\right] ;$ let $u(X) \varepsilon A[[X]]$, and let $\bar{u}(X)$ denote the power series obtained by applying $\varphi$ to the coefficients of $u(X)$. Since $A^{*}$ is complete and separated, the series $\bar{u}\left(a^{\prime}\right)$ $u\left(a_{1}, \ldots, a_{m}^{\prime}\right)$ converges in $A^{*} .$ The map $u(X) \mapsto \bar{u}\left(a^{\prime}\right) d e-$ Fines a surjective homomorphism $A[[X] \rightarrow A *$. Thus $A * \simeq$ $A[[X]] / J$ with some ideal $J$ of $A[[X]]$. As a consequence, $A^{*}$ is noetherian if $A$ is so.

(23.L) Let $A$ be a ring, I an ideal and $M$ an A-module, Let * denote the I-adic completion. Then $M *$ is an A*-module in a natural way, therefore there exists a canonical map $M \otimes A^{A} *$ $\rightarrow M^{*}$

THEOREM 55. When $A$ is noetherian and $M$ is finite over $A$, the map $M_{A^{*}} \rightarrow M^{*}$ is an isomorphism.

Proof. Take an exact sequence of $A$-modules $A^{p} \stackrel{f}{\rightarrow} A^{q} \stackrel{g}{\rightarrow} M \rightarrow 0 .$

Since completion commutes with direct sum, we get a commutative diagram

\includegraphics[max width=\textwidth]{2022_08_01_8d4eee36f1f42236b4f4g-094}

where the vertical arrows $v_{i}$ are the canonical maps and the horizontal sequences are exact by the right-exactness of tensor product and by Th.54. Since $v_{1}$ and $v_{2}$ are isomorphisms $v_{3}$ is also an isomorphism by the Five-Lemma.

COROLLARY 1. Let A be a noetherian ring and I an ideal of A.

Then the I-adic completion $A^{*}$ of $A$ is flat over A.

COROLLARY 2. Let A and I be as above and assume that A is I-adically complete and separated. Let $M$ be a finite A-module.

Then $M$ is complete and separated, and any submodule $N$ of $M$ is closed in $M$, for the I-adic topology. Proof. Since $A=A^{*}$ we have $M^{*}=M \otimes A *=M$, i.e. $M$ is its own completion. Similarly, a submodule $\mathrm{N}$ is complete in the I-adic topology, which coincides with the induced topology by Artin-Rees. Since a complete subspace of $M$ is necessarily closed, we are done.

COROLLARY 3. Let A be a noetherian ring, $M$ a finite A-module, $N$ a submodule of $M$ and $I$ an ideal of $A$. Let $q: M \rightarrow M^{*}$ be the canonical map to the I-adic completion $M^{*}$. Then we have

\includegraphics[max width=\textwidth]{2022_08_01_8d4eee36f1f42236b4f4g-094(1)}

Proof. Immediate from Th.54 and Th.55.

COROLLARY 4. Let $A$ and I be as in Cor.3. Then the topology of the I-adic completion $A^{*}$ of $A$ is the IA*-adic topology.

Proof. By construction, the topology of $A^{*}$ is defined by the ideals $\left(\varphi\left(I^{n}\right)\right.$ in $\left.A *\right)=I^{n} A *=\left(I A^{*}\right)^{n}$.

COROLLARY 5. Let $A, I$ and $A *$ be as above and suppose that $I=\sum_{1}^{m} a_{1} A . \quad$ Then $\quad A * \simeq A\left[\left[x_{1}, \ldots, x_{m}\right]\right] /\left(x_{1}-a_{1}, \ldots, x_{n}-a_{m}\right)$.

Proof. Put $B=A\left[X_{1}, \ldots, X_{m}\right], I^{\prime}=\sum X_{i} B$ and $J=\sum\left(X_{i}-a_{i}\right) B$. Then $B / J \simeq A$, and the $I^{\prime}$-adic topology on the $B-a 1 g e b r a ~ B / J$ corresponds to the I-adic topology on A. Denoting the I'- adic completion by $\Lambda$, we thus obtain
$$
A^{*} \simeq(B / J)^{\wedge}=\hat{B} / \hat{J}=\hat{B} / J \hat{B}=A\left[\left[X_{1}, \ldots X_{m}\right]\right] /\left(X_{1}-a_{1}, \ldots, X_{m}-a_{m}\right) .
$$

\section{Zariski Rings}
(24.A) DEFINITION. A Zariski ring is a noetherian ring equipped with an adic topology, such that every ideal is closed in it.

THEOREM 56. Let A be a noetherian ring with an adic topology, and let I be an ideal of definition. Then the following are equivalent.

(1) A is a Zariski ring;

(2) I $\subseteq \operatorname{rad}(\mathrm{A})$

(3) every finite A-module $M$ is separated in the I-adic

\section{topology}
(4) in every finite A-module M, every submodule is closed in the I-adic topology;

(5) the completion $A^{*}$ of $A$ is faithfully flat over A.

Proof. $(1) \rightarrow(2):$ Suppose that a maximal ideal $m$ does not contain I. Then $I^{n}$. 1 for all $n>0$, so that $M L+I^{n}=A$ and $\bigcap_{n}\left(M+I^{n}\right)=A \neq 111$. Therefore 112 is not closed, contradiction. (2) $\rightarrow(3):$ By the intersection theorem (11.D).

$(3) \Rightarrow(4):$ If $N$ is a submodule of $M$, then $M / N$ is separated by assumption so that $N$ is closed in $M . \quad(4) \Rightarrow(1) \operatorname{Trivial}$. $(2) \Rightarrow(5)$ Let $w$ be a maximal ideal of $A .$ Then $H I P I$, hence 111 is open in $A$ and so $A^{*} / 111 A^{*} \simeq A / A 11$. Thus in $A^{*} \neq A^{*}$. Since $A^{*}$ is flat over A by (23.L) Cor.l, this implies by (4.A) Th.2 that $A$ * is $f, f$, over A.

$(5) \Rightarrow(2)$ If $11 t$ is a maximal ideal of A then there exists, by assumption, a maximal ideal $\$ 1^{\prime}$ of $A^{*}$ lying over 14 . Since IA* $\subseteq 112^{\prime}$ by $(23 . G)$, we have $I \subseteq I A * \cap A \subseteq M H^{\prime} \cap A=M M$, Q.E.D.

COROLLARY. Let $A$ be a Zariski ring and $A *$ its completion. Then (1) $A$ is a subring of $A *$, and (2) the map $M \mapsto M A^{*}$ is a bijection from the set $\Omega(A)$ of all maximal ideals in A to $\Omega\left(A^{*}\right)$, and we have $A / M \simeq A * / M^{*} A^{*}$ and $M A * \wedge A=M$.

(24.B) A noetherian semi-local ring is a Zariski ring. A noetherian ring with an adic topology which is complete and separated is also a Zariski ring.

Let A be an arbitrary noetherian ring and I a proper ideal of A. Put $S=1+I=\{1+x \mid x \varepsilon I\}, A^{\prime}=S^{-1} A$ and $I^{\prime}=S^{-1} I$. Then all elements of $1+I^{\prime}$ are invertible in $A^{\prime}$, and so I' $\subseteq r a d\left(A^{\prime}\right)$. We equip A with the I-adic topology and A' with the I'-adic (or what is the same, the I-adic) topology. Then the canonical map $\psi: A \rightarrow A^{\prime}$ is continuous, and has the universal mapping property for continuous homomorphisms from A to Zariski rings. In fact, if $f: A \rightarrow B$ is such a homomorphism and if $J$ is an ideal of definition for $B$, then $f\left(I^{n}\right)$ $\subseteq J \subseteq \operatorname{rad}(B)$ for some $n$, hence $f(I) \subseteq \operatorname{rad}(B)$ and the elements of $f(S)$ are invertible in $B$. Therefore $f$ factors through $A^{\prime}$. In particular, the canonical map $\mathrm{A} \rightarrow \mathrm{A}^{*}$ of $\mathrm{A}$ into the completion $A^{*}$ of $A$ factors through $A^{\top}$, and it follows immediately that $\mathrm{A}^{*}$ is also the completion of $\mathrm{A}^{\prime}$.

For a prime ideal $p$ of $A$, we have $p \cap S=\emptyset$ iff $p+I$ $\neq(1)$, i.e. iff $V(p) \cap V(I) \neq \emptyset$. The localization $A \rightarrow A^{\prime}$ has, geometrically, the effect of considering only the "subvarieties" of Spec(A) which intersect the closed set $\mathrm{V}(\mathrm{I})$. Since $A^{*}$ is faithfully flat over $A^{\top}$, the set $\{P \varepsilon \operatorname{Spec}(A) \mid$ $p+I \neq(1)\}(\approx \operatorname{Spec}(\mathrm{A}))$ is also the image of $\operatorname{spec}\left(A^{*}\right)$ in $\operatorname{Spec}(\mathrm{A})$. The set of the maximal ideals of $\mathrm{A}^{*}$ (resp. the prime ideals of $A^{*}$ containing $\left.I^{*}\right)$ is in a natural $1-1$ correspondence with the set of the maximal ideals (resp. prime ideals) of A containing I.

(24.C) Let A be a semi-local ring and $m_{1}, \ldots, M_{r}$ be its

maximal ideals. Put $A_{i}=A_{M_{i}}, M_{i}^{\prime}=m_{i} A_{i}(i=1, \ldots, r)$, and $M=\operatorname{rad}(\mathrm{A})=M m_{1} \ldots M_{r}$. Then $m^{\mathrm{n}}=\Pi_{m_{i}^{n}}^{\mathrm{n}}=\bigcap m_{i}^{\mathrm{n}}$, hence $\mathrm{A} / \mathrm{m}^{\mathrm{n}}=\mathrm{A} / \mathrm{m}_{1}^{\mathrm{n}} \times \ldots \times \mathrm{A} / \mathrm{m}_{\mathrm{r}}^{\mathrm{n}}$ by (1.C). Moreover, $\mathrm{A} / m_{i}^{\mathrm{n}}$ $\simeq A_{i} / M H_{1}^{1^{n}}$ as $A / M_{i}^{n}$ is a local ring. Therefore $\mathrm{A}^{*}=1 \mathrm{im} \mathrm{A} / \mathrm{m}^{\mathrm{n}}=\mathrm{A}_{1} * \times \ldots \times \mathrm{A}_{\mathrm{r}}^{*}$

(24.D) Let $(A, M)$ be a noetherian local ring and $A^{*}$ its completion. Then $A / m^{n} \simeq A^{*} / m^{n} A *$ for all $n>0$, hence $m^{n} / m^{n+1} \simeq m^{n} A * m^{n+1} A *$ and $g r(A) \simeq \operatorname{gr}(A *) .$ It follows that i) $\operatorname{dim} A=\operatorname{dim} A^{*}$, and ii) $A$ is regular iff $A^{*}$ is so.

Next, let A be an arbitrary noetherian ring, I an ideal of $A$ and $A^{*}$ the I-adic completion of $A$. Let $p$ be a prime ideal of A containing $I$. Since $p$ is open in A, the ideal $p A^{*}$ $=p^{*}$ is open and prime in $A^{*}$ and $A / p^{\mathfrak{n}} \simeq A^{*} / p^{*^{n}}$ for all $n>0$. Localizing both sides with respect to $p / p^{n}$ and $p^{*} / p^{*}$ respectively, we get
$$
\mathrm{A}_{p} / p^{\mathrm{n}} \mathrm{A}_{p} \simeq \mathrm{A}^{*} p^{*} / p^{*^{\mathrm{n}} \mathrm{A}^{*}} p^{*}
$$
Therefore $\left(A_{p}\right)^{*}=\lim A_{p} / p^{n} A_{p} \simeq\left(A^{*} p^{*}\right) *$. Two local rings are said to be analytically isomorphic if their completions are isomorphic, Thus, if $p$ and $p^{*}$ are corresponding open prime ideals of $A$ and $A^{*}$, then the local rings $A_{P}$ and $A^{*} P^{*}$ are analytically isomorphic. Since all maximal ideals of $\mathrm{A}^{*}$ are open, it follows that

i') $\operatorname{dim} A^{*}=\sup \operatorname{dim} A_{p}$

ii') if $A_{p}$ is regular for every prime ideal $p$ containing I, then $A^{*}$ is regular.

As a corollary of ii'' $^{\prime}$ we have the following PROPOSITION. Let A be a regular noetherian ring. Then the ring of formal power series $A\left[\left[X_{1}, \ldots, X_{m}\right]\right]$ is also regular. PART 11

Proof. $\left.A[X]=A\left[X_{1}, \ldots, X_{m}\right]\right]$ is a regular $r$ ing by $(17 . J)$,

and $A[X]]$ is the $\sum X_{i} A[X]$-adic completion of $A[X]$.

CHAPTER 10. DERIVATION

(24.E) PROPOSITION. Let $A$ be a Zariski ring and $A *$ its

completion. Then:

i) If $O$ is an ideal of $A$ and if $\sigma^{*}$ is principal, then ol itself is principal.

Proof. i) Suppose or $A^{*}=\alpha A^{*}, \alpha \in A^{*}$. Then $\alpha=\sum a_{i} \xi_{i}$ with $a_{i} \in$ or, $\xi_{i} \varepsilon A^{*}$, Put $I^{*}=I^{*}$, where I is an ideal of definition of A. By Artin-Rees we have $\alpha A^{*} \cap I I^{n} \subseteq I * a A^{*}$ for $n$ sufficiently large. Take $x_{i} \varepsilon$ A such that $x_{i} \equiv \xi_{i}\left(I^{n}\right)$ and put $a=\sum a_{i} x_{i}$. Then $a \equiv \alpha\left(I^{n}\right)$, and a $\varepsilon$ or $\subseteq \alpha A *$. Therefore $\alpha=a+B$ with $\beta \varepsilon \alpha A^{*} \cap I^{*} \subseteq I^{*} \alpha A^{*}$, hence $\alpha A *$ $\subseteq \mathrm{aA}^{*}+I * \alpha \mathrm{A}^{*}$, and by $\mathrm{NAK}$ we get $\alpha^{*}=a^{*}$. Then $\sigma=$ $\alpha A^{*} \cap A=a A * \cap A=a A$

ii) is a consequence of faithful flatness and was already proved in (21.E.iii).

We shall see in Part II that noetherian local (or semi-local) rings have many good properties.

\section{Extension of a Ring by a Module}
(25. A) Let $C$ be a ring and $N$ an ideal of $C$ with $N^{2}=(0)$; put $C^{\prime}=C / N$. Then the $C$-module $N$ can be viewed as a $C^{\prime}-$ module. Conversely, suppose that we are given a ring $C^{\prime}$ and triple $(C, \varepsilon, i)$ of a ring $C$, a surjective homomorphism of rings $\varepsilon: C \rightarrow C$ and a map 1: $N \rightarrow C$, such that: (1) Ker $(\varepsilon)$ is an Ker $(\varepsilon))$, and (2) the map i is an isomorphism from $N$ onto Ker (E) as $C^{\text {'modules. Therefore, identifying } N \text { with }}$ ( $(\mathrm{N})$ we get $C^{\prime} \simeq C / N, N^{2}=(0)$. An extension is often represented by the exact sequence $0 \rightarrow \mathrm{N} \rightarrow \mathrm{C} \rightarrow \mathrm{C}^{\prime} \rightarrow 0$. Two extensions $(C, \varepsilon, i)$ and $\left(C_{1}, \varepsilon_{1},{ }^{1}{ }_{1}\right)$ are said to be isomorphic if there exists a ring homomorphism $f: C \rightarrow C_{1}$ such that $\varepsilon_{1} f=\varepsilon$

i1) If $A^{*}$ is normal, then $A$ is also normal. and $f i=i_{1}$. Such $f$ is necessarily unique.

(25.B) Given $C^{\prime}$ and $N$ we can always construct an extension as follows: take the additive group $C^{\prime} \oplus N$, and define a multiplication in this set by the formula
$$
(a, x)(b, y)=(a b, a y+b x) \quad\left(a, b \in c^{\dagger} ; x, y \varepsilon N\right) .
$$
This is bilinear and associative, and has $(1,0)$ as the unit element. Hence we get a ring structure on $C^{\prime} \oplus N^{\prime}$ We denote this ring by $C^{\prime} * N$. By the obvious definitions $E(a, x)=a$ and $i(x)=(0, x)$ the ring $C^{\prime} * N$ becomes an extension of $C^{\prime}$ by $\mathrm{N}$, which is called the trivial extension.

An extension (C, $\varepsilon, i)$ of $C^{\prime}$ by $N$ is isomorphic to $C^{\prime} * N$ iff there exists a section, i.e. a ring homomorphism $s: C^{\prime} \rightarrow$ $C$ satisfying $\varepsilon s=i d^{\prime}$. In this case the extension $(C, \varepsilon, i)$ is also said to be trivial, or to be split.

(25.C) Let us briefly mention the Hochschild extensions. An extension $(C, \varepsilon, i)$ is called a Hochschild extension if the exact sequence of additive groups $0 \rightarrow \mathrm{N} \rightarrow \mathrm{C} \rightarrow \mathrm{C}^{\prime} \rightarrow 0$ splits, i.e. if there exists an additive map $s: C^{\prime} \rightarrow C$ such that $\varepsilon s=i d C^{\prime}$ Then $C$ is isomorphic to $C^{\prime} \oplus N$ as additive group, while the multiplication is given by
$$
(a, x)(b, y)=(a b, a y+b x+f(a, b))\left(a, b \in C^{\prime} ; x, y \in N\right)
$$
where the map $\mathrm{f}: \mathrm{C}^{\prime} \times \mathrm{C}^{\prime} \rightarrow \mathrm{N}$ is symmetric and bilinear and satisfies the cocycle condition (corresponding to the associativity in C)
$$
a f(b, c)-f(a b, c)+f(a, b c)-f(a, b) c=0
$$
Conversely, any such function $f(a, b)$ gives rise to a Hochschild extension. Moreover, the extension is trivial iff there exists a function $g: C^{\prime} \rightarrow N$ satisfying
$$
f(a, b)=a g(b)-g(a b)+g(a) b .
$$
(25.D) Let $A$ be a ring, and let $0 \rightarrow N \rightarrow C \rightarrow C^{\prime} \rightarrow 0$ be an extension of a ring $C^{\prime}$ by a $C^{\prime}$-module $N$ such that $C$ and $C^{\prime}$ are A-algebras and $\varepsilon$ is a homomorphism of A-algebras. Then $C$ is called an extension of the A-algebra $C^{\prime}$ by $\mathrm{N} .$ The extension is said to be A-trivial, or to split over A, if there exists a homomorphism of A-algebras $s: C^{\prime} \rightarrow C$ with $E s={ }^{\prime}{ }^{1}{ }^{\prime}$

$(25 . E)$ Let $E: 0 \rightarrow M \stackrel{\varepsilon}{\rightarrow} \underset{\rightarrow}{\rightarrow} C^{\prime} \rightarrow 0$ be an extension and let $g: M \rightarrow N$ be a homomorphism of $C^{\prime}$-modules. Then there exists an extension $g_{*}(E): 0 \rightarrow N \rightarrow D \rightarrow C^{\dagger} \rightarrow 0$ of $C^{\prime}$ by $N$ and a ring homomorphism $f: C \rightarrow D$ such that

\includegraphics[max width=\textwidth]{2022_08_01_8d4eee36f1f42236b4f4g-098}

is commutative. Such an extension $g_{*}(E)$ is unique up to isomorphisms. The ring $D$ is obtained as follows: we view the $C^{\prime}-$ module $N$ as a $C$-module and form the trivial extension $C * N_{\bullet}$ Then $M^{\prime}=\{(x,-g(x)) \mid x \in M\}$ is an ideal of $C * N$, and we put $D=(C * N) / M^{\prime}$. Thus, as an additive grou $D$ is the amalgamated sum of $C$ and $N$ with respect to $M$. The uniqueness of $g_{*}(E)$ follows from this construction.

Similarly, if $h: C^{\prime \prime} \rightarrow C^{\prime}$ is a ring homomorphism then there exists an extension $\mathrm{h}^{*}(E): 0 \rightarrow \mathrm{M} \rightarrow \mathrm{E} \rightarrow \mathrm{C}^{\prime \prime} \rightarrow 0$ of $\mathrm{C}^{\prime \prime}$ by $M$ and a ring homomorphism $f: E \rightarrow C$ such that the diagram

\includegraphics[max width=\textwidth]{2022_08_01_8d4eee36f1f42236b4f4g-099}

is commutative. Moreover, such $h *(E)$ is unique up to isomorphisms.

\section{Derivations and Differentials}
(26. A) Let A be a ring and $M$ an A-module. A derivation $D$ of A into $M$ is defined as usual: it is an additive map from A to $M$ satisfying $D(a b)=a D b+b D a$. The set of all derivations of $A$ into $M$ is denoted by $\operatorname{Der}(A, M)$; it is an A-module in the natural way.

For any derivation $D, D^{-1}(0)$ is a subring of $A$ (in particular, $D(1)=0$ : this follows from $\left.l^{2}=1 .\right)$ If A is a field, then $D^{-1}(0)$ is a subfield. Let $k$ be a ring and A a k-algebra. Then derivations $A \rightarrow M$ which vanish on $\mathrm{k} \cdot 1_{A}$ are called derivations over $k$. The set of such derivations is denoted by $\operatorname{Der}_{k}(A, M)$. We write $\operatorname{Der}_{k}(A)$ for $\operatorname{Der}_{k}(A, A)$.

Suppose that $A$ is a ring whose characteristic is a prime number $\mathrm{p}$, and let $\mathrm{A}^{\mathrm{P}}$ denote the subring $\left\{\mathrm{a}^{\mathrm{P}} \mid \mathrm{a} \varepsilon \mathrm{A}\right\}$. Then any derivation $D: A \rightarrow M$ vanishes on $A^{P}$, for $D\left(a^{p}\right)=p^{p-1} D(a)$ $=0$

(26.B) Let $A$ and $C$ be rings and $N$ an ideal of $C$ with $\mathrm{N}^{2}=0$. Let $j: C \rightarrow C / N$ be the natural map, Let $u, u^{\prime}: A \rightarrow C$ be two homomorphisms (of rings) satisfying ju $=j u^{\prime}$, and put $D=$ $u^{\prime}-u$. Then $u$ and $u^{\prime}$ induce the same A-module structure on $N$, and $D: A \rightarrow N$ is a derivation. In fact, we have
$$
\begin{aligned}
u^{\prime}(a b)=u^{\prime}(a) u^{\prime}(b) &=(u(a)+D(a))(u(b)+D(b)) \\
&=u(a b)+a D(b)+b D(a)
\end{aligned}
$$
$$
\begin{aligned}
& =u(a b)+a D(b)+b D(a)
\end{aligned}
$$
Conversely, if u: $A \rightarrow C$ is a homomorphism and $D: A \rightarrow N$ is a derivation (with respect to the A-module structure on $N$ induced by $u$ ), then $u^{\prime}=u+D$ is a homomorphism.

(26.C) Let $k$ be a ring, $A$ a $k$-algebra and $B=A B_{k} A$. Consider the homomorphisms of k-algebras

$E: B \rightarrow A$ and $\lambda_{1}, \lambda_{2}: A \rightarrow B$

defined by $\varepsilon\left(a \otimes a^{\prime}\right)=a^{\prime}, \lambda_{1}(a)=a \otimes 1, \lambda_{2}(a)=1 \otimes a$. Once and for al1, we make $\mathrm{B}=\mathrm{A} \otimes \mathrm{A}$ an $\mathrm{A}-\mathrm{a} l$ gebra via $\lambda_{1}$. We denote the kernel of $E$ by $I_{A / k}$ or simply by $I$, anc we put $I / I^{2}=\Omega_{A} / k$. The B-modules I, $I^{2}$ and $\Omega_{A} / k$ are also viewed as A-modules via $\lambda_{1}: A \rightarrow B$. Then the A-module $\Omega_{A} / k$ is called the module of differentials (or of Kähler differentials) of A over $k$.

We have $\varepsilon \lambda_{1}=\varepsilon \lambda_{2}=i d_{A}$. Therefore, if we denote. the natural homomorphism $B \rightarrow B / I^{2}$ by $v$ and if we put $d *=\lambda_{2}-\lambda_{1}$ and $\mathrm{d}=\nu \mathrm{d}^{*}$, then we get a derivation $\mathrm{d}: \mathrm{A} \rightarrow \Omega_{\mathrm{A} / \mathrm{k}}$. Note that we have $B=\lambda_{1}(A) \oplus I$, hence $B / I^{2}=V \lambda_{1}(A) \oplus \Omega_{A / k}$ (as $A$ module). Identifying $v \lambda_{1}$ (A) with $A$, we get
$$
\mathrm{B} / \mathrm{I}^{2}=\mathrm{A} \oplus \Omega_{\mathrm{A}} / \mathrm{k}^{\circ}
$$
In other words, $\mathrm{B} / \mathrm{I}^{2}$ is a trivial extension of $\mathrm{A}$ by $\Omega \mathrm{A} / \mathrm{k}^{\text {* }}$

PROPOSITION. The pair $\left(\Omega_{A} / k^{\prime}\right.$ d) has the following universal property: if $D$ is a derivation of $A$ over $k$ into an A-module $M$, then there is a unique $A-1$ inear map $f: \Omega_{A / k} \rightarrow M$ such that $D=f d$

Proof. In $B=A \otimes A$ we have $x \otimes y=x y \otimes 1+x(1 \otimes y-y \otimes 1)=$ $\varepsilon(x \otimes y)+x d * y$. Therefore, if $\sum x_{i} \otimes y_{1} \varepsilon I=\operatorname{Ker}(\varepsilon)$ then $\sum x_{i} \otimes y_{i}=\sum x_{i} d^{*} y_{i}$. Since $d * y \bmod I^{2}=d y$, any element of $\Omega=I / I^{2}$ has the form $\sum x_{i} d y_{i}\left(x_{i}, y_{i} E A\right)$. In other words, $\Omega$ is generated by $\{d y \mid y \varepsilon A\}$ as A-module. This proves the uniqueness of $f$. As for the existence of $f$, take the trivial extension $A * M$ and define a homomorphism of A-algebras $\phi: B=A \otimes_{k} A \rightarrow A * M$ by $\phi(x \otimes y)=(x y, x D(y))$. Since $\phi(I) \subseteq$ $M$ and $\mathrm{M}^{2}=0$, we have $\phi\left(\mathrm{I}^{2}\right)=0$ so that $\phi$ induces a homomorphism $\bar{\phi}$ of $A$-algebras $B / I^{2}=A * \Omega \rightarrow A * M$ which maps $\mathrm{d} y \in \Omega$ to $\phi\left(\mathrm{d}^{*} \mathrm{y}\right)=\phi(1 \otimes y-\mathrm{y} \otimes 1)=(0, D y)$. Thus the restriction of $\bar{\phi}$ to $\Omega$ gives an A-linear map $f: \Omega \rightarrow M$ with $f \circ d=D$. Q.E.D.

As a consequence of the proposition we get a canonical isomorphism of A-modules
$$
\operatorname{Der}_{k}(A, M) \simeq \operatorname{Hom}_{\mathrm{A}}\left(\Omega_{\mathrm{A}} / \mathrm{k}, M\right) .
$$
In the categorical language, the pair $\left(\Omega_{A} / k^{\prime} d\right)$ represents the covariant functor $M \mapsto \operatorname{Der}_{k}(A, M)$ from the category of A-modules into itself. The map $\mathrm{d}: \mathrm{A} \rightarrow \Omega_{\mathrm{A} / \mathrm{k}}$ is called the canonical derivation and is denoted by $\mathrm{d}_{\mathrm{A} / \mathrm{k}}$ if necessary.

(26.D) Any ring A is a Z-algebra in a unique way. The module $\Omega_{A / Z}$ is simply written $\Omega_{A}$. If A contains a field $k$ and if $F$ is the prime field in $k$, then $\Omega_{A / F}=\Omega_{A}$ because $A \otimes_{Z} A=A \otimes_{F} A$.

The r-th exterior product $\bigwedge^{\mathrm{r}} \Omega_{\mathrm{A} / \mathrm{k}}$ is denoted by $\Omega^{\mathrm{r}} \mathrm{A} / \mathrm{k}$ and is called the module of differentials of degree $r$. In this notation we have $\Omega_{\mathrm{A} / \mathrm{k}}=\Omega^{1} \mathrm{~A} / \mathrm{k}^{\circ}$

(26.E) Example 1. Let $k$ be a ring, and let $A$ be a $k-a l g e b r a$ which is generated by a set of elements $\left\{x_{\lambda}\right\}$ over $k$. Then

$\Omega_{\mathrm{A} / \mathrm{k}}$ is generated by $\left\{\mathrm{dx} \mathrm{\lambda}_{\lambda}\right\}$ as A-module. This is clear since d is a derivation.

In particular, if $A$ is a polynomial ring over the ring $k$ in an arbitrary number of indeterminates $\left\{x_{\lambda}\right\}: A=k\left[\ldots, x_{\lambda}, \ldots\right]$, then $\Omega_{\mathrm{A} / \mathrm{k}}$ is a free $\mathrm{A}$-module with $\left\{\mathrm{dX}_{\lambda}\right\}$ as a basis. In fact, suppose $\sum P_{\lambda} d X_{\lambda}=0 \quad\left(P_{\lambda} \varepsilon A\right)$ and let $\partial / \partial x_{\lambda}$ denote the partial derivations. Then $\partial / \partial x_{\lambda} \varepsilon \operatorname{Der}_{k}(A)$, hence there exists a linear map $f: \Omega_{\mathrm{A} / \mathrm{k}} \rightarrow \mathrm{A}$ such that $\mathrm{f}\left(\mathrm{dX}_{\mu}\right)=\partial \mathrm{X}_{\mu} / \partial \mathrm{x}_{\lambda}=\delta_{\lambda \mu} \cdot$ Applying $f$ to $\sum P_{\mu} d X_{\mu}=0$ we find $P_{\lambda}=0$. As $\lambda$ is arbitrary we see that the $\mathrm{dx}_{\lambda}{ }^{\prime}$ s are linear1y independent over A. Q.E.D.

(Note that $\operatorname{Der}_{k}(A)=\operatorname{Hom}_{A}\left(\Omega_{A} / k, A\right) \simeq \prod_{\lambda} A_{\lambda}$, where $A_{\lambda} \simeq A_{0}$ )

(26.F) Example 2. Let $k$ be a field of characteristic $p>0$, and let $k^{\prime}$ be a subfield such that $k=k^{\dagger}(t), t^{p}=a \varepsilon k^{\prime}$ $t \neq k^{\prime}$. Then $k=k^{\prime}[X] /\left(X^{p}-a\right)$, and since $\partial\left(X^{p}-a\right) / \partial X=$ 0 the derivation $\partial / \partial X$ of $k^{\prime}[X]$ maps the ideal $\left(X^{P}-a\right) k^{\prime}[X]$ into itself. It thus induces a derivation $D$ of $k$ over $k^{\prime}$ such that $D(t)=1$.

Next, let $k^{\prime}$ be an arbitrary subfield such that $k^{p} \subseteq k^{\prime}$ $\subseteq k$. A family of elements $\left(x_{\lambda}\right)$ of $k$ is said to be p-independent over $k^{\eta}$ if, for any finite subset $\left\{x_{\lambda_{1}}, \ldots, x_{\lambda_{n}}\right\}$, we have $\left[k^{\prime}\left(x_{\lambda_{1}}, \ldots, x_{\lambda_{n}}\right): k^{1}\right]=p^{n}$. A family $\left(x_{\lambda}\right)$ is called a pbasis of $k$ over $k^{\prime}$ if it is p-independent over $k^{\prime}$ and if $k^{\prime}\left(\ldots, x_{\lambda}, \ldots\right)=k$. The existence of a p-basis of $k$ over $k^{\prime}$ can be easily proved by Zorn's lemma. Moreover, any p-indep. family over $k^{\prime}$ can be extended to a p-basis. Suppose that we are given a p-basis $\left(x_{\lambda}\right)$. Then $\Omega_{k} / k^{\prime}$ is a free $k$-module with $\left(\mathrm{dx} \lambda_{\lambda}\right)$ as a basis. In fact, putting $\mathrm{k}_{\lambda}^{\prime}=\mathrm{k}^{\prime}\left(\left\{\mathrm{x}_{\mu} \mid \mu \neq \lambda\right\}\right)$ we have $k_{\lambda}^{\prime}\left(x_{\lambda}\right)=k, x_{\lambda}^{p} \varepsilon k_{\lambda}^{\prime}$ and $x_{\lambda} \notin k_{\lambda}^{\prime}$, so there exists a derivation $D_{\lambda}$ of $k$ over $k_{\lambda}^{\prime}$ such that $D_{\lambda}\left(x_{\lambda}\right)=1$. Therefore $D_{\lambda} \varepsilon \operatorname{Der}_{k^{\prime}}(k)$ and $D_{\lambda}\left(x_{\mu}\right)=\delta_{\lambda \mu}$. From this we conclude the linear independence of the $\mathrm{dx} \lambda^{\prime} \mathrm{s}$ as in Example $1 .$

If $k^{p} \subseteq k^{\prime} \subseteq k$ and $\left[k: k^{\prime}\right]=p^{m}<\infty$, then $\Omega k / k^{\prime}$ and $\operatorname{Der}_{k},(k)$ are vector spaces of rank $m$, dual to each other.

In general, if $k^{\prime}$ is an arbitrary subfield of $k$ and $x_{1}$, $\ldots, x_{n} \varepsilon k_{2}$. then the differentials $\mathrm{dx}_{1}, \ldots, \mathrm{d} \mathrm{x}_{n}$ in $\Omega_{k} / k^{\prime}$ are linearly independent over $k$ iff the $f a m i l y\left(x_{i}\right)$ is $p$-indep. over $k^{\prime}\left(k^{\mathrm{P}}\right)$. Proof is left to the reader.

(26.G) Example 3. Let $k$ be a field and $\mathrm{K}$ a separable algebraic extension field of $k$. Then $\Omega_{K} / k=0$. In fact, for any $\alpha \varepsilon \mathrm{K}$ there is a polynomial $\mathrm{f}(\mathrm{X}) \varepsilon \mathrm{k}[\mathrm{X}]$ such that $\mathrm{f}(\alpha)=0$ and $f^{\prime}(\alpha) \neq 0$. 'Since $d: k \rightarrow \Omega_{K / k}$ is a derivation we have $0=d(f(\alpha))=f^{\prime}(\alpha) d \alpha$, whence $d \alpha=0$. As $\Omega_{K / k}$ is generated by the da's we get $\Omega_{\mathrm{K} / \mathrm{k}}=0$

\section{Exercises.}
\begin{enumerate}
  \item If $A \longrightarrow A^{\prime}$ homomorphisms, then there is a natural homomorphism of Amodules $\Omega_{\mathrm{A} / \mathrm{k}} \rightarrow \Omega_{\mathrm{A}^{1}} / \mathrm{k}^{1}$, hence also a natural homomorphism of $A^{\prime}$-modules $\Omega_{A / k} \otimes_{A^{\prime}} \rightarrow \Omega_{A^{\prime}} / k^{1}$

  \item If $A^{\prime}=A_{k} B^{\prime}$ in 1 , then the last homomorphism is an isomorphism: $\Omega_{A^{\prime}} / k^{\prime}=\Omega_{A / k} \otimes_{k} k^{\prime}=\Omega_{A / k} \otimes_{A^{\prime}} A^{\prime}$.

  \item If $S$ is a multiplicative set in a k-algebra $A$ and if $A^{\prime}=S^{-1} A$, then $\Omega_{A^{\prime} / k}=\Omega_{A / k} \otimes A^{A^{\prime}}=S^{-1} \Omega_{A / k}$

\end{enumerate}
(26.H) THEOREM 57. (The first fundamental exact sequence)

Let $k, A$ and $B$ be rings and let $k \rightarrow A \rightarrow B$ be homomorphisms.

Then i) there is an exact sequence of natural homomorphisms

\section{of $\mathrm{B}-$ modules}
$$
\Omega_{\mathrm{A} / \mathrm{k}} \otimes_{\mathrm{A}} \stackrel{\mathrm{v}}{\rightarrow} \Omega_{\mathrm{B} / \mathrm{k}} \rightarrow \Omega_{\mathrm{B} / \mathrm{A}} \rightarrow 0 ;
$$
ii) the map $v$ has a left inverse (or what amounts to the

same, $v$ is injective and $\operatorname{Im}(v)$ is a direct summand of $\Omega_{B / A}$

as $B$-module) iff any derivation of $A$ over $k$ into any $B$-module

$T$ can be extended to a derivation $\mathrm{B} \rightarrow \mathrm{T}$.

Proof. 1) The map $v$ is defined by $v\left(d_{A / k}(a) \otimes b\right)=b \cdot d_{B / k} \psi(a)$,

and the map $u$ by $\left.u\left(b \cdot d_{B / k}\left(b^{\prime}\right)\right)=b \cdot d_{B / A}\left(b^{\prime}\right)\right)\left(a \varepsilon A ; b, b^{\prime} \varepsilon B\right)$.

It is clear that $u$ is surjective. Since $\left.d_{B / A} \psi(a)\right)=0$ we

have $u v=0 .$ It remains to prove that $\operatorname{Ker}(u)=\operatorname{Im}(v)$. To

do this, it is enough to show that
$$
\operatorname{Hom}_{B}\left(\Omega_{A} / k \otimes A B, T\right)+\operatorname{Hom}_{B}\left(\Omega_{B} / k, T\right)+\operatorname{Hom}_{B}\left(\Omega_{B} / A, T\right)
$$
is exact for any $B$ module $T$ (take $T=\operatorname{Coker}(v)$ ). But we have canonical isomorphisms $\operatorname{Hom}_{B}\left(\Omega / k_{A} B, T\right) \simeq \operatorname{Hom}_{A}\left(\Omega_{A} / k, T\right) \simeq$ $\operatorname{Der}_{k}(A, T)$ etc., so we can identify the last sequence with
$$
\operatorname{Der}_{k}(A, T)+\operatorname{Der}_{k}(B, T)+\operatorname{Der}_{A}(B, T)
$$
where the first arrow is the map $\mathrm{D} \mapsto \mathrm{D} \bullet \psi$. This sequence

is exact by the definitions.

ii) A homomorphism of $B$-modules $M^{\prime} \rightarrow M$ has a left inverse iff the induced map $\operatorname{Hom}_{B}\left(M^{\top}, T\right)+\operatorname{Hom}_{B}(M, T)$ is surjective for any B-module $T$. Thus, $v$ has a left inverse iff the natural map $\operatorname{Der}_{k}(A, T)+\operatorname{Der}_{k}(B, T)$ is surjective for any B-module $\mathrm{T}$.

Q.E.D.

COROLLARY. The map $v: \Omega_{A / k} \otimes_{A} B \rightarrow \Omega_{B / k}$ is an isomorphism iff any derivation of $A$ over $k$ into any B-module $T$ can be extended uniquely to a derivation $\mathrm{B} \rightarrow \mathrm{T}$.

(26.I) Let $k$ be a ring, A a $k$-algebra, $m$ an ideal of $A$ and $B=A / M$. Define a map $m \rightarrow \Omega_{A} / k_{A} \otimes_{A}$ by $x \mapsto d_{A / k} x \otimes 1$ $(x \in M)$. It sends $M^{2}$ to 0 , hence induces a $B-1$ inear map $\delta: 1 น / \mathrm{ML}^{2} \rightarrow \Omega_{\mathrm{A} / \mathrm{k}} \otimes \mathrm{A}_{\mathrm{B}}$

THEOREM 58 (The second fundamental exact sequence). Let the notation be as above.

i) The sequence of B-modules () $\mu \mathrm{mL}{ }^{2} \stackrel{\delta}{\rightarrow} \Omega_{\mathrm{A} / \mathrm{k}} \otimes_{\mathrm{A}}^{\mathrm{B}} \stackrel{\mathrm{v}}{\rightarrow} \Omega_{\mathrm{B} / \mathrm{k}} \rightarrow 0$

is exact.

ii) Put $A_{1}=A / H^{2}$. Then $\Omega_{A / k} \otimes_{A}{ }^{B} \simeq \Omega_{A_{1}} / k \otimes_{A_{1}}{ }^{B}$.

iii) The homomorphism $\delta$ has a left inverse iff the extension

$0 \rightarrow m / m^{2}+A_{1} \rightarrow B \rightarrow 0$ of the k-algebra $B$ by $M M_{R^{2}}$ is tri-

vial over $k$.

Proof. i) The surjectivity of $v$ follows from that of $A \rightarrow B$.

Obviously the composite $v \delta=0$, So, as in the proof of the

preceding theorem, it is enough to prove the exactness of

$\operatorname{Hom}_{B}\left(1 i / m m^{2}, T\right)+\operatorname{Hom}_{B}\left(\Omega_{A / k} \otimes_{A} B, T\right)+\operatorname{Hom}_{B}\left(\Omega_{B} / A, T\right)$

for any B-module $T$. But we can rewrite it as follows:

$\operatorname{Hom}_{A}(M, T)+\operatorname{Der}_{k}(A, T)+\operatorname{Der}_{k}(A / M r, T)$

where the first arrow is the map $D \mapsto D \mid \mathbb{H}\left(D \varepsilon \operatorname{Der}_{k}(A, T)\right.$ ).

Then the exactness is obvious.

ii) A homomorphism of $B$-modules $N^{\prime} \rightarrow N$ is an isomorphism iff

the induced map $\operatorname{Hom}_{B}\left(N^{\prime}, T\right) \leftarrow \operatorname{Hom}_{B}(N, T)$ is an isomorphism

for every B-module T. Applying this to the present situation

we are led to prove that the natural $\operatorname{map}^{-\operatorname{Der}_{k}}(A, T) \leftarrow \operatorname{Der}_{k}\left(A / H^{2}\right.$,

T) is an isomorphism for every A/M-module $T$, which is obvious.

iii) By ii) we may replace $A$ by $A_{1}$ in (*), so we assume $m^{2}$

$=0$. Suppose that $\delta$ has a left inverse $w: \Omega_{A / k} \otimes_{A} B \rightarrow m$.

Putting Da $=w(\mathrm{da} \otimes 1)$ for a $\varepsilon$ A we obtain a derivation $D: A$

$\rightarrow M$ over $k$ such that $D x=x$ for $x \in M$. Then the map $\mathrm{f}: \mathrm{A} \rightarrow \mathrm{A}$ given by $\mathrm{f}(\mathrm{a})=\mathrm{a}-\mathrm{Da}$ is a homomorphism of $\mathrm{k}-$ algebras and satisfies $\mathrm{f}(M M)=0$, hence induces a homomorphism $\bar{f}: B=A / t n+A$. Since $f(a) \equiv$ a mod $t h$, the homomorphism $\bar{f}$ is a section of the ring extension $0 \rightarrow M \rightarrow A \rightarrow B \rightarrow 0$. The converse is proved by reversing the argument.

(26. J) Example. Let $k$ be a ring, A a k-algebra and $B=A\left[X_{1}\right.$, $\left.\ldots, X_{n}\right] .$ Let $T$ be an arbitrary $B$-module and let $D \in \operatorname{Der}_{k}(A, T)$. Then we can extend it to a derivation $B \rightarrow T$ by putting $D(P(X))=P^{D}(X)$, where $P^{D}$ is obtained from $P(X)$ by applying D to the coefficients. Thus the natural map $\Omega_{A / k} \otimes_{A} B \rightarrow \Omega_{B} / k$ has a left inverse, and we have

$\Omega_{B / k} \simeq\left(\Omega_{A / k} \otimes_{A} B\right) \oplus \mathrm{BdX}_{1} \oplus \ldots \oplus B d X_{n}$

Let $M$ be an ideal of $B=A\left[x_{1}, \ldots, x_{n}\right]$, and put $C=B / M, x_{i}$ $=\mathrm{X}_{i} \bmod M$. Then we have the second fundamental exact seque quence $m / m^{2} \rightarrow \Omega_{B} / \mathrm{k} \otimes_{\mathrm{B}} \mathrm{C}=\left(\Omega_{\mathrm{A}} / \mathrm{k} \otimes_{\mathrm{A}} \mathrm{C}\right) \oplus \Sigma \mathrm{Cdx}{ }_{\mathrm{i}} \rightarrow \Omega_{\mathrm{C} / \mathrm{k}} \rightarrow 0$ with

$\delta(P(X))=(d P)(x)+\sum_{i=l}^{n} \partial P / \partial x_{i}(x) d X_{i}(P(X) \varepsilon M V)$, where $(\mathrm{dP})(\mathrm{x})$ is obtained by applying $\mathrm{d}_{\mathrm{A} / \mathrm{k}}$ to the coefficients

of $P(X)$ and then reducing the result modulo M.

Exercise 4. Let $B=k[X, Y] /\left(Y^{2}-X^{3}\right)=k[x, y] \Leftrightarrow$ the affine ring of the plane curve $y^{2}=x^{3}$, which has a cusp at the

\includegraphics[max width=\textwidth]{2022_08_01_8d4eee36f1f42236b4f4g-103}\\
torsion.

\section{Separability}
(27.A) Let $k$ be a field and $k$ an extension 1 ) of $k$. A transcendency basis $\left\{x_{\lambda}\right\} \lambda \in \Lambda$ of $K$ over $k$ is called a separating transcendency basis if $\mathrm{K}$ is separably algebraic over the field $k\left(\ldots, x_{\lambda}, \ldots\right)$. We say that $k$ is separably generated over $k$ if it has a separating transcendency basis.

Put $r(K)=\operatorname{rank}_{K} \Omega_{K / k}$ Let L be a finitely generated extension of $K$. We want to compare $r(L)$ and $r(K)$. Suppose

first that $\mathrm{L}=\mathrm{K}(\mathrm{t})$. There are four typical cases.

Case 1. $t$ is transcendental over $K$. Then $\Omega_{K}[t] / k=$ $\left(\Omega_{K / k} \otimes_{K} K[t]\right) \oplus K[t] d t$ by $(26 . J)$, so by localization we get $\Omega_{\mathrm{L} / \mathrm{k}}=\left(\Omega_{\mathrm{K} / \mathrm{k}} \otimes \mathrm{K}_{\mathrm{L}}\right) \oplus \mathrm{Ld} t$, hence $\mathrm{r}(\mathrm{L})=\mathrm{r}(\mathrm{K})+1$

Case 2. $t$ is separably algebraic cver $k$. Let $f(X)$ be the irreducible equation of $t$ over $K$. Then $L=K[t]=K[X] /(f)$, $f(t)=0$ and $f^{\prime}(t) \neq 0$. By $(26 . J)$ we have $\Omega L / k=$ $\left(\left(\Omega_{K} / k \otimes_{K} L+L d X\right) / L \delta f\right.$, where $\delta f=(d f)(t)+f^{\prime}(t) d X$ in the notation of $(26 . J)$. As $f^{\prime}(t)$ is invertible in L we have $\Omega_{\mathrm{K} / \mathrm{k}} \otimes_{\mathrm{K}}^{\mathrm{L}} \simeq \Omega_{\mathrm{L} / \mathrm{k}}$. Whence $\mathrm{r}(\mathrm{L})=\mathrm{r}(\mathrm{K})$. From this, or by a direct computation, one sees that any derivation of $\mathrm{K}$ into $\mathrm{L}$

can be extended uniquely to a derivation of L.

Case 3. $\underline{\operatorname{ch}(k)=p}, \quad t^{p}=a \varepsilon K, \quad t \notin K, \quad d_{K / k}(a)=0$.

Then $L=K[t]=K[X] /\left(X^{p}-a\right)$. We have $\delta\left(X^{p}-a\right)=0$,

\begin{enumerate}
  \item By an extension of a field we mean an extension field; by a finite extension, a finite algebraic extension. therefore $\Omega_{\mathrm{L} / \mathrm{k}} \simeq \Omega_{\mathrm{K}}[\mathrm{X}] / \mathrm{k} \otimes \mathrm{L} \simeq\left(\Omega_{\mathrm{K} / \mathrm{k}} \otimes \mathrm{K}_{\mathrm{L}}\right) \oplus \mathrm{Ldt}$ and $r(L)=r(K)+1$
\end{enumerate}
Case 4. Same as in case 3 with the exception that $d_{K / k}$ a $\neq 0$. Then $\delta\left(X^{p}-a\right) \neq 0$, and so $r(L)=r(K)$.

(27.B) THEOREM 59. i) Let $k$ be a field, $k$ an extension of

$k$ and $L$ a finitely generated extension of $k$. Then
$$
\operatorname{rank}_{\mathrm{L}} \Omega_{\mathrm{L} / \mathrm{k}} \geqslant \operatorname{rank}_{\mathrm{K}} \Omega_{\mathrm{K} / \mathrm{k}}+\mathrm{tr} \cdot \operatorname{deg}_{\mathrm{K}} \mathrm{L} \text {. }
$$
ii) The equality holds in i) if $L$ is separably generated

over $K$.

iii) Let L be a finitely generated extension of a field $k$. Then $\operatorname{rank}_{L} \Omega_{L / k} \geqslant t r . \operatorname{deg}_{k} L$, where the equality holds iff $\mathrm{L}$ is separably generated over $\mathrm{k}$. In particular, $\Omega_{\mathrm{L} / \mathrm{k}}=0$ iff $L$ is separably algebraic over $k$.

Proof. Since any finitely generated extension of $\mathrm{K}$ is obtained by repeating extensions of the four types just discussed, the assertions i) and ii) are now obvious. As for iii), the inequality is a special case of i). Suppose that $\Omega_{L / k}=0$, 1.e. that $r(E)=0$. Then $r(K)=0$ for any $k \subseteq K \subseteq L$. Therefore the cases 1,3 and 4 of $(27 . A$ ) cannot happen for $L$ and K. This means that $L$ is separably algebraic over $k$. Suppose, next, that $r(L)=t r \cdot \operatorname{deg}_{k} L=r$, Let $x_{1}, \ldots, x_{r} \varepsilon$ L be such that $\left\{d x_{1}, \ldots, d x_{r}\right\}$ is a basis of $\Omega_{L / k}$ over L. Then we have $\Omega_{L} / k\left(x_{1}, \ldots, x_{r}\right)=0$ by Th.57, so $L$ is separably algebraic over $k\left(x_{1}, \ldots, x_{r}\right)$. Since $r=\operatorname{tr}_{r} \operatorname{deg}_{k} L$ the elements $x_{i}$ must form a transcendency basis of L over $k$.

Remark, Let $L=k\left(x_{1}, \ldots, x_{n}\right)$ and $t r, \operatorname{deg}_{k} L=r$, and put $p=\left\{f(\mathrm{X}) \varepsilon k\left[\mathrm{x}_{1}, \ldots, \mathrm{x}_{\mathrm{n}}\right] \mid \mathrm{f}\left(\mathrm{x}_{1}, \ldots, \mathrm{x}_{\mathrm{n}}\right)=0\right\}$. Let $\mathrm{f}_{1}, \ldots, \mathrm{f}_{\mathrm{s}}$ generate the ideal $p$. Then $\mathrm{L}$ is separably generated over $k$ iff the Jacobian matrix $\partial\left(f_{1}, \ldots, f_{s}\right) / \partial\left(x_{1}, \ldots, x_{n}\right)$ has rank n - $r$, as one can easily check. If this is the case, and if the minor determinant $\partial\left(f_{1}, \ldots, f_{n-r}\right) / \partial\left(x_{r+1}, \ldots, x_{n}\right) \neq 0$, then $\mathrm{dx}_{1}, \ldots, \mathrm{dx} \mathrm{r}_{\mathrm{r}}$ form a basis of $\Omega_{\mathrm{L} / \mathrm{k}}$, and the above proof shows that $\left\{x_{1}, \ldots, x_{r}\right\}$ is a separating transcendency basis of $\mathrm{L} / \mathrm{k}$.

(27. C) LEMMA 1 . Let $k$ be a field and $k$ an algebraic extension of $k$. Then the following are equivalent:

(1) $K$ is separably algebraic over $k$;

(2) the ring $k \otimes_{k} k^{\prime}$ is reduced for any extension $k^{\prime}$ of $k$;

(3) ditto for any algebraic extension $k^{\prime}$ of $k$;

(4) ditto for any finite extension $k^{\prime}$ of $k$.

Proof. Each of these properties holds iff it holds for any finite extension $K^{\prime}$ of $k$ contained in $K$. So we may assume that $[K: k]<\infty$.

$(1) \Rightarrow(2)$ : If $K$ is finite and separable over $k$ then $K=$ $k(t)$ with some $t \varepsilon K$. Let $f(X)$ be the irreducible equation of $t$ over $k$. Then $k \simeq k[X] /(f)$, hence $k \otimes k^{\prime} \simeq k^{\prime}[X] /(f)$, and since $f(X)$ has no multiple factors in $k^{\prime}[X]$ (because it decomposes into distinct linear factors in $\bar{k}[X]$, where $\bar{k}$ is the algebraic closure of $k$ ), $K \otimes k^{1}$ is reduced. (More precise$1 y$, it is a direct product of finite separable extensions of $\left.k^{\prime} .\right)(2) \Rightarrow(3) \Rightarrow(4)$ is trivial.

$(4) \Rightarrow(1)$ : Suppose that $\operatorname{ch}(k)=p$ and that $k$ contains an inseparable element $t$ over $k$. Then the irreducible equation $f(X)$ of $t$ over $k$ is of the form $f(X)=g\left(X^{P}\right)$ with some $g \varepsilon$ $k[X]$. Let $a_{0}, \ldots, a_{n}$ be the coefficients of $g(X)$ and put $k^{\prime}=k\left(a_{0}^{1 / p}, \ldots, a_{n}^{1 / p}\right)$. Then $f(X)=g\left(X^{p}\right)=h(X)^{p}$ with $h(X) \varepsilon k^{\prime}[X]$, and $k(t) \otimes_{k} k^{\prime}=k^{\prime}[X] /\left(h(X){ }^{p}\right)$ has nilpotent elements. Since $k$ is a field we can view $k(t) \otimes_{k} k^{\prime}$ as a subring of $K \otimes_{k} k^{\prime}$, so the condition (4) does not hold.

(27.D) DEFINITION. Let $k$ be a field and A a k-algebra. We say that A is separable (over k) if, for any algebraic extension $k^{\prime}$ of $k$, the ring $A \otimes k^{\prime}$ is reduced.

The following. properties are immediate consequences of the definition.

\begin{enumerate}
  \item If A is separable, then any subalgebra of A is also separable.

  \item If all finitely generated subalgebras of A are separable, then A is separable. 3) If, for any finite extension $k^{\prime}$ of $k$, the ring $A \otimes_{k} k^{\prime}$ is reduced, then $A$ is separable.

\end{enumerate}
(27. E) LEMMA 2. If $k^{\prime}$ is a separably generated extension of

a field $k$, and if $A$ is a reduced $k$-algebra, then $A \otimes_{k} k^{\prime}$ is reduced

Proof. Enough to consider the case of a separably algebraic

extension and the case of a purely transcendental extension.

We may also assume that A is finitely generated over $k$. Then

$A$ is noetherian and reduced, so the total quotient ring $\Phi \mathrm{A}$

of $A$ is a direct product of a finite number of fields, and

$A \otimes_{k} k^{\prime} \subseteq \Phi A \otimes_{k} k^{\top} \cdot$ Thus we may assume that $A$ is a field.

Then $A \otimes_{k} k^{\prime}$ is reduced by Lemma 1 in the separably algebraic

case, and is a subring of a rational function field over A

in the purely transcendental case.

COROLLARY. If $k$ is a perfect field, then a $k$-algebra $A$ is separable iff it is reduced. In particular, any extension field $k$ of $k$ is separable over $k$.

(27.F) LEMMA 3. Let $k$ be a field of characteristic $p$, and

$K$ be a finitely generated extension of $k$. Then the following

are equivalent: (1) $K$ is separable over $k$;

(2) the ring $K \otimes_{k} k^{1 / p}$ is reduced;

(3) $\mathrm{K}$ is separably generated over $\mathrm{k}$.

Proof. (3) $\Rightarrow(1):$ If $K$ is separably generated over $k$, then $k^{\prime} \otimes k^{K}$ is reduced for any extension $k^{\prime}$ of $k$ by Lemma 2 .

$(1) \Rightarrow(2)$ : Trivial. $(2) \Rightarrow(3): \quad$ Let $k=k\left(x_{1}, \ldots, x_{n}\right)$. We may suppose that $\left\{x_{1}, \ldots, x_{r}\right\}$ is a transcendency basis of $k / k$. Suppose that $x_{r+1}, \ldots, x_{q}$ are separable over $k\left(x_{1}, \ldots, x_{r}\right)$ while $\mathrm{x}_{\mathrm{q}+1}$ is not. Put $y=\mathrm{x}_{\mathrm{q}+1}$ and let $f\left(\mathrm{Y}^{\mathrm{p}}\right)$ be the irreducible equation of $y$ over $k\left(x_{1}, \ldots, x_{r}\right)$. Clearing the denominators of the coefficients of $f$ we obtain a polynomial $\mathrm{F}\left(\mathrm{X}_{1}, \ldots, \mathrm{X}_{\mathrm{r}}, \mathrm{Y}^{\mathrm{p}}\right)$, irreducible in $\mathrm{k}\left[\mathrm{X}_{1}, \ldots, \mathrm{X}_{\mathrm{r}}, \mathrm{Y}\right]$, such that $F\left(x_{1}, \ldots, x_{r}, y^{p}\right)=0$. Then there must be at least one $x_{i}$ such that $\partial \mathrm{F} / \partial \mathrm{X}_{i} \neq 0$, for otherwise we would have $F\left(X, \mathrm{Y}^{\mathrm{p}}\right)$ $=G(X, Y)^{p}$ with $G \varepsilon k^{1 / p_{1}}\left[x_{1}, \ldots, X_{r}, Y\right]$, so that $k\left(x_{1}, \ldots, x_{r}, y\right)$ $\otimes_{\mathrm{k}} \mathrm{k}^{1 / \mathrm{p}} \simeq \mathrm{k}^{1 / \mathrm{p}}\left(\mathrm{X}_{1}, \ldots, \mathrm{X}_{\mathrm{r}}\right)[\mathrm{Y}] /\left(\mathrm{G}(\mathrm{X}, \mathrm{Y})^{\mathrm{p}}\right)$ would have nilpotent elements. Therefore we may suppose that $\partial F / \partial x_{1} \neq 0$. Then $x_{1}$ is separably algebraic over $k\left(x_{2}, \ldots, x_{r}, y\right)$, hence the same holds for $x_{r+1}, \ldots, x_{q}$ also. Exchanging $x_{1}$ with $y=x_{q+1}$ we have that $\mathrm{x}_{\mathrm{r}+1}, \ldots, \mathrm{x}_{\mathrm{q}+1}$ are separable over $\mathrm{k}\left(\mathrm{x}_{1}, \ldots, \mathrm{x}_{\mathrm{r}}\right)$. By induction on $q$ we see that we can choose a separating

transcendency basis of $\mathrm{k} / \mathrm{k}$ from the $\operatorname{set}\left\{\mathrm{x}_{1}, \ldots, \mathrm{x}_{n}\right\}$. (27.G) PROPOSITION. Let $k$ be a field and A a separable $k$ -

algebra. Then, for any extension $k^{\prime}$ of $k$ (algebraic or not),

the ring $A \otimes_{k} k^{\prime}$ is reduced and is a separable $k^{\prime}-a l g e b r a .$

CHAPTER 11. FORMAL SMOOTHNESS

Proof. Enough to prove that $A Q_{k} k^{\prime}$ is reduced. We may assume

that $k^{\prime}$ contains the algebraic closure $\bar{k}$ of $k$. Since $A \otimes \bar{k}$ is

reduced by assumption, and since any finitely generated exten-

sion of $\bar{k}$ is separably generated by Lemma 3 , the ring $A \otimes_{k} k^{\prime}$

$=\left(\mathrm{A} \otimes_{\mathrm{k}} \overline{\mathrm{k}}\right) \otimes \frac{\mathrm{k}^{\prime}}{}$ is reduced by Lemma $2 .$

Exercises. 1 (MacLane). Let $k$ be a field of characteristic

$p$ and $k$ an extension of $k$. Then $k$ is separable over $k$ iff

$K$ and $k^{1 / p}$ are linearly disjoint over $k$, that is, iff the

canonical homomorphism from $K \otimes_{k} k^{1 / p}$ onto the subfield $K\left(k^{1 / p}\right)$ of $\mathrm{K}^{1 / \mathrm{p}}$ is an isomorphism.

\begin{enumerate}
  \setcounter{enumi}{2}
  \item Let $k$ and $k$ be as above, and suppose that $k$ is finttely
\end{enumerate}
generated over $k$. Then there exists a finite extension $k^{\prime}$ of

$k$, contained in $k^{p^{-\infty}}$, such that $k\left(k^{\prime}\right)$ is separable over $k^{\prime}$.

\section{Formal Smoothness 1}
(28. A) The notion of formal smoothness is due to Grothendieck (EGA Ch.IV, 1964). 

\section{Jacobion Criterio}
(29.A) Let $k$ he a field, and I be an ideal of $k\left[x_{1}, \ldots, x_{n}\right]$ Let $P$ be a prime ideal containing $I$, and put $A=k\left[X_{1}, \ldots, X_{n}\right]$, $B=A / I$ and $P=P / I$. Then $B_{P}=A_{P} / I A_{P} ;$ let $k$ denote the common residue field of $A_{P}$ and $B_{P}$, Put $\operatorname{dim~} A_{P}=m$ and ht $\left(I A_{p}\right)=r$. Since $A$ is catenarian we have $\operatorname{dim} B_{p}=m-r$. We know that $A_{P}$ is a regular local ring, and that $B_{p}$ is regular iff IA $P$ is a prime ideal generated by a subset of a regular system of parameters of $A_{P}(c f .(17 . F)$ Th.36). We have $\operatorname{rank}_{K}\left(\mathrm{P} / \mathrm{P}^{2} \bigotimes_{\mathrm{A}} \mathrm{K}\right)=\mathrm{m}$, and $\operatorname{rank}_{K}\left(p / p^{2} \otimes_{B} K\right)=m-\operatorname{rank}_{K}\left(\left(P^{2}+I\right) / P^{2} \otimes_{A} K\right) \geqslant \operatorname{dim} B_{p}=m-r .$

\section{Therefore}
$$
\operatorname{rank}_{K}\left(\left(P^{2}+I\right) / P^{2} \otimes A^{K}\right) \leqslant r \text {, }
$$
and the equality holds iff $B_{p}$ is regular. The left hand side is the rank of the image of the natural map $v: I / I^{2} \otimes \mathrm{A}^{K} \rightarrow$ $P / P^{2} \otimes K$

To each polynomial $f(X) \varepsilon P$ we assign the vector in $K^{n}$ $\left(\partial f / \partial x_{1}, \ldots, \partial f / \partial x_{n}\right) \bmod P$. Then we get a $k$-IInear map $P / P^{2} \otimes A^{K} \rightarrow K^{n}$. If we identify $K^{n}$ with $\Omega_{A / k} \otimes_{A}^{K}=\Omega_{A_{P}} / k \otimes_{A_{P}} K$ $=\sum_{1}^{n} \mathrm{KdX}_{1}$, the map just defined is nothing but the map $\delta$ of the second fundamental exact sequence (cf.(26.I))
$$
\mathrm{P} / \mathrm{P}^{2} \otimes \mathrm{K}=\mathrm{PA}_{\mathrm{P}} / \mathrm{P}^{2} \mathrm{~A}_{\mathrm{P}} \rightarrow \Omega_{\mathrm{A}_{\mathrm{P}} / \mathrm{k}} \otimes \mathrm{K} \rightarrow \Omega_{\mathrm{K} / \mathrm{k}} \rightarrow 0
$$
If $I=\left(f_{1}(x), \ldots, f_{S}(X)\right)$, then the image of $\delta v: I / I^{2} \otimes K \rightarrow$ ${ }^{\Omega} \mathrm{A} / \mathrm{k} \otimes \mathrm{K}$ is generated by the vectors $\left(\partial \mathrm{f}_{\mathrm{i}} / \partial \mathrm{x}_{1}, \ldots, \partial \mathrm{f}_{\mathrm{i}} / \partial \mathrm{x}_{\mathrm{n}}\right) \bmod$

P, $1 \leqslant i \leqslant s$, so that $\operatorname{rank}_{K}(\operatorname{Im}(\delta v))=\operatorname{rank}\left(\partial\left(f_{1}, \ldots, f_{s}\right) /\right.$

$\left.\partial\left(X_{1}, \ldots, x_{n}\right) \bmod P\right)$, where the right hand side is the rank

of the Jacobian matrix evaluated at the point $P$; we write the

matrix $(\partial(f) / \partial(X))(P)$ for short. Thus, if we have

(*) $\operatorname{rank}\left(\partial\left(f_{1}, \ldots, f_{s}\right) / \partial\left(x_{1}, \ldots, x_{n}\right)\right)(P)=\ldots$

then we must have rank $\operatorname{Im}(\nu)=r$ also, and hence $B_{p}$ is re-

gular. When the residue field $k$ is separable over $k$ we have

$\operatorname{rank}_{K} R_{k} / k=t r \cdot d e g_{k} k=n-h t(P)=n-m$

by (27.B) Th.59, while rank $\mathrm{P} / \mathrm{P}^{2} \otimes k=\mathrm{m}$, So the map $\delta$ :

$\mathrm{P} / \mathrm{P}^{2} \otimes K+\Omega_{\mathrm{A}} / \mathrm{k}^{\otimes}$ is injective, In this case the condition

(*) is equivalent to the regularity of $B{ }^{\text {. }}$

The condition (*) is nothing but the classical definition

of a simple point, The above consideration shows that, when

$k$ is perfect, the point $p$ is simple on Spec(B) iff its local

ring $B_{p}$ is regular. In the general case note that $(*)$ is

invariant under any extension of the ground field $k$. Thus,

if $k^{\prime}$ denotes the algebraic closure of $k$ and if $P^{\prime}$ is a prime ideal of $A^{\prime}=k^{2}\left[X_{1}, \ldots, X_{n}\right]$ lying over $P$, then $p$ is simple on Spec(B) iff the local ring $B^{\prime} p^{\prime}=\left(A^{\prime} / I^{\prime}\right)_{P^{\prime}} / I A^{\prime}$ is regular. Since $k$ is finitely generated over $k$, it is also easy to see that $(*)$ is equivalent to the geometrical regularity of $B_{p}$ over $k$. (29.B) The results of the preceding paragraph can be more fully described by the notion of formal smoothness. We begin by proving lemmas.

LEMMA 1. Let $\mathbf{k} \rightarrow \mathrm{B}$ be a continuous homomorphism of topological rings and suppose $B$ is formally smooth over $k$. Then, for any open ideal $J$ of $B, \Omega_{B / k} \otimes(B / J)$ is a projective $B / J-$

(In such case we say that the B-module $\Omega_{B / k}$ is formally projective.)

Proof. Let $u: L+M$ be an epimorphism of $B / J$-modules. We have to prove that $\operatorname{Hom}_{B}\left(\Omega_{B} / k, L\right) \rightarrow \operatorname{Hom}_{B}\left(\Omega_{B} / k, M\right)$ is surjective, i.e. that $\operatorname{Der}_{k}(B, L) \rightarrow \operatorname{Der}_{k}(B, M)$ is surjective, Let D $\varepsilon \operatorname{Der}_{k}(B, M)$, and consider the commutative diagram

\includegraphics[max width=\textwidth]{2022_08_01_8d4eee36f1f42236b4f4g-116}

where $j(x, y)=(x, u y)$ and $v(b)=(b \bmod J, D(b))$. Let $v^{\prime}: B+(B / J) * L$ be a lifting of $v .$ Then we have $v^{\prime}(b)=$

(b mod $J, D^{\prime}(b)$ ) with a derivation $D^{\prime} \varepsilon \operatorname{Der}_{k}(B, L)$, and $u D^{\prime}=D$.

LEMMA 2. Let $B$ be a ring, $J$ an ideal of $B$ and $u: L \rightarrow M a$

homomorphism of $B$-modules. Suppose $M$ is projective. Further- more, assume either that $(\alpha) \mathrm{J}$ is nilpotent, or that (B) $\mathrm{L}$ is a finite $B$-module and $J \subseteq \operatorname{rad}(\mathrm{B})$. Then u is leftinvertible iff $\bar{u}: \mathrm{L} / \mathrm{JL} \rightarrow \mathrm{M} / \mathrm{JM}$ is so.

Proof. "Only $1 f^{\prime \prime}$ is trivial, so suppose $\bar{u}$ has a left-inverse $\bar{v}: M / J M \rightarrow L / J L$. Since $M$ is projective we can lift $\bar{v}$ to $v:$ $M \rightarrow L ;$ put $w=$ vu. Then $L=w(L)+J L$, hence $L=w(L)$ by NAK. Then $w$ is an automorphism. [In fact, it is generally true that a surjective endomorphism $f$ of a finite $B$-module $\mathrm{L}$ is an automorphism. Here is an elegant proof due to Vasconcelos: Let $B[T]$ operate on $L$ by $\mathrm{T} \xi=\mathrm{f}(\xi)$. Then $\mathrm{L}=\mathrm{TL}$, hence by NAK there exists $\phi(T) \varepsilon B[T]$ such that $(1+T \phi(I)) L$ $=0 ;$ then $\mathrm{T} \xi=0$ implies $\xi=0 .]$ Therefore $\mathrm{w}^{-1} \mathrm{v}$ is a leftinverse of $u$

THEOREM 63. Let $k$ and $A$ be topological rings

(cf. 28.B) and $g: k \rightarrow A$ a continuous homomorphism. Let $Q$ be an ideal of definition of A, let I be an ideal of A and put
$$
\mathrm{B}=\mathrm{A} / \mathrm{I}, \quad q=(\mathrm{Q}+\mathrm{I}) / \mathrm{I} .
$$
Suppose that $A$ is noetherian and formally smooth over $k$.

Then the following are equivalent:

(1) B (with the q-adic topology) is $f . s$. over $k$;

(2) the canonical maps $\delta_{n}:\left(I / I^{2}\right) \otimes_{B}\left(B / q^{n}\right) \rightarrow \Omega_{A / k} \otimes_{A}\left(B / q^{n}\right) \quad(n=1,2, \ldots)$ derived from the map $\delta: I / I^{2} \rightarrow \Omega_{A / k} \otimes B$ of Th.58 are leftinvertible;

(3) the map $\delta_{1}:\left(I / I^{2}\right) \otimes(B / q) \rightarrow \Omega_{A / k} \otimes(B / q)$ is leftinvertible. (When $q$ is a maximal ideal, this condition says simply that $\delta_{1}$ is injective.)

Proof. $(2) \Rightarrow(3)$ is trivial, while $(3) \Rightarrow(2)$ follows from the preceding lemmas. (2) $=.3$ (1) is easy and left to the reader. We prove $(1) \Rightarrow(2)$. Put $B_{n}=B / q^{n}$. The map $\delta_{n}$ is leftinvertible iff, for any $\mathrm{B}_{n}-$ module $N_{\text {, the }}$ induced map
$$
\operatorname{Hom}\left(I / I^{2}, N\right)+\operatorname{Der}_{k}(A, N)
$$
is surjective, So fix a $B_{n}$-module $N$ and a homomorphism $g \varepsilon$ $\operatorname{Hom}_{B}\left(I / I^{2}, N\right)$. Since A is noetherian there exists, by ArtinRees, an integer $v>n$ such that $I \cap Q^{\nu} \subseteq Q^{n} I$. Then $g$ induces $a \operatorname{map} g_{V}:\left(I+Q^{V}\right) /\left(I^{2}+Q^{V}\right) \rightarrow I /\left(I^{2}+\left(Q^{V} \cap I\right)\right)+I /\left(I^{2}+Q^{n} I\right)$ $+\mathrm{N}$, which is a homomorphism of $\mathrm{B}_{V}$-modules. Let $E$ denote the extension
$$
0 \rightarrow\left(I+Q^{V}\right) /\left(I^{2}+Q^{V}\right) \rightarrow A /\left(I^{2}+Q^{V}\right) \rightarrow B_{V} \rightarrow 0
$$
of the discrete $k$-algebra $\mathrm{B}{ }^{\prime}$, and let
$$
\mathrm{O} \rightarrow \mathrm{N} \rightarrow \mathrm{C} \rightarrow \mathrm{B}_{V} \rightarrow 0
$$
be the extension $g_{\nu^{*}}(E)(c f, 25 . E)$. The ring $C$ is a discrete $k$-algebra. Since $B$ is $f . s$. over $k$, there exists a continuous

\includegraphics[max width=\textwidth]{2022_08_01_8d4eee36f1f42236b4f4g-117}\\
homomorphism $v: \mathrm{B} \rightarrow \mathrm{C}$ such that is commutative. On the other hand, by the definition of $g_{U^{*}}(E)$ we have a canonical homomorphism of $\mathrm{k}$-algebras $\mathrm{u}$ : $A \rightarrow A /\left(I^{2}+Q^{\nu}\right) \rightarrow C$ such that

\includegraphics[max width=\textwidth]{2022_08_01_8d4eee36f1f42236b4f4g-118}

commutes. Denoting the natural map $A \rightarrow B=A / I$ by $r$, we get a derivation $D=u$ - vr $\varepsilon \operatorname{Der}_{k}(A, N)$. It is easy to check that $\mathrm{D}(\mathrm{x})=\mathrm{u}(\mathrm{x})=\mathrm{g}\left(\mathrm{x} \bmod \mathrm{I}^{2}\right)$ for $\mathrm{x} \varepsilon \mathrm{I}$. Q.E.D.

COROLLARY. If, in the notation of $T h .63, B$ is also $f . s$. over $k$, then the B-module $I / I^{2}$ is formally projective.

(29.D) LEMMA 3 (EGA $\left.O_{\text {IV }} 19.1 .12\right)$. Let $B$ be a ring, $L$ a finite B-module, $M$ a projective $B$-module and $u: L \rightarrow M$ a B-linear map. Then the following conditions on $p \varepsilon \operatorname{spec}(B)$ are equivalent, and the set of the points $p$ satisfying the conditions is open in $\operatorname{Spec}(B)$.

(1) $u_{p}: L_{p}=L \otimes B_{p} \rightarrow M_{p}=M \otimes B_{p}$ is left-invertible.

(2) there exist $x_{1}, \ldots, x_{m} \varepsilon L$ and $v_{1} \ldots, v_{m} \varepsilon \operatorname{Hom}_{B}(M, B)$ such that $L_{p}=\sum x_{i} B_{p}$ and $\operatorname{det}\left(v_{i}\left(u\left(x_{j}\right)\right)\right) \notin p$.

(3) there exists $f \varepsilon B-p$ such that $u_{f}: L_{f}=L \otimes B_{f}$ $\rightarrow \quad M_{f}=M \otimes B_{f}$ is left-invertible.

Proof. The module $M$ is a direct summand of a free B-module $F$.

Since I is finitely generated $u(\mathrm{~L})$ is contained in a free submodule $F^{\prime}$ of $F$ of finite rank which is a direct summand of $F$. Now the conditions (1), (2), (3) are not affected if we replace $M$ by $F$, and then $F$ by $F^{F}$. Therefore we may assume that $M$ is free of finite rank.

(1) $\Rightarrow$ (2): The assumption (1) implies that $L_{p}$ is $B_{p}$-projective, hence $B_{p}$-free. Let $x_{i} \varepsilon L(1 \leqslant i \leqslant m)$ be such that their images in $L_{p}$ (which are denoted by the same letters $x_{i}$ ) form a basis. Then $\left\{u_{p}\left(x_{1}\right), \ldots, u_{p}\left(x_{m}\right)\right\}$ is a part of a basis of $M_{p}$ ' so there exist linear forms $v_{i}^{1}: M_{p} \rightarrow B_{p}$ such that $v_{i}^{\prime}\left(u_{p}\left(x_{j}\right)\right)=\delta_{i j}$. Since $M$ is free of finite rank we can write $v_{i}^{\prime}=s_{i}^{-1} v_{i}, \quad s_{i} \varepsilon B-p, v_{f} \in \operatorname{Hom}_{B}(M, B) .$ Then $\operatorname{det}\left(v_{i}\left(u\left(x_{j}\right)\right)\right) \notin p$

(2) $\Rightarrow$ (3); Since $L$ is finite over $B$ and since $L_{p}=\sum_{p x_{i}{ }^{B}}^{p}$ it is easy to find $g \varepsilon B-p$ such that $L_{g}=\sum x_{i} B_{g}$. Put $d=$ $\operatorname{det}\left(v_{i}\left(u\left(x_{j}\right)\right)\right)$ and $f=g d$. Then $L_{f}=\sum x_{i}{ }^{B}{ }_{f}$, and $d$ is a unit in $\mathrm{B}_{\mathrm{f}}$. It follows that $\mathrm{M}_{\mathrm{f}}=\mathrm{u}_{\mathrm{f}}\left(\mathrm{L}_{\mathrm{f}}\right)+\mathrm{V}$ with $\mathrm{V}=$ $\cap \operatorname{Ker}\left(v_{i}\right)$. Moreover, $u\left(x_{i}\right)(1 \div i \leqslant m)$ are linearly independent over $B_{f}$, so that $u_{f}$ is injective. Thus $u_{f}$ is leftinvertible.

(3) $\rightarrow$ (1) : Trivial. Lastly, the set of the points $p$ which satisfy (3) is obviously open in $\operatorname{spec}(B)$. Q.E.D.

(29.E) THEOREM 64. Let $k$ be a ring, and A be a noetherian, smooth $k$-a1gebra. Let $I$ be an ideal of $A, B=A / I, P \varepsilon$ $\operatorname{Spec}(B), P=$ the inverse image of $p$ in $A, q=P \cap k$ and $k(p)$

$=$ the residue field of $B_{p}$ and $A_{P}$. Then the following are equivalent:

(1) $B_{p}$ is smooth over $k$ (or what amounts to the same, over $k_{q}$ );

(2) the local ring $B_{p}$ (with the topology as a local ring) is formally smooth over the discrete ring $k$ or $k_{q}$;

(2' $^{\prime}$ the local ring $B_{p}$ is $f_{0}$, over the local ring $k_{q}$;

(3) $\left(I / I^{2}\right) \otimes_{B} K(p) \rightarrow \Omega_{A / k} \otimes_{A} K(p)$ is injective;

(4) $\left(I / I^{2}\right) \otimes_{B} B_{P} \rightarrow \Omega_{A / K} \otimes_{A} B_{P}$ is left-invertible;

(5) there exist $F_{1}, \ldots, F_{r} \in I$ and $D_{1}, \ldots, D_{r} \in \operatorname{Der}_{k}(A, B)$ such that $\sum_{1} F_{1} A_{P}=I_{P}$ and $\operatorname{det}\left(D_{i} F_{j}\right) \notin p$;

(6) there exists $f \varepsilon B-P$ such that $B_{f}$ is smooth over $k$.

Consequently, the set $\left\{p \in \operatorname{Spec}(B) \mid B_{p}\right.$ is smooth over $\left.k\right\}$ is open in $\operatorname{Spec}(\mathrm{B})$.

Proof. $(1) \Rightarrow(2)$ : trivial. (2) $\Rightarrow\left(2^{\prime}\right)$ is also trivial (cf.

28.C). (2) $\Rightarrow(3)$ : we know that the local ring A $P$ is (smooth,

hence a fortiori) f.s, over $k$, and we have $B_{p}=A_{P} / I A_{P}$ and $\Omega_{\mathrm{A}} / \mathrm{k}=\Omega_{\mathrm{A} / \mathrm{k}} \otimes_{\mathrm{A}} \mathrm{A}_{\mathrm{P}}$. So apply Th.63.

(3) $\Rightarrow(4)$ : since $\Omega_{\mathrm{A} / \mathrm{k}}$ is A-projective by Lemma $1, \Omega_{\mathrm{A} / \mathrm{k}} \otimes \mathrm{B}_{P}$ is $B_{p}$-projective. Apply Lemma 2 .

(4) $\Rightarrow$ (5): apply Lemma 3 to the B-linear map $I / I^{2}+\Omega_{A / k} \otimes_{A}^{B}$. (5) $\Rightarrow(6)$ : by Lemma 3 and Th. $63 .$

$(6) \Rightarrow(1)$ : trivial.

Remark 1. The theorem has two important consequences. First, if, in the theorem, $k$ is a field, then $A$ is smooth over the prime field $k_{0}$ in $k$ also, and $B_{p}$ is smooth over $k_{0}$ iff it is regular. Therefore the set $\left\{p \mid B_{p}\right.$ is regular $\}$ is open in $\operatorname{Spec}(\mathrm{B})$

Secondly, let $k$ be a noetherian ring and $B$ a k-algebra of finite type. Then $B_{p}(p \varepsilon \operatorname{Spec}(B))$ is smooth over $k$ iff it is $f . s$. over $k$. In fact $B$ is of the form $A / I, A=k\left[X_{1}\right.$, $\left.\ldots, x_{n}\right]$, so we can apply the theorem.

Remark 2. When the conditions of Th.64 hold, the number $r$ of (5) is equal to the height of IA ${ }^{\text {. }}$.

(29.F) Nagata gave a similar Jacobian criterion for rings of the form $B=k\left[\left[x_{1}, \ldots, x_{n}\right]\right] / I$, where $k$ is a field (IIl. J. Math. vol.1 (1957), 427-432). By lack of space we just quote the main result in the form found in EGA:

THEOREM (cf. EGA $0_{\text {IV }}$ 22.7.3). Let $k$ be a field, and let $(A, M, K)$ be a noetherian complete local ring. Let I be an ideal of $\mathrm{A}, \mathrm{B}=\mathrm{A} / \mathrm{I}, \mathrm{P}$ a prime ideal containing $\mathrm{I}$ and $P=$ P/I. Suppose that

(1) $\left[k: k^{p}\right]<\infty$ if $\operatorname{ch}(k)=p>0$ (2) $K$ is a finite extension of a separable extension $k_{0}$ of $k$, and

(3) A has a structure of a formally smooth $\mathrm{K}_{0}$-algebra.

Then the local rin $\operatorname{rin}_{P}$ is $f_{.}$. over $k$ iff there exist $F_{1}, \ldots$,

$F_{m} \varepsilon I$ and $D_{1}, \ldots, D_{m} \varepsilon \operatorname{Der}_{k}(A)$ such that $I_{P}=\sum F_{i} A_{P}$ and such that $\operatorname{Det}\left(\mathrm{D}_{i}\left(F_{j}\right)\right) \notin P$.

COROLLARY (cf. EGA ${ }_{\text {IV }}$ 22.7.6). Let B be a noetherian complete local ring containing a field. Then the set

$\left\{p \in \operatorname{Spec}(B) \mid B_{p}\right.$ is regular $\}$ is open in $\operatorname{Spec}(B) .$

\section{Formol Smoothness II}
\includegraphics[max width=\textwidth]{2022_08_01_8d4eee36f1f42236b4f4g-120}\\
isms of topological rings (cf. 28.B). We say that $A$ is formally smooth over $k$ relative to $\Lambda$ (f.s. over $k$ rel. 1 , for short) if, given any commutative diagram

\includegraphics[max width=\textwidth]{2022_08_01_8d4eee36f1f42236b4f4g-120(1)}

where $C$ and $C / N$ are discrete rings, $N$ an ideal of $C$ with $N^{2}$ $=0$ and the homomorphisms are continuous, the map $v$ can be lifted to a k-algebra homomorphism $A \rightarrow C$ whenever it can be lifted to a $\Lambda$-algebra homomorphism $A \rightarrow C$. THEOREM 65. Let $\Lambda \rightarrow \mathrm{g} \underset{\mathrm{f}}{\rightarrow} \mathrm{A}$ be as above. Then the following are equivalent:

(1) A is $f . s$, over $k$ rel. $A$;

(2) for any A-module $N$ such that $I N=0$ for some open ideal I of $A$, the map $\operatorname{Der}_{A}(A, N) \rightarrow \operatorname{Der}_{A}(k, N)$ induced by f is surjective;

(3) $\Omega_{k} / \AA_{k}(\mathrm{~A} / \mathrm{I})+\Omega_{\mathrm{A} / \Lambda} \otimes_{\mathrm{A}}(\mathrm{A} / \mathrm{I})$ is left-invertible for any open ideal I of $A$.

Proof. $(1) \Rightarrow(2):$ Put $C=(A / I) * N, \quad$ take $D \varepsilon \operatorname{Der}_{\Lambda}(k, N)$ and define $i: k \rightarrow C$ by $i(\alpha)=(v f(\alpha), D(\alpha))(\alpha \varepsilon k)$ where $v: A \rightarrow A / I$ is the natural map. Then $v$ can be lifted to the $\Lambda$-homomorphism a $\longrightarrow(v(a), 0) \varepsilon C$, hence it can also be lifted to a k-homomorphism a $\mapsto\left(v(a), D^{\prime}(a)\right)$, and then $D^{\prime}: A \rightarrow N$ is a derivation satisfying $D=D^{\prime} f \cdot(2) \Rightarrow(1)$ is also easy, and $(2) \leftrightarrow(3)$ is obvious.

(30. B) THEOREM 66. Let $\Lambda \rightarrow \mathrm{k} \rightarrow \mathrm{A}$ be as above, let $\mathrm{J}$ be an ideal of definition of $A$ and suppose A is formally smooth over $A$. Then $A$ is $f . s$, over $k$ iff
$$
\Omega_{k / \Lambda} \otimes_{k}(A / J) \rightarrow \Omega_{A / \Lambda} \otimes_{A}(A / J)
$$
is left-invertible.

Proof. By assumption, $A$ is $f . s$. over $k$ iff it is $f . s$. over $k$ re1. $\Lambda$. On the other hand, for any open ideal I of $A$ the A/I-module $\Omega_{\mathrm{A} / \Lambda} \otimes(\mathrm{A} / \mathrm{I})$ is projective by (29.B) Lemma 1 .

Thus the condition (3) of the preceding theorem is equivalent

to the present condition by (29.B) Lemma $2 .$

COROLLARY. Let $(\mathrm{A}, M, \mathrm{~K})$ be a regular local ring containing

a field $k$. Then $A$ is f.s. over $k$ iff
$$
\Omega_{\mathrm{k}} \otimes_{\mathrm{k}} \mathrm{K} \rightarrow \Omega_{\mathrm{A}} \otimes_{\mathrm{A}} \mathrm{K}
$$
is injective.

Proof. Since $A$ is f.s. over the prime field in $k$, the assertion follows from the theorem.

(30.C) LEMMA 1. Let $k$ be a field of characteristic p. Let $F=\left\{k_{\alpha}\right\}$ be a family of subfields of $k$, directed downwards (i.e.for any two members of $F$ there exists a third which is contained in both of them), such that $k^{p} \subseteq k_{\alpha} \subseteq k, \bigcap_{\alpha} k_{\alpha}=k^{p}$. Let $u_{\alpha}: \Omega_{k} \rightarrow \Omega_{k} / k$ be the canonical homomorphisms.

Then $\bigcap_{\alpha} \operatorname{Ker}\left(u_{\alpha}\right)=(0)$.

Proof. Let $\left(x_{i}\right)$ be a p-basis of $k$. Then $\Omega_{k}$ is a free $k-$ module with $\left(\mathrm{d} \mathrm{x}_{1}\right)$ as a basis. Suppose that $0 \neq \sum_{i}^{n} c_{i} \mathrm{~d} \mathrm{x}_{1} \varepsilon$ $\bigcap_{\alpha} \operatorname{Ker}\left(u_{\alpha}\right)$. Then the monomials $\left\{x_{1}{ }^{1} \ldots x_{n}{ }_{n} \mid 0 \leqslant v_{1}<p\right\}$ must be linearly dependent over $k_{\alpha}$ for all $\alpha$. But since they are linearly independent over $k^{p}$ and since $\cap_{\alpha}=k^{p}$, it is easily seen that they are linearly indep. over some $k_{\alpha}$. THEOREM 67. Let $(A, M, K)$ be a regular local ring containing a field $k$ of characteristic p. Let $F=\left\{k_{\alpha}\right\}$ be as in the above lemma. Then A is f.s. over $k$ iff A is f.s. over k rel. $k_{\alpha}$ for all $\alpha$

Proof. "On1y-if" is trivial. Conversely, suppose the condition holds, and look at the commutative diagram

\includegraphics[max width=\textwidth]{2022_08_01_8d4eee36f1f42236b4f4g-121}

Here $w_{\alpha}$ is injective by Th.65 and $u_{\alpha}^{\prime}=u_{\alpha} \otimes 1_{K}$. Thus Ker(w) $\subseteq \bigcap \operatorname{Ker}\left(u_{\alpha}^{\prime}\right)=\left(\cap \operatorname{Ker}\left(u_{\alpha}\right)\right) \otimes K=(0)$

(30.D) THEOREM 68 (Grothendieck). Let A be a noetherian complete local ring and $P$ a prime ideal of $A ;$ put $B=A_{p}$ and let $B^{*}$ denote the completion of $B$. Let $q^{\prime} E \operatorname{Spec}(B)$ and put $L=k\left(q^{\prime}\right)=B_{q^{\prime}} / q^{\prime} B q^{\prime}$ Then, for any prime ideal $Q$ of $B^{*}$ lying over $q^{\prime}$, the 'local ring of $Q$ on the fibre' $B^{*}{ }_{Q}{ }_{B}{ }^{L}$ $={ }^{B}{ }_{Q} / Q^{\prime} B_{Q}^{*}(\mathrm{cf} \cdot 21 . \mathrm{A})$ is formally smooth (hence geometrically regular) over L.

Proof. Step I. Put $q=q^{\prime} \cap \mathrm{A}, \overline{\mathrm{A}}=\mathrm{A} / q, \quad \overline{\mathrm{B}}=\mathrm{B} / q \mathrm{~B}=\mathrm{B} / q^{\prime}$, $\vec{B} *=($ the completion of the local ring $\bar{B})=B * / q^{\prime} B^{*}$ and $\bar{Q}=Q / Q^{\prime} B^{*}$. Then the 'local ring of $Q$ on the fibre' remains the same when we replace $A, B, B *, Q$ by $\bar{A}, \bar{B}, \bar{B} *, \bar{Q}$ respectively. Thus we may assume that $\mathrm{A}$ is an integral domain and $Q \cap B=q^{\prime}=(0)$.

Step II (Reduction to the case that $B$ is regular). Take a complete regular local ring $\mathrm{R} \subseteq \mathrm{A}$ over which $\mathrm{A}$ is finite. Put $P_{0}=P \cap R, \quad S=R_{P_{0}}$ and $B^{\prime}=A_{P_{0}}$. Then $B^{\prime}$ is finite over $S$, and $B=A_{p}$ is a localization of the semi-local ring $B^{\prime}$ by a maximal idea1. Hence $B^{*}$ is a localization (and a direct factor) of $\mathrm{B}^{\prime *}=\mathrm{B}^{\prime} \otimes_{\mathrm{S}^{\mathrm{S}}}$. Let $L($ resp. $K$ ) be the quotient field of $A, B^{\prime}$ and $B$ (resp. $R$ and $S$ ).

\includegraphics[max width=\textwidth]{2022_08_01_8d4eee36f1f42236b4f4g-122}

We are given $Q . E \operatorname{Spec}\left(B^{*}\right)$ such that $Q \cap B=(0)$. Then $B^{*}{ }_{Q}$ is a localization of $L \otimes_{B^{\prime}} B^{\prime *}=\mathrm{L} \otimes_{S^{*}} S^{*}=L_{K}\left(K \otimes_{S} S^{*}\right)$, and $\mathrm{L}$ is a finite extension of the field $K$. In general if $T$ is a $\mathrm{K}$-algebra, if $M \in \operatorname{Spec}\left(\mathrm{L} \otimes_{\mathrm{K}} \mathrm{T}\right)$ and $m=M \simeq T$, and if $\mathrm{T}_{m}$ is I.s. over $K$, then $(L \otimes T)_{M}$ is a localization of $L_{K} K_{m}$ and hence is $f . s$, over $L$. Thus it suffices to show that $S^{*} Q \cap S^{*}$ is f.s. over $K$. Thus the problem is reduced to proving that, if $R$ is a complete regular local ring with quotient field $k$, if $P \varepsilon \operatorname{Spec}(R)$ and $S=R_{p}$, and if $Q$ is a prime ideal of $S^{*}$ such that $Q \cap S=(0)$, then $S^{*} Q^{\text {is }} f . S$. over $K$. Step III. The local ring $S_{*}$ is regular, so if $\operatorname{ch}(K)=0$ we are done. If $\operatorname{ch}(K)=p$ we apply the preceding theorem. In this case $R$ is an equicharacteristic complete regular local ring, hence $R=k\left[\left\{x_{1}, \ldots, X_{n}\right]\right]$ for some subfield $k$ of $R$. Let $\left\{k_{\alpha}\right\}$ be the family of all subfields $k_{\alpha}$ of $k$ such that $\left[k: k_{\alpha}\right]<\infty$ and $k^{p} \subseteq k_{\alpha} \subseteq k$. Put $R_{\alpha}=k_{\alpha}\left[\left[\mathrm{x}_{1}{ }^{p}, \ldots, \mathrm{X}_{n}{ }^{\mathrm{p}}\right]\right]$, $p_{\alpha}=\mathrm{R}_{\alpha} \cap p, \quad \mathrm{~S}_{\alpha}=\left(\mathrm{R}_{\alpha}\right)_{p_{\alpha}}$ and $\mathrm{K}_{\alpha}=\Phi \mathrm{R}_{\alpha}=\mathrm{k}_{\alpha}\left(\left(\mathrm{X}_{1} \mathrm{P}, \ldots, \mathrm{X}_{\mathrm{n}}^{\mathrm{P}}\right)\right)$ Then $\bigcap_{\alpha} k_{\alpha}=k^{p}$, hence it is elementary to see that $\bigcap_{\alpha} k_{\alpha}=K^{p}$ (see below). By the preceding theorem we have only to show that, for each $\alpha, S^{*}{ }_{Q}$ is $f, s$, over $K$ rel. $K_{\alpha}$.

Since $R^{P} \subseteq R_{\alpha} \subseteq R, P$ is the only prime Ideal of $R$ lying over $p_{\alpha}$. Hence $S=R_{p}=R_{p}=R \otimes_{R_{\alpha}} S_{\alpha}$, and so $S$ is finite over $S_{\alpha}$. Therefore $S^{*}=S^{*} S_{\alpha} S_{\alpha}^{*}$ Suppose we are given diagram

\includegraphics[max width=\textwidth]{2022_08_01_8d4eee36f1f42236b4f4g-122(1)}

where $\mathrm{N}^{2}=(0)$ and $u$ and $v$ are homomorphisms, and a lifting $v^{\prime}: s^{*} \rightarrow C$ of $v$ over $s_{\alpha} \cdot$ Put $v^{*}=v^{\prime} \mid s_{\alpha^{*}}$ and $v^{\prime \prime}=u \otimes v^{*}$ : $\mathrm{S}^{*}=\mathrm{S}_{\mathrm{S}_{\alpha}} \mathrm{S}_{\alpha}^{*} \rightarrow \mathrm{C} .$ Then $\mathrm{v}^{\prime \prime}$ is a lifting of $v$ over $\mathrm{S}^{\prime}$ Thus $S^{*}$ is formally smooth over $S_{\text {rel. }} S_{\alpha}$ with respect to the discrete topology. Then it follows immediately from the definition that $S^{*}{ }_{Q}$ is $f_{0} s_{\text {o over }} k$ rel. $K_{\alpha}$ as a discrete ring, hence a fortiori as a local ring. Q.E.D. (30.E) A Digression. Let $A$ be a ring and $M$ an A-module.

We say that $M$ is injectively free if, for any non-zero element $x$ of $M$, there exists a linear form $f \in \operatorname{Hom}_{A}(M, A)$ with $f(x) \neq$ 0 (in other words, if the canonical map from $M$ to its double dual is injective)

LEMMA 2. Let $B$ be an A-algebra which is injectively free as A-module. Then $B\left[X_{1}, \ldots, X_{n}\right]$ (resp. $B\left[\left[X_{1}, \ldots, X_{n}\right]\right]$ ) is injectively free over $A\left[X_{1}, \ldots, X_{n}\right]$ (resp. $A\left[\left[X_{1}, \ldots, X_{n}\right]\right]$ ).

Proof. Just extend a suitable A-1inear map $\ell: B \rightarrow A$ to $B\left[X_{1}, \ldots, X_{n}\right]$ (resp...) by letting it operate on the coefficients.

LEMMA 3. Let $A \subset B$ be integral domains, and suppose $B$ is injectively free over $A$. Let $K$ and $L$ be the quotient fields of $A$ and $B$ respectively, and $X$ be an indeterminate. Then
$$
\Phi(B[[X]]) \cap K((X))=\Phi(A[[X]]) .
$$
Proof. 2 is trivial. To see $\subseteq$, let $\xi \varepsilon \Phi(B[[X]]) \cap K((X))$. As an element of $K((X)$ ) we can write (the Laurent expansion)
$$
\xi=x^{m}\left(r_{0}+r_{1} x+r_{2} x^{2}+\ldots\right), \quad m \varepsilon Z, \quad r_{i} \varepsilon k_{0}
$$
We may assume $m \neq 0$. Since $\xi \varepsilon \Phi(B[[X]])$ there exists $0 \neq \phi$ $\varepsilon B[[X]]$ such that $\phi \xi=\psi \varepsilon B[[X]]$. Write
$$
\phi=\sum_{0}^{\infty} \alpha_{1} \mathrm{X}^{i}, \quad \psi=\sum_{0}^{\infty} \beta_{k} \mathbf{X}^{\mathbf{k}}, \quad \alpha_{i}, \beta_{j} \in B .
$$
Then $\sum_{i+j=k} \alpha_{i}{ }^{i} j=B_{k}$. Take a linear map $\ell: B \rightarrow A$ with $\ell\left(\alpha_{i}\right)$ $\neq 0$ for some 1 . Then $\sum_{f+j} \ell\left(\alpha_{1}\right) r_{j}=\beta_{k}$. Writing $\ell(\phi)=$ $\Sigma \ell\left(\alpha_{1}\right) x^{1}$ and $\ell(\psi)=\sum \ell\left(\beta_{k}\right) x^{k}$ we therefore get $\ell(\phi) \neq 0$ and $\xi=\ell(\psi) / \ell(\phi) \varepsilon \Phi(\mathrm{A}[\mathrm{X}]])$

PROPOSITION. Let $k$ be a fleld and $\left\{k_{\alpha}\right\}$ a family of subfields of $k$. Put $k_{0}=\bigcap_{\alpha}$. Then we have
$$
\bigcap_{\alpha} k_{\alpha}\left(\left(x_{1}, \ldots, x_{n}\right)\right)=k_{0}\left(\left(x_{1}, \ldots, x_{n}\right)\right)
$$
Proof. When $n=1$, the uniqueness of the Laurent expansion proves the assertion. Induction on $\mathrm{n}$. Put
$$
\begin{array}{ll}
A=k_{0}\left[\left[x_{1}, \ldots, X_{n-1}\right]\right], & B_{\alpha}=k_{\alpha}\left[\left[X_{1}, \ldots, X_{n-1}\right]\right], \\
K=\Phi A=k_{0}\left(\left(x_{1}, \ldots, X_{n-1}\right)\right), & L_{\alpha}=\Phi B_{\alpha}=k_{\alpha}\left(\left(x_{1}, \ldots, X_{n-1}\right)\right) .
\end{array}
$$
Then we have
$$
\bigcap_{\alpha} k_{\alpha}\left(\left(x_{1}, \ldots, x_{n}\right)\right) \subseteq \bigcap_{\alpha}\left(\left(x_{n}\right)\right)=\left(\bigcap_{\alpha} L_{\alpha}\right)\left(\left(x_{n}\right)\right)=k\left(\left(x_{n}\right)\right)
$$
by the induction hypothesis, whence
$$
\begin{aligned}
\bigcap_{\alpha} k_{\alpha}\left(\left(x_{1}, \ldots, x_{n}\right)\right) & \subseteq k_{\alpha}\left(\left(x_{1}, \ldots, x_{n}\right)\right) \cap K\left(\left(x_{n}\right)\right) \\
&=\Phi\left(B_{\alpha}\left[\left[x_{n}\right]\right]\right) \cap K\left(\left(x_{n}\right)\right) \\
&=\Phi\left(A\left[\left[x_{n}\right]\right]\right) \\
&=k_{0}\left(\left(x_{1}, \ldots, x_{n}\right)\right) .
\end{aligned}
$$

\section{Nagata Rings}
(31.A) 

\section{Closedness of the Singular Locus}
(32. A) Let $A$ be a noetherian ring; put $X=\operatorname{Spec}(A), \operatorname{Reg}(X)$

$=\left\{p \varepsilon \times \mid A_{p}\right.$ is regular $\}$ and $\operatorname{Sing}(X)=X-\operatorname{Reg}(X)$. We ask whether $\operatorname{Reg}(\mathrm{X})$ is open in $\mathrm{X}$.

LEMMA 1. In order that $\operatorname{Reg}(X)$ is open in $X$,

(1) it is necessary and sufficient that for each $p \in \operatorname{Reg}(X)$, the set $V(p) \cap \operatorname{Reg}(X)$ contains a non-empty open set of $V(p)$; and (ii) it is sufficient that, if $p \varepsilon \operatorname{Reg}(X)$ and $Y=\operatorname{Spec}($ $A / p)$, then $\operatorname{Reg}(Y)$ contains a non-empty open set of $Y$.

Proof. (i) This follows from (22.B) Lemma 2.

(i1) We derive the condition of (1) from (ii). Let $p E$ $\operatorname{Reg}(X)$, and choose $a_{1}, \ldots, a_{r} \in p$ which form a regular system of parameters of $A_{p}$; put $I=\sum_{i} A$. As $I A_{p}=p A_{p}$, there exists $f \varepsilon A$ such that $I A_{f}=p A_{f} . \quad$ Then $D(f)=X-V(f) \simeq \operatorname{Spec}\left(A_{f}\right)$ is an open neighborhood of $P$ in $X$. So, replacing $A$ by $A_{f}$ we may assume that $I=p$. Now put $Y=\operatorname{Spec}(A / p)$, and identify it with the closed subset $V(p)$ of $X$. By assumption, there exists a non-empty open set $Y_{0}$ of $Y$ contained in $\operatorname{Reg}(Y)$. If $q \varepsilon Y_{0^{\prime}}$ then $A_{q} / p A_{q}$ is regular and $p A_{q}=\sum_{1} a_{i} A_{q}$ is generated by an $A_{q}-$ regular sequence. Thus $\operatorname{dim} A_{q}=\operatorname{dim~} A_{q} / p A_{q}+r$, so that $A_{q}$ is regular. Therefore $Y_{0} \subseteq Y \wedge \operatorname{Reg}(X)$, and the condition (i) is proved.

(32. B) Let $A$ be a noetherian ring. We say that $A$ is $J-0$ if Reg(Spec(A)) contains a non-empty open set of Spec(A), and that $A$ is $J-1$ if $\operatorname{Reg}(\operatorname{Spec}(A))$ is open in $\operatorname{Spec}(A) .$ Thus $J-1$ implies J-0 if $A$ is a domain, but not in general. We say that A is J-2 if the conditions of the following theorem are satisfied.

THEOREM 73. For a noetherian:ring $A$, the following conditions are equivalent:

(1) any finitely generated A-algebra B is J-1;

(2) any finite A-algebra $B$ is $J-1$;

(3) for any $p \varepsilon \operatorname{Spec}(A)$, and for any finite radical extension $K^{\prime}$ of $K(P)$, there exists a finite A-algebra $A^{\prime}$ satisfying $A / p \subseteq A^{\prime} \subseteq K^{\prime}$ which is $J-0$ and whose quotient field is $K^{\prime}$. Proof. $(1) \Rightarrow(2) \Rightarrow(3):$ trivia1. $(3) \Rightarrow(1)$ : Step I. Let $p$ and $A^{\prime}$ be as in $(3)$, and 1 et $w_{1}, \ldots, \omega_{n} \in A^{\prime}$ be a 1 inear basis of $K^{\prime}$ over $K(P)$. Then there exists $0 \neq f \varepsilon A / P$ such that $A^{\prime}=\sum^{n}$ $(A / p) w_{i}$. From this and from Th.51 (i) it follows easily that $A / p$ is $J-0$. Therefore $A / p$ (and $A$ itself) is $J-1$ by Lemma 1 .

Step II. In view of Lemma 1, the condition (1) is equivalent to (1'): Let B be a domain which is finitely generated over $A / p$ for some $p \in \operatorname{Spec} A$. Then $B$ is $\mathrm{J}-0$.

We will prove $\left(1^{\prime}\right)$, Replacing A by $A / P$ we may assume $A \subseteq B$. Since A is $J-0$ by Step I we may also assume that A is regular. Let $K$ and $K^{\prime}$, be the quotient fields of $A$ and $B$ respectively.

Case 1. $K^{\prime}$ is separable over $K$. In this case we use only the assumption that $A$ is regular. Let $t_{1}, \ldots, t_{n} \in B$ be a separating transcendency basis of $K^{\prime}$ over $K$, and put $A_{1}=$ $A\left[t_{1}, \ldots, t_{n}\right], k_{1}=K\left(t_{1}, \ldots, t_{n}\right)$. Then $A_{1}$ is a regular ring. There exists a basis $\omega_{1}, \ldots, w_{r}$ of $K^{\prime}$ over $K_{1}$ such that each $\omega_{i} \varepsilon B_{0}$ Replacing A by some $\left(A_{1}\right)_{f}\left(f \in A_{1}\right)$ and $B$ by $B_{f}$, we may assume $B$ is finite and free over $A: B=\sum_{1} A \cdot$ Put $d=$ $\operatorname{det}\left(t r_{K} \prime / K\left(\omega_{i} \omega_{j}\right)\right)$. Then $d \neq 0$ as $K^{\prime}$ is separably algebraic over $K$. We claim that $B_{d}$ is a regular ring. Indeed, if $d \notin p^{\prime} \varepsilon \operatorname{Spec}(B)$ and $p=p^{\prime} \cap A$, then $B_{p}=\sum_{1} \sum \omega_{i} A_{p}$, and putting $\bar{B}=B \otimes K(p)=\sum_{1} \bar{\omega}_{i} K(p)$ we get $\operatorname{det}\left(t r_{B} / K(p)\left(\bar{\omega}_{i} \bar{\omega}_{j}\right)\right)$ fields, and so $B_{p}, \otimes K(p)=B_{p}, / p B_{p}$, is a field. Since $A_{p}$ is regular and dim $A_{p}=\operatorname{dim} B_{p}$, it follows that $B_{P}$ is regular.

Case 2. General case. We may suppose $c h(K)=p$. There exists a finite purely inseparable extension $K_{1}$ of $K$ such that $K_{1}^{\prime}=K^{p}\left(K_{1}\right)$ is separable over $K_{1}$. Choose $A_{1} \leqslant K_{1}$ as in

(3). Then $A_{1}$ is $J-0$, and so $A_{1}[B]$ is $J-0$ by Case 1. Since $A_{1}[B]$ is finite over $B, B$ itself is $J-0$ as in Step I. Q.E.D.

Remark. The condition (3) is satisfied if A is a Nagata ring of dimension 1. Indeed, $A / p$ is either a field - in which

case (3) is trivial - or a Nagata domain of dimension 1 , and then the integral closure $A^{\prime}$ of $A$ in $K^{\prime}$ is finite over $A$ and is a regular ring.

(32.C) THEOREM 74. Let A be a noetherian complete local

ring. Then A is J-2.

Proof. Any finite A-algebra B is a finite product of complete local rings: $B=B_{1} \times \ldots \times B_{s}$, and $B$ is $J-1$ iff each $B_{i}$ is

so. Therefore, by Th.73 and Lemma 2, it suffices to prove

that a noetherian complete local domain A is $\mathrm{J}-0$.

Case I. $\operatorname{ch}(\mathrm{A})=0$. The ring $A$ is finite over a suitable subring B which is a regular local ring, and by the case 1

of Step II of the preceding proof we see that $A$ is $J-0$.

Case II. $\operatorname{ch}(A)=p$, Then A contains the prime field, hence also a coefficient field $k$, so that $A$ is of the form $K\left[\left[X_{1}, \ldots, X_{n}\right]\right] / I$. Therefore $A$ is $J-1$ by the Jacobian criterion of Nagata $(29 . F)$.

\section{Formal Fibres and G-Rings}


\section{Excellent Rings}
(34.A) DEFINITION. We say that a ring A is excellent (resp. quasi-excellent) if the following conditions (resp. (1), (3) and (4)) are satisfied:

(1) A is noetherian;

(2) A is universally catenary (cf. pp.84-86);

(3) A is a G-ring (cf. 33.B);

(4) A is $\mathrm{J}-2$ (cf. 32.B Th.73).

Each of these conditions is stable under the two important operations on rings: the localization and the passage to a finitely generated algebra. (Stability of $J-2$ under localization follows from the criterion (3) of Th.73.) Thus the class of (quasi-)excellent rings is stable under these operations. Note also that $(2),(3),(4)$ are conditions on $\mathrm{A} / \mathrm{P}$, $P \varepsilon \operatorname{Spec}(\mathrm{A})$. Thus a noetherian ring $A$ is (quasi-)excellent iff A red is so.

A quasi-excellent ring is a Nagata ring (Th.78).

If A is a local ring and if it satisfies (1) and (3) then it is quasi-excellent (Th.76, Th.77 and Th.73). In the general case, note that the conditions (2) and (3) are of local nature (in the sense that if they hold for $A_{p}$ for all $p \varepsilon \operatorname{Spec}(A)$, then they hold for $A$ ), while (4) is not. (34.B) Noetherian complete semi-local rings are excellent $((28 . P), T h .68, T h .74)$. In particular, formal power series rings over a field are excellent. Convergent power series rings over $R$ or $C$ are excellent (cf. Th.102 and the remark after that). It is easy to see that a Dedekind domain (i.e. noetherian normal domain of dimension one) of characteristic zero is excellent. On the other hand, there exists a regular local ring of dimension one and of characteristic $p$ which is not excellent. [Take a field $k$ of char. $p$ with $\left[k: k^{p}\right]=\infty$, put $R=k[[x]]$ and let $A$ be the subring of $R$ consisting of the power series $\sum a_{1} x^{1}$ such that $\left[k^{p}\left(a_{0}, a_{1}, \ldots\right): k^{p}\right]<\infty$. Then $A$ is regular and $A^{*}=R$. Since $R^{P} \subseteq A$ the quotient field $\Phi \mathrm{R}$ is purely inseparable over $\Phi \mathrm{A}$. Thus $A$ is not a Gring, not even a Nagata ring by Th.71.]

Let $K$ be a field, $\operatorname{ch}(K) \neq 2$. Then there exist a regular local ring $R$ of dimension 2 containing $K$ and a prime element $z$ of $R$ such that $S=R\left[z^{1 / 2}\right]$ is a normal local ring whose completion $S *$ has zero-divisors. (Nagata, LOCAL RINGS p.210, $(E 7.1))$. Thus $R$ is not Nagata.

C. Rotthaus (Math. Z. 152 (1977), 107-125) constructed a regular local ring $R$ of dimension three which contains a field and which is Nagata but not quasi-excellent.

The ring A of p.88 is a G-ring which is not u.c..

(34.C) One can ask the following questions:

(A) If $A$ is quasi-excellent, is $A[X]]$ quasi-excellent ?

(A') If $A$ is as above and $I$ is an ideal, is the I-adic completion $A^{*}$ of $A$ qusi-excellent ?

(B) If $(\mathrm{A}, \mathrm{I})$ is a complete Zariski ring with $\mathrm{A} / \mathrm{I}$ quasiexcellent, is A also quasi-excellent ?

Of course (A) and ( $A^{\prime}$ ) are equivalent, and (B) is stronger. These questions are still open in the general case, cf. $\$ 43$. Appendix

\begin{enumerate}
  \setcounter{enumi}{35}
  \item Eakin's Theorem

  \item A Flatness Theorem

  \item Coefficient Rings

  \item p-Basis

  \item Cartier's Equality and Geometric Regularity

  \item Jacobian Criteria and Excellent Rings

  \item Krul1 Rings and Marot's Theorem

  \item Kunz' Theorems

  \item Complement

\end{enumerate}
\section{Eakin's Theorem}
A module is said to be noetherian (resp. artinian) if the ascending (resp. descending) chain condition for submodules holds. It is easy to see that if $0 \rightarrow M^{\prime} \rightarrow M \rightarrow M^{11}$ $\rightarrow 0$ is exact and if $M^{\prime}$ and $M^{\prime \prime}$ are noetherian (resp. artinian), so is M. A module is noetherian iff all submodules are finitely generated.

A module is called faithful if Ann $(M)=(0)$.

LEMMA. Let A be a ring and $M$ an A-module. If $M$ is faithful and noetherian, then $A$ is a noetherian ring.

Proof. Let $M=A \omega_{1}+\ldots+A \omega_{n} \cdot$ Then $A$ is embedded in $M^{n}$ as A-module by the map $a \rightarrow\left(a \omega_{1}, \ldots, a \omega_{n}\right)$. Since $M^{n}$ is noetherian, so is A. THEOREM 80 (E. Formanek, Proc. AMS $41(1973), 381-383) .$

Let $A$ be a ring and $B$ be a faithful and finite A-module. If the ascending chain condition holds for the submodules of the form IB, where I is an ideal of $A$, then $A$ is noetherian.

Proof. It suffices to prove that $B$ is a noetherian A-module.

Assume the contrary. Then the set $\{\mathrm{IB} \mid$ I is an ideal of A and B/IB is a non-noetherian A-module\} is not empty, hence it has a maximal element $\mathrm{I}_{0} \mathrm{~B} \cdot$ Replacing $\mathrm{B}$ and $\mathrm{A}$ by $\mathrm{B} / \mathrm{I}_{0} \mathrm{~B}$ and $A / A n n\left(B / I_{0} B\right)$, we may assume that $B$ is not noetherian but $B / I B$ is noetherian for every non-zero ideal I of A. Put $\Gamma=$ $\{N \mid N$ is a submodule of $B$ and $B / N$ is faithful $\}$. If $B=A \omega_{1}$ $+\ldots+A \omega_{n}$ then a submodule $N$ of $B$ belongs to $\Gamma$ iff $\left\{a \omega_{1}\right.$, $\left.. ., a w_{n}\right\} \& N$ for every $0 \neq$ a $\varepsilon$ A. Therefore we can use Zorn to conclude that $\Gamma$ has a maximal element $\mathrm{N}_{0} \cdot$ Replacing $\mathrm{B}$ by $\mathrm{B} / \mathrm{N}_{0}$ we get the situation where (1) $\mathrm{B}$ is not noetherian (for, otherwise A and our original B would be noetherian), (2) B/IB is noetherian for every non-zero ideal I of $A$, and (3) $B / N$ is not faithful for every non-zero submodule $\mathrm{N}$ of $\mathrm{B} .$ But this is absurd. In fact, there exists by (1) a submodule $\mathrm{N}$ of $B$ which is not finite over $A$. Then there exists $0 \neq a$ $\varepsilon \mathrm{A}$ such that $a \mathrm{~B} \subset \mathrm{N}$ by (3). Since $\mathrm{B} / \mathrm{aB}$ is noetherian, the A-module $\mathrm{N} / \mathrm{aB}$ must be finitely generated. Therefore $\mathrm{N}$ itself is finite over A, contradiction. COROLLARY (Eakin). If $B$ is a noetherian ring and A is a subring of B such that B is finite over A, then A is noetherian.

\section{A Flatness Theorem}
(36.A) LEMMA. Let $A$ be a ring and $M$ be an A-module. Let $x$ be an element of $A$ which is M-regular and A-regular, and $N$ be an A-module with $x N=.0$. Put $A^{\prime}=A / x A, M^{\prime}=M / x M$. Then:

(1) $\operatorname{Tor}_{n}^{A}(M, N) \simeq \operatorname{Tor}_{n}^{A^{\prime}}\left(M^{\prime}, N\right)$ for all $n \geqslant 0$,

(2) $\operatorname{Ext}_{A}^{n}(M, N) \simeq \operatorname{Ext}_{A}^{n},\left(M^{\top}, N\right)$ for all $n \geqslant 0$,

(3) $\operatorname{Ext}_{A}^{n+1}(N, M) \simeq \operatorname{Ext}_{A^{p}}^{n}\left(N, M^{r}\right)$ for al1 $n \geqslant 0$, and $\operatorname{Hom}_{A}(\mathrm{~N}, \mathrm{M})=0$

Proof. (1) and (2): The exact sequence $0 \rightarrow A \rightarrow A \rightarrow A^{\prime} \rightarrow 0$ is a free resolution of $A^{\prime}$. Since $0 \rightarrow M \rightarrow M \rightarrow M \otimes_{A^{\prime}} \rightarrow 0$ is also exact, we have $\operatorname{Tor}_{i}^{A}\left(M, A^{*}\right)=0$ for all i $>0$. Let L. $\rightarrow$ $M \rightarrow 0$ be a free resolution of $M$. Since $H_{i}\left(L . \otimes A^{\top}\right)=$ $\operatorname{Tor}_{i}^{A}\left(M, A^{\prime}\right)=0(1>0), L, \otimes_{A} A^{\prime}$ is a free resolution of the $A^{\prime}$-module $M^{\dagger}$. Now (1) and (2) are immediate.

(3): $\operatorname{Hom}_{A}(N, M)=0$ is obvious. For $n \geqslant 0$, put $T^{n}(N)=$ $\operatorname{Ext}_{\mathrm{A}}^{\mathrm{n}+1}(\mathrm{~N}, \mathrm{M})$ and view them as functors on $\mathrm{A}^{\prime}$-modules. From $0 \rightarrow M \rightarrow M \rightarrow M^{\prime} \rightarrow 0$ we get $T^{0}(N)=\operatorname{Hom}_{A},\left(N, M^{\prime}\right)$. Since proj.dim $A^{\prime}=1$ we have $T^{n}\left(A^{v}\right)=0$ for $n>0$, hence $T^{n}(N)$ $=0$ for $n>0$ if $N$ is projective over $A^{\prime} \cdot$ If $0 \rightarrow N^{\prime} \rightarrow N \rightarrow$ $N^{\prime \prime} \rightarrow 0$ is an exact sequence of $A^{\prime}$-modules, then we have the long exact sequence $0 \rightarrow \mathrm{T}^{0}\left(\mathrm{~N}^{\prime \prime}\right) \rightarrow \mathrm{T}^{0}(\mathrm{~N}) \rightarrow \mathrm{T}^{0}\left(\mathrm{~N}^{\prime}\right) \rightarrow \mathrm{T}^{1}\left(\mathrm{~N}^{\prime \prime}\right) \rightarrow$ $T^{1}(N) \rightarrow T^{1}\left(N^{\prime}\right) \rightarrow T^{2}\left(N^{\prime \prime}\right) \rightarrow \ldots$. Thus $T^{i}(-)$ are the derived

\includegraphics[max width=\textwidth]{2022_08_01_8d4eee36f1f42236b4f4g-141}

(36.B) Let $(A, m)$ and $(B, n)$ be noetherian local rings and $\phi: A \rightarrow B$ be a local homomorphism. Put $F=B / m B$. If $B$ is flat over $A$, we have $\operatorname{dim} B=\operatorname{dim} A+\operatorname{dim} F$ by Th.19.

The converse is also true in some case. (Cf, Th.46.)

THEOREM 81. Let the notation be as above. Assume that A is regular, $B$ is Cohen-Macaulay and $\operatorname{dim} B=\operatorname{dim} A+\operatorname{dim} F$.

Then $B$ is flat over A.

Proof. Induction on $\operatorname{dim} A$. If $\operatorname{dim} A=0$ then $A$ is a field. Suppose $\operatorname{dim} A>0$, and take $x \in m-m^{2}$. Put $A^{\prime}=A / x A$, $B^{\prime}=B / x B$. Then $\operatorname{dim} B^{\prime} \leqslant \operatorname{dim} A^{\prime}+\operatorname{dim} F=\operatorname{dim} A-1+\operatorname{dim} F$ $=\operatorname{dim} B-1$ by Th.19, but $\operatorname{dim} B^{\prime} \geqslant \operatorname{dim} B-1$ (by (12.F), or consider system of parameters of $B^{\top}$ ). Therefore $\operatorname{dim} B^{\prime}$ $=\operatorname{dim} B-1, \mathrm{x}$ is $\mathrm{B}$-regular and $\mathrm{B}^{\prime}$ is $\mathrm{CM}$. Hence $\mathrm{B}^{\top}$ is flat over $A^{\prime}$ by induction hypothesis, and so $\operatorname{Tor}_{1}^{A^{\prime}}\left(A / m, B^{\top}\right)=0$. Since $x$ is $A-r e g u l a r$ and $B$-regular, we have $\operatorname{Tor}_{1}^{A}(A / m, B)=$ $\operatorname{Tor}_{1}^{A^{\prime}}\left(A / m, B^{\prime}\right)=0$. Therefore $B$ is flat over $A$ by $T h .49 .$

(Cf. EGA IV $(6.1 .5) \cdot)$

\section{Coefficient Rings}
In this section we will prove the Cohen structure theorem ( $p .211)$ in the unequal characteristic case by the method of Grothendieck.

THEOREM 82. Let $(A, M, k)$ be a local ring and let $B$ be a flat $A-a l g e b r a$. Put $B_{0}=B / w B=B \otimes_{A}$. If $B_{0}$ is smooth over $k$ then $B$ is formally smooth over A with respect to the m B-adic topology.

Proof. By the definition of formal smoothness we have only to show that $B / m^{i} B$ is smooth over $A / m^{i}$ for every $i$. Thus we can assume that $m$ is nilpotent. Then B is free over A by (3.G), and so any A-algebra extension of B by a B-module is a Hochschild extension, cf. $(25 . C)$. Therefore the proof of smoothness of $B$ reduces, as in $(28.1)$, to showing that every symmetric $2-$ cocycle $f: B \times B \rightarrow N$ with values in a B-module $N$ is a coboundary. Suppose first that $N$ satisfies $m N=0 .$ In this case $f$ is essentially a cocycle on $B_{0}$ namely, there exists a symmetric $2-\operatorname{cocycle} \mathrm{I}_{0}: \mathrm{B}_{0} \times \mathrm{B}_{0} \rightarrow \mathrm{N}$ such that $f(x, y)=f_{0}(x, y)$. Since $B_{0}$ is smooth over $k$ we have $\mathrm{f}_{0}=\delta_{\mathrm{g}}$ for some $\mathrm{k}$-linear map $\mathrm{g}_{0}: \mathrm{B}_{0}+\mathrm{N} .$ Putting $g(x)=g_{0}(x)$ we have $f=\delta g$. In the general case let $\phi:$ $N \rightarrow N / m N$ denote the natural map. Then $\phi \circ f: B \times B \rightarrow N / m N$

\includegraphics[max width=\textwidth]{2022_08_01_8d4eee36f1f42236b4f4g-142}

THEOREM 83. Let $(A, t A, k)$ be a principal valuation ring and $K$ be an extension field of $k$. Then there exists a principal valuation ring B containing A with maximal ideal generated by $t$ and with residue field $k$-isomorphic to $K$.

Proof. Let $\left\{x_{\lambda}\right\} \lambda \varepsilon \Lambda$ be a transcendency basis of $K$ over $k$ and put $k_{1}=k\left(\left\{x_{\lambda}\right\}\right)$. Let $\left\{x_{\lambda}\right\} \lambda \in \Lambda$ be a set of independent indeterminates and put $A\left[\left\{x_{\lambda}\right\}\right]=A^{\prime}, A_{1}=A_{t}^{\prime} A^{\prime .}$ Then $A^{\prime}$ is a free $A$-module, so that $A^{\prime}$ and $A_{1}$ are separated in the $t-$ adic topology. Therefore $A_{1}$ is a principal valuation ring with residue field $k_{1}$. So we can assume that $k$ is algebraic over $k$. Let $L$ be the algebraic closure of the quotient field of A. Let $f$ denote the set of the pairs $(B, \phi)$ of a subring $B$ of $L$ containing A and an A-algebra homomorphism $\phi: B \rightarrow K$ Ker $\phi=t B$, and define an order in fo by $(\mathrm{B}, \phi)<(\mathrm{C}, \psi) \Longleftrightarrow \mathrm{B} \subset \mathrm{C}$ and $\phi=\psi \mid \mathrm{B} .$

One can easily check that Fi satisfy the condition of Zorn's lemma, therefore there exists a maximal element $(B, \phi)$ in $\mathcal{f}$. If $\phi(B) \neq K$, take an element a $\varepsilon K-\phi(B)$, let $\bar{f}(X)$ be the irreducible equation of a over $\phi(B)$ and lift it to a monic polynomial $f(X) \varepsilon B[X]$. Since $B$ is normal, $f$ is irreducible over the quotient field of $B$. Let $x$ be a root of $f$ in $L$ and put $B^{\prime}=B[\alpha] ;$ then $B^{\prime}=B[X] /(I)$, so that we have $B^{\top} / t B^{\prime}=$ $B[X] /(t, f)=\phi(B)(a)$. Since $B^{\prime}$ is integral over B all maximal ideals of $B^{\prime}$ must contain $t B^{\prime}$, therefore $B^{\prime}$ is a local Ing with $\mathrm{tB}^{\prime}$ as maximal ideal. Clearly $\mathrm{B}^{\prime}$ is a noetherian domain, so B must be a principal valuation ring. This contradicts the maximality of $(B, \phi)$ in 3 . Thus $\phi(B)$ $=\mathrm{K}$

Remark 1. If $(A, t A)$ is a principal valuation ring and $M$ is an A-module, then $M$ is flat over A iff $t$ is M-regular. This is an immediate consequence of (3.A) Th.1 (3). In particular the ring $B$ of the above theorem is flat over A.

Remark 2. In EGA $0_{\text {III }}(10.3 .1)$ the following more general theorem is proved: if $(\mathrm{A}, \ldots, \mathrm{k})$ is a noetherian local ring and $k$ is an extension field of $k$, then one can find a noetherian local ring B containing A and flat over A such that $\operatorname{rad}(\mathrm{B})=m \mathrm{~B}, \mathrm{~B} / \operatorname{mB} \simeq \mathrm{K}$ THEOREM 84. Let $(A, m, K)$ be a complete, separated local ring, $(\mathrm{R}, \mathrm{pR}, \mathrm{k})$ be a principal valuation ring and $\phi_{0}: \mathrm{k} \rightarrow \mathrm{K}$ be a homomorphism of fields. Then there exists a local homomorphism $\phi: R \rightarrow A$ which induces $\phi_{0} \cdot$

Proof. Put $S=Z_{p Z}$ and let $k_{0}$ be the prime field in $k$. Since $\operatorname{ch}(K)=\operatorname{ch}(k)=p$, the canonical homomorphism $Z \rightarrow A$ can be extended to a local homomorphism $\mathrm{S} \rightarrow$ A. Similarly $\mathrm{R}$ is an S-algebra, which is flat by Remark 1. Since $\mathrm{R} / \mathrm{PR}=$ $k$ is separable (hence smooth) over $k_{0}, R$ is formally smooth over $\mathrm{S}$ in the pR-adic topology by Th.82. Therefore we can lift the map $\mathrm{R} \rightarrow \mathrm{k} \rightarrow \mathrm{K}$ to $\phi: \mathrm{R} \rightarrow \mathrm{A}$.

\includegraphics[max width=\textwidth]{2022_08_01_8d4eee36f1f42236b4f4g-143}

THEOREM 85. A complete separated local ring has a coefficient ring. (Cf. p.211.)

Proof. This follows from Th.83 and Th.84. 38. $p$ - Basis

(38.A) Let $R$ be a ring of characteristic $p>0$, and let $R^{p}$ denote the subring $\left\{x^{P} \mid x \in R\right\}$. Let $S$ be a subring of $R$. A subset $B \subseteq R$ is said to be p-independent (in R) over $S$ if the monomials $b_{1}{ }_{1} \ldots b_{n}^{n}$, where $b_{1}, \ldots, b_{n}$ are distinct elements of $B$ and $0 \leqslant e_{i}<p$, are linearly independent over $R^{p}[S]$. When A is a ring of characteristic $p$, a polynomial (or a monomia1) f $\varepsilon \mathrm{A}\left[X_{1}, \ldots, X_{n}\right]$ is said to be reduced if it is of degree < $p$ in each variable $X_{f} \cdot \quad B$ is called a $p$ basis of $R$ over $S$ if it is $p$-independent over $S$ and $R^{P}[S, B]$ $=R$, i.e. if every element a of $R$ can be written uniquely as a reduced polynomial $a=f\left(b_{1}, \ldots, b_{n}\right)$ in distinct elements $b_{i}$ of $B$ with coefficients in $R^{p}[S]$.

If $B$ is a p-basis and $M$ is an $R$-module, then any map $\phi: B \rightarrow M$ is uniquely extended to a derivation $D: R \rightarrow M$ over $S$ by $D(a)=D(f(b))=\sum \partial f / \partial b_{i} \phi\left(b_{i}\right)$, where $a=f(b)$ is the unique representation of $a \varepsilon R$ as a reduced polynomial in elements of $\mathrm{B}$ with coefficients in $\mathrm{R}^{\mathrm{P}}[\mathrm{S}]$. It follows that $\Omega_{R / S}$ is a free $R$-module with basis $\{d b \mid b \varepsilon B\}$.

(38.B) If $k, k^{\prime}$ are subfields of a field $k$, the subfield generated by them will be denoted by $k^{\prime}$; thus $k^{\prime}=k\left(k^{\prime}\right)=$ $k^{\prime}(k)$. Let $k$ be a field of characteristic $p$ and $K^{\prime}$ be a subfield containing $\mathrm{K}^{\mathrm{p}}$. If $\left[\mathrm{K}^{\mathrm{p}}{ }^{\mathrm{p}}\right]$ is finite it is a power $\mathrm{p}^{\mathrm{n}}$ of $\mathrm{p}$; its exponent $\mathrm{n}$ is called the $\mathrm{p}$-degree of $\mathrm{K} / \mathrm{K}^{\prime}$ and . will be denoted by $\left(K: K^{\prime}\right)_{p}$. This is equal to the smallest number of generators of $K$ over $K^{\prime}$, and also equal to the rank of the $\mathrm{K}$-module $\Omega_{\mathrm{K} / \mathrm{K}^{\prime}}$

Let $k$ be a field of characteristic $p$ and $k$ be a subfield. Since $\mathrm{K}^{\mathrm{p}}[\mathrm{k}]=\mathrm{K}^{\mathrm{p}}(\mathrm{k})=\mathrm{K}^{\mathrm{p}} \mathrm{k}$, a subset $\mathrm{B}$ of $\mathrm{K}$ is $\mathrm{p}-$ independent over $k$ iff, for every finite subset $B^{\prime}$ of $B$, we have $\left(\mathrm{K}^{\mathrm{P}} \mathrm{k}\left(\mathrm{B}^{\prime}\right): \mathrm{K}^{\mathrm{P}} \mathrm{k}\right)=\operatorname{Card}\left(\mathrm{B}^{\prime}\right)$. Also $\mathrm{B}$ is a p-basis of $\mathrm{K} / \mathrm{k}$ Iff it is p-independent over $k$ and $\mathrm{K}^{\mathrm{P}} \mathrm{k}(\mathrm{B})=\mathrm{K}$. By Zorn's

lemma any p-independent subset is contained in a p-basis.

THEOREM 86. Let $K$ and $k$ be as above, $B$ be a subset of $K$ and let $\mathrm{dB}$ denote the subset $\{\mathrm{db} \mid \mathrm{b} \in \mathrm{B}\}$ in $\Omega_{\mathrm{K} / \mathrm{k}}$. Then:

i) $B$ is $p$-independent over $k \Leftrightarrow d B$ is linearly indep. $/ K$,

ii) $B$ is a p-basis of $\mathrm{K} / \mathrm{K} \Leftrightarrow \mathrm{dB}$ is a basis of $\Omega_{\mathrm{K} / \mathrm{k}}$ over $\mathrm{K}$.

Proof. If $B$ is a p-basis we have already seen that $\Omega_{K / k}$ is

a free $\mathrm{K}$-module with basis $\mathrm{dB}$. If $\mathrm{B}$ is $\mathrm{p}$-independent then

there exists a p-basis containing $B$, hence $\mathrm{dB}$ is lin. indep.

over $\mathrm{K}$. On the other hand if $\mathrm{B}$ is not $\mathrm{p}$-independent then there exist $b, b_{1}, \ldots, b_{n} \in B$ such that $b \varepsilon K^{p_{k}}\left(b_{1}, \ldots, b_{n}\right)$, and then $\mathrm{db} \varepsilon \sum \mathrm{Kdb}_{i}$. Therefore if $\mathrm{dB}$ is linearly independent then $B$ is $p$-independent, and there exists a $p$-basis $B^{\prime}$ containing $B$. If $d B$ is a basis of $\Omega_{K} / k$ then $B=B^{\prime}$.

(38.C) Let $K$ be an arbitrary field and $k$ be a subfield. The $\mathrm{K}$-module $\Omega_{\mathrm{K} / \mathrm{k}}$ is generated over $\mathrm{K}$ by $\mathrm{dK}$, therefore there exists a subset $B$ such that $\mathrm{dB}=\{\mathrm{db} \mid \mathrm{b} \varepsilon \mathrm{B}\}$ is a basis of $\mathrm{K} / \mathrm{k}^{\cdot}$ Such a subset $\mathrm{B}$ is called a differential basis of $\mathrm{K} / \mathrm{k}$. The concept of differential basis coincides with that of pbasis in the case of characteristic $p$ as we have just seen. In case $\operatorname{ch}(K)=0$ it coincides with that of transcendency basis by the following theorem.

THEOREM .87. Let $\mathrm{K} \supset \mathrm{k}$ be fields of characteristic 0 . Then:

i) $B \subset K$ is algebraically dependent over $k$ iff $\mathrm{dB}$ is linearly independent over $k$ in $\Omega_{K} / k^{\prime}$

ii) $B \subset K$ is a transcendency basis of $K / k$ iff $d B$ is a linear basis of $\Omega_{\mathrm{K} / \mathrm{k}}$ over $\mathrm{K}$.

Proof. Similar to the proof of the preceding theorem.

(38.D) THEOREM 88. Let $K / k$ be a field extension. Then the following is equivalent:

(1) $\mathrm{K}$ is separable over $\mathrm{k}$,

(2) for any subfield $k^{\prime}$ of $k$, the canonical map $\Omega_{k} / k^{\prime} k^{K} \rightarrow$ $\Omega_{K} / k^{\prime}$ is injective, (3) the canonical map $\Omega_{k} \otimes_{k} K \rightarrow \Omega_{K}$ is injective,

(4) any derivation $D$ from $k$ to a K-module $M$ can be extended to a derivation $K \rightarrow M$.

Proof. It is clear that (2) and (4) are equivalent. But (4) is also equivalent to (3). If $\operatorname{ch}(K)=0$ then (3) holds by the preceding theorem, so (1), (2), (3) and (4) are al1 true. If $\operatorname{ch}(\mathrm{K})=\mathrm{p}$, (1) is equivalent to $\mathrm{K} \otimes_{\mathrm{k}} \mathrm{k}^{-1} \simeq \mathrm{kk}^{\mathrm{p}^{-1}}$ by Maclane's theorem $(p .196)$, or what is the same, to linear disjointness of $K^{p}$ and $k$ over $k^{P}$. Therefore, $k$ is separable over $\mathrm{k} \Leftrightarrow$ the reduced monomials in the elements of a p-basis B of $\mathrm{k} / \mathrm{k}^{\mathrm{p}}$ are linearly independent over $\mathrm{K}^{\mathrm{p}} \Leftrightarrow \mathrm{dB}$ is linearly independent over $K$ in $\Omega_{K} \Leftrightarrow \Omega_{k} \otimes K \rightarrow \Omega_{K}$ is injective.

THOREM 89. Let $K$ be a separable extension of a field $k$ of characteristic $p$, and let $B$ be a p-basis of $\mathrm{K} / \mathrm{k}$. Then $\mathrm{B}$ is algebraically independent over $k$.

Proof. Assume the contrary and suppose $b_{1}, \ldots, b_{n} \in B$ are algebraically dependent over $k$. Take an algebraic relation
$$
f\left(b_{1}, \ldots, b_{n}\right)=0, \quad \text { f } \varepsilon k\left[x_{1}, \ldots, x_{n}\right]
$$
of lowest possible degree. Put deg $f=d$. Write

\includegraphics[max width=\textwidth]{2022_08_01_8d4eee36f1f42236b4f4g-145}

where $g_{(v)}$ are polynomials with coefficients in $k$. Since $b_{1}, \ldots, b_{n}$ are $p$-independent over $k$, we must have $g_{(v)}\left(b^{p}\right)=0$ for all $(\nu)$. By the choice of $f$ this happens only if
$$
f\left(x_{1}, \ldots, x_{n}\right)=g_{0, \ldots, 0}\left(x_{1}^{p}, \ldots, x_{n}^{p}\right) .
$$
But then we would have $f(X)=h(X)^{p}$ with $h \in k^{p}\left[X_{1}, \ldots, X_{n}\right]$.

Hence $h(b)=0$. By MacLane's theorem $(p .196)$, however, $K$ and $\mathrm{k}^{\mathrm{p}^{-1}}$ are linearly disjoint over $k$. The monomials of degree

$<\mathrm{d}$ in $\mathrm{b}_{1}, \ldots, \mathrm{b}_{\mathrm{n}}$ are linearly independent over $\mathrm{k}$, hence they must be linearly independent over $k^{p^{-1}}$ a1so. This is a contradiction.

(38.E) We defined formal smoothness (p.199) by the condition of liftability (FS). If we further require that the lifting $v^{\prime}$ of $v$ is unique, then we say that A is formally etale over k. Here we are mainly concerned with field extensions, so that we consider only discrete topologies.

Let $\mathrm{K} / \mathrm{k}$ be an extension of flelds. If $\operatorname{ch}(\mathrm{K})=0$, then "formally smooth" and "separably algebraic" are the same thing. If $\operatorname{ch}(\mathrm{K})=\mathrm{p}$, however, "formally etale" is weaker than "separably algebraic". (Consider the case where both $K$ and $k$ are perfect. Then $\mathrm{k}$ is formally etale over $k_{.}$) In any case, the following are easily seen to be equivalent:

(1) $\mathrm{K}$ is formally etale over $k$,

(2) $\mathrm{K}$ is smooth over $\mathrm{k}$ and $\Omega_{\mathrm{K} / \mathrm{k}}=0$,

(3) $\Omega_{k} \otimes_{k} K \simeq \Omega_{k}$, (4) for any subfield $k^{\prime}$ of $k, \Omega_{k / k^{\prime}} \otimes k \simeq \Omega_{K} / k^{\prime}$,

(5) any derivation from $k$ into a $K$-module $M$ can be uniquely extended to a derivation $\mathrm{K} \rightarrow \mathrm{M}$.

THEOREM 90. Let $K$ be a separable extension field of a field $k$, and let $B$ be a differential basis of $K / k$. Then $k(B)$ is purely transcendental over $k$ and $k$ is formally etale over $k(B)$.

Proof. Immediate from Th.87 and Th.89.

(38.F) Let $(A, m, K)$ be a local ring and $k$ be a subfield of A such that $\mathrm{K} / \mathrm{k}$ is formally etale. In this case we call $k$ a quasi-coefficient field of A.

THEOREM 91. Every local ring containing a field contains quasi-coefficient fields. If $k$ is a quasi-coefficient field of a local ring $A$, then the completion $A^{*}$ of A contains a unique coefficient field $\mathrm{K}$ containing $\mathrm{k}$.

Proof. If $(A, W, K)$ is a local ring and $k_{0}$ is a perfect field (e.g. the prime field) contained in A, then let B be a differential basis of $k$ over $k_{0}$ and choose a representative $x_{i}$ in A for each $b_{i} \varepsilon B$. Since $B$ is algebraica11y independent over $k_{0}$ by Th.89, A contains the quotient field $k^{\prime}$ of $k_{0}\left[\left\{x_{i}\right\}\right]$, and $k^{\prime} \simeq k_{0}(B)$. Then $K$ is formally etale over $k^{\prime}$. By the definition of formal etaleness, the identity map $\mathrm{K} \rightarrow \mathrm{A} / \mathrm{m}$ can be uniquely lifted to a homomorphism $\mathrm{K} \rightarrow 1 \mathrm{im} A / \mathrm{m}^{\mathrm{V}}=\mathrm{A}^{*}$ over $k^{\prime}$, which proves the second half of the theorem.

One can define "quasi-coefficient rings" in the unequal characteristic case as follows: a subring I of a local ring $(A, m, K)$ with $c h(K)=p$ is a quasi-coefficient ring of A if (1) I is a noetherian local ring with $\operatorname{rad}(I)=p I$, and (2) $\mathrm{K}$ is formally etale over I/pI. One can prove that any local ring of unequal characteristic has quasi-coefficient rings. Cf. H.Matsumura, Nagoya Math. J. 68 (1977).

(38.G) Not much is known about p-bases for rings. If $k$ is a field of characteristic $p$ and $A$ is a reduced local ring containing $k$, and if $A$ has a p-basis over $A^{P}$, then $A$ must be regular by a theorem of Kunz which will be discussed later. If $A$ is a regular local ring essentially of finite type over $k$, then A has a p-basis over $A^{p}$ (cf. Kimura-Niitsuma, to appear in J. Japan Math. Soc.). The following interesting conjecture of Kunz (1975) is stil1 open in the general case.

Let $R$ be a regular local ring of characteristic $p$ and $S$ be a regular subring of $R$ over which $R$ is finite. Does $\mathrm{R}$ have a p-basis over $\mathrm{S}$ ?

The answer is yes if $p=2$ or 3 (proof is easy). If $\operatorname{dim} R$ $=2$ there is a geometric proof by Rudakov-Shafarevich (Izves.tija Akad. Nauk SSSR, t.40, No.6, 1976). The following proposition is a converse of (38.A) in the case of noetherian local rings.

PROPOSITION. Let $\left(R, m_{R}\right)$ be a noetherian local ring of characteristic $p$, and $s$ be a subring of $R$ containing $R^{p}$ such that $R$ is finite over $S$. Put $m_{S}=m_{R} \cap S, K=R / m_{R}$ and $K^{\prime}=S / m_{S}$. If $\Omega_{R} / s^{\text {is a }}$ free $R$-module with $\mathrm{dx}_{1}, \ldots, \mathrm{dx}{ }_{r}\left(\mathrm{x}_{1} \varepsilon \mathrm{R}\right)$ as a basis, then $\mathrm{x}_{1}, \ldots, \mathrm{x}_{\mathrm{r}}$ form a p-basis of $R$ over $S$.

Proof. First we consider the case $\Omega_{R / S}=0$. Suppose $K \neq K^{\prime}$. Then, since $K^{\prime} \supseteq K^{p}$, there would exist $0 \neq \bar{D} \varepsilon \operatorname{Der}_{K^{\prime}}(K)$, and composing it with the natural homomorphism $\mathrm{R} \rightarrow \mathrm{K}$ we would have a derivation $0 \neq D \in \operatorname{Der}_{S}(R, K) . \quad$ Therefore $K=K^{\prime}, 1 .$ e. $R=S+m_{R} \cdot$ Then $R /\left(m_{S} R+m_{R}^{2}\right)=k+m_{R} /\left(m_{S} R+m_{R}^{2}\right)$, and the right-hand side is a direct sum. Let $\mathrm{p}_{2}$ de-ote the projection onto the second summand. Then the composition $R \rightarrow$ $R /\left(m_{S} R+m_{R}^{2}\right) \stackrel{P_{2}}{\rightarrow} m_{R} /\left(m_{S} R+m_{R}^{2}\right)$ is a derivation of $R$ over $S$, which must be zero. Therefore $m_{R}=m_{S} R+m_{R}^{2}$, and by NAK we have $m_{R}=m_{S} R$. Therefore $R=S+m_{S} R$, hence $R=S$ by NAK.

In the general case put $T=S\left[x_{1}, \ldots, x_{r}\right]$ If $x_{1}, \ldots$, $x_{r}$ are not $p$-independent over $s$, take a reduced polynomial $f\left(X_{1}, \ldots, x_{r}\right) \varepsilon S[X]$ of lowest degree such that $f\left(x_{1}, \ldots, x_{r}\right)$

$=0$. Then $\Sigma\left(\partial f / \partial \mathrm{x}_{i}\right) \mathrm{dx}_{i}=0$ in $\Omega_{R / S}$, contradiction. Thus $x_{1}, \ldots, x_{r}$ is a p-basis of $T$ over $S$ and $\Omega_{T / S}$ is a free $T-$ module with $\mathrm{dx}_{\mathrm{i}}$ as basis, so that $\Omega_{\mathrm{T} / \mathrm{S}}{ }_{\mathrm{T}} \mathrm{R} \simeq \Omega_{\mathrm{R}} / \mathrm{S}^{\cdot}$. Then $\Omega_{R / T}=0$, and so $R=T$ by what we have already seen.

Remark. In connection with the above proof, it is worthwhile to note the following more general result of Berger and Kunz. Let $(R, m, K)$ be a local ring, $S$ a subring of $R, n=\operatorname{se} n S$, $k=S / w$. If $K / k$ is separable then the following sequence is exact: $0 \rightarrow \mathrm{m} /\left(m \mathrm{R}+\mathrm{m}^{2}\right) \rightarrow \Omega_{\mathrm{R} / \mathrm{S}} \boldsymbol{O} \mathrm{K} \rightarrow \Omega_{\mathrm{K} / \mathrm{K}} \rightarrow 0$. If $\operatorname{ch}(\mathrm{R})=p$ then put $m^{\prime}=m \cap \mathrm{R}^{\mathrm{P}}[\mathrm{S}] \cdot$ Then the following sequence is exact: $0 \rightarrow M /\left(M^{\prime} \mathrm{R}+\mathrm{m}^{2}\right) \rightarrow \Omega_{\mathrm{R} / \mathrm{S}^{\otimes} \mathrm{K}} \rightarrow \Omega_{\mathrm{K} / \mathrm{K}}+0$. For the proof, cf, Berger-Kunz, Math. Z. 77 (1961).

\section{Cartier's Equality and Geometric Regularity}
(39.A) Let $k \subseteq K \subseteq L$ be fields. The kernel of the natural

$\operatorname{map} \Omega_{\mathrm{K} / \mathrm{k}} \otimes \mathrm{L} \rightarrow \Omega_{\mathrm{L} / \mathrm{k}}$ is denoted by $\Gamma_{\mathrm{L} / \mathrm{K} / \mathrm{k}}$ and is called

the module of imperfection for $L / k / k$. Thus we have the

following exact sequence:
$$
0 \rightarrow \Gamma_{L / K / k}+\Omega_{K / k} \otimes L+\Omega_{L / k} \rightarrow \Omega_{L / K} \rightarrow 0 .
$$
LEMMA. If $k \subseteq K \subseteq L^{\prime} \subseteq L$ are fields, we have the following

exact sequence.
$$
\begin{aligned}
0 &+\Gamma_{L^{\prime} / K / k} \otimes_{L^{\prime}} L \rightarrow \Gamma_{L / K / k} \rightarrow \Gamma_{L / L^{\prime} / k} \rightarrow \Omega_{L^{\prime} / K} \otimes_{L^{\prime}} L \\
& \rightarrow \Omega_{L / K} \rightarrow \Omega_{L^{\prime} / L^{\prime}} \rightarrow 0
\end{aligned}
$$
Proof. Consider the following commutative diagram with

\includegraphics[max width=\textwidth]{2022_08_01_8d4eee36f1f42236b4f4g-148}

For simplicity we write $0+\mathrm{X} \rightarrow \mathrm{Z} \rightarrow \mathrm{A} \rightarrow \mathrm{B} \rightarrow \mathrm{O}$
$$
\begin{aligned}
&\downarrow \quad \| \quad \downarrow f \quad+g \\
&0 \rightarrow \mathrm{Y} \rightarrow \mathrm{Z} \rightarrow \mathrm{A}^{9} \rightarrow \mathrm{B}^{\prime} \rightarrow 0
\end{aligned}
$$
Applying the 'snake lemm' (cf. e.g. Bourbaki, Alg. Comm.,

Ch.1) to the induced diagram we get the exact sequence $0 \rightarrow \mathrm{Y} / \mathrm{X} \rightarrow \operatorname{Ker} \mathrm{f} \rightarrow \operatorname{Ker} \mathrm{g} \rightarrow 0$, which shows the exactness of $0 \rightarrow \mathrm{X} \rightarrow \mathrm{Y} \rightarrow \operatorname{Ker} \mathrm{f} \rightarrow \mathrm{B} \rightarrow \mathrm{B}^{\prime} \rightarrow$ Coker $\mathrm{g}$ $\rightarrow 0$. This is what we wanted.

THEOREM 92 (Cartier's equality). Let L be a finitely generated extension of a field $\mathrm{K}$. Then
$$
\operatorname{rank}_{\mathrm{L}} \Omega_{\mathrm{L} / \mathrm{K}}=\operatorname{tr} \cdot \operatorname{deg}_{\cdot \mathrm{K}} \mathrm{L}+\operatorname{rank}_{\mathrm{L}} \mathrm{I}_{\mathrm{L} / \mathrm{K}} \cdot
$$
Proof. If $\mathrm{L}^{\text {Pro }} \mathrm{L}^{\prime} \supseteq \mathrm{K}$ and if the theorem holds for $\mathrm{L} / \mathrm{L}^{\prime}$ and for $L^{\prime} / K$, then the validity of the theorem for $L / K$ is an immediate consequence of the lemma. On the other hand any finitely generated extension is composed of simple extensions of the following types: (1) $\mathrm{L}=\mathrm{K}(\alpha)$ with $\alpha$ transcendental over $K,(2) L=K(\alpha)$ with $\alpha$ separably algebraic over $K$, (3) $I=K(\alpha), \operatorname{ch}(K)=p, \alpha^{P}=a \varepsilon K, \alpha \notin K$. Therefore it suffices to prove the theorem in each of these cases. Cases (1) and

(2) are easy; cf. p.190. In case (3) we have $L=K[X] /\left(X^{P}-a\right)$, and then $\Omega_{L}=\left(\Omega_{K[X]} \otimes \mathrm{L}\right) / \mathrm{Lda}=\left(\Omega_{\mathrm{K}} / \mathrm{Kda}\right) \otimes \mathrm{L}+\mathrm{Ld} \alpha$, $d \alpha \neq 0$. Since $d a \neq 0$ in $\Omega_{K}$, we have $\operatorname{rank} \Gamma_{L / K}=\operatorname{rank} \Omega_{L / K}$

$=1$ and the theorem holds in this case also. $(39.0)$ THEOREM 93. Let $(A, M, K)$ be a noetherian local ring containing a field $k$. Then $A$ is formally smooth over $k$ in the $m$-adic topology iff A is geometrically regular over $k$.
$$
\begin{aligned}
& 0 \rightarrow \mathrm{Z} / \mathrm{X} \rightarrow \mathrm{A} \rightarrow \mathrm{B} \rightarrow \mathrm{O}
\end{aligned}
$$
\includegraphics[max width=\textwidth]{2022_08_01_8d4eee36f1f42236b4f4g-148(1)}

Proof. The 'only if' part is known $(28 . N)$. In order to prove the 'if' part we may assume, by $(28 . N)$, that $\operatorname{ch}(k)=p$. According to Cor. of Th.66 it suffices to show that $\Omega_{k} \otimes \mathrm{K} \rightarrow$ $\Omega_{A} \otimes \mathrm{K}$ is injective. Therefore let $\mathrm{x}_{1}, \ldots, \mathrm{x}_{\mathrm{r}}$ be p-independent elements in $k$. We will show that $\mathrm{dx}_{1}, \ldots, \mathrm{d} \mathrm{x}_{\mathrm{r}}$ are linearly independent in $\Omega_{A} \otimes k$ over $k$. Put $\alpha_{1}=x_{1}^{1 / p}, k^{\prime}=k\left(\alpha_{1}, \ldots\right.$, $\left.\alpha_{r}\right) \cdot$ Then $\mathrm{B}=\mathrm{A} \otimes_{k} k^{\prime}=\mathrm{A}\left[\mathrm{T}_{1}, \ldots, \mathrm{T}_{\mathrm{r}}\right] /\left(\mathrm{T}_{1}{ }^{\mathrm{P}}-\mathrm{x}_{1}, \ldots, \mathrm{T}_{\mathrm{r}}{ }^{\mathrm{P}}-\mathrm{x}_{\mathrm{r}}\right)$ is a noetherian local ring. Let $W$ and $L$ denote its maximal ideal and its residue field respectively. Since $\mathrm{L}$ is smooth over the prime field the sequence $0 \rightarrow \mu / N^{2} \rightarrow \Omega_{B} \otimes L \rightarrow \Omega_{L} \rightarrow 0$ is exact by Th.58. Similarly the sequence $0 \rightarrow m / m^{2}+\Omega_{A} \otimes \mathrm{K}$ $\rightarrow \Omega_{1} \rightarrow 0$ is exact. Consider the following commutative diagram:

\includegraphics[max width=\textwidth]{2022_08_01_8d4eee36f1f42236b4f4g-149}

By the snake lemma we get an exact sequence of L-modules $0 \rightarrow \operatorname{Ker} \psi_{1} \rightarrow \operatorname{Ker} \psi_{2} \rightarrow \operatorname{Ker} \psi_{3} \rightarrow \operatorname{Coker} \psi_{1} \rightarrow \operatorname{Coker} \psi_{2} \rightarrow \operatorname{Coker} \psi_{3}$ $\rightarrow 0 .$ Since $A$ and $B$ are regular by hypothesis and have the same dimension, we have rank $w / w^{2}=\operatorname{dim} A=\operatorname{rank} \mathrm{m} / \mathrm{m}^{2}$, so that rank Ker $\psi_{1}=\operatorname{rank}$ Coker $\psi_{1}<\infty .$ Since L is finite algebraic over $k$ we also have rank Ker $\psi_{3}=$ rank Coker $\psi_{3}<\infty$ by Cartier's equality. It follows from these and from the above exact sequence that rank $\operatorname{Ker} \psi_{2}=\operatorname{rank}$ Coker $\psi_{2}<\infty$. On the other hand, we have coker $\psi_{2}=\Omega_{B / A} \otimes L$ and $\Omega_{B / A}=$ $\mathrm{BdT}_{1}+\ldots+\mathrm{BdT}_{r} \simeq \mathrm{B}^{\mathrm{r}}$ by Th.58, hence rank Ker $\psi_{2}=\mathrm{r}$. Putting $J=\left(T_{1}^{p}-x_{1}, \ldots, T_{r}^{P}-x_{r}\right)$ we have the exact sequence $\mathrm{J} / \mathrm{J}^{2} \rightarrow \Omega_{\mathrm{A}}\left[\mathrm{T}_{1}, \ldots, \mathrm{T}_{\mathrm{r}}\right] \mathrm{B}=\Omega_{\mathrm{A}} \otimes \mathrm{B}+\sum \mathrm{BdT}_{\mathrm{i}} \rightarrow \Omega_{\mathrm{B}} \rightarrow 0$.

It remains exact after tensoring with $\mathrm{L}$ over $B$, so $\operatorname{Ker} \psi_{2}$ is generated by $\mathrm{d} \mathrm{x}_{1}, \ldots, \mathrm{d}_{\mathrm{r}}$. Therefore $\mathrm{d} \mathrm{x}_{1}, \ldots, \mathrm{dx} \mathrm{r}_{\mathrm{r}}$ are linear1y independent in $\Omega_{A} \otimes L$ over $L$, and a fortiori so in $\Omega_{A} \otimes \mathrm{K}$ over K. QED.

(This proof is due to G.Faltings, Arch. Math. 30 (1978.)

\section{Jacobian Criteria and Excellent Rings}
(40.A) Let $A$ be a ring and let $x_{1}, \ldots, x_{r} \varepsilon A, D_{1}, \ldots, D_{S} \varepsilon$ $\operatorname{Der}(A)$. We shall denote the Jacobian matrix $\left(D_{i} x_{j}\right)$ by $J\left(x_{1}, \ldots, x_{r} ; D_{1}, \ldots, D_{s}\right)$. If $P$ is an ideal of $A$, we shall write $J\left(x_{1}, \ldots, x_{r} ; D_{1}, \ldots, D_{s}\right)(P)$ for $\left(D_{1} x_{j}\right.$ mod $\left.P\right)$. When $P$ Is a prime ideal containing the $x^{\prime} s$, the rank of the above matrix depends on the ideal $I=\sum A x_{i}$ rather than the elements $x_{i}$ thenselves, so we denote it by $\operatorname{rank} J\left(I ; D_{1}, \ldots, D_{s}\right)(P)$.

If $\Delta$ is a set of derivations of $A$ we define rank $J(I ; \Delta)(P)$ to be the supremum of rank $J\left(I ; D_{1}, \ldots, D_{s}\right)(P)$ when $\left\{D_{1}, \ldots, D_{s}\right\}$ runs over the set of all finite subset of $\triangle$.

When $A$ is an integral domain with quotient field $K$ and $M$ is an $A$-module, by rank $M$ we understand rank $M \otimes_{K} K$. THEOREM 94. Let $(R, m)$ be a regular local ring, $P$ be a prime ideal of height $r$ and $\Delta$ be a subset of $\operatorname{Der}(R)$. Then:

i) rank $J(P ; \Delta)(m) \leqslant \operatorname{rank} J(P ; \Delta)(P) \leq r$

ii) if $\operatorname{rank} J\left(f_{1}, \ldots, f_{r} ; D_{1}, \ldots, D_{r}\right)(m)=r$ and $f_{1}, \ldots$, $f_{r} \varepsilon P$, then $P=\left(f_{1}, \ldots, f_{r}\right)$ and $R / P$ is regular.

Proof, i) The first inequality is trivial, and the second is a consequence of the fact that $P R_{P}$ is generated by $r$ elements. ii) The condition implies that the images of $f_{i}^{\prime} / 8$ are linearly independent over $R / \mathrm{m}$ in $\mathrm{s} / \mathrm{m}^{2}$, hence the $f_{i}{ }^{\prime}$ s generate a prime ideal of height $r$. Our assertion follows.

THEOREM 95. Let $R, P$ and $\Delta$ be as in the preceding theorem. Then the following two conditions are equivalent:

(1) rank $J(P ; \Delta)(P)=h t P$,

(2) let $Q$ be a prime ideal contained in $P$, then $R_{P} / Q R_{P}$ is regular iff rank $J(Q ; \Delta)(P)=$ ht $Q$.

Proof. (1) is the special case $Q=P$ of (2). Conversely, suppose (1) holds. If rank $J(Q ; \Delta)(P)=h t Q$ then $R_{P} / Q R_{P}$ is regular by the preceding theorem. If $\mathrm{R}_{\mathrm{P}} / \mathrm{QR}_{\mathrm{P}}$ is regular then there exists $f_{1}, \ldots, f_{r} \varepsilon P$ such that $\left(f_{1}, \ldots, f_{r}\right) R_{P}=P R_{P}$, $\left(f_{1}, \ldots, f_{s}\right) R_{P}=Q R_{P}, r=h t P, s=h t Q . \quad$ Then rank $J\left(f_{1}, \ldots,\right.$, $\left.f_{r} ; \Delta\right)(P)=r$, and so rank $J\left(f_{1}, \ldots, f_{s} ; \Delta\right)(P)=s$. (40.B) We shall say that a subfield $k^{\prime}$ of a field $k$ is cofinite if $\left[k: k^{\prime}\right]<\infty$.

LEMMA 1. Let $k \subseteq k$ be fields of characteristic $p$ and let $F=\left\{k_{\alpha}\right\}_{\alpha \varepsilon I}$ be a downwards-directed family of cofinite subfields of $K$ containing $k$. Then the following are equivalent:

\includegraphics[max width=\textwidth]{2022_08_01_8d4eee36f1f42236b4f4g-150}

(2) The natural map $\Omega_{K / k} \rightarrow \lim _{\leftarrow} \Omega_{K}$ is injective.

(3) For every finite subset $\left\{u_{1}, \ldots, u_{n}\right\}$ of $K$ which is p-independent over $k$, there exists $k_{\alpha} \varepsilon F$ over which this set is $\mathrm{p}$-independent.

(4) There exists a p-basis $B$ of $k$ over $k$ such that for each finite subset $F$ of $B$ there exists $k_{\alpha} \varepsilon F$ over which $F$ is p-independent.

Proof. $(2) \Leftrightarrow(3)$ is easy, and $(3) \Rightarrow(4)$ is trivia1.

$(1) \Rightarrow(3)$ : The proof of (30.C) Lemma I applies mutatis mutandis. $(4) \Rightarrow(2):$ Let $0 \neq \omega \varepsilon \Omega_{\mathrm{K}} / \mathrm{k} \cdot \quad$ Then $\omega=\mathrm{c}_{1} \mathrm{db}_{1}+\ldots+$ $c_{n} \mathrm{db}_{n}, b_{i} \in B, 0 \neq c_{i} \varepsilon K$, and if $b_{1}, \ldots, b_{n}$ are p-independent over $k_{\alpha}$ then the image of $w$ in $\Omega_{K / k_{\alpha}}$ is not 0 . (3) $\Rightarrow$ (1): Suppose a $\notin \mathrm{kK}^{\mathrm{P}}$. Then a is $\mathrm{p}$-independent over $\mathrm{k}$, therefore it is so over some $k_{\alpha}, 1 . e, a \& k_{\alpha}{ }^{p}$.

LEMMA 2. Let $k, K$ and $F$ be as in lemma 1 and let $L$ be a finitely generated extension over $K .$ If $\bigcap_{\alpha} k_{\alpha} \mathrm{K}^{\mathrm{p}}=\mathrm{kK}^{\mathrm{P}}$ holds,

\section{then $\bigcap_{\alpha} k_{\alpha} \mathrm{L}^{\mathrm{p}}=\mathrm{kL}^{\mathrm{p}}$ holds also.}
Proof. It suffices to check the 4 cases of (27.A). i) If $L=K(t)$ with $t$ transcendenta1, then $\cap k_{\alpha}{ }^{\mathrm{L}}=\cap k_{\alpha} \mathrm{K}^{\mathrm{p}}\left(t^{\mathrm{p}}\right)=$ ${k K^{p}}^{p}\left(t^{p}\right)=k L^{P}$ is obvious. ii) If $L$ is separably algebraic over $k$ then a p-basis of $k$ over $k$ is also a p-basis of L over $\mathrm{k}$, and we can use the criterion (4) of Lemma 1. iii) $\mathrm{L}=$ $K(t), t^{p}=a \varepsilon K, d_{K / k}$ a $=0$. Then $\Omega_{L / k}=\Omega_{K / k} \otimes L+L d t$, and $\Omega_{L / k_{\alpha}}=\Omega_{K / k_{\alpha}} \odot \mathrm{L}+\mathrm{Ld}$. Therefore $\Omega_{\mathrm{L} / \mathrm{k}} \rightarrow \lim _{\mathrm{L}} \Omega_{\mathrm{L}}$ is injective. iv) $L=K(t), t^{p}=a \varepsilon K, d_{K} / k^{a} \neq 0 . \quad$ Then $\Omega_{L} / k$ $=\left(\Omega_{\mathrm{K} / \mathrm{k}} \odot \mathrm{L}\right) / \mathrm{Ld}_{\mathrm{K} / \mathrm{k}} \mathrm{a}^{\mathrm{a}}+\mathrm{Ldt}$; if $\mathrm{B}^{\prime} \mathrm{CK}$ is such that $\{\mathrm{a}\}^{\sim} \mathrm{B}^{\prime}$ is a p-basis of $\mathrm{K} / \mathrm{k}$ and a $q \mathrm{~B}$, then $\{t\} \mathrm{B}^{\text {th }}$ is a p-basis of $\mathrm{L} / \mathrm{k}$. So if $\mathrm{b}_{1}, \ldots, \mathrm{b}_{\mathrm{m}} \in \mathrm{B}^{\prime}$ and $\left\{\mathrm{a}, \mathrm{b}_{1}, \ldots, \mathrm{b}_{\mathrm{m}}\right\}$ is $\mathrm{p}$-indep. in $K$ over $k_{\alpha}$, then $\left\{t, b_{1}, \ldots, b_{m}\right\}$ is $p$-indep. in $L$ over $k_{\alpha}{ }^{\circ}$

(40. C) Let $k$ be a field of characteristic $p, R=k\left[\left[X_{1}, \cdots,\right.\right.$, $\left.X_{n}\right]$ ], $P \in \operatorname{Spec}(R)$ and $A=R / P, \quad$ Let $y_{1}, \ldots, y_{r}(r=\operatorname{dim} A)$ be a system of parameters of $A$ and put $B=k\left[\left[y_{1}, \ldots, y_{r}\right]\right]$.

Then A is finite over B. Let $k^{\prime}$ be a cofinite subfield of $k$ and put $C^{\prime}=k^{\prime}\left[\left[y_{1}^{p}, \ldots, y_{r}^{p}\right]\right]$. Since every derivation $D \varepsilon$ $\operatorname{Der}(A)$ is continuous (in any ideal-adic topology), we have $\operatorname{Der}_{k^{\prime}}(A)=\operatorname{Der}_{C^{\prime}}(A)$, and $A$ is finite over $C^{\prime}$. Let $L, K, K^{\prime}$ denote the quotient fields of $A, B, C^{\prime}$. Then it is easy to see that rank $\operatorname{Der}_{k^{\top}}(A)=\left(L: K^{\prime}\right)_{p}=\operatorname{rank} \Omega_{L / K} /$, and similarly rank $\operatorname{Der}_{k}(B)=\left(K: K^{\prime}\right)_{p}=\operatorname{rank} \Omega_{K / K^{\prime}}$. If $E$ is a p-basis of $k$ over $k^{\prime}$ then $E \cup\left\{y_{1}, \ldots, y_{r}\right\}$ is a p-basis of $B$ over $C^{\prime}$. Therefore rank $\Omega_{K / K^{\prime}}=\operatorname{dim} A+\left(k: k^{\prime}\right)$, and in general we have by Th. 59

$\operatorname{rank} \operatorname{Der}_{k^{\prime}}(A)=\operatorname{rank} \Omega_{L / K^{\prime}} \geqslant \operatorname{rank} \Omega_{K / K^{\prime}}=\operatorname{dim} A+\left(k: k^{\prime}\right){ }_{p} \cdot$

THEOREM 96. Let $k, R$ and $A$ be as above, and let $\left.F=k_{\alpha}\right\}_{\alpha E I}$ be a family of cofinite subfields of $k$, directed downwards, such that $\cap k_{\alpha}=k^{p}$. Then there exists $k_{\alpha} \varepsilon F$ such that, for every cofinite subfield $k^{\prime}$ of $k_{\alpha}$, we have

$\operatorname{rank} \operatorname{Der}_{k},(A)=\operatorname{dim} A+\left(k: k^{\prime}\right)_{p^{\circ}}$

Proof. If $L=K$ then the theorem is obvious, so we will prove the existence of $\alpha$ such that $\left(L: K^{\prime}\right)_{p}=\left(K: K^{\prime}\right)$ for $k^{\prime} \subseteq k_{\alpha}$ by induction on $(L: K)$. Suppose that our claim is proved for every proper subfield $L^{\prime}$ of $L$ containing $K$, and let $L^{\prime}$ be maximal among such subfields. If L is separable over $L^{\prime}$ then $\Omega_{L / K^{\prime}}=\Omega_{L^{\prime} / K^{\prime}} \otimes L$ and we are done. So we can suppose $L=L^{\prime}(t), t^{p}=a \varepsilon L^{\prime}$. Then $a \notin L^{\prime} P$. Put $K_{\alpha}=$ $k_{\alpha}\left(\left(y_{1}{ }^{p}, \ldots, y_{r}{ }^{p}\right)\right) .$ Then $\cap k_{\alpha}=k^{p}\left(\left(y_{1}{ }^{p}, \ldots, y_{r}{ }^{p}\right)\right)=k^{p}$ by p.229, hence $\cap \mathrm{K}_{\alpha} \mathrm{L}^{\prime} \mathrm{P}=\mathrm{L}^{\prime} \mathrm{P}$ by Lemma 2. Therefore there exists $\alpha$ such that a $\notin K_{\alpha} L^{\prime P}$ and such that $\left(L^{\prime}: K^{\prime}\right)_{p}=\left(K: K^{\prime}\right)_{p}$ for $k^{\prime} \subseteq k_{\alpha^{\circ}}$ Then for $k^{\prime} \subseteq k_{\alpha}$ we have a $\notin K^{\prime} L^{\prime}, i \cdot e \cdot d_{L^{\prime} / K^{\prime}}$, $\neq 0$, hence $\Omega_{L / K^{\prime}}=\left(\Omega_{L^{\prime}} / K^{\prime} \otimes \mathrm{L}\right) / \mathrm{Ld}_{\mathrm{L}^{\prime} / \mathrm{K}^{\prime} \mathrm{a}}+\mathrm{Ldt}$, and so $\operatorname{rank} \Omega_{L / K^{\prime}}=\operatorname{rank} \Omega_{L^{\prime}} / K^{\prime}=\operatorname{rank} \Omega_{K / K^{\prime}}$ THEOREM 97 (Nagata). Let $k$ be a field, $R=k\left[\left[X_{1}, \ldots, X_{n}\right]\right]$

and $P \varepsilon \operatorname{Spec}(R)$. Then rank $J(P ; \operatorname{Der}(R))(P)=h t P$.

Proof. Here we consider only the case $\operatorname{ch}(\mathrm{k})=\mathrm{p}$. The case

$\operatorname{ch}(k)=0$ is easier, and we will prove a much more general

result soon.

Put $\mathbf{A}=\mathrm{R} / \mathrm{P}$ and $\mathrm{r}=\operatorname{dim} \mathrm{A}$. $\mathrm{By}$ the preceding theorem

there exists a cofinite subfield $k^{\prime}$ of $k$ such that

rank $\operatorname{Der}_{\mathrm{k}^{\prime}}(\mathrm{A})=\mathrm{r}+\left(\mathrm{k}: \mathrm{k}^{\prime}\right)_{\mathrm{p}} \cdot$

Put $s=\left(k: k^{\prime}\right) p_{p}$. If $\left\{u_{1}, \ldots, u_{s}\right\}$ is a p-basis of $k / k^{\prime}$ then

$\left\{u_{1}, \ldots, u_{s}, x_{1}, \ldots, x_{n}\right\}$ is a p-basis of $R$ over $k^{\prime}\left[\left[x_{1}^{p}, \ldots, x_{n}^{p}\right]\right]$.

Let $\phi: R \rightarrow A$ denote the natural map and put $X_{i}=u_{s+i}, D_{i}$

$=\phi \cdot \partial / \partial u_{i} \quad(1 \leqslant i \leqslant n+s)$. Then $\operatorname{Der}_{k^{\prime}}(R, A)$ is a free A-module

of rank $n+s$ with $D_{1}, \ldots, D_{n+s}$ as a basis. Let now $\bar{D}$ be an

arbitrary element of $\operatorname{Der}_{k^{\prime}}(A)$, and put $\bar{D}\left(\phi u_{i}\right)=\bar{c}_{i} \in A$.

Then $\bar{D}$ is induced by $D=\bar{c}_{i} D_{i} \varepsilon \operatorname{Der}_{k},(R, A)$ in the sense that

$\bar{D} \cdot \phi=D$. The derivation $\bar{D}$ is determined by $\bar{c}_{i}(1 \leqslant i \leqslant n+s)$,

and these must satisfy
$$
\sum_{i=1}^{n+s} \bar{c}_{i} D_{i}(f)=0 \quad \text { for all } f \in P .
$$
Conversely, if $\bar{c}_{\dot{i}}$ satisfy these linear equations then $\mathrm{D}=$

$\overline{\Sigma \bar{c}_{i}} D_{i}$ induces a derivation of $A$ over $k^{\prime}$. Therefore $r+s=$

rank $\operatorname{Der}_{k}(A)=n+s-\operatorname{rank} J\left(P ; \operatorname{Der}_{k}(R)\right)(P)$, whence we get

rank $J\left(P ; \operatorname{Der}_{k},(R)\right)(P)=n-r=h t P$. Since rank $J(P ; \operatorname{Der}(R))$

(P) $\leqslant$ ht $P$ by Th.94, we are done. (40.D) Let $(\mathrm{A}, W)$ be a noetherian complete local ring containing a field. Let $k$ be a coefficient field of $A$ and let $m=\left(x_{1}, \ldots, x_{n}\right) .$ Putting $R=k\left[\left[x_{1}, \ldots, x_{n}\right]\right]$ we then have $A=R / I$ with some ideal I of $R$. Let $P=P / I E \operatorname{Spec}(A)$. If $A_{p}=R_{P} / I R_{P}$ is regular, then $I R_{P}=Q R_{P}$ for some $Q \varepsilon \operatorname{Spec}(R)$, $Q \subseteq P$, and we have $\operatorname{rank} J(I ; \operatorname{Der}(R))(P)=h t Q=h t R_{P}$ by Th.95 and Th.97. Put $r=h t R_{p}$ and 1 et $f_{1}, \ldots, f_{r} \varepsilon I$ and $D_{1}, \ldots, D_{r} \varepsilon \operatorname{Der}(R)$ be such that $\operatorname{det}\left(D_{i} f_{j}\right) \& P$. Then $I R_{P}=$ $\mathrm{QR}_{\mathrm{P}}=\sum \mathrm{f}_{1} \mathrm{R}_{\mathrm{P}}$, hence there exists $\mathrm{g} \varepsilon \mathrm{R}-\mathrm{P}$ such that IR $\mathrm{I}_{\mathrm{g}}=$ $\sum_{1}^{r} \mathrm{f}_{i} \mathrm{R}_{\mathrm{g}} \cdot$ Put $\mathrm{h}=\operatorname{det}\left(\mathrm{D}_{i} \mathrm{f}_{\mathrm{j}}\right)$. If $\mathrm{P}^{\prime} / \mathrm{I}=p^{\prime} \varepsilon$ Spec(A) is such that $h g \& P^{\prime}$, then $\mathrm{R}_{\mathrm{P}^{\prime}} / \mathrm{IR}_{\mathrm{P}^{\prime}}=\mathrm{A}_{p^{\prime}}$ is regular by Th.94 (note that IR $\mathrm{P}^{\prime}$ is generated by r elements). Thus $\operatorname{Reg}(A)$ contains the open neighborhood $\left\{p^{\prime} \mid \overline{h g} \notin p^{\prime}\right\}$ of $p$. Therefore $\operatorname{Reg}(A)$ is open in $\operatorname{Spec}(\mathrm{A})$, and we have proved the Cor. on $\mathrm{p} .222$.

THEOREM 98. Let $(A, m)$ be a noetherian local domain containing $Q$. Let $k$ be a quasi-coefficient field of A, i.e. a subfield of $A$ such that $A / m$ is algebraic over $k$. Then:

$\operatorname{rank}^{\operatorname{Der}}(\mathrm{A}) \leqslant \operatorname{dim} A$.

Proof, We will prove that $\operatorname{Der}_{k}(\mathrm{~A})$ is isomorphic to a submodule of $A^{n}$, where $n=\operatorname{dim} A . \quad$ Take a system of parameters $x_{1}, \ldots$, $x_{n}$ of A. We claim that the map $\phi: \operatorname{Der}_{k}(A) \rightarrow A^{n}$ defined by $\phi(D)=\left(D x_{1}, \ldots, D x_{n}\right)$ is injective. Suppose that $D \varepsilon \operatorname{Der}_{k}(A)$ and $\mathrm{Dx}_{1}=\ldots=\mathrm{Dx}_{\mathrm{n}}=0$. By continuity $\mathrm{D}$ is uniquely extended to the completion $A^{*}$. Now $A^{*}$ is finite over the subring $k\left[\left[x_{1}, \ldots, x_{n}\right]\right]$, on which $D$ vanishes. Let a $\varepsilon$ A. As an element of $A^{*}$ it satisfies a polynomial relation $f(a)=0$ with coefficients in $k\left[\left[x_{1}, \ldots, x_{n}\right]\right]$. Choose such a polynomial $f(T)$ of lowest degree. Then $0=D(f(a))=f^{\prime}(a) D a$ and $f^{\prime}(a)$ $\neq 0$. Since Da $\varepsilon$ A and since the non-zero elements of A are not zero divisors in $A *$, we must have $\mathrm{Da}=0$. Thus $\mathrm{D}=0$.

Letepe $(\mathrm{R}, \mathrm{m})$ be a regular local ring of dimension nontaining a ffeld. Let $R^{*}$ be the completion of $R$ and $k$ be a coefficient field of $R^{*}$ containing a quasi-coefficient field $k$ of $R$. Let $x_{1}, \ldots, x_{n}$ be a regular system of parameters of $R$. Then $R^{*}=k\left[\left[x_{1}, \ldots, x_{n}\right]\right]$, a formal power series ring over $k$, and $\operatorname{Der}_{k}\left(R^{*}\right)$ is a free $R^{*}$-module with the partial derivations $\partial / \partial x_{1}, \ldots, \partial / \partial x_{n}$ as a basis. Then the following conditions are all equivalent:

(1) $\partial / \partial x_{i}(1 \leqslant i \leqslant n) \operatorname{map} R$ into $R, i, e . \partial / \partial x_{1} \varepsilon \operatorname{Der}_{k_{0}}$ (R);

(2) there exist $D_{1}, \ldots, D_{n} \varepsilon \operatorname{Der}_{k_{0}}$ (R) and $a_{1}, \ldots, a_{n} \varepsilon R$ such that $\mathrm{D}_{i} \mathrm{a}_{j}=\delta_{i j}$;

(3) there exist $D_{1}, \ldots, D_{n} \in \operatorname{Der}_{k_{0}}$ (R) and $a_{1}, \ldots, a_{n} \in R$ such that $\operatorname{det}\left(D_{i^{j}}\right)^{\&} \boldsymbol{m}$;

(4) $\operatorname{Der}_{k_{0}}(R)$ is a free $R$-module of rank $n$;

(5) $\operatorname{rank} \operatorname{Der}_{\mathrm{k}_{0}}(\mathrm{R})=\mathrm{n}$. (Remark. Since $\operatorname{Der}_{k_{0}}(k)=0$ we have $\operatorname{Der}_{k_{0}}(R)=\operatorname{Der}_{k}(R *) \cap$ $\operatorname{Der}(R)$. If we define $\operatorname{Der}_{k}(R)$ by $\operatorname{Der}_{k}\left(R^{*}\right) \cap \operatorname{Der}(R)$ then Th.98 and Th.99 hold for any coefficient field $k$ of $R^{*}$ and the mention to quasi-coefficient field is superfluous.)

Proof. Let $K$ and $L$ denote the quotient fields of $R$ and $R *$. The implications $(1) \Rightarrow(2) \Rightarrow(3)$ and $(4) \Rightarrow(5)$ are trivial.

(3) $\Rightarrow(4)$ : Clearly $D_{1}, \ldots, D_{n}$ are linearly independent over $R$ as well as over $R *$. So every $D \varepsilon \operatorname{Der}_{k_{0}}$ (R) can be written as $D=\sum c_{i} D_{i}$ with $c_{i} \varepsilon$ L. Solving the equations $D_{j}=$ $\Sigma c_{i} D_{i} a_{j}$, we get $c_{i} \varepsilon$ R.

(5) $\Rightarrow(1):$ Let $D_{1}, \ldots, D_{n}$ be linearly independent over $R$. This means that there exists $a_{1}, \ldots, a_{n} \varepsilon R$ with $\operatorname{det}\left(D_{f} a_{f}\right) \neq 0$. Therefore $D_{1}, \ldots, D_{n}$ are linearly independent over $R^{*}$ also. Hence $\partial / \partial \mathrm{x}_{i}=\Sigma_{j} c_{i j} D_{j}$ with $c_{i j}$ in L. Then $\delta_{i k}=\Sigma_{j} c_{i j} D_{j} x_{k}$, therefore the matrix $\left(c_{i j}\right)$ is the inverse of $\left(D_{j} x_{k}\right)$ and so $c_{i f} \in K . \quad$ Then $\left(\partial / \partial x_{i}\right)(R) \subseteq K \cap R^{*}=R .$

(40.F) We will say that (WJ) (= weak Jacobian condition) holds in a regular ring $R$ if rank $J(P ; \operatorname{Der}(R))(P)=$ ht $P$ for every $P \varepsilon \operatorname{Spec}(\mathrm{R})$. The reasoning of (40.D) and Th.95 show that, if $A$ is a homomorphic image of a regular ring $R$ in which (WJ) holds, then $\operatorname{Reg}(A)$ is open in $\operatorname{Spec}(A)$. For the definition and the theory of the strong Jacobian condition (SJ), we refer to our article 'Noetherian rings with many deriva- tions', in Contributions to Algebra (dedicated to E. Kolchin) et by Bass et a1., Academic Press, $1977 .$

THEOPEM 100 . Let $(R, M, K)$ be a regular local ring of dimenion $n$ containing a field $k$ of characteristic 0 . Assume that

(1) $\mathrm{y}$ is algebraic over $k$, and (2) rank $\operatorname{Der}_{k}(R)=n$. Then:

i) (WJ) holds in R,

ii) if $P \varepsilon \operatorname{Spec}(R)$ then every element of $\operatorname{Der}_{k}(R / P)$ is induced by an element of $\operatorname{Der}_{k}(R)$,

iii) $\operatorname{rank} \operatorname{Der}_{k}(\mathrm{R} / \mathrm{P})=\operatorname{dim} \mathrm{R} / \mathrm{P}$.

We use the notation of Th.99. Then there exists $D_{1}, \ldots, D_{n}$ e $\operatorname{Der}_{k}(R)$ and $x_{1}, \ldots, x_{n} \varepsilon m$ such that $D_{i} x_{j}=\delta_{i j}$, and $\operatorname{Der}_{k}(R)$

is a free $R$-module with $D_{1}, \ldots, D_{n}$ as a basis. Put $A=R / P$ and let $\phi: R \rightarrow A$ denote the natural map. Then $\operatorname{Der}_{k}(R, A)$ is

a free A-module with $\phi \cdot D_{i}(1 \leqslant 1 \leqslant n)$ as a basis. If $D \varepsilon \operatorname{Der}_{k}(A)$ let $c_{i} \varepsilon R$ be such that $\phi\left(c_{i}\right)=\bar{D} \phi\left(x_{i}\right)$. Then $D=\sum c_{i} D_{1} \varepsilon$ Der $(R)$ induces $\vec{D}$ in the sense that $\phi \cdot D=\bar{D} \bullet \phi$. Let $\left(u_{1}, \ldots, u_{n}\right)$

$\varepsilon A^{n}$. Then $\sum u_{i} \phi \circ D$, induces a derivation $D \varepsilon \operatorname{Der}_{k}$ (A) iff $\sum u_{i} \phi\left(D_{i} f\right)=0$ for all $f \varepsilon P$. Thus

$\operatorname{rank} \operatorname{Der}_{k}(A)=n-\operatorname{rank} J\left(P ; \operatorname{Der}_{k}(R)\right)(P)$.

The $1 e f t-h a n d ~ s i d e ~ i s \leqslant \operatorname{dim} A=\mathrm{A}-$ ht $\mathrm{P}$ by Th.98, and the The lett i) and iii). THEOREM 101. Let $R$ be a regular ring containing $Q$. If (WJ) holds in $R$, then $R$ is excellent.

Proof. Since $R$ is Cohen-Macaulay it is universally catenary. We have already remarked that (WJ) implies the openness of $\operatorname{Reg}(R / P)$ in $\operatorname{Spec}(R / P)$ for every $P \in \operatorname{Spec}(R)$, and as $R$ contains Q this proves that $R$ is $J-2$ by $T h .73(3), p .246$. To prove that $R$ is a $G$ ring we can assume that $R$ is a regular local ring, and we have to show that the formal fibres of $R$ are regular. Let $P$ be a pime ideal of the conpletion $R *$ and $p u t p=P \cap R$ rank $J\left(p ; D_{1}, D^{\prime}\right)(p)=r \cdot D^{r} E$ Der $(\mathrm{R})$ such that $D_{i}$ to $\mathrm{R}^{*}$ and view the matrix $J\left(p ; D^{\prime} 1^{\prime} \cdot D^{\prime}\right)(p)$ as $J\left(p R^{*}\right.$ $1, D^{1}$ to $R *$ and view the matrix $J\left(p ; D^{1}, \ldots, D^{\prime}\right)(p)$ as $J(p R * ;$ by $(13 . B)$. Therefore $R^{*} P / P R{ }_{P}$ is regular, Q.E.D.

THEOREM 102. Let $k$ be a field of characteristic 0 , and $R$ be a regular ring containing $k$. Suppose that (1) for any maximal ideal in of $R$, the residue field $R / m$ is algebraic over $k$ and $h t w=n$, and (2) there exist $D_{1}, \ldots, D_{n} \varepsilon \operatorname{Der}_{k}(R)$ and $x_{1}, \ldots, x_{n} \varepsilon R$ such that $D_{i} x_{j}=\delta_{i j} .$ Then $R$ is excellent.

Proof. By Th. 100 it is clear that (WJ) holds in R. Q.E.D.

Remark. Convergent power series rings over $R$ or $C$, formal power series rings over a field $k$ of characteristic 0 , and more generally the rings of type $k\left[x_{1}, \ldots, X_{n}\right]\left[\left[Y_{1}, \ldots, Y_{m}\right]\right]$ where $k$ is a field of char. 0 , are examples of regular rings to which the theorem applies. Formal power series rings over a convergent power series ring also belong to the class. On the other hand there are excellent regular rings containing a coefficient field $k$ of char. 0 , such that $\operatorname{Der}_{k}(R)=0$ Example: Let $k$ be a field of char. 0 and let $f(X)$ be a formal power series such that $f(X), f^{\prime}(X)$ and $X$ are algebraically independent over $k(e . g . f=\exp (\exp (X))$ will do), Let $f=$ $\sum a_{i} x^{i}, a_{i} E k$, and put $y_{i}=\sum_{j=i}^{\infty} a_{j} x^{j-1}(i=0,1,2, \ldots)$. Then $y_{0}=f(X)$ and $y_{i}=a_{i}+x y_{i+1}$. Put $R=k\left[X, y_{0}, y_{1}, \ldots\right]$. Then $R / X R=k$, so that $X R$ is a prime ideal. Put $A=R_{X R}$. Since $A$ is a subring of $k[[X]]$ it is $X$-adically separated, so it is a regular local ring of dimension 1 and $\operatorname{ch}(A)=0$, hence $A$ is excellent. Its completion $A^{*}$ is $k[[X]]$ and $d / d X$ maps $f$ to $\mathrm{F}^{\prime}$ which is not in $k\left(X, y_{0}\right)$, hence not in A. By Th.99 we see that $\operatorname{Der}_{k}(A)=0$

THEOREM 103. Let $R$ be a regular ring. If (WJ) holds in $R\left[X_{1}, \ldots, X_{n}\right]$ for every $n \geqslant 0$, then $R$ is excellent.

Proof. The condition implies that $\operatorname{Reg}(B)$ is open in Spec (B) for every finitely generated $R$-algebra $B$, $1 .$ e. that $R$ is $J-2$. To prove that $R$ is a G-ring we may assume that $R$ is local, and we have to prove that the formal fibres are geometrically regular. By (33.E) Lemma 3 , it suffices to prove that, if (is a localization of a finite R-algebra which is a domain, and if is a prime ideal of $C^{*}$ such that $Q \cap C=(0)$, then Q is regular. Now is a homomorphic image of a localization of some $R\left[x_{1}, \ldots, X_{n}\right]$, and our assertion is proved by the

Remark. It is easy to see that, if $R$ contains $Q$, then (WJ) in $R$ implies (WJ) in $R[X]$, But this is not so in the case of characteristic p. In fact, the ring A of $(34 . B)$ is a counterexample.







\includegraphics[max width=\textwidth]{2022_08_01_8d4eee36f1f42236b4f4g-164}

trivial $\leadsto 178$

\includegraphics[max width=\textwidth]{2022_08_01_8d4eee36f1f42236b4f4g-165}

Zariski ring 172

\includegraphics[max width=\textwidth]{2022_08_01_8d4eee36f1f42236b4f4g-166}


\end{document}
