\documentclass[../main]{subfiles}
\begin{document}

\section{Jacobian Criteria}\label{sec:29}
\newparagraph Let $k$ be a field, and $I$ be an ideal of $k[X_1, \ldots, X_n]$. Let $P$ be a prime ideal containing $I$, and put $A = k[X_1, \ldots, X_n]$, $B = A/I$ and $\ideal p = P/I$. Then $B_{\ideal p} = A_P/I A_P$; let $\kappa$ denote the common residue field of $A$ and $B_{\ideal p}$. Put $\dim A_P = m$ and $\Ht(I A_P) = r$. Since $A$ is catenarian we have $\dim B_{\ideal p} = m - r$. We know that $A_P$ is a regular local ring, and that $B_{\ideal p}$ is regular iff $IA_P$ is a prime ideal generated by a subset of a regular system of paramters of $A_P$ (cf.\ref{17.F} Th.\ref{thm:036}). We have $\rank_K(P/P^2 \otimes_A \kappa) = m$, and
\[
\rank_K(\ideal p/\ideal p^2 \otimes_B \kappa) = m - \rank_K ((P^2 + I)/P^2 \otimes_A \kappa) \ge \dim B_{\ideal p} = m - r.
\]
Therefore
\[
\rank_K ((P^2 + I)/P^2 \otimes_A \kappa) \le r,
\]
and the equality holds iff $B_{\ideal p}$ is regular. The left hand side is the rank of the image of the natural map $\nu : I/I^2 \otimes_A \kappa \longrightarrow P/P^2 \otimes_A K$. 

To each polynomial $f(X) \in P$ we assign the vector in $\kappa^n$ $(\partial f/\partial X_1, \ldots, \partial f/\partial X_n) \mod P$. Then we get a $\kappa$-linear map $P/P^2 \otimes_A \kappa \longrightarrow \kappa^n$. If we identify $\kappa^n$ with \[\Omega_{A/k} \otimes_A \kappa = \Omega_{A_P/k} \otimes_{A_P} \kappa = \sum\limits_1^n \kappa \dd X_i,\] the map just defined is nothing but the map $\delta$ of the second fundamental exact sequence (cf.\ref{26.I})
\[
P/P^2 \otimes \kappa = P A_P/P^2 A_P \varrightarrow{\delta} \Omega_{A_P/k} \otimes \kappa \longrightarrow \Omega_{\kappa/k} \longrightarrow 0.
\]

If $I = (f_1(X), \ldots, f_s(X))$, then the image of $\delta \nu : I/I^2 \otimes \kappa \longrightarrow \Omega_{A/k} \otimes \kappa$ is generated by the vectors $(\partial f_i/\partial X_1, \ldots, \partial f_i/\partial X_n) \mod p\for{1 \le i \le s}$, so that \[\rank_K(\Im(\delta \nu)) = \rank\bigg(\dfrac{\partial(f_1, \ldots, f_s)}{\partial (X_1, \ldots, X_n)}\bigg) \mod P,\] where the right hand side is the rank of the Jacobian matrix evaluated at the point $P$; we write the matrix $(\partial(f)/\partial(X))(P)$ for short. Thus, if we have 
\begin{equation}
\label{eqn:s29}
\tag{29.*}
\rank\bigg(\dfrac{\partial(f_1, \ldots, f_s)}{\partial (X_1, \ldots, X_n)}\bigg)(P) = r,
\end{equation}

then we must have $\rank \, \Im(\nu) = r$ also, and hence $B_{\ideal p}$ is regular. When the residue field $\kappa$ is separable over $k$ we have 
\[
\rank_\kappa \Omega_{\kappa/k} = \TrDeg_k \kappa = n - \Ht(P) = n - m
\]
by \ref{27.B} Th.\ref{thm:059}, while $\rank \, P/P^2 \otimes \kappa = m$. So the map $\delta : P/P^2 \otimes \kappa \longrightarrow \Omega_{A/k} \otimes \kappa$ is injective. In this case the condition \eqref{eqn:s29} is equivalent to the regularity of $B_{\ideal p}$. 

The condition \eqref{eqn:s29} is nothing but the classical definition of a simple point. The above consideration shows that, when $k$ is perfect, the point $\ideal p$ is simple on $\Spec(B)$ iff its local ring $B_{\ideal p}$ is regular. In the general case note that \eqref{eqn:s29} is invariant under any extension of the ground field $k$. Thus, if $k'$ denotes the algebraic closure of $k$ and if $P'$ is a prime ideal of $A' = k'[X_1, \ldots, X_n]$ lying over $P$, then $\ideal p$ is simple on $\Spec(B)$ iff the local ring $B'_{\ideal p'} = (A'/IA')_{P'/IA'}$ is regular. Since $\kappa$ is finitely generated over $k$, it is also easy to see that \eqref{eqn:s29} is equivalent to the geometrical regularity of $B_{\ideal p}$ over $k$. 

\newparagraph The results of the preceding paragraph can be more fully described by the notion of formal smoothness. We begin by proving lemmas. 

\begin{lemma}
\label{lem:29.1}
Let $k \longrightarrow B$ be a continuous homomorphism of topological rings and suppose $B$ is formally smooth over $k$. Then, for any open ideal $J$ of $B$, $\Omega_{B/k} \otimes (B/J)$ is a projective $B/J$-module. 
\end{lemma}

(In this case we say that the $B$-module $\Omega_{B/k}$ is \defemph{formally projective}\index{formally!\indexline projective}.)

\begin{proof}
Let $u : L \longrightarrow M$ be an epimorphism of $B/J$-modules. We have to prove that $\Hom_B(\Omega_{B/k}, L) \longrightarrow \Hom_B(\Omega_{B/k}, M)$ is surjective, i.e. that $\Der_K(B, L) \longrightarrow \Der_k(B,M)$ is surjective. Let $D \in \Der_k(B,M)$, and consider the commutative diagram

\begin{center}
\begin{tikzcd}
B \arrow[rr, "v"]      &  & (B/J) \ast M                 \\
k \arrow[u] \arrow[rr] &  & (B/J) \ast L \arrow[u, "j"']
\end{tikzcd}
\end{center}

where $j(x,y) = (x, uy)$ and $v(b) = (b \mod J, D(b))$. Let $v : B \longrightarrow (B/J) \ast L$ be a lifting of $v$. Then we have $v'(b) = (b \mod J, D'(b))$ with a derivation $D' \in \Der_k(B, L)$ and $u D' = D$.
\end{proof}

\begin{lemma}
\label{lem:29.2}
Let $B$ be a ring, $J$ an ideal of $B$ and $u : L \longrightarrow M$ a homomorphism of $B$-modules. Suppose $M$ is projective. Furthermore, assume either ($\alpha$) $J$ is nilpotent, or that ($\beta$) $L$ is a finite $B$-module and $J \subseteq \rad(B)$. Then $u$ is left-invertible iff $\overline u : L/JL \longrightarrow M/JM$ is also. 
\end{lemma}

\begin{proof}
``Only if'' is trivial, so suppose $\overline u$ has a left-inverse $\overline v : M/JM \longrightarrow L/JL$. Since $M$ is projective we can lift $\overline v$ to $v : M \longrightarrow L$; put $w = vu$. Then $L = w(L) + JL$, hence $L = w(L)$ by \hyperref[NAK]{NAK}. Then $w$ is an automorphism. [In fact, it is generally true that \defemph{a surjective endomomorphism $f$ of a finite $B$-module $L$ is an automorphism}. Here is an elegant proof due to Vasconcelos: Let $B[T]$ operate on $L$ by $T \xi = f(\xi)$. Then $L = TL$, hence by \hyperref[NAK]{NAK} there exists $\phi(T) \in B[T]$ such that $(1 + T \phi(T))L = 0$; then $T \xi = 0$ implies $\xi = 0$.] Therefore $w^{-1}v$ is a left-inverse of $u$. 
\end{proof}

\begin{partheorem}
\label{thm:063}
Let $k$ and $A$ be topological rings (cf. \ref{28.B}) and $g : k \longrightarrow A$ a continuous homomorphism. Let $Q$ be an ideal of definition of $A$, let $I$ be an ideal of $A$ and put
\[
B = A/I, \quad \ideal q = (Q + I)/I.
\]
Suppose that $A$ is Noetherian and formally smooth over $k$. Then the following are equivalent: 

\begin{enumerate}
    \item[(1)] $B$ (with the $\ideal q$-adic topology) is f.s. over $k$;
    \item[(2)] the canonical maps \[\delta_n : (I/I^2) \otimes_B (B/{\ideal q}^n) \longrightarrow \Omega_{A/k} \otimes_A (B/{\ideal q}^n) \quad (n = 1, 2, \ldots)\] derived from the map $\delta : I/I^2 \longrightarrow \Omega_{A/k} \otimes B$ of Th.\ref{thm:058} are left-invertible;
    \item[(3)] the map \[\delta_1 : (I/I^2) \otimes (B/\ideal q) \longrightarrow \Omega_{K/k} \otimes (B/\ideal q)\] is left-invertible. (When $\ideal q$ is a maximal ideal, this condition says simply that $\delta_1$ is injective.) 
\end{enumerate}
\end{partheorem}

\begin{proof}
$(2) \implies (3)$ is trivial, while $(3) \implies (2)$ follows from the preceding lemmas. $(2) \implies (1)$ is easy and left to the reader. 

We prove $(1) \implies (2)$. Put $B_n = B/{\ideal q}^n$. The map $\delta_n$ is left-invertible iff, for any $B_n$-module $N$, the induced map \[\Hom(I/I^2, N) \longleftarrow \Der_k(A, N)\] is surjective. So fix a $B_n$-module $N$ and a homomorphism $g \in \Hom_B(I/I^2, N)$. Since $A$ is Noetherian there exists, by Artin-Rees, an integer $\nu > n$ such that $I \cap Q^\nu \subseteq Q^n I$. Then $g$ induces a map \[g_\nu : (I + Q^\nu)/(I^2 + Q^\nu) \longrightarrow I/(I^2 + (Q^\nu \cap I)) \longrightarrow I/(I^2 + Q^n I) \longrightarrow N,\] which is a homomorphism of $B_\nu$-modules. Let $E$ denote the extension
\[
0 \longrightarrow (I + Q^\nu)/(I^2 + Q^\nu) \longrightarrow A/(I^2 + Q^\nu) \longrightarrow B_\nu \longrightarrow 0
\]
of the discrete $k$-algebra $B_\nu$, and let
\[
0 \longrightarrow N \longrightarrow C \longrightarrow B_\nu \longrightarrow 0
\]
be the extension $g_{\nu^\ast}(E)$ (cf. \ref{25.E}). The ring $C$ is a discrete $k$-algebra. Since $B$ is f.s. over $k$, there exists a continuous homomorphism $v : B \longrightarrow C$ such that

\begin{center}
\begin{tikzcd}
B \arrow[rrd, "v"] \arrow[rr] &  & B_\nu       \\
k \arrow[rr] \arrow[u]        &  & C \arrow[u]
\end{tikzcd}
\end{center}

is commutative. On the other hand, by the definition of $g_{\nu^\ast}(E)$ we have a canonical homomorphism of $k$-algebras $u : A \longrightarrow A/(I^2 + Q^\nu) \longrightarrow C$ such that

\begin{center}
\begin{tikzcd}
B \arrow[rr]                &  & B_\nu       \\
A \arrow[rr, "u"] \arrow[u] &  & C \arrow[u]
\end{tikzcd}
\end{center}

commutes. Denoting the natural map $A \longrightarrow B = A/I$ by $r$, we get a derivation $D = u - vr \in \Der_k(A, N)$. It is easy to check that \[D(x) = u(x) = g(x \mod I^2)\text{ for }x \in I.\] 
\end{proof}

\begin{corollary}
If, in the notation of Th.\ref{thm:063}, $B$ is also f.s. over $k$, then the $B$-module $I/I^2$ is formally projective. 
\end{corollary}

\begin{parlemma}[EGA IV 19.1.12 \cite{egaIV}]
\label{lem:29.3}
Let $B$ be a ring, $L$ a finite $B$-module, $M$ a projective $B$-module and $u : L \longrightarrow M$ a $B$-linear map. Then the following conditions on $\ideal p \in \Spec(B)$ are equivalent, and the set of the points $\ideal p$ satisfying the conditions is open in $\Spec(B)$. 

\begin{enumerate}
    \item[(1)] $u_{\ideal p} : L_{\ideal p} = L \otimes B_{\ideal p} \longrightarrow M \otimes B_{\ideal p}$ is left-invertible.
    \item[(2)] there exists $x_1, \ldots, x_m \in L$ and $v_1, \ldots, v_m \in \Hom_B(M, B)$ such that \newline $L_{\ideal p} = \sum x_i B_{\ideal p}$ and $\det(v_i(u(x_j))) \not \in \ideal p$.
    \item[(3)] there exists $f \in B - \ideal p$ such that \[u_f : L_f = L \otimes B_f \longrightarrow M_f = M \otimes B_f\] is left-invertible. 
\end{enumerate}
\end{parlemma}

\begin{proof}
The module $M$ is a direct summand of a free $B$-module $F$. Since $L$ is finitely generated $u(L)$ is contained in a free submodule $F'$ of $F$ of finite rank which is a direct summand of $F$. Now the conditions (1), (2), (3) are not affected if we replace $M$ by $F$, and then $F$ by $F'$. Therefore we may assume that $M$ is free of finite rank. 
\begin{implyenumerate}
    \item[$(1) \implies (2)$] The assumption (1) implies that $L_{\ideal p}$ is $B_{\ideal p}$-projective, hence $B_{\ideal p}$-free. Let $x_i \in L\for{1 \le i \le m}$ be such that their images in $L_{\ideal p}$ (which are denoted by the same letters $x_i$) form a basis. Then $\{u_{\ideal p}(x_1), \ldots, u_{\ideal p}(x_m)\}$ is a part of a basis of $M_{\ideal p}$, so there exists linear forms $v_i' : M_{\ideal p} \longrightarrow B_{\ideal p}$ such that $v_i' (u_{\ideal p}(x_j)) = \delta_{ij}$. Since $M$ is free of finite rank we can write $v_i' = s_i^{-1} v_i$, $s_i \in B - \ideal p$, $v_i \in \Hom_B(M, B)$. Then $\det(v_i(u(x_j)) \not \in \ideal p$. 
    \item[$(2) \implies (3)$] Since $L$ is finite over $B$ and since $L_{\ideal p} = \sum\limits_1^m x_i B_{\ideal p}$ it is easy to find $g \in B - \ideal p$ such that $L_g = \sum x_i B_g$. Put $d = \det(v_i(u(x_j)))$ and $f = gd$. Then $L_f = \sum x_i B_f$, and $d$ is a unit in $B_f$. It follows that $M_f = u_f(L_f) + V$ with $V = \bigcap \Ker(v_i)$. Moreover, $u(x_i)\for{1 \le i \le m}$ are linearly independent over $B_f$, so that $u_f$ is injective. Thus $u_f$ is left-invertible. 
    \item[$(3) \implies (1)$] Trivial. Lastly, the set of the points $\ideal p$ which satisfy (3) is obviously open in $\Spec(B)$. 
\end{implyenumerate}
\end{proof}

\begin{partheorem}
\label{thm:064}
Let $k$ be a ring, and $A$ be a Noetherian, smooth $k$-algebra. Let $I$ be an ideal of $A$, $B = A/I$, $\ideal p \in \Spec(B)$, $P = $ the inverse image of $\ideal p$ in $A$, $\ideal q = P \cap k$ and $\kappa(\ideal p) = $ the residue field of $B_{\ideal p}$ and $A_P$. Then the following are equivalent: 

\begin{enumerate}
    \item[(1)] $B_{\ideal p}$ is smooth over $k$ (or what amounts to the same, over $k_{\ideal q})$;
    \item[(2)] the local ring $B_{\ideal p}$ (with the topology as a local ring) is formally smooth over the discrete ring $k$ or $k_{\ideal q}$;
    \item[($2'$)] the local ring $B_{\ideal p}$ is f.s. over the local ring $k_{\ideal q}$;
    \item[(3)] $(I/I^2) \otimes_B \kappa(\ideal p) \longrightarrow \Omega_{A/k} \otimes_A \kappa(\ideal p)$ is injective;
    \item[(4)] $(I/I^2) \otimes_B B_{\ideal p} \longrightarrow \Omega_{A/k} \otimes_A B_{\ideal p}$ is left-invertible;
    \item[(5)] there exists $F_1, \ldots, F_r \in I$ and $D_1, \ldots, D_r \in \Der_k(A,B)$ such that \newline $\sum_1^r F_i A_P = IA_P$ and $\Det(D_i F_j) \not \in \ideal p$;
    \item[(6)] there exists $f \in B - \ideal p$ such that $B_f$ is smooth over $k$.
\end{enumerate}

Consequently, the set $\{p \in \Spec(B) \mid B_{\ideal p} \text { is smooth over } k\}$ is open in $\Spec(B)$.
\end{partheorem}

\begin{proof}\phantom{,}
\begin{implyenumerate}
    \item[$(1) \implies (2)$] trivial.
    \item[$(2) \implies (2')$] is also trivial (cf. \ref{28.C}).
    \item[$(2) \implies (3)$] we know that the local ring $A_P$ is (smooth, hence a fortiori) f.s. over $k$, and we have $B_{\ideal p} = A_P/I A_P$ and $\Omega_{A_P/k} = \Omega_{A/k} \otimes_A A_P$. So apply Th.\ref{thm:063}.
    \item[$(3) \implies (4)$] since $\Omega_{A/k}$ is $A$-projective by Lemma~\ref{lem:29.1}, $\Omega_{A/k} \otimes B_{\ideal p}$ is $B_{\ideal p}$-projective. Apply Lemma~\ref{lem:29.2}.
    \item[$(4) \implies (5)$] apply Lemma~\ref{lem:29.3} to the $B$-linear map $I/I^2 \longrightarrow \Omega_{A/k} \otimes_A B$.
    \item[$(5) \implies (6)$] by Lemma~\ref{lem:29.3} and Th.\ref{thm:063}.
    \item[$(6) \implies (1)$] trivial.
\end{implyenumerate}

\end{proof}

\begin{remark}
The theorem has two important consequences. First, if, in the theorem, $k$ is a field, then $A$ is smooth over the prime field $k_0$ in $k$ also, and $B_{\ideal p}$ is smooth over $k_0$ iff it is regular. Therefore the set $\{\ideal p \mid B_{\ideal p} \text { is regular}\}$ is open in $\Spec(B)$.

Secondly, let $k$ be a Noetherian ring and $B$ a $k$-algebra of finite type. Then $B_{\ideal p}\for{\ideal p \in \Spec(B)}$ is smooth over $k$ iff it is f.s. over $k$. In fact $B$ is of the form $A/I$, $A = k[X_1, \ldots, X_n]$, so we can apply the theorem.
\end{remark}

\begin{remark}
When the conditions of Th.\ref{thm:064} hold, the number $r$ of (5) is equal to the height of $I A_P$.
\end{remark}

\newparagraph Nagata gave a similar Jacobian criterion for rings of the form $B = k[[X_1, \ldots, X_n]]/I$ where $k$ is a field \plscite{Cf. \cite{nagata1957a}}. By lack of space we just quote the main result in the form found in EGA:

\begin{theorem*}[cf. EGA IV 22.7.3 \cite{egaIV}]
Let $k$ be a field, and let $(A, \ideal m, K)$ be a Noetherian complete local ring. Let $I$ be an ideal of $A$, $B = A/I$, $P$ a prime ideal containing $I$ and $\ideal p = P/I$. Suppose that: 

\begin{enumerate}
    \item[(1)] $[k : k^p] < \infty$ if $\ch(k) = p > 0$,
    \item[(2)] $K$ is a finite extension of a separable extension $K_0$ of $k$, and
    \item[(3)] $A$ has a structure of a formally smooth $K_0$-algebra. Then the local ring $B_{\ideal p}$ is f.s. over $k$ iff there exist $F_1, \ldots, F_m \in I$ and $D_1, \ldots, D_m \in \Der_k(A)$ such that $I A_P = \sum F_i A_P$ and such that $\Det(D_i(F_j)) \ne 0$.
\end{enumerate}
\end{theorem*}

\begin{corollary}[EGA IV 22.7.6 \cite{egaIV}]\label{cor:29.02}
Let $B$ be a Noetherian complete local ring containing a field. Then the set $\{\ideal p \in \Spec(B) \mid B_{\ideal p} \text { is regular}\}$ is open in $\Spec(B)$. 
\end{corollary}
\end{document}