\documentclass[../main]{subfiles}
\begin{document}

\section{Closeness of Singular Locus}\label{sec:32}

\newparagraph Let $A$ be a Noetherian ring; put $X = \Spec(A)$, $\Reg(X) = \{\ideal p \in X \mid A_{\ideal p} \text { is regular}\}$ and $\Sing(X) = X - \Reg(X)$. We ask whether $\Reg(X)$ is open in $X$. 

\begin{lemma}
\label{lem:32.1}
In order that $\Reg(X)$ is open in $X$, (i) it is necessary and sufficient that for each $\ideal p \in \Reg(X)$, the set $V(\ideal p) \cap \Reg(X)$ contains a non-empty open set of $V(\ideal p)$; and (ii) it is sufficient that, if $\ideal p \in \Reg(X)$ and $Y = \Spec(A/\ideal p)$, then $\Reg(Y)$ contains a non-empty open set of $Y$. 
\end{lemma}

\begin{proof}
\begin{enumerate}
    \item This follows from \ref{22.B} Lemma \ref{lem:22.02}. 
    \item We derive the condition of (i) from (ii). Let $\ideal p \in \Reg(X)$, and choose $a_1, \ldots, a_r \in \ideal p$ which form a regular system of parameters of $A_{\ideal p}$; put $I = \sum a_i A$. As $I A_{\ideal p} = \ideal p A_{\ideal p}$, there exists $f \in A$ such that $I A_f = \ideal p A_f$. Then \[D(f) = X - V(f) \simeq \Spec(A_f)\] is an open neighborhood of $\ideal p$ in $X$. So, replacing $A$ by $A_f$ we may assume that $I = \ideal p$. Now put $Y = \Spec(A/\ideal p)$, and identify it with the closed subset $V(\ideal p)$ of $X$. By assumption, there exists a non-empty open set $Y_0$ of $Y$ contained in $\Reg(Y)$. If $\ideal q \in Y_0$, then $A_{\ideal q}/\ideal p A_{\ideal q}$ is regular and $\ideal p A_{\ideal q} = \sum\limits_1^r a_i A_{\ideal q}$ is generated by a $A_{\ideal q}$-regular sequence. Thus $\dim A_{\ideal q} = \dim A_{\ideal q}/\ideal p A_{\ideal q} + r$, so that $A_{\ideal q}$ is regular. Therefore $Y_0 \subseteq Y \cap \Reg(X)$, and the condition (i) is proved. 
\end{enumerate}
\end{proof}

\newparagraph Let $A$ be a Noetherian ring. We say that $A$ is J-0\index{J-0, J-1, J-2} if $\Reg(\Spec(A))$ contains a non-empty open set of $\Spec(A)$, and that $A$ is J-1 if $\Reg(\Spec(A))$ is open in $\Spec(A)$. Thus J-1 implies J-0 if $A$ is domain, but not in general. We say that $A$ is J-2 if the conditions of the following theorem are satisfied.

\begin{theorem}
\label{thm:073}
For a Noetherian ring $A$, the following conditions are equivalent:

\begin{enumerate}[label = (\arabic*)]
    \item any finitely generated $A$-algebra $B$ is J-1;
    \item any finite $A$-algebra $B$ is J-1;
    \item for any $\ideal p \in \Spec(A)$, and for any finite radical extension $K'$ of $\kappa(\ideal p)$, there exists a finite $A$-algebra $A'$ satisfying $A/\ideal p \subseteq A' \subseteq K'$ which is J-0 and whose quotient field is $K'$. 
\end{enumerate}
\end{theorem}

\begin{proof}
\begin{implyenumerate}
    \item[$(1) \implies (2) \implies (3)$] trivial
    \item[$(3) \implies (1)$]
    \begin{enumerate}[label= Step \Roman*.]
        \item Let $\ideal p$ and $A'$ be as in (3), and let $\omega_1, \ldots, \omega_n \in A'$ be a linear basis of $K'$ over $\kappa(\ideal p)$. Then there exists $0 \ne f \in A/\ideal p$ such that $A_f' = \sum\limits_1^n (A/\ideal p)_f \omega_i$. From this and from Th.\ref{thm:051} (i) it follows easily that $A/\ideal p$ is J-0. Therefore $A/\ideal p$ (and $A$ itself) is J-1 by Lemma~\ref{lem:32.1}. 
        \item In view of Lemma~\ref{lem:32.1}, the condition $(1)$ is equivalent to $(1')$: Let $B$ be a domain which is finitely generated over $A/\ideal p$ for some $\ideal p \in \Spec(A)$. Then $B$ is J-0.
\end{enumerate}
\end{implyenumerate}

We will prove $(1')$. Replacing $A$ by $A/\ideal p$ we may assume $A \subseteq B$. Since $A$ is J-0 by Step I we may also assume that $A$ is regular. Let $K$ and $K'$ be the quotient fields of $A$ and $B$ respectively. 
\begin{enumerate}[label=Case \arabic*.]
    \item $K'$ is separable over $K$. In this case we use only the assumption that $A$ is regular. Let $t_1, \ldots, t_n \in B$ be a separating transcendency basis of $K'$ over $K$, and put $A_1 = A[t_1, \ldots, t_n]$, $K_1 = K(t_1, \ldots, t_n)$. Then $A_1$ is a regular ring. There exists a basis $\omega_1, \ldots, \omega_r$ of $K'$ over $K_1$ such that each $\omega_i \in B$. Replacing $A$ by some $(A_1)_f\for{f \in A_1}$ and $B$ by $B_f$, we may assume $B$ is finite and free over $A$: $B = \sum\limits_1^r \omega_i A$. Put $d = \det(\Tr_{K'/K} (\omega_i \omega_j))$. Then $d \ne 0$ as $K'$ is separately algebraic over $K$. We claim that $B_d$ is a regular ring. Indeed, if $d \not \in {\ideal p}' \in \Spec(B)$ and $\ideal p = \ideal p' \cap A$, then $B_{\ideal p} = \sum\limits_1^r \omega_i A_{\ideal p}$, and putting \[\overline B = B \otimes \kappa(\ideal p) = \sum\limits_1^r \overline \omega_i \kappa(\ideal p)\] we get \[\det(\Tr_{\overline B/\kappa(\ideal p)} (\overline \omega_i \overline \omega_j)) = \overline d \ne 0\text{ in }\kappa(\ideal p).\] Therefore $\overline B = B \otimes \kappa(\ideal p)$ is a product of fields, and so $B_{\ideal p'} \otimes \kappa(\ideal p) = B_{\ideal p'}/\ideal p B_{\ideal p'}$ is a field. Since $A_{\ideal p}$ is regular and $\dim A_{\ideal p} = \dim B_{\ideal p'}$, it follows that $B_{\ideal p}$ is regular. 
    \item General case. We may suppose $\ch(K) = p$. There exists a finite purely inseparable extension $K_1$ of $K$ such that $K_1' = K'(K_1)$ is separable over $K_1$. Choose $A_1 \subseteq K_1$ as in (3). Then $A_1$ is J-0, and so $A_1[B]$ is J-0 by Case 1. Since $A_1[B]$ is finite over $B$, $B$ is itself J-0 as in Step I.
\end{enumerate}
\end{proof}

\begin{remark}
The condition (3) is satisfied if $A$ is a Nagata ring of dimension $1$. Indeed, $A/\ideal p$ is either a field -- in which case (3) is trivial -- or a Nagata domain of dimension $1$, and then the integral closure $A'$ of $A$ in $K'$ is finite over $A$ and is a regular ring. 
\end{remark}

\begin{partheorem}
\label{thm:074}
Let $A$ be a Noetherian complete local ring. Then $A$ is J-2. 
\end{partheorem}

\begin{proof}
Any finite $A$-algebra $B$ is a finite product of complete local rings: \newline $B = B_1 \times \ldots \times B_s$ and $B$ is J-1 iff each $B_i$ is so. Therefore, by Th.\ref{thm:073} and Lemma~\ref{lem:32.1}, it suffices to prove that a Noetherian complete local domain $A$ is J-0.
\begin{enumerate}[label = Case \Roman*. ]
    \item $\ch(A) = 0$. The ring $A$ is finite over a suitable subring $B$ which is a regular local ring, and by the case 1 of Step II of the preceding proof we see that $A$ is J-0.
    \item $\ch(A) = p$. Then $A$ contains the prime field, hence also a coefficient field $K$, so that $A$ is of the form $K[[X_1, \ldots, X_n]]/I$. Therefore $A$ is J-1 by the Jacobian criterion of Nagata \ref{29.F}.
\end{enumerate}
\end{proof}
\end{document}