\documentclass[../main]{subfiles}
\begin{document}

\section{Local Criteria of Flatness}\label{sec:20}
\newparagraph In \ref{18.B} Lemma \ref{lem:18.04} % fix reference?
we proved the following.

\begin{quote}
    Let $(A,\ideal M)$ be a Noetherian local ring and $M$ a finite $A$-module. Then $M$ is flat iff $\Tor_1(M,A/\ideal M) = 0$.
\end{quote}
The condition that $M$ is finite over $A$ is too strong; in geometric application it is often necessary to prove flatness of infinite modules. In this section we shall learn several criteria of flatness, due to Bourbaki, which are very useful. 

Let $A$ be a ring, $I$ an ideal of $A$ and $M$ an $A$-module. We say that $M$ is \defemph{idealwise separated}\index{idealwise separated} (i.s. for short) for $I$ if, for each finitely generated ideal $\ideal q$ of $A$, the $A$-module $\ideal q\otimes_A M$ is separated in the $I$-adic topology.

\begin{example}
Let $B$ be a Noetherian $A$-algebra such that $IB\subseteq \rad(B)$, and let $M$ be a finite $B$-module. Then $M$ is i.s. for $I$ as an $A$-module: since $\ideal q\otimes_A M$ is a finite $B$-module and since the $I$-adic topology on $\ideal q\otimes M$ is nothing but the $IB$-adic topology, we can apply \ref{11.D} %fix reference
Cor. \ref{cor:11.01}.
\end{example}

\begin{example}
When $A$ is a principal ideal domain, any $I$-adically separated $A$-module $M$ is i.s. for $I$.
\end{example}

\begin{example}
Let $M$ be an $I$-adically separated flat $A$-module. Then $M$ is i.s. for $I$. In fact we have $\ideal q\otimes M\cong \ideal q M\subseteq M$.
\end{example}

\newparagraph Put \[\gr(A) = \gr^I(A) = \bigoplus_{n=0}^\infty I^n/I^{n+1},\] \[\gr(M) = \gr^I(M) = \bigoplus_{n=0}^\infty I^nM/I^{n+1}M,\] $A_0 = \gr_0(A) = A/I$ and $M_0 = \gr_0(M) = M/IM$. Then $\gr(M)$ is a graded $\gr(A)$-module. There are canonical epimorphisms \[\gamma_n\colon I^n/I^{n+1}\otimes_{A_0}M_0\longrightarrow I^nM/I^{n+1}M\] for $n = 0,1,2,\ldots$. In other words, there is a degree-preserving epimorphism $\gamma\colon \gr(A)\otimes_{A_0}M_0\longrightarrow \gr(M)$.

\begin{partheorem}[Local criteria of flatness]\label{thm:049}
Let $A$ be a ring, $I$ an ideal of $A$ and $M$ an $A$-module. Assume that either
\begin{enumerate}
    \item[$(\alpha)$] $I$ is nilpotent, or
    \item[$(\beta)$] $A$ is Noetherian and $M$ is ideal-wise separated for $I$.
\end{enumerate}
Then the following are equivalent:
\begin{enumerate}[label=(\arabic*)]
    \item $M$ is $A$-flat;
    \item $\Tor_1^A(N,M)=0$ for all $A_0$-modules $N$;
    \item $M_0$ is $A_0$-flat, and $I\otimes_A M\cong IM$ by the natural map, (note that, if $I$ is a maximal ideal, the flatness over $A_0$ is trivial);
    \item[$(3')$] $M_0$ is $A_0$-flat and $\Tor_1^A(A_0,M)=0$;
    \setcounter{enumi}{3}
    \item $M_0$ is $A_0$-flat, and the canonical maps \[\gamma_n\colon I^n/I^{n+1}\otimes_{A_0}M_0\longrightarrow I^nM/I^{n+1}M\] are isomorphisms;
    \item $M_n=M/I^{n+1}M$ is flat over $A_n=A/I^{n+1}$, for each $n\geqslant0$.
\end{enumerate}
(The implications $(1)\implies (2)\iff (3)\iff (3') \implies (4) \implies (5)$ are true without any assumption on $M$.)
\end{partheorem}

\begin{proof}
We first prove the equivalence of (1) and (5) under the assumption $(\alpha)$ or $(\beta)$. 
\begin{implyenumerate}
    \item[$(1)\implies (5)$] just a change of base (cf.\ref{3.C}).
    \item[$(5)\implies (1)$] The nilpotent case $(\alpha)$ is trivial ($A = A_n$ for some $n$.) In the case $(\beta)$, we prove the flatness of $M$ by showing that, for every ideal $\ideal q$ of $A$, the canonical map $j\colon \ideal q \otimes M\longrightarrow M$ is injective. Since $\ideal q \otimes M$ is $I$-adically separated it suffices to prove that $\ker(j)\subseteq I^n(\ideal q\otimes M)$ for all $n>0$. Fix an $n$. Then there exists, by Artin-Rees, an integer $k>n$ such that $\ideal q \cap I^k \subseteq I^n\ideal q$. Consider the natural maps \[\ideal q \otimes M \varrightarrow{f} \ideal q/(I^k\cap \ideal q)\otimes M\varrightarrow{g} \ideal q/I^n\ideal q = (\ideal q \otimes M)/I^n(\ideal q\otimes M).\]
    Since $M_{k-1}$ is $A_{k-1}$-flat, the natural map \[\ideal q/(I^k\cap \ideal q)\otimes_A M = \ideal q/(I^k\cap \ideal q)\otimes_{A_{k-1}} M_{k-1}\longrightarrow M_{k-1}\] 
    is injective. Therefore $\ker(j)\subseteq \ker(f)$, and a fortiori \[\ker(j)\subseteq \ker(gf) = I^n(\ideal q\otimes M).\] Thus our assertion is proved.
\end{implyenumerate}

Next we prove $(1)\implies (2)\iff (3)\iff (3') \implies (4) \implies (5)$ for arbitrary $M$. $(1)\implies (2)$ is trivial.
\begin{implyenumerate}
    \item[$(2)\implies (3)$]Let $0\longrightarrow N' \longrightarrow N \longrightarrow N'' \longrightarrow 0$ be an exact sequence of $A_0$-modules. Then \[0 = \Tor_1^A(N'',M) \longrightarrow N' \otimes_A M = N' \otimes_{A_0} M_0\longrightarrow N\otimes_A M = N\otimes_{A_0} M_0 \] is exact, so $M_0$ is $A_0$-flat. From the exact sequence $0\longrightarrow I\longrightarrow A\longrightarrow A_0\longrightarrow 0$ we get $0 = \Tor_1^A(A_0,M)\longrightarrow I\otimes M\longrightarrow M$ exact, which proves $I\otimes M\cong IM$.
    \item[$(3)\implies (3')$] Immediate.
    \item[$(3')\implies (2)$] Let $N$ be an $A_0$-module and take an exact sequence of $A_0$-modules $0\longrightarrow R\longrightarrow F_0\longrightarrow N\longrightarrow 0$ where $F_0$ is $A_0$-free. Then \[\Tor_1^A(F_0,M) = 0 \longrightarrow \Tor_1^A(N,M)\longrightarrow R\otimes_{A_0} M_0\longrightarrow F_0\otimes_{A_0} M_0\] is exact and $M_0$ is $A_0$-flat, hence $\Tor_1^A(N,M) = 0$.
    \item[$(2)\implies (4)$] Consider the exact sequences \[0\longrightarrow I^{n+1}\longrightarrow I^n\longrightarrow I^n/I^{n+1}\] and the commutative diagrams \[\begin{tikzcd}
	0 & {I^{n+1}\otimes M} & {I^n\otimes M} & {I^n/I^{n+1}\otimes M} & 0 \\
	0 & {I^{n+1} M} & {I^n M} & {I^nM/I^{n+1}M} & 0
	\arrow[from=1-1, to=1-2]
	\arrow[from=1-2, to=1-3]
	\arrow[from=1-3, to=1-4]
	\arrow[from=1-4, to=1-5]
	\arrow["{\alpha_{n+1}}", from=1-2, to=2-2]
	\arrow["{\alpha_n}", from=1-3, to=2-3]
	\arrow["{\gamma_n}", from=1-4, to=2-4]
	\arrow[from=2-1, to=2-2]
	\arrow[from=2-2, to=2-3]
	\arrow[from=2-3, to=2-4]
	\arrow[from=2-4, to=2-5]
    \end{tikzcd}\] where $\alpha_1,\alpha_2,\dots$ are the natural epimorphisms, the first row is exact by (2) and the second row is of course exact. Since $\alpha_1$ is injective by (3) we see inductively that all $\alpha_n$ are injective. Thus they are isomorphisms, and consequently the $\gamma_n$ are also isomorphisms.
\end{implyenumerate}

Before proving $(4)\implies (5)$ we remark the following fact: if (2) holds then, for any $n \geq 0$ and for any $A_n$-module $N$, we have $\Tor_1^A(N,M) = 0$. In fact, if $N$ is an $A_n$-module and $n>0$, then $IN$ and $N/IN$ are $A_{n-1}$-modules, so that the assertion is proved by induction on $n$.
\begin{implyenumerate}
    \item[$(4)\implies (5)$] We fix an integer $n\geq 0$ and we are going to prove that $M_n$ is $A_n$-flat. For $n = 0$ this is included in the assumptions, so we suppose $n>0$. Put $I_n = I/I^{n+1}$.
    Consider the commutative diagrams with exact rows: 
\[    
\adjustbox{scale = 0.9, center}{
\begin{tikzcd}
	& {{I^{i+1}/I^{n+1} }\otimes M} & {I^i/I^{n+1}\otimes M} & {I^i/I^{i+1}\otimes M} & 0 \\
	0 & {I^{i+1}M_n = I^{i+1}M/I^{n+1}M} & {I^iM_n = I^iM/I^{n+1}M} & {I^iM/I^{i+1}M} & 0
	\arrow[from=1-4, to=1-5]
	\arrow[from=1-3, to=1-4]
	\arrow[from=1-2, to=1-3]
	\arrow[from=2-1, to=2-2]
	\arrow[from=2-2, to=2-3]
	\arrow[from=2-3, to=2-4]
	\arrow[from=2-4, to=2-5]
	\arrow["{\overline{\alpha}_{i+1}}", from=1-2, to=2-2]
	\arrow["{\overline{\alpha}_i}", from=1-3, to=2-3]
	\arrow["{\gamma_i}", from=1-4, to=2-4]
\end{tikzcd}}\]
for $i = 1,2,\dots,n$. Since the $\gamma_i$ are isomorphisms by assumption, and since $\overline{\alpha}_{n+1} = 0$, we see by descending induction on $i$ that all $\overline{\alpha}_i$ are isomorphisms. In particular, \[\overline{\alpha}_1\colon I/I^{n+1}\otimes M = IA_n\otimes_{A_n} M_n \varrightarrow IM_n\] is an isomorphism. Therefore the condition (3) (hence also (2)) holds for $A_n$, $IA_n$, and $M_n$. From this and from what we have just remarked it follows that $\Tor_1^{A_n}(N,M_n) = 0$ for all $A_n$-modules $N$, hence $M_n$ is $A_n$-flat.
\end{implyenumerate}

\end{proof}

\begin{parapplication}[Hartshorne] Let $(B,\ideal n)$ be a Noetherian local ring containing a field $k$ and let $x_1, \dots, x_n$ be a $B$-regular sequence in $\ideal n$. Then the subring $k[x_1,\dots,x_n]$ of $B$ is isomorphic to the polynomial ring $A = k[X_1,\dots,X_n]$, and $B$ is flat over it.
\end{parapplication}
\begin{proof}
Considering the $k$-algebra homomorphism $\phi\colon A\longrightarrow B$ such that $\phi(X_i) = x_i$, we view $B$ as an $A$-algebra. It suffices to prove $B$ is flat over $A$. In fact, any non-zero element $y$ of $A$ is $A$-regular, so under the assumption of flatness it is also $B$-regular, hence $\phi(y)\neq 0$.

We apply the criterion $(3')$ of \ref{thm:049} to $A$, $I = \sum_1^n X_iA$ and $M = B$. The $A$-module $B$ is idealwise separated for $I$ as $IB\subseteq \rad(B)$. Since $A/I = k$ is a field we have only to prove $\Tor_1^A(k,B) = 0$. Now the Koszul complex $K.(X_1,\dots,X_n;A)$ is a free resolution of the $A$-module $k = A/I$ by Cor. to \ref{thm:043} So we have \[\Tor_i^A(k,B) = H_i(K.(X_1,\dots,X_n;A)\otimes_A B)= H_i(K(x_1,\dots,x_n;B)),\] which is zero for $i>0$ as $x_1,\dots,x_n$ is a $B$-regular sequence.
\end{proof}

\begin{parapplication}[EGA III (10.2.4)\cite{egaIII}]. Let $(A,\ideal m, k)$ and $(B,\ideal n, k')$ be Noetherian local rings and $A\longrightarrow B$ a local homomorphism. Let $u\colon M\longrightarrow N$ be a homomorphism of finite $B$-modules, and assume that $N$ is $A$-flat. Then the following are equivalent: \begin{enumerate}[label=(\alph*)]
    \item $u$ is injective, and $N/u(M)$ is $A$-flat;
    \item $\overline{u}\colon M\otimes_A k\longrightarrow N\otimes_A k$ is injective.
\end{enumerate} \end{parapplication}
\begin{proof} (a)$\implies$(b). Immediate. (b)$\implies$(a). Let $x\in \ker(u)$. Then $x\otimes 1= 0$ in $M\otimes k = M/\ideal m M$, therefore $x\in \ideal m M$. We will show $x\in \bigcap_n \ideal m^n M = (0)$ by induction. Suppose $x\in \ideal m^n M$, let $\{a_1,\dots,a_p\}$ be a minimal basis of the ideal $\ideal m^n$ and write $x = \sum a_ix_i, x_i\in M$. Then $u(x) = \sum a_i u(x_i) = 0$ in $N$. By flatness of $N$ there exists $c_{ij}\in A$ and $x_j'\in N$ such that $\sum a_ic_{ij} = 0$ (for all $j$) and such that $u(x_i) = \sum_j c_{ij}x_j'$ (for all $i$). By the choice of $a_1,\dots,a_p$ all the $c_{ij}$ must belong to $\ideal m$. Thus $u(x_i)\in \ideal m N$, in other words $\overline{u}(x_i\otimes 1) = 0$. Since $u$ is injective we get $x_i\in \ideal m M$, hence $x\in \ideal m^{n+1} M$. Thus $u$ is injective and we get an exact sequence \[0\varrightarrow{}M\overline{u} \varrightarrow{}N\varrightarrow{} N/u(M)\varrightarrow{} 0.\] From this and from the hypotheses it follows that $\Tor_1^A(k,n/u(M)) = 0$m which shows the flatness of $N/u(M)$ by \ref{thm:049}.
\end{proof}

\begin{parcorollary} Let $A$ be a Noetherian ring, $B$ a Noetherian $A$-algebra, $M$ a finite $B$-module, and $f\in B$. Suppose that (i) $M$ is $A$-flat, and (ii) for each maximal ideal $P$ of $B$, the element $f$ is $M/(P\cap A)M$-regular. Then $f$ is $M$-regular and $M/fM$ is $A$-flat.\end{parcorollary}
\begin{proof}
If $K$ denotes the kernel of $M\varrightarrow{f} M$, then $K = 0$ iff $K_P = 0$ for all maximal ideals $P$ of $B$. Similarly, by an obvious extension of \ref{3.J}, $M/fM$ is $A$-flat iff $M_P/fM_P$ is flat over $A_{P\cap A}$ for all maximal $P$. The assumptions are also stable under localization. So we may assume that $(A,\ideal m, k)$ and $(B,\ideal n, k')$ are Noetherian local rings and $A\longrightarrow B$ is a local homomorphism. Then the assertion follows from \ref{20.E}.
\end{proof}

\begin{parcorollary} Let $A$ be a Noetherian ring and $B = A[X_1,\dots,X_n]$ a polynomial ring over $A$. Let $f(X)\in B$ such that its coefficients generate over $A$ the unit ideal $A$. Then $f$ is not a zero-divisor of $B$, and $B/fB$ is $A$-flat.\end{parcorollary}

\begin{parapplication} Let $A\longrightarrow B \longrightarrow C$ be local homomorphisms of Noetherian local rings and $M$ be a finite $C$-module. Suppose $B$ is $A$-flat. Let $k$ denote the residue field of $A$. Then $M$ is $B$-flat $\iff$ $M$ is $A$-flat and $M\otimes_A k$ is $B\otimes_A k$-flat. \end{parapplication}

\begin{proof}\phantom{,}
\begin{implyenumerate}
    \item[$\implies$] Trivial.
    \item[$\impliedby$] Use the criterion (4) of Th.\ref{thm:049}.
\end{implyenumerate}
\end{proof}

For more applications of Th.\ref{thm:049}, cf. EGA III \cite{egaIII}.
\end{document}