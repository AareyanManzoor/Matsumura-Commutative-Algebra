\documentclass[../main]{subfiles}
\begin{document}

\section{Graded Rings and Modules}\label{sec:10}

\newparagraph A \defemph{graded ring}\index{graded!\indexline ring} is a ring $A$ equipped with a direct decomposition of the underlying additive group, $A=\bigoplus_{n\geqslant0}A_n$, such that $A_nA_m\subseteq A_{n+m}$. A \defemph{graded $A$-module}\index{graded!\indexline module} is an $A$-module $M$, together with a direct decomposition as a group $M=\bigoplus_{n\in Z}M_n$ such that $A_nM_m\subseteq M_{n+m}$. Elements of $A_n$ (or $M_n$) are called homogeneous elements\index{homogeneous!\indexline element} of degree $n$. A submodule $N$ of $M$ is said to be a \defemph{graded} (or \defemph{homogeneous}\index{homogeneous!\indexline submodule}) \defemph{submodule} if $N = \bigoplus(N\cap M_n)$. It is easy to see that this condition is equivalent to 
\begin{enumerate}[label = (10.$*$), ref = cond:10.$*$]
    \item $N$ is generated over $A$ by homogeneous elements, 
\end{enumerate}
and also to 
\begin{enumerate}[label = (10.$**$), ref = cond:10.$**$]
    \item if $x = x_r + x_{r+1}+\cdots+x_s\in N$ (all $i$), then each $x_i$ is in $N$. 
\end{enumerate}

If $N$ is a graded submodule of $M$, then $M/N$ is also a graded $A$-module, in fact $M/N=\bigoplus M_n/N\cap M_n$.

\begin{parproposition}
Let $A$ be a Noetherian graded ring, and $M$ a graded $A$-module. Then
\begin{enumerate}[label=\roman*)]
    \item any associated prime $\ideal{p}$ of $M$ is a graded ideal, and there exists a homogeneous element $x$ of $M$ such that $\ideal{p}=\Ann(x)$;
    \item one can choose a $\ideal{p}$-primary graded submodule $Q(\ideal{p})$ for each $\ideal{p} \in \Ass (M)$ in such a way that $(0)=\displaystyle\prod_{\ideal{p} \in \Ass(M)} Q(\ideal{p})$.
\end{enumerate}
\end{parproposition}

\begin{proof}
\begin{enumerate}[label = \roman*)]
    \item Let $\ideal{p} \in \Ass(M)$. Then $\ideal{p}=\Ann(x)$ for some $x \in M$. Write \[x=x_e+x_{e-1}+\cdots+x_0\for{x_i \in M_i}.\] Let \[f=f_r+f_{r-1}+\cdots+f_0\in \ideal{p}\for{f_i \in A_i}.\] We shall prove that all $f_i$ are in $\ideal{p}$. We have
    \[0=fx=f_rx_e+(f_{r-1}x_e+f_rx_{e-1})+\cdots+\Big(\sum_{i+j=p} f_i x_j\Big)+\cdots+f_0 x_0.\] 
    Hence 
    \[f_r x_e=0, \quad f_{r-1}x_e+f_rx_{e-1}=0,\quad \dots, \quad f_{r-e}x_e+\cdots+f_r x_0=0\] (we put $f_i=0$ for $i<0$). It follows that $f_r^ex_i=0$ for $0\leqslant i\leqslant e$. Hence $f_r^ex=0$, $f_r^e \in \ideal{p}$, therefore $f_r\in\ideal{p}$. By descending induction we see that all $f_i$ are in $\ideal{p}$, so that $\ideal{p}$ is a graded ideal. Then $\ideal{p} \in \Ann(x_i)$ for all $i$, and clearly $\ideal{p}=\bigcap_{i=0}^e \Ann(x_i)$. Since $\ideal{p}$ is prime this means $\ideal{p}=\Ann(x_i)$ for some $i$.

    \item A slight modification of the proof of \ref{8.G} Th.\ref{thm:011} proves the assertion. Alternatively, we can derive it from Th.\ref{thm:011} and from the following Lemma:
    \begin{lemma}
    Let $\ideal{p}$ be a graded ideal and let $Q \subset M$ be a $\ideal{p}$-primary submodule. Then the largest graded submodule $Q'$ contained in $Q$ (i.e. the submodule generated by the homogeneous elements in $Q$) is again $\ideal{p}$-primary 
    \end{lemma}
    \begin{proof} let $\ideal{p}'$ be an associated prime of $M / Q'$. Since both $\ideal{p}$ and $\ideal{p}'$ are graded, $\ideal{p}'=\ideal{p}$ iff $\ideal{p}'\cap H=\ideal{p} \cap H$ where $H$ is the set of homogeneous elements of $A$. If $a \in \ideal{p} \cap H$ then $a$ is locally nilpotent on $M / Q'$. If $a \in H$, $a \notin \ideal{p}$, then for $x \in M$ satisfying $ax \in Q', x=\sum x_i\for{x_i \in M_i}$, we have $ax_i \in Q' \subseteq Q$ for each $i$, hence $x_i \in Q$ for each $i$, hence $x \in Q'$. Thus $a \notin \ideal{p}'$.
    \end{proof}
\end{enumerate}

\end{proof}

\newparagraph In this book we define a \defemph{filtration}\index{filtration} of a ring $A$ to be a descending sequence of ideals
\[\tag{10.$\dagger$}\label{cond:10.dag}A=J_0 \supseteq J_1 \supseteq J_2 \supseteq \cdots \]

satisfying $J_n J_m \subseteq J_{n+m}$. Given a filtration $(*)$, we construct a graded ring $A'$ as follows. The underlying additive group is \[A'=\bigoplus_{n=0}^\infty J_n / J_{n+1},\] and if $\xi \in A_n'=J_n / J_{n+1}$ and $\eta \in A_m'=J_m / J_{m+1}$, then choose $x \in J_n$ and $y \in J_m$ such that $\xi=x \mod J_{n+1}$ and $\eta=y\mod J_{m+1}$ and put $\xi\eta= xy \mod J_{n+m+1}$. This multiplication is well defined and makes $A'$ a graded ring.

When $I$ is an ideal of $A$, its powers define a filtration $A=I^0\supseteq I \supseteq I^2 \supseteq \cdots$. This is called the \defemph{$I$-adic filtration}, and its associated graded ring is denoted by $\gr^I(A)$.

\begin{parproposition}
If $A$ is a Noetherian ring and $I$ an ideal, then $\gr^I(A)$ is Noetherian.
\end{parproposition}

\begin{proof}
Write $\gr^I(A)=\bigoplus_{n=0}^\infty A'_n$, $A_n'=I^n / I^{n+1}$. Then $A_0'=A / I$ is a Noetherian ring. Let $I=a_1 A+\cdots+a_r A$ and let $\overline{a_i}$ denote the image of $a_i$ in $I / I^2$. Then $\gr^I(A)$ is generated by $\overline{a_1}, \ldots, \overline{a_r}$ over $A_0'$, therefore is Noetherian.
\end{proof}

\newparagraph Let $A$ be an Artinian ring, and $B=A[X_1, \ldots, X_m]$ the polynomial ring with its natural grading. Let $M=\bigoplus_{n=0}^\infty M_n$ be a finitely generated, graded $B$-module. Put $F_M(n)=\length(M_n)$ for $n \geqslant 0$, where $\length(\,\cdot\,)$ denotes the length of $A$-module. The numerical function $F_M$ measures the largeness of $M$. The number $F_M(n)$ is finite for any $n$, because there exists a degree-preserving epimorphism of $B$-modules \[\bigoplus_{i=1}^pB(d_i)\varrightarrow{f}M\] where $B(d)=B$ as a module but $B(d)_n=B_{n-d}$ (in fact, if $M$ is generated over $B$ by homogeneous elements $\xi_1, \ldots, \xi_p$ with $\deg(\xi_i)=d_i$ then the map $f: \bigoplus B(d_i) \longrightarrow M$ such that $f(b_1, \ldots, b_p)=\sum b_i \xi_i$ satisfies the requirement), so that \[\length(M_n) \leqslant \sum \length(B_{n-d_i})<\infty.\]
Note that, since the number of the monomials of degree $n$ in $X_1, \ldots, X$ is $\binom{n+m-1}{m-1}$, we have $F_B(n)=\length(B_n)=\binom{n+m-1}{m-1}\length(A)$.

\begin{partheorem}\label{thm:014}
Let $A, B$ and $M$ be as above. Then there is a polynomial $f_M(x)$ in one variable with rational coefficients such that $F_M(n)=f_M(n)$ for $n \gg 0$ (i.e. for all sufficiently large $n$)
\end{partheorem}

\begin{proof}
Let $P(M)$ denote the assertion for $M$. We consider the graded submodules $N$ of $M$ and we will prove $P(M/N)$ by induction on the largeness of $N$ (note that $M$ satisfies the maximum condition for submodules). For $N=M$ the assertion is obvious. Supposing $P(M / N')$ is true for any graded submodule $N'$ of $M$ properly containing $N$, we prove $P(M / N)$.
\begin{enumerate}[label = Case \arabic*.]
    \item If $N=N_1 \cap N_2$ with $N_i \supset N\for{i=1,2}$, then using $N_1+N_2 / N_1 \cong N_2 / N$ we get
    \[\begin{aligned}
    F_{M / N} &=F_{M / N_2}+F_{N_1+N_2 / N_1} \\
    &=F_{M / N_2}+F_{M / N_1}-F_{M / N_1+N_2}
    \end{aligned}\]
    and the assertion $P(M / N)$ follows from $P(M / N_1)$, $P(M / N_2)$ and\newline $P(M / N_1+N_2)$
    
    \item If $N$ is \defemph{irreducible} (in the sense that it is not the intersection of two larger submodules) then $N$ is a primary submodule of $M$; let $\Ass(M / N)=\{\ideal{p}\}$. Put $I=X_1 B+\cdots+X_mB$ and $M'=M / N$. If $I \subseteq \ideal{p}$ then we claim that $M_n'= 0$ for large $n$. In fact, if $\{\xi_1, \ldots, \xi_p\}$ is a set of homogeneous generators of $M'$ over $B$ and if $d=\max(\deg \xi_i)$, then $M_{d+n}'=I^nM_d'$. On the other hand we have $\ideal{p}^pM'=(0)$ for some $p>0$. Thus $M_n'=0$ for $n>p+d$, and $P(M')$ holds with $f_{M'}=0$. It remains to show the case $I \nsubseteq \ideal{p}$. We may suppose that $X_1 \notin \ideal{p}$. Then the sequence
\[0\longrightarrow(M/N)_{n-1}\varrightarrow{X_1}(M/N)_n\longrightarrow(M/(N+X_1M))_n\longrightarrow0\]
is exact for $n>0$. Since $N+X_1 M \supset N$ there is a polynomial \newline $f(x)=a_d x^d+\cdots+a_0$ with rational coefficients satisfying $P(M/(N+X_1M))$. Thus there is an integer $n_0>0$ such that \[ F_{M / N}(n)-F_{M / N}(n-1)=a_d n^d+\cdots+a_0\quad(n>n_0).\] Then
\[\begin{aligned}
F_{M / N}(n)=& a_d\bigg(\sum_{i=n_0+1}^n i^d\bigg)+a_{d-1}\bigg(\sum_{i=n_0+1}^n i^{d-1}\bigg)+\\
&\cdots+a_0(n-n_0)+F_{M / N}(n_0) \for{n>n_0},
 \end{aligned}\]
which means (cf. the remark below) that $F_{M / N}(n)$ is a polynomial of degree $d+1$ in $n$ for $r>n_0$, as wanted.
\end{enumerate}

\end{proof}

\begin{remark}
Put \[\binom{x}{r}=\dfrac{x(x-1)\cdots(x-r+1)}{r!},\, \binom{x}{0}=1.\] Then any polynomial $f(x)$ of degree $d$ in $\bQ[x]$ can be written \[f(x)=c_d\binom{x+d}{d}+c_{d-1}\binom{x+d-1}{d-1}+\cdots+c_0\binom{x}{0}\for{c_i\in\bQ}.\] Moreover, since $\displaystyle\binom{x+r}{r}-\binom{x+r-1}{r}=\binom{x+r-1}{r-1}$, we have \[f(x)-f(x-1)=c_d\binom{x+d-1}{d-1}+\cdots+c_1\binom{x}{0}.\] It follows by induction on $d$ that, if $f(n) \in \bZ$ for $n \gg 0$, we have $c_i \in \bZ$ for all $i$ (and so $f(n) \in \bZ$ for all $n \in \bZ$). It also follows that, if $F(n)$ is a numerical function such that \[F(n)-F(n-1)=f(n) \text{ for }n>n_0,\] then $\displaystyle F(n)=c_d\binom{n+d+1}{d+1}+\cdots+c_0\binom{n+1}{1}+\text{const}$ for $n>n_0$.
\end{remark}

\begin{remark}
The polynomial $f_M(x)$ of the theorem is called the \defemph{Hilbert polynomial}\index{Hilbert!\indexline polynomial} or the \defemph{Hilbert characteristic function of $M$}\index{Hilbert!\indexline characteristic function}.
\end{remark}

\end{document}