\documentclass[../main]{subfiles}
\begin{document}

\section{Artin-Rees Theorem}\label{sec:11}

\newparagraph Let $A$ be a ring, $I$ an ideal of $A$ and $M$ an $A$-module. We define a \defemph{filtration}\label{def:filtration} of $M$ to be a descending sequence of submodules
\[\tag{11.*}\label{cond:11.*}
    M=M_0 \supseteq M_1 \supseteq M_2 \supseteq \cdots
\]
The filtration is said to be \defemph{$I$-admissible} if $IM_i \subseteq M_{i+1}$ for all $i$, \defemph{$I$-adic} if $M_1=I^iM$, and \defemph{essentially $I$-adic} if it is $I$-admissible and if there is an integer $i_0$ such that $IM_i=M_{i+1}$ for $i>i_0$

Given a filtration (\ref{cond:11.*}), we can define a topology on $M$ by taking \newline $\{x+M_n \mid n=1,2, \ldots\}$ as a fundamental system of neighborhoods of $x$ for each $x \in M$. This topology is separated iff $\bigcap^{\infty} M_n=(0)$. The topology defined by the $I$-adic filtration is called the \defemph{$I$-adic topology}\index{adic topology} of $M$. An essentially $I$-adic filtration defines the $I$-adic topology on $M$, since \[I^i M \subseteq M_i \subseteq I^{i-i_0}M_{i_0} \subseteq I^{i-i_0} M.\]

\begin{parlemma}\label{lem:11.01}
Let $A, I$ and $M$ be as above. Let $M=M_0 \supseteq M_1 \supseteq M_2 \supseteq \cdots$ be an $I$-admissible filtration such that all $M_i$ are finite $A$-modules, let $X$ be an indeterminate and put $A'=\sum I^n X^n$ and $M'=\sum M_n X^n$. Then the filtration is essentially $I$-adic iff $M'$ is finitely generated over $A'$.
\end{parlemma}

\begin{proof}
$A'$ is a graded subring of $A[X]$ and $M'$ is a subgroup of $M \otimes_A A[X]$ such that $A' M' \subseteq M'$, hence $M'$ is a graded $A'$-module. If \newline $M'=A' \xi_1+\cdots+A'\xi_r\for{\xi_i \in M_{d_i}'}$, then $M_n'=(IX) M_{n-1}'$ (hence $M_n=IM_{n-1}$) for $n>\max d_i$. Conversely, if $M_n=IM_{n-1}$ for $n>d$, then $M'$ is generated over $A'$ by $M_{d-1} X^{d-1}+\cdots+M_1X+M_0$, which is, in turn, generated by a finite number of elements over $A$.
\end{proof}

\begin{partheorem}[Artin-Rees]\label{thm:015}
Let $A$ be a Noetherian ring, $I$ an ideal, $M$ a finite $A$-module and $N$ a submodule. Then there exists an integer $r>0$ such that \[I^n M \cap N=I^{n-r}(I^r M \cap N) \quad\text{for }n>r.\]
\end{partheorem}

\begin{proof}
In other words, the theorem asserts that the filtration \newline $I^nM \cap N\for{n=0,1,2, \ldots}$ of $N$ (induced on $N$ by the $I$-adic filtration of $M$) is essentially $I$-adic. The filtration is $I$-admissible, and $N'=\sum(I^n M \cap N) X^n$ is a submodule of the finite $A'$-module $M'=\sum I^n M X^n$, where $A'=\sum I^n X^n$. If $I=a_1 A+\cdots+a_rA$ then $A'=A[a_1 X, \ldots, a_r X]$, so that $A'$ is Noetherian. Therefore $N'$ is finite over $A'$. Thus the assertion follows from the preceding lemma.
\end{proof}

\begin{remark}
It follows that the $I$-adic topology on $M$ induces the $I$-adic topology on $N$. This is not always true if $M$ is infinite over $A$.
\end{remark}

\begin{partheorem}[Intersection theorem]\label{thm:016}
Let $A$, $I$ and $M$ be as in the preceding theorem, and put $N=\bigcap^\infty I^nM$, Then we have $IN=N$.
\end{partheorem}

\begin{proof}
For sufficiently large $n$ we get \[N=I^n M \cap N=I^{n-r}(I^r M \cap N) \subseteq IN \subseteq N.\]
\end{proof} 

\begin{corollary}\label{cor:11.01}
If $I \subseteq \rad(A)$ then $\displaystyle \bigcap^\infty I^n M=(0)$. In other words $M$ is $I$-adically separated in that case.
\end{corollary}

\begin{corollary}[Krull]\label{cor:11.02}
Let $A$ be a Noetherian ring and $I=\rad(A)$. Then $I^n=(0)$. 
\end{corollary}

\begin{corollary}[Krull]\label{cor:11.03}
Let $A$ be a Noetherian domain and let $I$ be any proper ideal. Then $\displaystyle \bigcup^\infty I^n=(0)$.
\end{corollary}

\begin{proof}
Putting $N=\bigcap I^n$ we have $IN=N$, whence there exists $x \in I$ such that $(1+x) N=(0)$ by \ref{1.M}. Since $A$ is an integral domain and since $1+x\neq 0$, we have $N=(0)$.
\end{proof}

\begin{parproposition}\label{pro:11.01}
Let $A$ be a Noetherian ring, $M$ a finite $A$-module, $I$ and ideal, and $J$ an ideal generated by $M$-regular elements. Then there exists $r>0$ such that \[I^nM: J=I^{n-r}(I^r M: J) \quad\text{for } n>r.\]
\end{parproposition}

\begin{proof}
Let $J=a_1A+\cdots+a_pA$ where $a_i$ are $M$-regular. Let $S$ be the multiplicative subset of $A$ generated by $a_1,\ldots,a_p$, and consider the $A$-submodules $a_j^{-1}M$ of $S^{-1}M$. Put $L=a_1^{-1}M \oplus\cdots\oplus a_p^{-1}M$ and let $\Delta_M$ be the image of the diagonal map $x \mapsto (x, x, \ldots, x)$ from $M$ to $L$. Then $M\cong\Delta_M$, and \[I^n M: J=\bigcap_j(I^nM:a_j)=\bigcap(I^na_j^{-1} M \cap M) \cong I^nL\cap\Delta_M,\] so that the assertion follows from the Artin-Rees theorem applied to $L$ and $\Delta_M$.
\end{proof}

\end{document}