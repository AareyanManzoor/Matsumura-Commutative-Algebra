\documentclass[../main]{subfiles}
\begin{document}

\section*{Remarks to Chapter 7}
\begin{enumerate}[label=\arabic*]
    \item As Th.\ref{thm:035} suggests, regular local rings are similar to polynomial rings or power series rings in many aspects. In particular, the inequality on the dimension \ref{14.I} can be extended to an arbitrary regular local ring. Namely, in the non-local form one has the following theorem (due to Serre): Let $A$ be a regular ring, $P_{i}\for{i=1,2}$ prime ideals of $A$ and $Q$ a minimal prime over-ideal of $P_{1}+P_{2}$. Then
    \[\Ht(Q)\leq \Ht(P_1) +\Ht(P_2).\]
    For the proof see \cite[Ch.V, p.18.]{serre2000local}\footnote{The original book cites \cite{serre2009algèbre},the original french version of this.} 
    \item A normal domain $A$ is called a Krull ring if \begin{enumerate}[label=(\arabic*)]
        \item for any non-zero element $x$ of $A$, the number of prime ideals of $A$ of height one containing $x$ is finite, and
        \item $A= \displaystyle\bigcap_{\Ht(\ideal{p})=1} A_\ideal{p}.$
    \end{enumerate}
    Noetherian normal rings are Krull, but not conversely. If $A$ is a Noetherian domain, then the integral closure of $A$ in the quotient field of $A$ is a Krull ring (Theorem of Y. Mori, cf. \cite{nagata1975local}). On Krull rings, cf. \cite{bourbaki1998commutative}.
    \item P. Samuel has made an extensive study on the subject of unique factorization. Cf.\cite{samuel1964lectures}
    \item We did not discuss valuation theory. On this topic the following paper contains important results in connection with algebraic geometry: \cite{abhyankar1956on}.
\end{enumerate}

\end{document}