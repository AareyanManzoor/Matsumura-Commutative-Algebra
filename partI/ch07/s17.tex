\documentclass[../main]{subfiles}
\begin{document}

\section{Classical Theory}\label{sec:17}

\newparagraph Let $A$ be an integral domain, and $K$ be its quotient field, We say that $A$ is \defemph{normal}\index{normal domain} if it is integrally closed in $K$. If $A$ is normal, so is the localization $S^{-1}A$ for every multiplicatively closed subset $S$ of $A$ not containing $0$,
Since $A = \bigcap_{\text{all max }\ideal{p}} A_\ideal{p}$. by \ref{1.H}, the domain $A$ is normal iff $A_\ideal{p}$ is normal for every maximal ideal $\ideal{p}$. 

An element $u$ of $K$ is said to be \defemph{almost integral over $A$}\index{almost integral} if there exists an element $a$ of $A$ $(a\neq 0)$ such that $au^n \in A$ for all $n>0$. If $u$ and $v$ are almost integral over $A$, so are $u+v$ and $uv$. If $u\in K$ is integral over $A$ then it is almost
integral over $A$, The converse is also true when $A$ is Noetherian. In fact, if $a\neq 0$ and $au^n\in A\for{n=1,2,\dots}$, then $A[u]$ is a submodule of the finite $A$-module $a^{-1}A$, whence $A[u]$ itself is finite over $A$ and $u$ is integral over $A$, We say that $A$ is completely normal if every element $u$ of $K$ which is
almost integral over $A$ belongs to $A$. For a Noetherian domain normality and complete normality coincide. Valuation rings of rank (= Krull dimension) greater than one (cf.\cite{nagata1975local} and \cite{zariski2014commutative}) are normal but not completely normal. 

We say (in accordance with the usage of \cite{egaIII}) that a ring $B$ is \defemph{normal}\index{normal ring} if $B_\ideal{p}$ is a normal domain for every prime ideal $\ideal{p}$ of $B$. A Noetherian normal ring is a direct product of a finite number of normal domains.

\begin{parproposition}
\begin{enumerate}[label=(\arabic*)]
    \item Let $A$ be a completely normal domain. Then a polynomial ring $A[X_1,\dots,X_n]$ over $A$ is also completely normal. Similarly for a formal power series $A[[X_1,\dots,X_n]]$. 
    \item Let A be a normal ring. Then $A[X_1,\dots,X_n]$ is normal.
\end{enumerate}
\end{parproposition}

\begin{proof}
\begin{enumerate}[label = (\arabic*)]
    \item Enough to treat the case of $n= 1$. Let $K$ denote the quotient field of $A$, Then the quotient field of $A[X]$ is $K(X)$. Let $u\in K(X)$ be almost integral over $A[X]$. Since $A[X]\subseteq K[X]$ and since $K[X]$ is completely normal (because of unique factorization), the element $u$ must belong to $K[X]$. Write \[u=\alpha_r X^r +\alpha_{r+1}X^{r+1}+\dots+\alpha_dX^d\for{\alpha_r\neq 0}\]
    Let \[f(X)=b_sX^s+b_{s+1}X^{s+1}+\dots+b_tX^t\in A[X]\] be such that $fu^m\in A[X]$ for all $n$. Then $b_s\alpha_r^n \in A$ for all $n$ so that $\alpha_r\in A$. Then $u-\alpha_rX^r = \alpha_{r+1}X^{r+1}+\cdots$ is almost integral over $A[X]$, so we get $\alpha_{r+1}\in A$ as before, and so on. Therefore $u\in A[X]$. The case of $A[[X]]$ is proved similarly.
    \item Let $P$ be a prime ideal and let $\ideal{p} = P\cap A$, Then $A[X]_P$, is a localization of $A_\ideal{p}[X]$ and $A_\ideal{p}$ is a normal domain. So we may assume that $A$ is a normal domain with quotient field $K$. Let $u= P(X)/Q(X) \for{P,Q \in A[X]}$ be such that \[u^d+f_1(X)u^{d-1}+\dots+f_d(X)=0\text{ with }f_i\in A[X]\] In order to prove that $u\in A[X]$, we consider the subring $A_0$ of $A$ generated by $1$ and by the coefficients of $P,Q$ and all the $f_i(X)$. Then $u$ is in the quotient field of $A_0 [x]$ and is integral over $A_0 [X]$. The proof of (1) shows that $u$ is a polynomial in $X$: $u = \alpha_r X^r +\dots + \alpha_dX^d$, and that each coefficient $\alpha_i$ is almost integral over $A_0$. As $A_0$ is Noetherian,$\alpha_i$ is integral over $A_0$ and a fortiori over $A$. Therefore $\alpha_i\in A$, as wanted.
\end{enumerate}
\end{proof}

\begin{remark}
There exists a normal ring $A$ such that $A[[X]]$ is
not normal (\cite{seidenberg1966derivations}).
\end{remark}

\newparagraph Let $A$ be a ring and $I$ an ideal with $\bigcap_{n=1}^\infty I^n=(0)$. Then for each non-zero element a of $A$ there is an integer $n \geq 0$ such that $a \in I^n$ and $a \notin I^{n+1}$. We then write $n=\ord (a)$ (or $\ord_I(a)$) and call it the \defemph{order of $a$}\index{order} (with respect to $I$).
We have \[\ord(a+b)\geq \min (\ord(a),\ord(b))\text{ and } \ord(ab)\geq \ord(a)+\ord(b).\]

Put \[A'=\gr^I(A) \oplus \bigoplus_{n\geq 0} I^n/I^{n+1}.\] For an element $a$ of $A$ with $\ord(a)=n$, we call the image of $a$ in $I^n / I^{n+1}=A_n'$ the \defemph{leading form}\index{leading form} of $a$ and denote it by $a^{\ast}$. We define $0^{\ast}=0$ ($\in A'$). The map a $H a^{\star}$ is in general neither additive nor multiplicative, but if $a^{\ast} b^{\ast} \neq 0$ (i.e. if $\ord(a b)=\ord(a)+$ ord (b)) then we have $(a b)^\ast=a^\ast b^\ast$, and if $\ord(a)=\ord(b)$ and $a^{\ast}+b^{\ast} \neq 0$ then we have $(a+b)^{\ast}=a^{\ast}+b^{\ast}$. It follows that, for any ideal $Q$ of $A$, the set $Q^{\ast}$ of the leading forms of the elements of $Q$ is a graded ideal of $A'$. Warning: if $Q=\sum a_iA$ it does not necessarily follow that $Q^{\ast}=\sum a_i^{\ast} A'$.But if $Q$ is a principal ideal $aA$ and if $A'$ is a domain, then we have $Q^{\ast}=a^{\ast} A'$.

%will continue here tomorrow
Put $\overline{A} = A/Q$ and $\overline{I}= (I+Q)/Q$. Then it holds that $\gr^{\overline{I}}(\overline{A}) = \gr^I(A)/Q^\ast$. In fact, we have \[\overline{I}^n/\overline{I}^{n+1}=(I^n+Q)/(I^{n+1}+Q)\cong I^n/(I^n\cap (I^{n+1}+Q))= I^n/(I^n\cap Q+ I^{n+1}) = A'_n/Q^\ast_n.\] 


\begin{partheorem}[Krull]\label{thm:034}
 Let $A, I$ and $A'$ be as above. Then
 \begin{enumerate}
     \item if $A'$ is a domain, so is $A$;
     \item suppose that $A$ is Noetherian and that $I \subseteq \rad(A)$. Then, if $A'$ is a normal domain, so is A.
 \end{enumerate}
\end{partheorem}


\begin{proof}
    \begin{enumerate}
        \item Let $a$ and $b$ be non-zero-elements of $A$. Then $a^\ast \neq 0$ and $b^{\ast} \neq 0$, hence $(a b)^{\ast}=a^{\ast} b^\ast \neq 0$ and so $a b \neq 0$.
        \item The ring $A$ is a domain by 1). Tet $a, b \in A$, $b \neq 0$, and suppose that $a/b$ is integral over $A$. We have to prove $a \in bA$. The $A$-module $A/bA$ is separated in the $I$-adic topology by \ref{11.D}, in other words $b A=\bigcap_{n=1}^\infty(bA+I^n)$. Therefore it suffices to prove that $a \in bA +I^n$ for all $n$. Suppose that $a \in bA +I^{n-1}$ is already proved. Then $a=b r+a'$ with $r \in A$ and $a' \in I^{n-1}$, and $a' / b=a / b-r$ is integral over $A$. So we can replace $a$ by $a'$ and assume that $a \in I^{n-1}$. We are to prove $a \in bA+I^n$. Since $a/b$ is almost integral over $A$ there exists $0 \neq c \in A$ such that $c a^m \in b^m A$ for all $m$. As $A'$ is a domain the map $a \mapsto a^{\ast}$ is multiplicative, hence we have $c^{\ast} a^{\ast^m} \in b^{\ast^m} A'$ for all m, and since $A'$ is Noetherian (by \ref{10.D}) and normal we have $a^\ast\in b^\ast A'$. Let $c\in A$ be such that $a^\ast = b^\ast c^\ast$. Then \[n-1<\ord(a)<\ord(a-bc),\] whence, $a-bc\in I^n$ so that $a\in bA+I^n$.
    \end{enumerate}
\end{proof}
\begin{remark}
 Even when $A$ is a normal domain it can happen that $A'$ is not a domain. Example: \[A=k[x, y, z]=k[X, Y, Z] / (Z^2-X^2-Y^3),\] where $k$ is a field of characteristic $\neq 2$, and $I=(x, y, z)$. We have \[A'=\gr^I(A) \simeq k[X, Y, Z] /(Z^2-X^2),\] so $(x^{\ast}-z^{\ast})(x^{\ast}+z^{\ast})=0$. On the other hand $A$ is normal. In general, a ring of the form $k[x_1, \ldots, x_n, Z] /(Z^2-f(X))$ is normal provided that $f(X)$ is square-free.
\end{remark}

\newparagraph Let $(A,\ideal{m},k)$ be a Noetherian local ring of dimension $d$. Recall that the ring $A$ is said to be regular if $\ideal{m}$ is generated by $d$ elements, or what amounts to the same, if $d=\rank_k \ideal{m} / \ideal{m}^2$(cf. \ref{12.J}). A regular local ring of dimension $0$ is nothing but a field. The formal power series ring $k[[X_1, \dots, X_d]]$ over a field $k$ is a typical example of regular local ring.

\begin{theorem}\label{thm:035}
  Let $(A, \ideal{m}, k)$ be a Noetherian local ring. Then A is regular iff the graded ring $\gr(A)=\bigoplus \ideal{m}^n / \ideal{m}^{n+1}$ associated to the $\ideal{m}$-adic filtration is isomorphic (as a graded $k$-algebra) to a polynomial ring $k[X_1, \ldots, X_d]$.
\end{theorem}

\begin{proof}
  Suppose $A$ is regular, and let $\{x_1, \dots, x_d\}$ be a regular system of parameters. Then $\gr(A) = k[x_1^\ast,\dots,x_d^\ast]$, hence $\gr(A)$ is of the form $k[X_1, \ldots, X_d] / I$ where $I$ is a graded ideal. If $I$ contains a homogeneous polynomial $F(X) \neq 0$ of degree $n_0$ then we would have, for $n>n_0$,
  \[\length(A/\ideal{m}^{n+1}) \leq \binom{n+d}d -\binom{n-n_0+d}d = \text{a polynomial of degree }d-1\text{ in }n\]
  But the Hilbert function $\length(\Lambda/\ideal{m}^n)$ of $A$ is a polynomial in $n$ (for large $n$) of degree $d$ by \ref{12.H}. Therefore the ideal $I$ must be $(0)$.
  
  Conversely, suppose $\gr(A) \simeq k[X_1, \ldots, X_d]$. Then we get $\dim A=d$ from the consideration of the Hilbert polynomial, while \[\rank_k \ideal{m} / \ideal{m}^2=\rank_k(k x_1+\ldots+k_d)=d.\] Thus $A$ is regular.
\end{proof}

\begin{partheorem}\label{thm:036}
  Let $A$ be a regular local ring and $\{x_1, \ldots, x_d\}$ a regular system of parameters. Then:
  \begin{enumerate}[label=(\arabic*)]
  \item $A$ is a normal domain;

  \item $x_1, \ldots, x_d$ is an $A$-regular sequence, and hence $A$ is a Cohen-Macaulay local ring;

  \item $(x_1, \ldots, x_i)=\ideal{p}_i$ is a prime ideal of height $i$ for each $1 \leqslant i \leqslant d$, and $A /\ideal{p}_i$ is a regular local ring of dimension $d-1$

  \item conversely, if $\ideal{p}$ is an ideal of $A$ and if $A / \ideal{p}$ is regular and has dimension $d-i$, then there exists a regular system of parameters $\{y_1, \ldots, y_d\}$ such that $\ideal{p}=(y_1, \ldots, y_i)$.

\end{enumerate}
\end{partheorem}

\begin{proof}
\begin{enumerate}[label=(\arabic*)]
    \item follows from \ref{thm:034} and \ref{thm:035}.
    
    \item follows from \ref{thm:027} as well as from 3) below.
    
    \item We have $\dim(A / \ideal{p}_i)=d-1$ by \ref{12.K}, while the maximal ideal $\ideal{m}/ \ideal{p}_1$ of $A/ \ideal{p}_i$ is generated by $d-i$ elements $\overline{x}_{i+1}, \dots, \overline{x}_d$. Therefore $A/\ideal{p}_i$ is regular, and hence $\ideal{p}_i$ is a prime by 1).
    
    \item Put $\overline{\ideal{m}}=\ideal{m} / \ideal{p}$. Then \[d-i=\rank_k(\overline{\ideal{m}}/\overline{\ideal{m}}^2) = \rank_k \ideal{m}/(\ideal{m}^2+\ideal{p}) = \rank_k \ideal{m}/\ideal{m}^2 = \rank_k (\ideal{m}^2+\ideal{p})/\ideal{m}^2\] hence $i=\rank_k (\ideal{m}^2+\ideal{p})/\ideal{m}^2$. Thus we can choose $i$ elements $y_1,\dots, y_i$ of $\ideal{p}$ which spans $\ideal{p}+\ideal{m}^2 \mod \ideal{m}^2$ over $k$, and $d-i$ elements $y_{i+1},\dots , y_d$ of $\ideal{m}$ which, together with $y_1,\dots,y_i$, span $\ideal{m}\mod \ideal{m}^2$ over $k$. Then $\{y_1,\dots y_d\}$ is a regular system of parameters of $A$, so that $(y_1,\dots,y_i) =\ideal{p}'$ is a prime ideal of height $i$ by 3). as $\ideal{p}\supseteq \ideal{p}'$ and $\dim(A/\ideal{p})=d-i$, we must have $\ideal{p}=\ideal{p}'$.
\end{enumerate}
\end{proof}

\newparagraph Let $A$ be a regular local ring of dimension $1$, and let $P= aA$ be the maximal ideal of $A$. Then the non-zero ideals of $A$ are the powers $P^n=a^nA$. (Proof: if $I$ is an ideal and $I \neq 0$, then there exists $n \geqslant 0$ such that $I \subseteq P^n=a^n A$ and $I \nsubseteq P^{n+1}$. Then $a^{-n} I$ is an ideal of $A$ not contained in the maximal ideal $P$, therefore $a^{-n} I=A$, i.e. $I=a^n A$, as claimed.) Thus $A$ is a principal ideal domain. Furthermore, any fractional ideal (that is, finitely generated non-zero $A$-submodule of the quotient field $K$ of $A$) is equal to some $a^n A\for{n\in\bZ}$. If $0 \neq x \in K$ and $x A=a^n A$, then we write $n=\ord(x)$. Then $x \mapsto \ord(x)$ is a valuation of $K$ with $\bZ$ as the value group, and $A$ is the ring of the valuation. Conversely, let $v$ be a valuation of $K$ whose value group is discrete and of rank $1$ (i.e. isomorphic to $\bZ$); then the valuation ring $R_v$ of $v$ is called a principal valuation ring or a discrete valuation ring of rank $1$, and is a regular local ring of dimension $1$. Thus a principal valuation ring and a one-dimensional regular local ring are the same thing. On the contrary, no other kinds of valuation rings are Noetherian.

In the next paragraph we shall learn another characterization (Th.\ref{thm:037}) of the one-dimensional regular local rings.

\newparagraph Let $A$ be a Noetherian domain with quotient field $K$. For any non-zero ideal $I$ of $A$ we put $I^{-1}=\{x \in K \mid x I \subseteq A\}$. We have $A \subseteq I^{-1}$ and $I-I^{-1} \subseteq A$

\begin{lemma}\label{lem:17.01}
Let $0 \neq a \in A$ and $P \in \Ass_A(A/aA)$. Then $P^{-1} \neq A$.
\end{lemma} 
\begin{proof}
By the definition of the associated primes there exists $b \in A$ such that $(a A: b)=P$. Then $(b / a) P \subseteq A$ and $b/a \notin A$.
\end{proof}

\begin{lemma}\label{lem:17.02}
Let $(A, P)$ be a Noetherian local domain such that $P \neq 0$ and $P P^{-1}=A$. Then $P$ is a principal ideal, and so $A$ is regular of dimension $1$.
\end{lemma} 
\begin{proof}
Since $\bigcap_{n=1}^{\infty} P^n=(0)$ by \ref{11.D} Cor.\ref{cor:11.03}, we have $P \neq P^2$. Take \newline $a\in P-P^2$. Then $aP^{-1} \subseteq A$, and if $aP^{-1} \subseteq P$ then $aA=a P^{-1} P \subseteq P^2$, contradicting the choice of a. Therefore we must have $aP^{-1}=A$, that is, \newline $aA=aP^{-1} P=P$.
\end{proof}

\begin{theorem}\label{thm:037}
   Let $(A, P)$ be a Noetherian local ring of dimension $1$. Then $A$ is regular iff it is normal.
\end{theorem}
\begin{proof}
Suppose $A$ is normal (hence a domain). By Lemma \ref{lem:17.02} it suffices to show $PP^{-1}=A$. Assume the contrary. Then $PP^{-1}=P$, and hence\newline $P(P^{-1})^n=P \subseteq A$ for any $n>0$. Therefore all the elements of $P^{-1}$ are integral over $A$, whence $P^{-1}=A$ by the normality. But, as $\dim A=1$, we have $P \in \Ass(A / aA)$ for any non-zero element a of $P$ so that $P^{-1} \neq A$ by Lemma \ref{lem:17.01}. Thus $PP^{-1}=P$ cannot occur.
\end{proof}

\begin{theorem}\label{thm:038}
   Let $A$ be a Noetherian normal domain. Then any non-zero principal ideal is unmixed, and it holds that \[A=\bigcap_{\Ht(\ideal{p})=1} A_{\ideal{p}}.\] If $\dim A \leqslant 2$ then $A$ is Cohen-Macaulay.
\end{theorem}

\begin{proof}
Let $a\neq 0$ be a non-unit of $A$ and let $P \in\Ass(A / a A) .$ Replacing $A$ by $A_P$ we may suppose that $(A, P)$ is local. Then we have $P^{-1} \neq A$ by Lemma \ref{lem:17.01}, and if $\Ht(P)>1$ we would have a contradiction as in the preceding proof. Thus $\Ht(P)=1$. This implies that $aA$ is unmixed. The other assertions of the theorem follow immediately from that.
\end{proof}

\newparagraph Let $A$ be a Noetherian ring. Consider the following conditions about $A$ for $k=0,1,2, \ldots$ : 
\begin{enumerate}
    \item[$(S_k)$] it holds $\depth (A_{\ideal{p}}) \geqslant \inf (k, \Ht(\ideal{p}))$ for all $\ideal{p} \in \Spec(A)$, and
    \item[$(R_k)$] if $\ideal{p} \in \Spec(A)$ and $\Ht(\ideal{p}) \leqslant k$, then $A_{\ideal{p}}$ is regular.
\end{enumerate}
 The condition $(S_0)$ is trivial. The condition $(S_1)$ holds iff $\Ass(A)$ has no embedded primes. The condition $(S_2)$, which is probably the most important, is equivalent to that not only $\Ass(A)$ but also $\Ass(A/fA)$ for every non-zero-divisor $f$ of $A$ have no embedded primes. The ring $A$ is C.M. iff it satisfies all $(S_k)$.
 
If $(R_0)$ and $(S_1)$ are satisfied then $A$ is reduced, and conversely, The following theorem is due to Krull(1931) in the case $A$ is a domain, and to Serre in the general case.
\begin{theorem}[Criterion of normality]\label{thm:039}
A Noetherian ring is normal iff it satisfies $(S_2)$ and $(R_1)$
\end{theorem}
\begin{proof}
(After EGA IV 2 p.108 \cite{egaIV}). Let $A$ be a Noetherian ring.
Suppose first that $A$ is normal, and let $\ideal{p}$ be a prime ideal. Then $A_{\ideal{p}}$ is a field for $\Ht(\ideal{p})=0$, and regular for $\Ht(\ideal{p})=1$ by Th.\ref{thm:037}, hence the condition $(R_1)$. Since a normal local ring is a domain, Th.\ref{thm:037} implies that $A$ satisfies $(S_2)$.

Next suppose that $A$ satisfies $(S_2)$ and $(R_1)$. Then $A$ is reduced. Let $\ideal{p}_1, \dots, \ideal{p}_r$ be the minimal prime ideals of $A$.Thus we have $(0)=\ideal{p}_1 \cap \ldots \cap \ideal{p}_{r}$. The total quotient ring $\Phi A$ (cf. \ref{1.O}) of $A$ is isomorphic to the direct product $K_1 \times \ldots \times K_r$, where $K_i$ is the quotient field of $A / \ideal{p}_i$; this follows from \ref{1.C} applied to $\Phi A$.

We shall prove that $A$ is integrally closed in $\Phi A$. Suppose this is done; then the unit element $e_i$ of $K_i$ belongs to $A$ since $e_i^2-e_i=0$, and we have $1=e_1+\ldots+e_r$ and $e_i e_j=0\for{i \neq j}$. Therefore $A=Ae_1 \times \dots \times Ae_r$, and $A e_i$ is a normal domain as it is integrally closed in $K_i$; thus $A$ is a normal ring. So suppose
\[(a / b)^n+c_1(a / b)^{n-1}+\ldots+c_n=0 \text { in } \Phi A\]
where $a, b$ and the $c_i$'s are elements of $A$ and $b$ is $A$-regular. This is equivalent to $a^n+\sum c_i a^{n-i_b}=0$. We want to prove $a\in bA$. Since $bA$ is unmixed of height $1$ by $(S_2)$, we have only to show that $a_{\ideal{p}} \in b_{\ideal{p}} A_{\ideal{p}}$ for all prime ideals $\ideal{p}$ of height $1$ (where $a_{\ideal{p}}$ and $b_{\ideal{p}}$ are the canonical images of $a$ and $b$ in $A_{\ideal{p}}$). But $A_{\ideal{p}}$ is normal by $(R_1)$ for such $\ideal{p}$, and we have \[a_{\ideal{p}}^n+\sum (c_i)_\ideal{p} a_\ideal{p}^{n-i} b_\ideal{p}^i=0,\] therefore $a_\ideal{p}\in b_\ideal{p} A_\ideal{p}$.
\end{proof}

\begin{partheorem}\label{thm:040}
Let $A$ be a ring such that for every prime ideal $\ideal{p}$ the localization $A_{\ideal{p}}$ is regular. Then the polynomial ring $A[X_1, \ldots, X_n]$ over $A$ has the same property.
\end{partheorem} 
\begin{proof}
As in the proof of \ref{16.D} Th.\ref{thm:033}, we are led to the following situation: $(A, \ideal{p})$ is a regular local ring, $n=1$ and $P$ is a prime ideal of $B=A[X]$ lying over $\ideal{p}$. And we have to prove $B_P$ is regular. In this circumstance we have $P \supseteq \ideal{p} B$ and $B / \ideal{p} B=k[X]$, where $k=A / \ideal{p}$ is a field. Therefore either $P=\ideal{p} B$, or $P=\ideal{p} B+f(X) B$ with a monic polynomial $f(X)$ in $B$. Put $d = \dim A$. Then $\ideal{p}$ is generated by $d$ elements, so $P$ is generated by $d$ elements over $B$ if $P=\ideal{p} B$, and by $d+1$ elements if $P=\ideal{p}B+fB$. Pn the other hand it is clear that $\Ht(\ideal{p}B)\geq d$, so we have $\Ht(P)=d$ in the former case and $\Ht(P)=d+1$ in the latter case by \ref{12.I} Th.\ref{thm:018}. Therefore $B_P$ is regular.
\end{proof}

In particular, all local rings of a polynomial ring $k[X_1, \ldots, X_n]$ over a fieid are regular.

\end{document}